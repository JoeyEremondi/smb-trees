% !TEX root =  main.tex
% !TeX spellcheck = en-US

\section{Discussion}
\label{sec:discussion}

\subsection{Comparison to Other Ordinal Systems}

In the literature, many different variations of ordinals have been presented.
To keep our comparison brief, we refer to the work of
\citet{KRAUS2023113843}.
They give a comprehensive overview
of ordinal notation systems in type theory, with a detailed
comparison of their comparative properties.
They define three different systems: Cantor normal forms
that represent ordinals as binary trees, restricted Brouwer trees that represent
ordinals as infinitely branching trees, and well-founded types
that represent ordinals as types with a certain sort of relation on
their elements.


The definitions \citeauthor{KRAUS2023113843} give are more restrictive than
ours. For example, for Brouwer trees they require that $\Lim$ only operate
on functions that are strictly increasing, preventing
the definition of $\limMax$. These restrictions
make their ordinals very well-behaved with respect to propositional equality,
so they can
examine their mathematical properties.
SMB-trees have less rich theory, but the properties
they do satisfy are specifically tailored to proving
termination of higher-order programs.

\paragraph{Transitivity, Extensionality and Well-foundedness}
\Citeauthor{KRAUS2023113843} show three properties for each system they present:
transitivity of the ordering, well-foundedness (as in \cref{subsec:wf}),
and \textit{extensionality}, the property that two ordinals are equal
iff their sets of smaller terms are equal. They also show a strict version
of extensionality for each system: to ordinals are equal iff their sets of
strictly smaller terms are equal.

SMB-trees satisfy each of the above properties: the transitivity of $\le$
is inherited from Brouwer trees, and we show well-foundedness in \cref{subsec:wf}.
Extensionality for $\le$ is trivially true for our setoid version of equivalence.
For propositional equality, extensionality cannot be proved without some form of quotient type.
We conjecture that the strict order $<$  is not extensional for SMB-trees,
since it does not hold for Brouwer trees without quotient types.

Well-founded types lack a basic transitivity property for the strict order:
without additional axioms, one cannot conclude $x < z$ from $x \le y$ and $y < z$.
So, though well-founded types have binary and infinitary suprema like SMB-trees,
they lack the basic principles for reasoning about strict orders, making them ill-suited
for defining recursive procedures.

\paragraph{Classifiability}
Classifiability is the property that each ordinal is either zero, a successor,
or a limit, and that exactly one of those properties holds.
Restricted Brouwer trees and Cantor normal forms both satisfy classifiability,
but SMB-trees do not. Even our version of Brouwer trees do not have this property:
since we allow non-increasing sequences, the limit of the constant-zero sequence
is equivalent to zero.

Not having classifiability does negatively affect the decidability properties of SMB-trees.
For example, for restricted Brouwer trees, it is decidable whether a tree is
infinite or not, but this is not the case for SMB-trees, since some limits are
actually finite. However, since SMB-trees are defined specifically around
well-founded recursion, losing decidability properties is an acceptable
compromise. Additionally, the ability to reason about SMB-trees using the
equational style reduces the need to pattern match on them.

\paragraph{Joins and Suprema}

The main novelty of SMB-trees is the existence of both binary suprema (joins) and infinitary suprema (limits)
that interact well with the strict ordering.
Cantor normal forms have binary joins and strict monotonicity (as a by-product of decidable ordering),
but lack infinitary joins.
Well-founded types have binary and infinitary suprema,
but without additional axioms even their successor function is not monotone,
so strict monotonicity is out of the question.
For restricted Brouwer trees,
binary joins cannot exist without further axioms. This is an artifact of
allowing $\Lim$ only on strictly increasing sequences, since it disallows
$\limMax$ or other similar constructs.
So even without strict monotonicity, the capability of SMB-trees exceeds that of
restricted Brouwer trees. The cost of this is that SMB-trees fulfill fewer nice properties
with respect to propositional equality. Since setoid reasoning is sufficient for well-founded recursion,
we find this tradeoff acceptable.


\paragraph{Conclusion}
Designing an ordinal library is an exercise in compromise, balancing the desired properties
with the limitations of decidability and constructive reasoning.
With SMB-trees, we have identified a point in the design space
well suited to proving termination. The algebraic framework of SMB-trees
lays the groundwork for future developments on reasoning mechanically
about ordinals. Beyond of our specific use-case, the development of
SMB-trees shows that sometimes careful design with dependent types
can avoid the need for additional axioms or language features.

\begin{acks}
  Thanks to Ron Garcia, \'Eric Tanter, Jonathan Chan, and Ohad Kammar for their feedback and support.
\end{acks}
