% !TeX root = main.tex
% !TeX spellcheck = en-US

Gradual dependent types can help with the incremental adoption of dependently typed code by providing a principled semantics for \textit{imprecise} types and proofs, where some parts have been omitted.
Current theories of gradual dependent types, though,  lack a central feature of type theory: propositional equality.
Lennon-Bertrand et al. show that, when the reflexive proof $\mathit{refl}$ is the only closed
value of an equality type, a gradual extension of the Calculus of Inductive Constructions (CIC) with propositional equality violates static observational equivalences.
Extensionally-equal functions should be indistinguishable at run time, but
they can be distinguished using
 a combination of equality and type imprecision.

This work presents a gradual dependently typed language that supports propositional equality.
We avoid the above issues by devising an equality type of which $\mathit{refl}$
is not the only closed inhabitant.
Instead, each equality proof is accompanied by a term that is at least as precise
as the equated terms, acting as a witness of their plausible equality.
These witnesses track partial type information as a program runs, raising errors
when that information shows that two equated terms are undeniably inconsistent.
Composition of type information is internalized as a construct of the language,
and is deferred for function bodies whose evaluation is blocked by variables.
We thus ensure that extensionally-equal functions compose without error,
thereby preventing contexts from distinguishing them.
We describe the challenges of designing consistency and precision relations
for this system, along with solutions to these challenges. Finally, we prove important metatheory: type safety, conservative embedding of CIC,
\change{weak canonicity}, and the gradual guarantees of Siek et al., which ensure that
reducing a program's precision introduces no new static or dynamic errors.
