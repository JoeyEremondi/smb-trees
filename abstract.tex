% !TeX root = main.tex
% !TeX spellcheck = en-US

Ordinals can be used to prove the termination of dependently typed programs.
Brouwer trees are a particular ordinal notation that
make it very easy to assign sizes to higher order data structures.
They extend unary natural numbers with a limit constructor,
so a function's size can be the least upper bound of the sizes of values from its image.
These can then be used to define well-founded recursion: any recursive calls are allowed
so long as they are on values whose sizes are strictly smaller than the current size.

Unfortunately, Brouwer trees are not algebraically well-behaved.
They can be characterized equationally as a join-semilattice, where the join takes the maximum
of two trees. However, this join does not interact well with
the successor constructor, so it does not interact properly with
the strict ordering used in well-founded recursion.

We present Strictly Monotone Brouwer trees (SMB-trees), a refinement of Brouwer trees
that are algebraically well-behaved. SMB-trees are built using functions with the same
signatures as Brouwer tree constructors, and they satisfy all Brouwer tree inequalities.
However,  their join operator distributes over the successor, making them
suited for well-founded recursion or equational reasoning.

This paper teaches how, using dependent pairs and careful definitions, an ill-behaved
definition can be turned into a well-behaved one.
Our approach is axiomatically lightweight:
it does not rely on Axiom K, univalence, quotient types, or Higher Inductive Types.
We implement a recursively-defined maximum operator for Brouwer trees that matches
on successors and handles them specifically.
Then, we define SMB-trees as the subset of Brouwer trees for which the recursive maximum
computes a least upper bound.
Finally, we show that every Brouwer tree can be transformed into a corresponding SMB-tree
by joining it with itself an infinite number of times.
All definitions and theorems are implemented in Agda.
