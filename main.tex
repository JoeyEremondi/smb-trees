% !Tee root = main.tex
% !TeX spellcheck = en-US
%%% TeX-command-extra-options: "-shell-escape"

%%
%% This is file `sample-acmsmall.tex',
%% generated with the docstrip utility.
%%
%% The original source files were:
%%
%% samples.dtx  (with options: `acmsmall')
%%
%% IMPORTANT NOTICE:
%%
%% For the copyright see the source file.
%%
%% Any modified versions of this file must be renamed
%% with new filenames distinct from sample-acmsmall.tex.
%%
%% For distribution of the original source see the terms
%% for copying and modification in the file samples.dtx.
%%
%% This generated file may be distributed as long as the
%% original source files, as listed above, are part of the
%% same distribution. (The sources need not necessarily be
%% in the same archive or directory.)
%%
%% The first command in your LaTeX source must be the \documentclass command.
% \documentclass[autogobble,dvipsnames,acmsmall,anonymous,review]{acmart}
\documentclass[dvipsnames,sigplan,10pt,anonymous,review]{acmart}\settopmatter{printfolios=true,printccs=false,printacmref=false}


% \def\@testdef #1#2#3{%
% \def\reserved@a{#3}\expandafter \ifx \csname #1@#2\endcsname
% \reserved@a  \else
% \typeout{^^Jlabel #2 changed:^^J%
% \meaning\reserved@a^^J%
% \expandafter\meaning\csname #1@#2\endcsname^^J}%
% \@tempswatrue \fi}



% \@ifclasswith{acmart}{review}{
%   \newcommand{\reviewmode}{}
%   % Draft only options
% }{
%   %Non-review only options
% }

% Joey's generic macros
% shared across documents
\usepackage{amsmath}
\usepackage{xparse}

\usepackage{mathpartir}
\usepackage[capitalise]{cleveref}
%% Some recommended packages.
\usepackage{booktabs}   %% For formal tables:
                        %% http://ctan.org/pkg/booktabs
% \usepackage{subcaption} %% For complex figures with subfigures/subcaptions
                        %% http://ctan.org/pkg/subcaption
\usepackage{csquotes}
\usepackage{fancyvrb}
\usepackage{stmaryrd}
\usepackage{adjustbox}
\usepackage{xspace}
\usepackage{relsize}
% \usepackage{listings}
\usepackage{comment}

\usepackage{savesym}
\savesymbol{program}
\usepackage{amsmath}
\usepackage{semantic}
\usepackage{mdframed}
% \usepackage[dvipsnames]{xcolor}
% \usepackage{proof-at-the-end}
% \usepackage{hyperref}
% \usepackage[capitalise]{cleveref}

\newtheorem{desiderata}{Desiderata}
\numberwithin{desiderata}{section}
\crefname{desiderata}{Desideratum}{Desiderata}

\usepackage[geometry]{ifsym}

%%% Proofs in the appendix

% \usepackage{bm}


% Custom mathBB

\newcommand\doubleplus{+\kern-1.3ex+\kern0.8ex}
\newcommand\mdoubleplus{\ensuremath{\mathbin{+\mkern-5mu+}}}

\newcommand\codeFont[1]{\mathtt{#1}}

\NewDocumentCommand\defineMeta{mmmm}{%
    \expandafter\DeclareDocumentCommand\csname#1#2\endcsname{ E{^_}{{}{}} }{%
    #3{#4^{#3{##1}}_{#3{##2}}}%
    }%
}
\NewDocumentCommand\sgMeta{mG{#1}}{
    \defineMeta{g}{#1}{\gradualstyle}{#2}
    \defineMeta{s}{#1}{\staticstyle}{#2}
    \defineMeta{r}{#1}{\surfstyle}{#2}
}


\NewDocumentCommand\defineStaticFun{mG{#1}}{%
    \expandafter\NewDocumentCommand\csname#1\endcsname{od()}{%
    \staticstyle{\mathtt{#2}\ {\IfNoValueF{##1}{##1}}{\IfNoValueF{##2}{(##2)}} }
    }%
}


%\sgMeta{x} Gives us \gx and \sx commands
%properly styled for the gradual and static languages
%Optional second argument is how to display the varible, defaults to first argument
\sgMeta{x}
\sgMeta{y}
\sgMeta{z}
\sgMeta{C}
\sgMeta{D}
\sgMeta{N}
\sgMeta{M}
\sgMeta{ff}{f}
\sgMeta{pf}{p}
\sgMeta{P}
\sgMeta{el}{\overline{e}}
\sgMeta{X}
\sgMeta{Y}
\sgMeta{Z}
\sgMeta{t}
\sgMeta{T}
\sgMeta{s}
\sgMeta{S}
\sgMeta{m}
\sgMeta{n}
\sgMeta{u}
\sgMeta{U}
\sgMeta{v}
\sgMeta{V}
\sgMeta{A}
\sgMeta{B}
\sgMeta{J}{\text{J}}
\sgMeta{Nat}{\text{Nat}}
\sgMeta{Int}{\text{Int}}
\sgMeta{Bool}{\text{Bool}}
\sgMeta{Zero}{0}
\sgMeta{NatElim}{\text{NatElim}}
\sgMeta{VecElim}{\text{EqElim}}
\sgMeta{EqElim}{\text{VecElim}}
\sgMeta{Refl}{\text{Refl}}
\sgMeta{subst}{\text{subst}}

\sgMeta{G}{\Gamma}
\sgMeta{ep}{\varepsilon}
\sgMeta{times}{\times}
\sgMeta{lambda}{\lambda}
\sgMeta{Unit}{\mathbf{1}}
\sgMeta{unit}{()}

\sgMeta{spine}{\seq{e}}

\mathlig{||}{\bnfalt}
\mathlig{::=}{\bnfdef}
\mathlig{+::=}{\bnfadd}
\mathlig{-::=}{\bnfsub}
\mathlig{|>}{\vartriangleright}
\mathlig{<|}{\vartriangleleft}
\mathlig{<=}{\Leftarrow}
\mathlig{=>}{\Rightarrow}
% \mathlig{~>}{\leadsto}
\mathlig{>->}{\rightarrowtail}
\mathlig{|->}{\mapsto}
% \mathlig{++}{\mdoubleplus}
\mathlig{|=>}{\Mapsto}
\mathlig{==}{=}
% \mathlig{=-=}{\cong}
\mathlig{**}{\times}
\mathlig{<<}{\langle}
\mathlig{>>}{\rangle}

\mathlig{|=}{\vDash}
\newcommand{\set}[1]{{\{ #1 \}}}
\newcommand{\ep}{\varepsilon}

\newcommand{\gmu}{\gradualstyle{\tilde{\mu}}}


% \NewDocumentCommand{\stt}{m}{\staticstyle{\ensuremath{\mathtt{#1}}}}
% \NewDocumentCommand{\gtt}{m}{\gradualstyle{\ensuremath{\mathtt{#1}}}}

\newcommand{\sto}{\staticstyle{\to}}
\newcommand{\gto}{\gradualstyle{\to}}

% \newcommand\gcode[1]{\gradualstyle{\texttt{#1}}}
% \newcommand\scode[1]{\staticstyle{\texttt{#1}}}
\renewcommand{\P}{\mathcal{P}}

\newcommand{\mynote}[3][red]
    {{\color{#1} \fbox{\bfseries\sffamily\scriptsize#2}
    {\small$\blacktriangleright$\textsf{\emph{#3}}$\blacktriangleleft$}}~}
\newcommand{\et}[1]{\mynote{ET}{#1}}
% \newcommand{\je}[1]{\mynote{JE}{#1}}
% \newcommand{\et}[1]{}
% \newcommand{\je}[1]{}
\newcommand{\ron}[1]{\mynote{RG}{#1}}

% \usepackage{inconsolata}
\usepackage{mathtools}

\usepackage{pifont}
\newcommand{\cmark}{{\color{Emerald}\ding{51}}}%
\newcommand{\xmark}{{\color{RedOrange}\ding{55}}}%


\newcommand{\gradualcolor}{\color{RoyalBlue}}
\newcommand{\surfcolor}{\color{Emerald}}
\newcommand{\staticcolor}{\color{RedOrange}}

\newcommand{\gradualstyle}[1]{{\ensuremath{\gradualcolor\mathbf{{#1}}}}}
\newcommand{\surfstyle}[1]{{\ensuremath{\surfcolor\mathit{{#1}}}}}
\newcommand{\staticstyle}[1]{{\ensuremath{\staticcolor\mathsf{{#1}}}}}

\newcommand{\staticdesc}{\staticstyle{\textsf{red sans-serif font}}\xspace}
\newcommand{\gradualdesc}{\gradualstyle{\textbf{blue, bold serif font}}\xspace}
\newcommand{\surfdesc}{\surfstyle{\textit{green, italic serif font}}\xspace}

\newcommand{\g}[1]{\gradualstyle{#1}}
\newcommand{\s}[1]{\staticstyle{#1}}
\renewcommand{\r}[1]{\surfstyle{#1}}
\newcommand{\ix}[1]{\color{black}{\mathit{#1}}}

\NewDocumentCommand{\sType}{m}{{\staticstyle{\textbf{\textsf{Type}}_{#1}}}}
\NewDocumentCommand{\gType}{m}{{\gradualstyle{\mathbf{Type}_{#1}}}}
\NewDocumentCommand{\rType}{m}{{\surfstyle{\textbf{\textit{Type}}_{#1}}}}


\newcommand{\stepstostar}{\longrightarrow^{*}}
\newcommand{\stepstoplus}{\longrightarrow^{+}}
\newcommand{\etasteps}{\longrightarrow^{*}_{\eta}}
\newcommand{\stepsto}{\longrightarrow}


\newcommand{\guarded}{\s{\triangleright}}
\newcommand{\tguarded}{\s{\blacktriangleright}}
\newcommand{\gapp}{\s{\circledast}}
\newcommand{\aeguarded}{\s{\triangleright_{\ae}}}
\newcommand{\aetguarded}{\s{\blacktriangleright_{\ae}}}
\newcommand{\aegapp}{\s{\circledast_{\ae}}}
\newcommand{\LAE}{\s{\mathcal{L}_\ae}}

%https://github.com/wilbowma/dissertation/blob/0454bdb171c95d826561703231930c23b8241d6e/defs.tex
\newenvironment{bnfgrammar}
{\begin{displaymath}\begin{array}{lrcl}}
{\end{array}\end{displaymath}}
\newcommand{\bnfnewline}{
  \\[6pt]
}
\newcommand{\bnflabel}[1]{\mbox{\textit{#1}}}
\newcommand{\bnfalt}{\mathbf{\,\,\mid\,\,}}
\newcommand{\bnfdef}{\mathbf{\ \Coloneqq\ }}
\newcommand{\defbnf}{\bnfdef} % fd is my vim-escape combo
\newcommand{\bnfadd}{{\mathbf{\ +\!\!\Coloneqq\ }}}
\newcommand{\bnfsub}{{\bf - : : =}}

\makeatletter
\newcommand{\mathboxed}[1]{\text{\fboxsep=.2em\fbox{\m@th$\displaystyle#1$}}}
\makeatother

\newcommand{\ie}{i.e.\xspace}
\newcommand{\eg}{e.g.\xspace}

\newcommand{\errsym}{\mho}
\newcommand\gqm{\gradualstyle{\normalfont{\textbf{?}}}}
\newcommand\rqm{\surfstyle{\normalfont{\textbf{?}}}}
\newcommand\rqmat[1]{\surfstyle{\normalfont{\textit{?}}_{@#1}}}
\newcommand\gqmat[1]{\gradualstyle{\normalfont{\textbf{?}}_{#1}}}
% \newcommand\qm{\mymathbb{?} }
\newcommand\err{\gradualstyle{\errsym}}
\newcommand\errat[1]{\gradualstyle{\errsym}_{\g{#1}}}
\newcommand\errl{\gradualstyle{\errsym_{\ell}}}
\newcommand{\bbb}{\mathbb{B}}
\newcommand{\If}{\mathbf{if}}
\newcommand{\true}{\mathbf{true}}
\newcommand{\false}{\mathbf{false}}
\newcommand{\Type}{\mathbf{Type}}



\newcommand{\K}{$\mathbf{K}$ }
\newcommand{\UIP}{$\mathbf{UIP}$ }

\newcommand{\abe}{{\alpha\beta\eta}}

\newcommand{\rrule}{\textsc}


\newcommand\blackoline[1]{\colorlet{temp}{.}\color{black}\overrightarrow{\color{temp}#1}\color{temp}}
\newcommand{\seq}[1]{\blackoline{#1}}

\newcommand{\ala}{\`a la }

\NewDocumentCommand{\tr}{}{\staticstyle{tr}}
\newcommand{\trUp}{\ensuremath{\tr^{\s\uparrow}}}
\newcommand{\trDown}{\ensuremath{\tr^{\s\downarrow}}}

% \newcommand{\sqube}{\sqsubseteq}
% \newcommand{\sqeq}{\sqsupseteq\sqsubseteq}
\newcommand{\bZero}{{\mymathbbZero}}
\newcommand{\bOne}{{\mymathbbOne}}
\newcommand{\bN}{{\mymathbbN}}
\newcommand{\bC}{{\mymathbbC}}
\newcommand{\bB}{{\mymathbbB}}


% Stolen from Felipe's \AGT paper
\definecolor{lightgray}{gray}{0.90}
\newcommand{\Gbox}[1]{\colorbox{lightgray}{$#1$}}

\newcommand{\gbox}[1]{%
  {\setlength{\fboxsep}{1pt}\colorbox{lightgray}{\tiny$#1$}}}

\usepackage{ulem}
\newcommand{\change}[1]{#1}
% \newcommand{\change}[1]{{\color{orange}\uline{#1}}}

\newcommand{\Godel}{G\"{o}del}


\newmdenv[
usetwoside=false,
topline=false,
bottomline=false,
rightline=false,
leftmargin=0.2in,
linewidth=0.75pt,
skipabove=\topsep,
skipbelow=\topsep,
nobreak=false
]{leftrule}


% stolen from https://arxiv.org/abs/1911.00583
\newcommand{\case}[2]{
  \noindent $\blacktriangleright$ \textbf{Case} \text{#1} \textbf{:}
  {
    % \setlength{\parskip}{0.7em}
    \begin{leftrule}
      % \vspace{0.35em}
      #2
    \end{leftrule}
  }
  \noindent \ignorespaces
}

\newcommand{\lcase}[2]{
  \noindent $\blacktriangleright$ \textbf{Case} \text{#1} \textbf{:} #2
}

\newcommand{\mcase}[2]{\case{$#1$}{#2}}


\newcommand{\rlcase}[2]{\lcase{\rrule{#1}}{#2}}
\newcommand{\mlcase}[2]{\lcase{$#1$}{#2}}

\newcommand{\rcase}[2]{\case{\rrule{#1}}{#2}}



\newsavebox{\saveboxedarray}
\newenvironment{inferbox}[0]
{\begin{minipage}{\textwidth}\mprset{center}\begin{mathpar}}
    {\end{mathpar}\end{minipage}}
\newenvironment{syntaxcategory}[1]
{\begin{array}{lcll} \multicolumn{4}{l}{\mbox{\textbf{#1}}}\\}
   {\end{array}}
 \newenvironment{boxedarray}[1]
 {\begin{lrbox}{\saveboxedarray}\begin{math}\begin{array}{#1}}
                                              {\end{array}\end{math}\end{lrbox}{\usebox{\saveboxedarray}}}
                                        \reservestyle{\command}{\textsf}

% \newsavebox{\saveboxedarray}
% \newenvironment{inferbox}[0]
% {\begin{minipage}{\textwidth}\mprset{center}\begin{mathpar}}
%     {\end{mathpar}\end{minipage}}
% \newenvironment{syntaxcategory}[1]
% {\begin{array}{lcll} \multicolumn{4}{l}{\mbox{\textbf{#1}}}\\}
%    {\end{array}}
%  \newenvironment{boxedarray}[1]{}{}
%                                         \reservestyle{\command}{\textsf}

\newcommand{\myparagrapht}[1]{\noindent{\bf #1}}
\newcommand{\myparagraph}[1]{\vspace{0.5em}\myparagrapht{#1.}}


% \newcommand*\circled[1]{\tikz[baseline=(char.base)]{
%   \node[shape=circle,draw,inner sep=1pt] (char) {#1};}}

\usepackage{wrapfig}

\newcommand{\cst}{\cong}

% \newcommand{\qm}{ { \textbf{?}TODO }}

%TODO: define this to link back?
\newcommand{\defn}[1]{\textit{\gls{#1}}}




% !TEX root = main.tex
% This file provides examples of some useful macros for typesetting
% dissertations.  None of the macros defined here are necessary beyond
% for the template documentation, so feel free to change, remove, and add
% your own definitions.
%
% We recommend that you define macros to separate the semantics
% of the things you write from how they are presented.  For example,
% you'll see definitions below for a macro \file{}: by using
% \file{} consistently in the text, we can change how filenames
% are typeset simply by changing the definition of \file{} in
% this file.
%
%% The following is a directive for TeXShop to indicate the main file

%Treat thesis like appendex i.e. show things in gory detail

% \newcommand{\ifapx}[1]{#1}
% \newcommand{\ifnotapx}[1]{}
%Maybe show rule names
% \newcommand{\mname}[1]{\ifapx{#1}}


\newcommand{\onlydiss}[1]{#1}
% \newcommand{\thesisProp}{dissertation}

% Some useful macros for typesetting terms.
\newcommand{\file}[1]{\texttt{#1}}
\newcommand{\class}[1]{\texttt{#1}}
\newcommand{\latexpackage}[1]{\href{http://www.ctan.org/macros/latex/contrib/#1}{\texttt{#1}}}
\newcommand{\latexmiscpackage}[1]{\href{http://www.ctan.org/macros/latex/contrib/misc/#1.sty}{\texttt{#1}}}
\newcommand{\env}[1]{\texttt{#1}}
\newcommand{\BibTeX}{Bib\TeX}

% Define a command \doi{} to typeset a digital object identifier (DOI).
% Note: if the following definition raise an error, then you likely
% have an ancient version of url.sty.  Either find a more recent version
% (3.1 or later work fine) and simply copy it into this directory,  or
% comment out the following two lines and uncomment the third.
\DeclareUrlCommand\DOI{}
\newcommand{\doi}[1]{\href{http://dx.doi.org/#1}{\DOI{doi:#1}}}
%\newcommand{\doi}[1]{\href{http://dx.doi.org/#1}{doi:#1}}

% Useful macro to reference an online document with a hyperlink
% as well with the URL explicitly listed in a footnote
% #1: the URL
% #2: the anchoring text
\newcommand{\webref}[2]{\href{#1}{#2}\footnote{\url{#1}}}



% epigraph is a nice environment for typesetting quotations
\makeatletter
\newenvironment{epigraph}{%
  \begin{flushright}
  \begin{minipage}{\columnwidth-0.75in}
  \begin{flushright}
  \@ifundefined{singlespacing}{}{\singlespacing}%
    }{
  \end{flushright}
  \end{minipage}
  \end{flushright}}
\makeatother

% \FIXME{} is a useful macro for noting things needing to be changed.
% The following definition will also output a warning to the console
\newcommand{\FIXME}[1]{\typeout{**FIXME** #1}\textbf{[FIXME: #1]}}

% \newcommand\codeLine[1]{\texttt{#1}}
\newcommand\Int{\mathsf{Int}}
\newcommand\Bool{\mathsf{Bool}}


\newcommand{\jform}[1]{\fbox{#1}\hspace{\fill}\\}
\newcommand{\jformlow}[1]{\fbox{#1}\hspace{\fill}\vspace{-1em}\\}
% \newcommand{\synth}{\Rightarrow}
\newcommand{\psynth}[1]{{\,\mathbin{\Rightarrow_{#1}}}\,}
\newcommand{\prepsynth}[1]{{\,\mathbin{\Rightarrow*_{#1}}}\,}
\newcommand{\psynthstar}[1]{{\,\mathbin{\Rightarrow^{*}_{#1}}}\,}
% \newcommand{\check}{\Leftarrow}
\newcommand{\ssorts}{\sType{ }}
% \newcommand{\ulev}[1]{^{@#1}}
\newcommand{\ind}{\mathsf{ind}}
\newcommand{\smatch}[4]{\s{\operatorname{\ind}_{#1}(#2,#3,#4)}}
\newcommand{\smatchnoarg}[1]{\s{\operatorname{\ind}_{#1}}}
\newcommand{\rmatch}[4]{\r{\operatorname{\ind}_{#1}(#2,#3,#4)}}
\newcommand{\gmatch}[4]{\g{\operatorname{\ind}_{#1}(#2,#3,#4)}}
\newcommand{\gmatchNoarg}[1]{\g{\operatorname{\ind}_{#1}}}

\newcommand{\fix}[3]{\operatorname{\mathtt{fix}}#1 : #2 := #3}
\newcommand{\pars}{\operatorname{\mathbf{\color{black} Params}}}
\newcommand{\indices}{\operatorname{\mathbf{\color{black} Indices}}}
\newcommand{\ivals}[3]{\operatorname{\mathbf{\color{black} IValsFor}}(#1,#2,#3)}
\newcommand{\iargs}[4]{\operatorname{\mathbf{\color{black} IArgsFor_{#1}}}(#2,#3,#4)}
\newcommand{\args}{\operatorname{\mathbf{\color{black} Args}}}
\newcommand{\parsub}[1]{[#1]}
\newcommand{\hlev}[1]{_{#1}}
\newcommand{\ulev}[1]{\scalebox{0.7}{@\{#1\}}}


\newcommand{\iswf}{\s{{iswf}}}
\newcommand{\WF}{\s{{WF}}}
\newcommand{\field}{\s{\textit{field}}}
\newcommand{\sSigma}{{\color{BrickRed}\mathrm\Sigma}}
\newcommand{\sPi}{{\color{BrickRed}\mathrm\Pi}}
\newcommand{\uttrans}[1]{{\llbracket \g{#1} \rrbracket}}
\newcommand{\interpCode}[1]{\s{\llparenthesis #1\rrparenthesis_\Code}}
\newcommand{\interp}[1]{\s{\llparenthesis #1 \rrparenthesis}}

\newcommand{\sind}[2]{\s{\ind_{#1}(}#2\s{)}}
\newcommand{\gind}[2]{\g{\ind_{#1}(}#2\g{)}}
\newcommand{\selim}{\s{\mathsf{elim}}}
\newcommand{\gelim}{\g{\mathsf{elim}}}

\newcommand{\varAE}{{\!\! \textit{\AE} }}

\newcommand{\elabsto}{\rightarrowtriangle}
\newcommand{\echeck}[3]{#1 \elabsto #3 <= #2}
\newcommand{\esynth}[3]{#1 \elabsto #3 => #2}
\newcommand{\epsynth}[4]{#2 \elabsto #4 =>_{#1} #3}
\newcommand{\etype}[3]{#1 \elabsto #3 : \gType{}_{=>\g{#2}} }
\newcommand{\redsto}{\leadsto}
\newcommand{\ifapxCaption}[1]{\caption{#1}}

% END


\newcommand{\DCat}[1]{{\g{D^C\langle #1\rangle}}}
\newcommand{\DCatnog}[1]{{\g{D^C\langle} #1\g{\rangle}}}
% \newcommand{\seqSqube}[1]{\sqube_{{#1}}}
\newcommand{\sqube}{\sqsubseteq}
\newcommand{\squbr}{\sqsubseteq_{\mathsf{Surf}}}
% \newcommand{\squbo}{\sqsubseteq_{obs}}
\newcommand{\squbo}{\sqsubseteq^{\Vdash}}
\newcommand{\squbB}{\sqsubseteq_{\bB}}
\newcommand{\squbG}{\sqsubseteq^{Ctx}}
\newcommand{\squbstar}{\sqsubseteq^{C*}}
\newcommand{\sqeqstar}{\sqsupseteq\sqsubseteq^{C*}}
\newcommand{\squbs}{\sqsubseteq_{\alpha}}
% \newcommand{\squbstep}{\sqube_{\leadsto}}
\newcommand{\equiprecstep}{\sqsupseteq\sqube_{\leadsto}}
\newcommand{\qmt}[1]{\gqm_{\g{#1}}}
\newcommand{\qml}{\gqm_{\gType{\ell}}}
\newcommand{\qmat}[1]{\g{\gqm_{#1}}}
\newcommand{\attagl}[2]{\g{\langle#1\rangle_{\g\ell}#2}}


% \usepackage{fontspec}
% \usepackage{inconsolata}
% \usepackage{sansmath}
% % \DeclareMathAlphabet{\mathbbm}{U}{bbm}{m}{n}
% % \DeclareMathAlphabet{\mymathbb}{U}{bbold}{m}{n}
% \DeclareSymbolFont{mymathbb}{U}{bbold}{m}{n}
% \SetMathAlphabet{\mathsfsl}{sans}{\sansmathencoding}{\sfdefault}{m}{n}
% \SetMathAlphabet{\mathsf}{sans}{\sansmathencoding}{\sfdefault}{m}{n}
% \SetMathAlphabet{\mathrm}{sans}{\sansmathencoding}{\rmdefault}{m}{n}
% \SetMathAlphabet{\mathbf}{sans}{\sansmathencoding}{\rmdefault}{m}{n}
% % \DeclareMathAlphabet{\mathbb}{U}{bbold}{m}{n}
% % \SetMathAlphabet{\mathbb}{sans}{U}{bbold}{m}{n}
% % \SetMathAlphabet{\mathbb}{normal}{U}{bbold}{m}{n}
% % \SetMathAlphabet{\mathbb}{bold}{U}{bbold}{m}{n}
% % \SetSymbolFont{largesymbols}{sans}{OMX}{bbold}{m}{n}
% % \SetSymbolFont{numbers}{mymathbb}{OMX}{bbold}{m}{n}
% \SetSymbolFont{largesymbols}{sans}{OMX}{libertine}{m}{n}
% \DeclareMathDelimiter{(}{\mathopen} {operators}{"28}{largesymbols}{"00}
% \DeclareMathDelimiter{)}{\mathclose}{operators}{"29}{largesymbols}{"01}
% \DeclareMathSymbol{\sansbigvee}{1}{largesymbols}{'137}
% \DeclareMathSymbol{\mymathbbOne}{3}{mymathbb}{"31}
% \DeclareMathSymbol{\mymathbbOne}{3}{mymathbb}{"31}
% \DeclareMathSymbol{\mymathbbZero}{3}{mymathbb}{"30}
% \DeclareMathSymbol{\mymathbbN}{3}{mymathbb}{'116}
% \DeclareMathSymbol{\mymathbbB}{3}{mymathbb}{"42}
% \DeclareMathSymbol{\mymathbbC}{3}{mymathbb}{"43}
% \DeclareMathSymbol{\mymathbbColon}{3}{mymathbb}{'072}
% \DeclareMathSymbol{\mymathbbSemiColon}{3}{mymathbb}{'073}
%   \usepackage{fonttable}
% \usepackage{sansmath} %TODO: re-enable
\usepackage{scalerel}
\usepackage{float}


\usepackage[capitalise]{cleveref}

\newcommand{\oblset}[1]{\textsc{#1}}
\newcommand{\GType}{\oblset{GType}}
\newcommand{\SType}{\oblset{SType}}
\newcommand{\Nat}{\mathsf{Nat}}
\newcommand{\VVec}{\mathsf{Vec}}
%
\newcommand{\dom}{{\mathbf{\mathrm{dom}}}}
\newcommand{\cod}{{\mathbf{\mathrm{cod}}}}


\newcommand{\upCast}[2]{\g{\langle #2 \nwarrow #1 \rangle}}
\newcommand{\downCast}[2]{\g{\langle #2 \swarrow  #1 \rangle}}
\newcommand{\genCast}[2]{\g{\langle #2 \mathrlap{\,\nwarrow}\swarrow  #1 \rangle}}
% plainly styled cast
\newcommand{\pcast}[2]{\g\langle #2 \g{<=} #1 \g\rangle}

\newcommand{\T}[1]{{\color{black}\mathcal{T}\llbracket} \g{#1} {\color{black}\rrbracket}}
\newcommand{\E}[1]{{\color{black}\mathcal{E}\llbracket} \g{#1} {\color{black}\rrbracket}}
% \newcommand{\V}[1]{{\color{black}\mathcal{V}\llbracket} \g{#1} {\color{black}\rrbracket}}
% \usepackage[literate]{myidrislang}

\newcommand{\naive}{na\"ive\xspace} %TODO check if this is in all of them
\newcommand{\Naive}{Na\"ive\xspace} %TODO check if this is in all of them



% \newcommand{\TypeType}{{\mathbf{TypeTODO} } } %TODO get rid of


\newcommand{\ottdrule}[3]{ \inferrule[#3]{#1}{#2} }
\newcommand{\AllTerms}{\oblset{Snf}}
\newcommand{\AllGTerms}{\oblset{Gnf}}
\newcommand{\AllNeut}{\oblset{Sne}}
\newcommand{\AllGNeut}{\oblset{Gne}}
\newcommand{\pto}{\rightharpoonup}

\newcommand{\emptyspine}{\cdot}
\newcommand\leadsfrom{\reflectbox{$\leadsto$ } }
\newcommand{\Pow}{\mathcal{P}}
\newcommand\myepsilon{\r{\varepsilon}}


\newcommand{\J}{\mathbf{J}}
\newcommand{\Kax}{\mathbf{K}}

\newcommand{\defprec}{{\sqube_{\stepsto}}}
\newcommand{\defsuprec}{{\sqube^{\longleftarrow}_{\stepsto}}}
\newcommand{\defcst}{{{\cong}_{\stepsto}}}
\newcommand{\acst}{{{\cong}_{\alpha}}}

\newcommand{\genprec}{\Gbox{\defprec}}
\newcommand{\gensuprec}{\Gbox{\defsuprec}}
\newcommand{\gencst}{\Gbox{\defcst}}

\DeclareMathOperator*{\bigamp}{\mathlarger{\&}}
\newcommand{\gcomp}[1]{\mathbin{\g{\&_{#1}}}}
\newcommand{\gcompop}{\g{\&}}
\newcommand{\itercomp}{\seq{\bigamp}}

\newcommand{\germ}{\mathsf{\color{black} germ}}
\newcommand{\head}{\mathsf{\color{black} head}}



\newcommand{\cast}[2]{\g{\langle #2 <= #1 \rangle}}
\newcommand{\castnog}[2]{\g{\langle} #2 \g{<=} #1 \g{\rangle}}
\newcommand{\castenv}[2]{{\langle} #2 {<=} #1 {\rangle}}
% \newcommand{\rmeet}[3]{\g{#1 \sqcap_{#3} #2 }}
% \newcommand{\rmeetnog}[3]{{#1 \g{\sqcap}_{#3} #2}}
% \newcommand{\grefl}[3]{\g{refl_{#1 |- #2 \cong  #3}}}
\newcommand{\grefl}[3]{\g{refl(#1)_{|- #2 \cong  #3}}}
\newcommand{\greflnog}[3]{\g{refl}(#1)_{\g{|-} #2 \g{\cong} #3}}

% % \newcommand{\upCast}[2]{\g{\langle #2 \nwarrow #1 \rangle}}
% \newcommand{\downCast}[2]{\g{\langle #2 \swarrow  #1 \rangle}}
% \newcommand{\genCast}[2]{\g{\langle #2 \mathrlap{\,\nwarrow}\swarrow  #1 \rangle}}

% Theorems taken from ACMART
  \newtheorem{theorem}{Theorem}[section]
  \newtheorem{conjecture}[theorem]{Conjecture}
  \newtheorem{proposition}[theorem]{Proposition}
  \newtheorem{lemma}[theorem]{Lemma}
  \newtheorem{corollary}[theorem]{Corollary}
  \newtheorem{example}[theorem]{Example}
  \newtheorem{definition}[theorem]{Definition}


\renewcommand{\em}{\PackageError{main}{Remove Em}{Deprecated}}




\newcommand{\nl}{\\&}
\let\oldAE\AE
\let\oldae\ae
\renewcommand{\AE}{\textsf{\oldAE}}
\renewcommand{\ae}{\textsf{\oldae}}

\newcommand{\lob}{\textsf{l\"{o}b}}
\newcommand{\Lob}{{L\"{o}b}}
\newcommand{\sqm}{\s{\textbf{?}}\:\!\!}
\newcommand{\sqmTy}{\s{\sqm Ty}}


% \usepackage{agda}
\usepackage{newunicodechar}
\newunicodechar{⨟}{ \ensuremath{{ \fatsemi }}}
% \newunicodechar{⟧}{ \ensuremath{\rrbracket} }
% \newunicodechar{≤}{ \ensuremath{\leq} }
% \newunicodechar{↝}{ \ensuremath{\arrowwaveright} }
% \newunicodechar{≟}{ \stackrel{?}{=} }
% \newunicodechar{⊑}{ \ensuremath{\sqsubseteq} }
% \newunicodechar{𝒟}{ \ensuremath{\mathscr{D}} }
% \newunicodechar{▹}{ \ensuremath{\triangleright} }
% \newunicodechar{⨟}{ \mymathbbSemiColon }
% \newunicodechar{≡}{ \ensuremath{\equiv} }
% \newunicodechar{𝟚}{ \ensuremath{\mathbb{2}} }
% \newunicodechar{◁}{ \triangleleft }
% \newunicodechar{ₚ}{_p}
% \newunicodechar{₃}{_3}
% \newunicodechar{ω}{ \ensuremath{\omega} }
% \newunicodechar{Δ}{ \ensuremath{\Delta} }
% \newunicodechar{̃}{^{\~}}
% \newunicodechar{λ}{ \ensuremath{\lambda} }
% \newunicodechar{∈}{ \ensuremath{\in} }
% \newunicodechar{θ}{ \ensuremath{\theta} }
% \newunicodechar{⇐}{ \ensuremath{\Leftarrow} }
% \newunicodechar{∨}{ \ensuremath{\vee} }
% \newunicodechar{Æ}{ {\AE} }
% \newunicodechar{β}{ \ensuremath{\beta} }
% \newunicodechar{⋯}{ \ensuremath{\cdots} }
% \newunicodechar{≔}{ \ensuremath{:=} }
% \newunicodechar{∧}{ \ensuremath{\wedge} }
% \newunicodechar{₂}{ _2 }
% \newunicodechar{⟨}{ \ensuremath{\langle} }
% \newunicodechar{𝟙}{ \ensuremath{\mathbb{1}} }
% \newunicodechar{ₛ}{ _s }
% \newunicodechar{⟦}{ \llbracket }
% \newunicodechar{▸}{ \ensuremath{\blacktriangleright} }
% \newunicodechar{₀}{ _0 }
% \newunicodechar{⁻}{ ^{-} }
% \newunicodechar{≢}{ \ensuremath{\not\equiv} }
% \newunicodechar{⇒}{ \ensuremath{\Rightarrow} }
% \newunicodechar{↦}{ \ensuremath{\mapsto} }
% \newunicodechar{∎}{ \ensuremath{\blacksquare} }
% \newunicodechar{ℕ}{ \ensuremath{\mathbb{N}} }
% \newunicodechar{Γ}{ \ensuremath{\Gamma} }
% \newunicodechar{ }{ ~ }
% \newunicodechar{⟩}{ \ensuremath{\rangle} }
% \newunicodechar{₄}{ _4 }
% \newunicodechar{∞}{ \ensuremath{\infty} }
% \newunicodechar{ℛ}{ \ensuremath{\mathscr{R}} }
% \newunicodechar{∷}{ \ensuremath{::} }
% \newunicodechar{⊥}{ \ensuremath{\perp} }
% \newunicodechar{ˡ}{ ^l }
% \newunicodechar{τ}{ \ensuremath{\tau} }
% \newunicodechar{⊛}{ \ensuremath{\circledast} }
% \newunicodechar{→}{ {\textrightarrow} }
% \newunicodechar{←}{ {\textleftarrow} }
% \newunicodechar{₁}{ _1 }
% \newunicodechar{∀}{ \ensuremath{\forall} }
% \newunicodechar{⦂}{ \mymathbbColon }
% \newunicodechar{⊎}{ \ensuremath{\uplus} }
% \newunicodechar{⊢}{ \ensuremath{\vdash} }
% \newunicodechar{¬}{ {\textlnot} }
% \newunicodechar{□}{ \ensuremath{\square} }
% \newunicodechar{◃}{ \ensuremath{\triangleleft} }
% \newunicodechar{δ}{ \ensuremath{\delta} }
% \newunicodechar{α}{ \ensuremath{\alpha} }
% \newunicodechar{≅}{ \ensuremath{\cong} }
% \newunicodechar{𝟘}{ \ensuremath{\mathbb{0}} }
% \newunicodechar{æ}{ {\ae} }
% \newunicodechar{ö}{ \"o }
% \newunicodechar{⟶}{ \ensuremath{\longrightarrow} }
% \newunicodechar{⦃}{ \{\{ }
% \newunicodechar{℧}{ {\textmho} }
% \newunicodechar{ℰ}{ \ensuremath{\mathscr{E}} }
% \newunicodechar{ℓ}{ \ensuremath{\ell} }
% \newunicodechar{Σ}{ \ensuremath{\Sigma} }
% \newunicodechar{μ}{ \ensuremath{\mu} }
% \newunicodechar{×}{ {\texttimes} }
% \newunicodechar{⁇}{ \ensuremath{\sqm} }
% \newunicodechar{Π}{ \ensuremath{\Pi} }
% \newunicodechar{⦄}{ \}\} }
% \newunicodechar{⊤}{ \ensuremath{\top} }
% \newunicodechar{∙}{ \ensuremath{\bullet} }
% \newunicodechar{𝒯}{ \ensuremath{\mathscr{T}} }
% \newunicodechar{ℂ}{ \ensuremath{\mathbb{C}} }
% \newunicodechar{φ}{ \ensuremath{\varphi} }

\usepackage{agda}


\ifdef{\reviewmode}{
  \settopmatter{printfolios=true,printccs=false,printacmref=false}
  \setcopyright{none}
  \renewcommand\footnotetextcopyrightpermission[1]{}
  % \pagestyle{plain}
  \raggedbottom
}{
  \settopmatter{printfolios=true,printccs=false,printacmref=false}
  \setcopyright{none}
  \renewcommand\footnotetextcopyrightpermission[1]{}
  \pagestyle{plain}
% \setcopyright{acmcopyright}
% \copyrightyear{2021}
% \acmDOI{10.1145/1122445.1122456}
}



%%
%% \BibTeX command to typeset BibTeX logo in the docs
\AtBeginDocument{%
  \providecommand\BibTeX{{%


%% Rights management information.  This information is sent to you
%% when you complete the rights form.  These commands have SAMPLE
%% values in them; it is your responsibility as an author to replace
%% the commands and values with those provided to you when you
%% complete the rights form.


%%
%% These commands are for a JOURNAL article.
% \acmJournal{PACMPL}
% \acmVolume{37}
% \acmNumber{4}
% \acmArticle{111}
% \acmMonth{7}

%%
%% Submission ID.
%% Use this when submitting an article to a sponsored event. You'll
%% receive a unique submission ID from the organizers
%% of the event, and this ID should be used as the parameter to this command.
%%\acmSubmissionID{123-A56-BU3}

%%
%% The majority of ACM publications use numbered citations and
%% references.  The command \citestyle{authoryear} switches to the
%% "author year" style.
%%
%% If you are preparing content for an event
%% sponsored by ACM SIGGRAPH, you must use the "author year" style of
%% citations and references.
%% Uncommenting
%% the next command will enable that style.
\citestyle{acmauthoryear}

%%% If you see 'ACMUNKNOWN' in the 'setcopyright' statement below,
%%% please first submit your publishing-rights agreement with ACM (follow link on submission page).
%%% Then please update our instructions page and copy-and-paste the NEW commands into your article.
%%% Please contact us in case of questions; allow up to 10 min for the system to propagate the information.
%%%
%%% The following is specific to ICFP '22 and the paper
%%% 'Propositional Equality for Gradual Dependently Typed Programming'
%%% by Joseph Eremondi, Ronald Garcia, and Éric Tanter.
%%%
% \setcopyright{rightsretained}
% \acmPrice{}
% \acmDOI{10.1145/3547627}
% \acmYear{2022}
% \copyrightyear{2022}
% \acmSubmissionID{icfp22main-p16-p}
% \acmJournal{PACMPL}
% \acmVolume{6}
% \acmNumber{ICFP}
% \acmArticle{96}
% \acmMonth{8}
%%
%% end of the preamble, start of the body of the document source.
\begin{document}

\title{Join Idempotent Brouwer Trees for Well Founded Recursion}
% \titlenote{This work is partially funded by CONICYT FONDECYT Regular
% Project 1190058. This work is partially funded by an NSERC Discovery Grant.}

% \subtitle{Exploring the Design Space of Gradual Propositional Equality}
%%
%% The "author" command and its associated commands are used to define
%% the authors and their affiliations.
%% Of note is the shared affiliation of the first two authors, and the
%% "authornote" and "authornotemark" commands
%% used to denote shared contribution to the research.
% \orcid{nnnn-nnnn-nnnn-nnnn}             %% \orcid is optional

  \author{Joseph Eremondi}
  % \authornote{with author1 note}          %% \authornote is optional;
                                          %% can be repeated if necessary

  \affiliation{
    \department{Laboratory for the Foundation of Computer Science}              %% \department is recommended
    \institution{University of Edinburgh}            %% \institution is required
    \country{Scotland, United Kingdom}                    %% \country is recommended
  }
  \email{{joey.eremondi@ed.ac.uk}}          %% \email is recommended




%%
%% By default, the full list of authors will be used in the page
%% headers. Often, this list is too long, and will overlap
%% other information printed in the page headers. This command allows
%% the author to define a more concise list
%% of authors' names for this purpose.
\renewcommand{\shortauthors}{Joseph Eremondi, Ronald Garcia, and \'Eric Tanter}


% \begin{acks}
  % \end{acks}

%%
%% The "title" command has an optional parameter,
%% allowing the author to define a "short title" to be used in page headers.

%%
%% The abstract is a short summary of the work to be presented in the
%% article.
\begin{abstract}
  % !TeX root = main.tex
% !TeX spellcheck = en-US

Ordinals can be used to prove the termination of dependently typed programs.
Brouwer trees are a particular ordinal notation that
make it very easy to assign sizes to higher order data structures.
They extend unary natural numbers with a limit constructor,
so a function's size can be the least upper bound of the sizes of values from its image.
These can then be used to define well founded recursion: any recursive calls are allowed
so long as they are on values whose sizes are strictly smaller than the current size.

Unfortunately, Brouwer trees are not algebraically well behaved.
They can be characterized equationally as a join-semilattice, where the join takes the maximum
of two trees. However, this join does not interact well with
the successor constructor, so it does not interact properly with
the strict ordering used in well founded recursion.

We present Strictly Monotone Brouwer trees (SMB-trees), a refinement of Brouwer trees
that are algebraically well behaved. SMB-trees are built using functions with the same
signatures as Brouwer tree constructors, and they satisfy all Brouwer tree inequalities.
However,  their join operator distributes over the successor, making them
suited for well founded recursion or equational reasoning.

This paper teaches how, using dependent pairs and careful definitions, an ill behaved
definition can be turned into a well behaved one.
Our approach is axiomatically lightweight:
it does not rely on Axiom K, univalence, quotient types, or Higher Inductive Types.
We implement a recursively-defined maximum operator for Brouwer trees that matches
on successors and handles them specifically.
Then, we define SMB-trees as the subset of Brouwer trees for which the recursive maximum
computes a least upper bound.
Finally, we show that every Brouwer tree can be transformed into a corresponding SMB-tree
by joining it with itself an infinite number of times.
All definitions and theorems are implemented in Agda.

\end{abstract}

\begin{CCSXML}
<ccs2012>
<concept>
<concept_id>10003752.10010124.10010125.10010130</concept_id>
<concept_desc>Theory of computation~Type structures</concept_desc>
<concept_significance>500</concept_significance>
</concept>
<concept>
<concept_id>10003752.10010124.10010131</concept_id>
<concept_desc>Theory of computation~Program semantics</concept_desc>
<concept_significance>500</concept_significance>
</concept>
</ccs2012>
\end{CCSXML}

\ccsdesc[500]{Theory of computation~Type structures}
\ccsdesc[500]{Theory of computation~Program semantics}



%%
%% Keywords. The author(s) should pick words that accurately describe
%% the work being presented. Separate the keywords with commas.
\keywords{dependent types, gradual types, propositional equality}

\def\acmBooktitle#1{\gdef\@acmBooktitle{#1}}
\acmBooktitle{Proceedings of \acmConference@name
       \ifx\acmConference@name\acmConference@shortname\else
         \ (\acmConference@shortname)\fi}

%%
%% This command processes the author and affiliation and title
%% information and builds the first part of the formatted document.
\maketitle

% !TEX root =  main.tex
% !TEX root =  main.tex
\section{Brouwer Trees: An Introduction}
\begin{code}[hide]%
%
\>[2]\AgdaKeyword{open}\AgdaSpace{}%
\AgdaKeyword{import}\AgdaSpace{}%
\AgdaModule{Data.Nat}\AgdaSpace{}%
\AgdaKeyword{hiding}\AgdaSpace{}%
\AgdaSymbol{(}\AgdaOperator{\AgdaDatatype{\AgdaUnderscore{}≤\AgdaUnderscore{}}}\AgdaSymbol{)}\<%
\\
%
\>[2]\AgdaKeyword{open}\AgdaSpace{}%
\AgdaKeyword{import}\AgdaSpace{}%
\AgdaModule{Relation.Binary.PropositionalEquality}\<%
\\
%
\>[2]\AgdaKeyword{open}\AgdaSpace{}%
\AgdaKeyword{import}\AgdaSpace{}%
\AgdaModule{Data.Product}\<%
\\
%
\>[2]\AgdaKeyword{open}\AgdaSpace{}%
\AgdaKeyword{import}\AgdaSpace{}%
\AgdaModule{Relation.Nullary}\<%
\\
%
\>[2]\AgdaKeyword{open}\AgdaSpace{}%
\AgdaKeyword{import}\AgdaSpace{}%
\AgdaModule{Iso}\<%
\\
%
\>[2]\AgdaKeyword{module}\AgdaSpace{}%
\AgdaModule{Tree}\AgdaSpace{}%
\AgdaKeyword{where}\<%
\\
\>[0]\<%
\end{code}

Brouwer trees  are a simple but elegant tool for proving termination of higher-order procedures.
Traditionally, they are defined as follows:
\begin{code}%
\>[0][@{}l@{\AgdaIndent{1}}]%
\>[2]\AgdaKeyword{data}\AgdaSpace{}%
\AgdaDatatype{SmallTree}\AgdaSpace{}%
\AgdaSymbol{:}\AgdaSpace{}%
\AgdaPrimitive{Set}\AgdaSpace{}%
\AgdaKeyword{where}\<%
\\
\>[2][@{}l@{\AgdaIndent{0}}]%
\>[4]\AgdaInductiveConstructor{Z'}\AgdaSpace{}%
\AgdaSymbol{:}\AgdaSpace{}%
\AgdaDatatype{SmallTree}\<%
\\
%
\>[4]\AgdaInductiveConstructor{↑'}\AgdaSpace{}%
\AgdaSymbol{:}\AgdaSpace{}%
\AgdaDatatype{SmallTree}\AgdaSpace{}%
\AgdaSymbol{→}\AgdaSpace{}%
\AgdaDatatype{SmallTree}\<%
\\
%
\>[4]\AgdaInductiveConstructor{Lim'}\AgdaSpace{}%
\AgdaSymbol{:}\AgdaSpace{}%
\AgdaSymbol{(}\AgdaDatatype{ℕ}\AgdaSpace{}%
\AgdaSymbol{→}\AgdaSpace{}%
\AgdaDatatype{SmallTree}\AgdaSymbol{)}\AgdaSpace{}%
\AgdaSymbol{→}\AgdaSpace{}%
\AgdaDatatype{SmallTree}\<%
\end{code}
Under this definition, a Brouwer tree is either zero, the successor of another Brouwer tree, or the limit of a countable sequence of Brouwer trees. However, these are quite weak, in that they can only take the limit of countable sequences.
To represent the limits of uncountable sequences, we can paramterize our definition over some Universe \ala Tarski:

\begin{code}%
%
\>[2]\AgdaKeyword{module}\AgdaSpace{}%
\AgdaModule{\AgdaUnderscore{}}\AgdaSpace{}%
\AgdaSymbol{\{}\AgdaBound{ℓ}\AgdaSymbol{\}}\<%
\\
\>[2][@{}l@{\AgdaIndent{0}}]%
\>[4]\AgdaSymbol{(}\AgdaBound{ℂ}\AgdaSpace{}%
\AgdaSymbol{:}\AgdaSpace{}%
\AgdaPrimitive{Set}\AgdaSpace{}%
\AgdaBound{ℓ}\AgdaSymbol{)}\<%
\\
%
\>[4]\AgdaSymbol{(}\AgdaBound{El}\AgdaSpace{}%
\AgdaSymbol{:}\AgdaSpace{}%
\AgdaBound{ℂ}\AgdaSpace{}%
\AgdaSymbol{→}\AgdaSpace{}%
\AgdaPrimitive{Set}\AgdaSpace{}%
\AgdaBound{ℓ}\AgdaSymbol{)}\<%
\\
%
\>[4]\AgdaSymbol{(}\AgdaBound{Cℕ}\AgdaSpace{}%
\AgdaSymbol{:}\AgdaSpace{}%
\AgdaBound{ℂ}\AgdaSymbol{)}\AgdaSpace{}%
\AgdaSymbol{(}\AgdaBound{CℕIso}\AgdaSpace{}%
\AgdaSymbol{:}\AgdaSpace{}%
\AgdaRecord{Iso}\AgdaSpace{}%
\AgdaSymbol{(}\AgdaBound{El}\AgdaSpace{}%
\AgdaBound{Cℕ}\AgdaSymbol{)}\AgdaSpace{}%
\AgdaDatatype{ℕ}\AgdaSpace{}%
\AgdaSymbol{)}\AgdaSpace{}%
\AgdaKeyword{where}\<%
\end{code}

We then generalize limits to any function whose domain is the interpretation of some code.
\begin{code}%
%
\>[4]\AgdaKeyword{data}\AgdaSpace{}%
\AgdaDatatype{Tree}\AgdaSpace{}%
\AgdaSymbol{:}\AgdaSpace{}%
\AgdaPrimitive{Set}\AgdaSpace{}%
\AgdaBound{ℓ}\AgdaSpace{}%
\AgdaKeyword{where}\<%
\\
\>[4][@{}l@{\AgdaIndent{0}}]%
\>[6]\AgdaInductiveConstructor{Z}\AgdaSpace{}%
\AgdaSymbol{:}\AgdaSpace{}%
\AgdaDatatype{Tree}\<%
\\
%
\>[6]\AgdaInductiveConstructor{↑}\AgdaSpace{}%
\AgdaSymbol{:}\AgdaSpace{}%
\AgdaDatatype{Tree}\AgdaSpace{}%
\AgdaSymbol{→}\AgdaSpace{}%
\AgdaDatatype{Tree}\<%
\\
%
\>[6]\AgdaInductiveConstructor{Lim}\AgdaSpace{}%
\AgdaSymbol{:}\AgdaSpace{}%
\AgdaSymbol{∀}%
\>[15]\AgdaSymbol{(}\AgdaBound{c}\AgdaSpace{}%
\AgdaSymbol{:}\AgdaSpace{}%
\AgdaBound{ℂ}\AgdaSpace{}%
\AgdaSymbol{)}\AgdaSpace{}%
\AgdaSymbol{→}\AgdaSpace{}%
\AgdaSymbol{(}\AgdaBound{f}\AgdaSpace{}%
\AgdaSymbol{:}\AgdaSpace{}%
\AgdaBound{El}\AgdaSpace{}%
\AgdaBound{c}\AgdaSpace{}%
\AgdaSymbol{→}\AgdaSpace{}%
\AgdaDatatype{Tree}\AgdaSymbol{)}\AgdaSpace{}%
\AgdaSymbol{→}\AgdaSpace{}%
\AgdaDatatype{Tree}\<%
\end{code}

\begin{code}[hiding]%
\>[0]\<%
\end{code}


Our module is paramterized over a universe level, a type $\bC$ of \textit{codes}, and an ``elements-of'' interpretation
function $\mathit{El}$, which computes the type represented by each code.
We require that there be a code whose interpretation is isomorphic to the natural numbers,
as this is essential to our construction in \cref{sec:TODO}.
Increasingly larger trees can be obtained by setting $\bC := \AgdaPrimitive{Set} \ \ell$ and
$\mathit{El} := \mathit{id}$ for increasing $\ell$.
However, by defining an inductive-recursive universe,
one can still capture limits over some non-countable types, since
 $\AgdaDatatype{Tree}$ is in $\AgdaPrimitive{Set}$ whenever $\bC$ is.

The small limit constructor can be recovered from the natural-number code
\begin{code}%
\>[0]\<%
\\
\>[0][@{}l@{\AgdaIndent{1}}]%
\>[4]\AgdaFunction{ℕLim}\AgdaSpace{}%
\AgdaSymbol{:}\AgdaSpace{}%
\AgdaSymbol{(}\AgdaDatatype{ℕ}\AgdaSpace{}%
\AgdaSymbol{→}\AgdaSpace{}%
\AgdaDatatype{Tree}\AgdaSymbol{)}\AgdaSpace{}%
\AgdaSymbol{→}\AgdaSpace{}%
\AgdaDatatype{Tree}\<%
\\
%
\>[4]\AgdaFunction{ℕLim}\AgdaSpace{}%
\AgdaBound{f}\AgdaSpace{}%
\AgdaSymbol{=}\AgdaSpace{}%
\AgdaInductiveConstructor{Lim}\AgdaSpace{}%
\AgdaBound{Cℕ}%
\>[21]\AgdaSymbol{(λ}\AgdaSpace{}%
\AgdaBound{cn}\AgdaSpace{}%
\AgdaSymbol{→}\AgdaSpace{}%
\AgdaBound{f}\AgdaSpace{}%
\AgdaSymbol{(}\AgdaField{Iso.fun}\AgdaSpace{}%
\AgdaBound{CℕIso}\AgdaSpace{}%
\AgdaBound{cn}\AgdaSymbol{))}\<%
\end{code}

Brouwer trees are a the quintessential example of a higher-order inductive type.%
\footnote{Not to be confused with Higher Inductive Types (HITs) from Homotopy Type Theory~\citep{hottbook}}:
Each tree is built using smaller trees or functions producing smaller trees, which is essentially
a way of storing a possibly infinite number of smaller trees.

\subsection{Ordering Trees}

Our ultimate goal is to have a well-founded ordering%
\footnote{Technically, this is a well-founded quasi-ordering because there are pairs of
  trees which are related by both $\leq$ and $\qeq$, but which are not propositionally equal.},
so we define a relation to order Brouwer trees.

\begin{code}%
%
\>[4]\AgdaKeyword{data}\AgdaSpace{}%
\AgdaOperator{\AgdaDatatype{\AgdaUnderscore{}≤\AgdaUnderscore{}}}\AgdaSpace{}%
\AgdaSymbol{:}\AgdaSpace{}%
\AgdaDatatype{Tree}\AgdaSpace{}%
\AgdaSymbol{→}\AgdaSpace{}%
\AgdaDatatype{Tree}\AgdaSpace{}%
\AgdaSymbol{→}\AgdaSpace{}%
\AgdaPrimitive{Set}\AgdaSpace{}%
\AgdaBound{ℓ}\AgdaSpace{}%
\AgdaKeyword{where}\<%
\\
\>[4][@{}l@{\AgdaIndent{0}}]%
\>[6]\AgdaInductiveConstructor{≤-Z}\AgdaSpace{}%
\AgdaSymbol{:}\AgdaSpace{}%
\AgdaSymbol{∀}\AgdaSpace{}%
\AgdaSymbol{\{}\AgdaBound{t}\AgdaSymbol{\}}\AgdaSpace{}%
\AgdaSymbol{→}\AgdaSpace{}%
\AgdaInductiveConstructor{Z}\AgdaSpace{}%
\AgdaOperator{\AgdaDatatype{≤}}\AgdaSpace{}%
\AgdaBound{t}\<%
\\
%
\>[6]\AgdaInductiveConstructor{≤-sucMono}\AgdaSpace{}%
\AgdaSymbol{:}\AgdaSpace{}%
\AgdaSymbol{∀}\AgdaSpace{}%
\AgdaSymbol{\{}\AgdaBound{t1}\AgdaSpace{}%
\AgdaBound{t2}\AgdaSymbol{\}}\<%
\\
\>[6][@{}l@{\AgdaIndent{0}}]%
\>[8]\AgdaSymbol{→}\AgdaSpace{}%
\AgdaBound{t1}\AgdaSpace{}%
\AgdaOperator{\AgdaDatatype{≤}}\AgdaSpace{}%
\AgdaBound{t2}\<%
\\
%
\>[8]\AgdaSymbol{→}\AgdaSpace{}%
\AgdaInductiveConstructor{↑}\AgdaSpace{}%
\AgdaBound{t1}\AgdaSpace{}%
\AgdaOperator{\AgdaDatatype{≤}}\AgdaSpace{}%
\AgdaInductiveConstructor{↑}\AgdaSpace{}%
\AgdaBound{t2}\<%
\\
%
\>[6]\AgdaInductiveConstructor{≤-cocone}\AgdaSpace{}%
\AgdaSymbol{:}\AgdaSpace{}%
\AgdaSymbol{∀}%
\>[20]\AgdaSymbol{\{}\AgdaBound{t}\AgdaSymbol{\}}\AgdaSpace{}%
\AgdaSymbol{\{}\AgdaBound{c}\AgdaSpace{}%
\AgdaSymbol{:}\AgdaSpace{}%
\AgdaBound{ℂ}\AgdaSymbol{\}}\AgdaSpace{}%
\AgdaSymbol{(}\AgdaBound{f}\AgdaSpace{}%
\AgdaSymbol{:}\AgdaSpace{}%
\AgdaBound{El}\AgdaSpace{}%
\AgdaBound{c}%
\>[43]\AgdaSymbol{→}\AgdaSpace{}%
\AgdaDatatype{Tree}\AgdaSymbol{)}\AgdaSpace{}%
\AgdaSymbol{(}\AgdaBound{k}\AgdaSpace{}%
\AgdaSymbol{:}\AgdaSpace{}%
\AgdaBound{El}\AgdaSpace{}%
\AgdaBound{c}\AgdaSymbol{)}\<%
\\
\>[6][@{}l@{\AgdaIndent{0}}]%
\>[8]\AgdaSymbol{→}\AgdaSpace{}%
\AgdaBound{t}\AgdaSpace{}%
\AgdaOperator{\AgdaDatatype{≤}}\AgdaSpace{}%
\AgdaBound{f}\AgdaSpace{}%
\AgdaBound{k}\<%
\\
%
\>[8]\AgdaSymbol{→}\AgdaSpace{}%
\AgdaBound{t}\AgdaSpace{}%
\AgdaOperator{\AgdaDatatype{≤}}\AgdaSpace{}%
\AgdaInductiveConstructor{Lim}\AgdaSpace{}%
\AgdaBound{c}\AgdaSpace{}%
\AgdaBound{f}\<%
\\
%
\>[6]\AgdaInductiveConstructor{≤-limiting}\AgdaSpace{}%
\AgdaSymbol{:}\AgdaSpace{}%
\AgdaSymbol{∀}%
\>[23]\AgdaSymbol{\{}\AgdaBound{t}\AgdaSymbol{\}}\AgdaSpace{}%
\AgdaSymbol{\{}\AgdaBound{c}\AgdaSpace{}%
\AgdaSymbol{:}\AgdaSpace{}%
\AgdaBound{ℂ}\AgdaSymbol{\}}\<%
\\
\>[6][@{}l@{\AgdaIndent{0}}]%
\>[8]\AgdaSymbol{→}\AgdaSpace{}%
\AgdaSymbol{(}\AgdaBound{f}\AgdaSpace{}%
\AgdaSymbol{:}\AgdaSpace{}%
\AgdaBound{El}\AgdaSpace{}%
\AgdaBound{c}\AgdaSpace{}%
\AgdaSymbol{→}\AgdaSpace{}%
\AgdaDatatype{Tree}\AgdaSymbol{)}\<%
\\
%
\>[8]\AgdaSymbol{→}\AgdaSpace{}%
\AgdaSymbol{(∀}\AgdaSpace{}%
\AgdaBound{k}\AgdaSpace{}%
\AgdaSymbol{→}\AgdaSpace{}%
\AgdaBound{f}\AgdaSpace{}%
\AgdaBound{k}\AgdaSpace{}%
\AgdaOperator{\AgdaDatatype{≤}}\AgdaSpace{}%
\AgdaBound{t}\AgdaSymbol{)}\<%
\\
%
\>[8]\AgdaSymbol{→}\AgdaSpace{}%
\AgdaInductiveConstructor{Lim}\AgdaSpace{}%
\AgdaBound{c}\AgdaSpace{}%
\AgdaBound{f}\AgdaSpace{}%
\AgdaOperator{\AgdaDatatype{≤}}\AgdaSpace{}%
\AgdaBound{t}\<%
\\
\>[0]\<%
\end{code}
The ordering is based on the one presented by \citet{KRAUS2023113843}, but we modify it
so that transitivity can be proven constructively, rather than adding it as a constructor
for the relation. This allows us to prove well-foundedness of the relation without needing
quotient types or other advanced features.

\begin{code}%
\>[0]\<%
\\
\>[0][@{}l@{\AgdaIndent{1}}]%
\>[4]\AgdaFunction{≤-refl}\AgdaSpace{}%
\AgdaSymbol{:}\AgdaSpace{}%
\AgdaSymbol{∀}\AgdaSpace{}%
\AgdaBound{t}\AgdaSpace{}%
\AgdaSymbol{→}\AgdaSpace{}%
\AgdaBound{t}\AgdaSpace{}%
\AgdaOperator{\AgdaDatatype{≤}}\AgdaSpace{}%
\AgdaBound{t}\<%
\\
%
\>[4]\AgdaFunction{≤-refl}\AgdaSpace{}%
\AgdaInductiveConstructor{Z}\AgdaSpace{}%
\AgdaSymbol{=}\AgdaSpace{}%
\AgdaInductiveConstructor{≤-Z}\<%
\\
%
\>[4]\AgdaFunction{≤-refl}\AgdaSpace{}%
\AgdaSymbol{(}\AgdaInductiveConstructor{↑}\AgdaSpace{}%
\AgdaBound{t}\AgdaSymbol{)}\AgdaSpace{}%
\AgdaSymbol{=}\AgdaSpace{}%
\AgdaInductiveConstructor{≤-sucMono}\AgdaSpace{}%
\AgdaSymbol{(}\AgdaFunction{≤-refl}\AgdaSpace{}%
\AgdaBound{t}\AgdaSymbol{)}\<%
\\
%
\>[4]\AgdaFunction{≤-refl}\AgdaSpace{}%
\AgdaSymbol{(}\AgdaInductiveConstructor{Lim}\AgdaSpace{}%
\AgdaBound{c}\AgdaSpace{}%
\AgdaBound{f}\AgdaSymbol{)}\AgdaSpace{}%
\AgdaSymbol{=}\AgdaSpace{}%
\AgdaInductiveConstructor{≤-limiting}\AgdaSpace{}%
\AgdaBound{f}\AgdaSpace{}%
\AgdaSymbol{(λ}\AgdaSpace{}%
\AgdaBound{k}\AgdaSpace{}%
\AgdaSymbol{→}\AgdaSpace{}%
\AgdaInductiveConstructor{≤-cocone}\AgdaSpace{}%
\AgdaBound{f}\AgdaSpace{}%
\AgdaBound{k}\AgdaSpace{}%
\AgdaSymbol{(}\AgdaFunction{≤-refl}\AgdaSpace{}%
\AgdaSymbol{(}\AgdaBound{f}\AgdaSpace{}%
\AgdaBound{k}\AgdaSymbol{)))}\<%
\\
\>[0]\<%
\end{code}

\begin{code}%
\>[0]\<%
\\
%
\\[\AgdaEmptyExtraSkip]%
\>[0][@{}l@{\AgdaIndent{1}}]%
\>[4]\AgdaFunction{≤-reflEq}\AgdaSpace{}%
\AgdaSymbol{:}\AgdaSpace{}%
\AgdaSymbol{∀}\AgdaSpace{}%
\AgdaSymbol{\{}\AgdaBound{t1}\AgdaSpace{}%
\AgdaBound{t2}\AgdaSymbol{\}}\AgdaSpace{}%
\AgdaSymbol{→}\AgdaSpace{}%
\AgdaBound{t1}\AgdaSpace{}%
\AgdaOperator{\AgdaDatatype{≡}}\AgdaSpace{}%
\AgdaBound{t2}\AgdaSpace{}%
\AgdaSymbol{→}\AgdaSpace{}%
\AgdaBound{t1}\AgdaSpace{}%
\AgdaOperator{\AgdaDatatype{≤}}\AgdaSpace{}%
\AgdaBound{t2}\<%
\\
%
\>[4]\AgdaFunction{≤-reflEq}\AgdaSpace{}%
\AgdaInductiveConstructor{refl}\AgdaSpace{}%
\AgdaSymbol{=}\AgdaSpace{}%
\AgdaFunction{≤-refl}\AgdaSpace{}%
\AgdaSymbol{\AgdaUnderscore{}}\<%
\\
\>[0]\<%
\end{code}

\begin{code}%
\>[0]\<%
\\
%
\\[\AgdaEmptyExtraSkip]%
\>[0][@{}l@{\AgdaIndent{1}}]%
\>[4]\AgdaFunction{≤-trans}\AgdaSpace{}%
\AgdaSymbol{:}\AgdaSpace{}%
\AgdaSymbol{∀}\AgdaSpace{}%
\AgdaSymbol{\{}\AgdaBound{t1}\AgdaSpace{}%
\AgdaBound{t2}\AgdaSpace{}%
\AgdaBound{t3}\AgdaSymbol{\}}\AgdaSpace{}%
\AgdaSymbol{→}\AgdaSpace{}%
\AgdaBound{t1}\AgdaSpace{}%
\AgdaOperator{\AgdaDatatype{≤}}\AgdaSpace{}%
\AgdaBound{t2}\AgdaSpace{}%
\AgdaSymbol{→}\AgdaSpace{}%
\AgdaBound{t2}\AgdaSpace{}%
\AgdaOperator{\AgdaDatatype{≤}}\AgdaSpace{}%
\AgdaBound{t3}\AgdaSpace{}%
\AgdaSymbol{→}\AgdaSpace{}%
\AgdaBound{t1}\AgdaSpace{}%
\AgdaOperator{\AgdaDatatype{≤}}\AgdaSpace{}%
\AgdaBound{t3}\<%
\\
%
\>[4]\AgdaFunction{≤-trans}\AgdaSpace{}%
\AgdaInductiveConstructor{≤-Z}\AgdaSpace{}%
\AgdaBound{p23}\AgdaSpace{}%
\AgdaSymbol{=}\AgdaSpace{}%
\AgdaInductiveConstructor{≤-Z}\<%
\\
%
\>[4]\AgdaFunction{≤-trans}\AgdaSpace{}%
\AgdaSymbol{(}\AgdaInductiveConstructor{≤-sucMono}\AgdaSpace{}%
\AgdaBound{p12}\AgdaSymbol{)}\AgdaSpace{}%
\AgdaSymbol{(}\AgdaInductiveConstructor{≤-sucMono}\AgdaSpace{}%
\AgdaBound{p23}\AgdaSymbol{)}\AgdaSpace{}%
\AgdaSymbol{=}\AgdaSpace{}%
\AgdaInductiveConstructor{≤-sucMono}\AgdaSpace{}%
\AgdaSymbol{(}\AgdaFunction{≤-trans}\AgdaSpace{}%
\AgdaBound{p12}\AgdaSpace{}%
\AgdaBound{p23}\AgdaSymbol{)}\<%
\\
%
\>[4]\AgdaCatchallClause{\AgdaFunction{≤-trans}}\AgdaSpace{}%
\AgdaCatchallClause{\AgdaBound{p12}}\AgdaSpace{}%
\AgdaCatchallClause{\AgdaSymbol{(}}\AgdaCatchallClause{\AgdaInductiveConstructor{≤-cocone}}\AgdaSpace{}%
\AgdaCatchallClause{\AgdaBound{f}}\AgdaSpace{}%
\AgdaCatchallClause{\AgdaBound{k}}\AgdaSpace{}%
\AgdaCatchallClause{\AgdaBound{p23}}\AgdaCatchallClause{\AgdaSymbol{)}}\AgdaSpace{}%
\AgdaSymbol{=}\AgdaSpace{}%
\AgdaInductiveConstructor{≤-cocone}\AgdaSpace{}%
\AgdaBound{f}\AgdaSpace{}%
\AgdaBound{k}\AgdaSpace{}%
\AgdaSymbol{(}\AgdaFunction{≤-trans}\AgdaSpace{}%
\AgdaBound{p12}\AgdaSpace{}%
\AgdaBound{p23}\AgdaSymbol{)}\<%
\\
%
\>[4]\AgdaCatchallClause{\AgdaFunction{≤-trans}}\AgdaSpace{}%
\AgdaCatchallClause{\AgdaSymbol{(}}\AgdaCatchallClause{\AgdaInductiveConstructor{≤-limiting}}\AgdaSpace{}%
\AgdaCatchallClause{\AgdaBound{f}}\AgdaSpace{}%
\AgdaCatchallClause{\AgdaBound{x}}\AgdaCatchallClause{\AgdaSymbol{)}}\AgdaSpace{}%
\AgdaCatchallClause{\AgdaBound{p23}}\AgdaSpace{}%
\AgdaSymbol{=}\AgdaSpace{}%
\AgdaInductiveConstructor{≤-limiting}\AgdaSpace{}%
\AgdaBound{f}\AgdaSpace{}%
\AgdaSymbol{(λ}\AgdaSpace{}%
\AgdaBound{k}\AgdaSpace{}%
\AgdaSymbol{→}\AgdaSpace{}%
\AgdaFunction{≤-trans}\AgdaSpace{}%
\AgdaSymbol{(}\AgdaBound{x}\AgdaSpace{}%
\AgdaBound{k}\AgdaSymbol{)}\AgdaSpace{}%
\AgdaBound{p23}\AgdaSymbol{)}\<%
\\
%
\>[4]\AgdaFunction{≤-trans}\AgdaSpace{}%
\AgdaSymbol{(}\AgdaInductiveConstructor{≤-cocone}\AgdaSpace{}%
\AgdaBound{f}\AgdaSpace{}%
\AgdaBound{k}\AgdaSpace{}%
\AgdaBound{p12}\AgdaSymbol{)}\AgdaSpace{}%
\AgdaSymbol{(}\AgdaInductiveConstructor{≤-limiting}\AgdaSpace{}%
\AgdaDottedPattern{\AgdaSymbol{.}}\AgdaDottedPattern{\AgdaBound{f}}\AgdaSpace{}%
\AgdaBound{x}\AgdaSymbol{)}\AgdaSpace{}%
\AgdaSymbol{=}\AgdaSpace{}%
\AgdaFunction{≤-trans}\AgdaSpace{}%
\AgdaBound{p12}\AgdaSpace{}%
\AgdaSymbol{(}\AgdaBound{x}\AgdaSpace{}%
\AgdaBound{k}\AgdaSymbol{)}\<%
\\
\>[0]\<%
\end{code}

\begin{code}%
\>[0]\<%
\\
%
\\[\AgdaEmptyExtraSkip]%
\>[0][@{}l@{\AgdaIndent{1}}]%
\>[4]\AgdaKeyword{infixr}\AgdaSpace{}%
\AgdaNumber{10}\AgdaSpace{}%
\AgdaOperator{\AgdaFunction{\AgdaUnderscore{}≤⨟\AgdaUnderscore{}}}\<%
\\
\>[0]\<%
\end{code}

\begin{code}%
\>[0]\<%
\\
%
\\[\AgdaEmptyExtraSkip]%
\>[0][@{}l@{\AgdaIndent{1}}]%
\>[4]\AgdaOperator{\AgdaFunction{\AgdaUnderscore{}≤⨟\AgdaUnderscore{}}}\AgdaSpace{}%
\AgdaSymbol{:}%
\>[12]\AgdaSymbol{∀}\AgdaSpace{}%
\AgdaSymbol{\{}\AgdaBound{t1}\AgdaSpace{}%
\AgdaBound{t2}\AgdaSpace{}%
\AgdaBound{t3}\AgdaSymbol{\}}\AgdaSpace{}%
\AgdaSymbol{→}\AgdaSpace{}%
\AgdaBound{t1}\AgdaSpace{}%
\AgdaOperator{\AgdaDatatype{≤}}\AgdaSpace{}%
\AgdaBound{t2}\AgdaSpace{}%
\AgdaSymbol{→}\AgdaSpace{}%
\AgdaBound{t2}\AgdaSpace{}%
\AgdaOperator{\AgdaDatatype{≤}}\AgdaSpace{}%
\AgdaBound{t3}\AgdaSpace{}%
\AgdaSymbol{→}\AgdaSpace{}%
\AgdaBound{t1}\AgdaSpace{}%
\AgdaOperator{\AgdaDatatype{≤}}\AgdaSpace{}%
\AgdaBound{t3}\<%
\\
%
\>[4]\AgdaBound{lt1}\AgdaSpace{}%
\AgdaOperator{\AgdaFunction{≤⨟}}\AgdaSpace{}%
\AgdaBound{lt2}\AgdaSpace{}%
\AgdaSymbol{=}\AgdaSpace{}%
\AgdaFunction{≤-trans}\AgdaSpace{}%
\AgdaBound{lt1}\AgdaSpace{}%
\AgdaBound{lt2}\<%
\\
\>[0]\<%
\end{code}

\begin{code}%
\>[0]\<%
\\
%
\\[\AgdaEmptyExtraSkip]%
\>[0][@{}l@{\AgdaIndent{1}}]%
\>[4]\AgdaOperator{\AgdaFunction{\AgdaUnderscore{}<o\AgdaUnderscore{}}}\AgdaSpace{}%
\AgdaSymbol{:}\AgdaSpace{}%
\AgdaDatatype{Tree}\AgdaSpace{}%
\AgdaSymbol{→}\AgdaSpace{}%
\AgdaDatatype{Tree}\AgdaSpace{}%
\AgdaSymbol{→}\AgdaSpace{}%
\AgdaPrimitive{Set}\AgdaSpace{}%
\AgdaBound{ℓ}\<%
\\
%
\>[4]\AgdaBound{t1}\AgdaSpace{}%
\AgdaOperator{\AgdaFunction{<o}}\AgdaSpace{}%
\AgdaBound{t2}\AgdaSpace{}%
\AgdaSymbol{=}\AgdaSpace{}%
\AgdaInductiveConstructor{↑}\AgdaSpace{}%
\AgdaBound{t1}\AgdaSpace{}%
\AgdaOperator{\AgdaDatatype{≤}}\AgdaSpace{}%
\AgdaBound{t2}\<%
\\
%
\\[\AgdaEmptyExtraSkip]%
%
\>[4]\AgdaFunction{≤↑t}\AgdaSpace{}%
\AgdaSymbol{:}\AgdaSpace{}%
\AgdaSymbol{∀}\AgdaSpace{}%
\AgdaBound{t}\AgdaSpace{}%
\AgdaSymbol{→}\AgdaSpace{}%
\AgdaBound{t}\AgdaSpace{}%
\AgdaOperator{\AgdaDatatype{≤}}\AgdaSpace{}%
\AgdaInductiveConstructor{↑}\AgdaSpace{}%
\AgdaBound{t}\<%
\\
%
\>[4]\AgdaFunction{≤↑t}\AgdaSpace{}%
\AgdaInductiveConstructor{Z}\AgdaSpace{}%
\AgdaSymbol{=}\AgdaSpace{}%
\AgdaInductiveConstructor{≤-Z}\<%
\\
%
\>[4]\AgdaFunction{≤↑t}\AgdaSpace{}%
\AgdaSymbol{(}\AgdaInductiveConstructor{↑}\AgdaSpace{}%
\AgdaBound{t}\AgdaSymbol{)}\AgdaSpace{}%
\AgdaSymbol{=}\AgdaSpace{}%
\AgdaInductiveConstructor{≤-sucMono}\AgdaSpace{}%
\AgdaSymbol{(}\AgdaFunction{≤↑t}\AgdaSpace{}%
\AgdaBound{t}\AgdaSymbol{)}\<%
\\
%
\>[4]\AgdaFunction{≤↑t}\AgdaSpace{}%
\AgdaSymbol{(}\AgdaInductiveConstructor{Lim}\AgdaSpace{}%
\AgdaBound{c}\AgdaSpace{}%
\AgdaBound{f}\AgdaSymbol{)}\AgdaSpace{}%
\AgdaSymbol{=}\AgdaSpace{}%
\AgdaInductiveConstructor{≤-limiting}\AgdaSpace{}%
\AgdaBound{f}\AgdaSpace{}%
\AgdaSymbol{λ}\AgdaSpace{}%
\AgdaBound{k}\AgdaSpace{}%
\AgdaSymbol{→}\AgdaSpace{}%
\AgdaFunction{≤-trans}\AgdaSpace{}%
\AgdaSymbol{(}\AgdaFunction{≤↑t}\AgdaSpace{}%
\AgdaSymbol{(}\AgdaBound{f}\AgdaSpace{}%
\AgdaBound{k}\AgdaSymbol{))}\AgdaSpace{}%
\AgdaSymbol{(}\AgdaInductiveConstructor{≤-sucMono}\AgdaSpace{}%
\AgdaSymbol{(}\AgdaInductiveConstructor{≤-cocone}\AgdaSpace{}%
\AgdaBound{f}\AgdaSpace{}%
\AgdaBound{k}\AgdaSpace{}%
\AgdaSymbol{(}\AgdaFunction{≤-refl}\AgdaSpace{}%
\AgdaSymbol{(}\AgdaBound{f}\AgdaSpace{}%
\AgdaBound{k}\AgdaSymbol{))))}\<%
\\
%
\\[\AgdaEmptyExtraSkip]%
%
\\[\AgdaEmptyExtraSkip]%
%
\>[4]\AgdaFunction{<-in-≤}\AgdaSpace{}%
\AgdaSymbol{:}\AgdaSpace{}%
\AgdaSymbol{∀}\AgdaSpace{}%
\AgdaSymbol{\{}\AgdaBound{x}\AgdaSpace{}%
\AgdaBound{y}\AgdaSymbol{\}}\AgdaSpace{}%
\AgdaSymbol{→}\AgdaSpace{}%
\AgdaBound{x}\AgdaSpace{}%
\AgdaOperator{\AgdaFunction{<o}}\AgdaSpace{}%
\AgdaBound{y}\AgdaSpace{}%
\AgdaSymbol{→}\AgdaSpace{}%
\AgdaBound{x}\AgdaSpace{}%
\AgdaOperator{\AgdaDatatype{≤}}\AgdaSpace{}%
\AgdaBound{y}\<%
\\
%
\>[4]\AgdaFunction{<-in-≤}\AgdaSpace{}%
\AgdaBound{pf}\AgdaSpace{}%
\AgdaSymbol{=}\AgdaSpace{}%
\AgdaFunction{≤-trans}\AgdaSpace{}%
\AgdaSymbol{(}\AgdaFunction{≤↑t}\AgdaSpace{}%
\AgdaSymbol{\AgdaUnderscore{})}\AgdaSpace{}%
\AgdaBound{pf}\<%
\\
%
\\[\AgdaEmptyExtraSkip]%
%
\\[\AgdaEmptyExtraSkip]%
%
\>[4]\AgdaComment{--\ https://cj-xu.github.io/agda/constructive-ordinals-in-hott/BrouwerTree.Code.Results.html\#3168}\<%
\\
%
\>[4]\AgdaComment{--\ TODO:\ proper\ credit}\<%
\\
%
\>[4]\AgdaFunction{<∘≤-in-<}\AgdaSpace{}%
\AgdaSymbol{:}\AgdaSpace{}%
\AgdaSymbol{∀}\AgdaSpace{}%
\AgdaSymbol{\{}\AgdaBound{x}\AgdaSpace{}%
\AgdaBound{y}\AgdaSpace{}%
\AgdaBound{z}\AgdaSymbol{\}}\AgdaSpace{}%
\AgdaSymbol{→}\AgdaSpace{}%
\AgdaBound{x}\AgdaSpace{}%
\AgdaOperator{\AgdaFunction{<o}}\AgdaSpace{}%
\AgdaBound{y}\AgdaSpace{}%
\AgdaSymbol{→}\AgdaSpace{}%
\AgdaBound{y}\AgdaSpace{}%
\AgdaOperator{\AgdaDatatype{≤}}\AgdaSpace{}%
\AgdaBound{z}\AgdaSpace{}%
\AgdaSymbol{→}\AgdaSpace{}%
\AgdaBound{x}\AgdaSpace{}%
\AgdaOperator{\AgdaFunction{<o}}\AgdaSpace{}%
\AgdaBound{z}\<%
\\
%
\>[4]\AgdaFunction{<∘≤-in-<}\AgdaSpace{}%
\AgdaBound{x<y}\AgdaSpace{}%
\AgdaBound{y≤z}\AgdaSpace{}%
\AgdaSymbol{=}\AgdaSpace{}%
\AgdaFunction{≤-trans}\AgdaSpace{}%
\AgdaBound{x<y}\AgdaSpace{}%
\AgdaBound{y≤z}\<%
\\
%
\\[\AgdaEmptyExtraSkip]%
%
\>[4]\AgdaComment{--\ https://cj-xu.github.io/agda/constructive-ordinals-in-hott/BrouwerTree.Code.Results.html\#3168}\<%
\\
%
\>[4]\AgdaComment{--\ TODO:\ proper\ credit}\<%
\\
%
\>[4]\AgdaFunction{≤∘<-in-<}\AgdaSpace{}%
\AgdaSymbol{:}\AgdaSpace{}%
\AgdaSymbol{∀}\AgdaSpace{}%
\AgdaSymbol{\{}\AgdaBound{x}\AgdaSpace{}%
\AgdaBound{y}\AgdaSpace{}%
\AgdaBound{z}\AgdaSymbol{\}}\AgdaSpace{}%
\AgdaSymbol{→}\AgdaSpace{}%
\AgdaBound{x}\AgdaSpace{}%
\AgdaOperator{\AgdaDatatype{≤}}\AgdaSpace{}%
\AgdaBound{y}\AgdaSpace{}%
\AgdaSymbol{→}\AgdaSpace{}%
\AgdaBound{y}\AgdaSpace{}%
\AgdaOperator{\AgdaFunction{<o}}\AgdaSpace{}%
\AgdaBound{z}\AgdaSpace{}%
\AgdaSymbol{→}\AgdaSpace{}%
\AgdaBound{x}\AgdaSpace{}%
\AgdaOperator{\AgdaFunction{<o}}\AgdaSpace{}%
\AgdaBound{z}\<%
\\
%
\>[4]\AgdaFunction{≤∘<-in-<}\AgdaSpace{}%
\AgdaSymbol{\{}\AgdaBound{x}\AgdaSymbol{\}}\AgdaSpace{}%
\AgdaSymbol{\{}\AgdaBound{y}\AgdaSymbol{\}}\AgdaSpace{}%
\AgdaSymbol{\{}\AgdaBound{z}\AgdaSymbol{\}}\AgdaSpace{}%
\AgdaBound{x≤y}\AgdaSpace{}%
\AgdaBound{y<z}\AgdaSpace{}%
\AgdaSymbol{=}\AgdaSpace{}%
\AgdaFunction{≤-trans}\AgdaSpace{}%
\AgdaSymbol{(}\AgdaInductiveConstructor{≤-sucMono}\AgdaSpace{}%
\AgdaBound{x≤y}\AgdaSymbol{)}\AgdaSpace{}%
\AgdaBound{y<z}\<%
\\
%
\\[\AgdaEmptyExtraSkip]%
%
\>[4]\AgdaComment{--\ underLim\ :\ ∀\ \ \ \{c\ :\ ℂ\}\ (k\ :\ ℂ)\ t\ →\ \ (f\ :\ El\ c\ →\ Tree)\ →\ (∀\ k\ →\ t\ ≤\ f\ k)\ →\ t\ ≤\ Lim\ c\ f}\<%
\\
%
\>[4]\AgdaComment{--\ underLim\ \{c\ =\ c\}\ k\ t\ f\ all\ =\ ≤-trans\ (≤-cocone\ (λ\ \AgdaUnderscore{}\ →\ t)\ \{!!\}\ (≤-refl\ t))\ (≤-limiting\ (λ\ \AgdaUnderscore{}\ →\ t)\ (λ\ k\ →\ ≤-cocone\ f\ k\ (all\ k)))}\<%
\\
%
\\[\AgdaEmptyExtraSkip]%
%
\>[4]\AgdaFunction{extLim}\AgdaSpace{}%
\AgdaSymbol{:}\AgdaSpace{}%
\AgdaSymbol{∀}%
\>[17]\AgdaSymbol{\{}\AgdaBound{c}\AgdaSpace{}%
\AgdaSymbol{:}\AgdaSpace{}%
\AgdaBound{ℂ}\AgdaSymbol{\}}\AgdaSpace{}%
\AgdaSymbol{→}%
\>[28]\AgdaSymbol{(}\AgdaBound{f1}\AgdaSpace{}%
\AgdaBound{f2}\AgdaSpace{}%
\AgdaSymbol{:}\AgdaSpace{}%
\AgdaBound{El}\AgdaSpace{}%
\AgdaBound{c}\AgdaSpace{}%
\AgdaSymbol{→}\AgdaSpace{}%
\AgdaDatatype{Tree}\AgdaSymbol{)}\AgdaSpace{}%
\AgdaSymbol{→}\AgdaSpace{}%
\AgdaSymbol{(∀}\AgdaSpace{}%
\AgdaBound{k}\AgdaSpace{}%
\AgdaSymbol{→}\AgdaSpace{}%
\AgdaBound{f1}\AgdaSpace{}%
\AgdaBound{k}\AgdaSpace{}%
\AgdaOperator{\AgdaDatatype{≤}}\AgdaSpace{}%
\AgdaBound{f2}\AgdaSpace{}%
\AgdaBound{k}\AgdaSymbol{)}\AgdaSpace{}%
\AgdaSymbol{→}\AgdaSpace{}%
\AgdaInductiveConstructor{Lim}\AgdaSpace{}%
\AgdaBound{c}\AgdaSpace{}%
\AgdaBound{f1}\AgdaSpace{}%
\AgdaOperator{\AgdaDatatype{≤}}\AgdaSpace{}%
\AgdaInductiveConstructor{Lim}\AgdaSpace{}%
\AgdaBound{c}\AgdaSpace{}%
\AgdaBound{f2}\<%
\\
%
\>[4]\AgdaFunction{extLim}\AgdaSpace{}%
\AgdaSymbol{\{}\AgdaArgument{c}\AgdaSpace{}%
\AgdaSymbol{=}\AgdaSpace{}%
\AgdaBound{c}\AgdaSymbol{\}}\AgdaSpace{}%
\AgdaBound{f1}\AgdaSpace{}%
\AgdaBound{f2}\AgdaSpace{}%
\AgdaBound{all}\AgdaSpace{}%
\AgdaSymbol{=}\AgdaSpace{}%
\AgdaInductiveConstructor{≤-limiting}\AgdaSpace{}%
\AgdaBound{f1}\AgdaSpace{}%
\AgdaSymbol{(λ}\AgdaSpace{}%
\AgdaBound{k}\AgdaSpace{}%
\AgdaSymbol{→}\AgdaSpace{}%
\AgdaInductiveConstructor{≤-cocone}\AgdaSpace{}%
\AgdaBound{f2}\AgdaSpace{}%
\AgdaBound{k}\AgdaSpace{}%
\AgdaSymbol{(}\AgdaBound{all}\AgdaSpace{}%
\AgdaBound{k}\AgdaSymbol{))}\<%
\\
%
\\[\AgdaEmptyExtraSkip]%
%
\\[\AgdaEmptyExtraSkip]%
%
\>[4]\AgdaFunction{existsLim}\AgdaSpace{}%
\AgdaSymbol{:}\AgdaSpace{}%
\AgdaSymbol{∀}%
\>[19]\AgdaSymbol{\{}\AgdaBound{c1}\AgdaSpace{}%
\AgdaSymbol{:}\AgdaSpace{}%
\AgdaBound{ℂ}\AgdaSymbol{\}}\AgdaSpace{}%
\AgdaSymbol{\{}\AgdaBound{c2}\AgdaSpace{}%
\AgdaSymbol{:}\AgdaSpace{}%
\AgdaBound{ℂ}\AgdaSymbol{\}}\AgdaSpace{}%
\AgdaSymbol{→}%
\>[40]\AgdaSymbol{(}\AgdaBound{f1}\AgdaSpace{}%
\AgdaSymbol{:}\AgdaSpace{}%
\AgdaBound{El}\AgdaSpace{}%
\AgdaBound{c1}%
\>[53]\AgdaSymbol{→}\AgdaSpace{}%
\AgdaDatatype{Tree}\AgdaSymbol{)}\AgdaSpace{}%
\AgdaSymbol{(}\AgdaBound{f2}\AgdaSpace{}%
\AgdaSymbol{:}\AgdaSpace{}%
\AgdaBound{El}%
\>[71]\AgdaBound{c2}%
\>[75]\AgdaSymbol{→}\AgdaSpace{}%
\AgdaDatatype{Tree}\AgdaSymbol{)}\AgdaSpace{}%
\AgdaSymbol{→}\AgdaSpace{}%
\AgdaSymbol{(∀}\AgdaSpace{}%
\AgdaBound{k1}\AgdaSpace{}%
\AgdaSymbol{→}\AgdaSpace{}%
\AgdaFunction{Σ[}\AgdaSpace{}%
\AgdaBound{k2}\AgdaSpace{}%
\AgdaFunction{∈}\AgdaSpace{}%
\AgdaBound{El}%
\>[105]\AgdaBound{c2}\AgdaSpace{}%
\AgdaFunction{]}\AgdaSpace{}%
\AgdaBound{f1}\AgdaSpace{}%
\AgdaBound{k1}\AgdaSpace{}%
\AgdaOperator{\AgdaDatatype{≤}}\AgdaSpace{}%
\AgdaBound{f2}\AgdaSpace{}%
\AgdaBound{k2}\AgdaSymbol{)}\AgdaSpace{}%
\AgdaSymbol{→}\AgdaSpace{}%
\AgdaInductiveConstructor{Lim}%
\>[132]\AgdaBound{c1}\AgdaSpace{}%
\AgdaBound{f1}\AgdaSpace{}%
\AgdaOperator{\AgdaDatatype{≤}}\AgdaSpace{}%
\AgdaInductiveConstructor{Lim}%
\>[145]\AgdaBound{c2}\AgdaSpace{}%
\AgdaBound{f2}\<%
\\
%
\>[4]\AgdaFunction{existsLim}\AgdaSpace{}%
\AgdaSymbol{\{}\AgdaBound{æ1}\AgdaSymbol{\}}\AgdaSpace{}%
\AgdaSymbol{\{}\AgdaBound{æ2}\AgdaSymbol{\}}\AgdaSpace{}%
\AgdaBound{f1}\AgdaSpace{}%
\AgdaBound{f2}\AgdaSpace{}%
\AgdaBound{allex}\AgdaSpace{}%
\AgdaSymbol{=}\AgdaSpace{}%
\AgdaInductiveConstructor{≤-limiting}%
\>[50]\AgdaBound{f1}\AgdaSpace{}%
\AgdaSymbol{(λ}\AgdaSpace{}%
\AgdaBound{k}\AgdaSpace{}%
\AgdaSymbol{→}\AgdaSpace{}%
\AgdaInductiveConstructor{≤-cocone}\AgdaSpace{}%
\AgdaBound{f2}\AgdaSpace{}%
\AgdaSymbol{(}\AgdaField{proj₁}\AgdaSpace{}%
\AgdaSymbol{(}\AgdaBound{allex}\AgdaSpace{}%
\AgdaBound{k}\AgdaSymbol{))}\AgdaSpace{}%
\AgdaSymbol{(}\AgdaField{proj₂}\AgdaSpace{}%
\AgdaSymbol{(}\AgdaBound{allex}\AgdaSpace{}%
\AgdaBound{k}\AgdaSymbol{)))}\<%
\\
%
\\[\AgdaEmptyExtraSkip]%
%
\\[\AgdaEmptyExtraSkip]%
%
\>[4]\AgdaFunction{¬Z<↑o}\AgdaSpace{}%
\AgdaSymbol{:}\AgdaSpace{}%
\AgdaSymbol{∀}%
\>[15]\AgdaBound{t}\AgdaSpace{}%
\AgdaSymbol{→}\AgdaSpace{}%
\AgdaOperator{\AgdaFunction{¬}}\AgdaSpace{}%
\AgdaSymbol{((}\AgdaInductiveConstructor{↑}\AgdaSpace{}%
\AgdaBound{t}\AgdaSymbol{)}\AgdaSpace{}%
\AgdaOperator{\AgdaDatatype{≤}}\AgdaSpace{}%
\AgdaInductiveConstructor{Z}\AgdaSymbol{)}\<%
\\
%
\>[4]\AgdaFunction{¬Z<↑o}\AgdaSpace{}%
\AgdaBound{t}\AgdaSpace{}%
\AgdaSymbol{()}\<%
\\
%
\\[\AgdaEmptyExtraSkip]%
%
\\[\AgdaEmptyExtraSkip]%
%
\>[4]\AgdaKeyword{open}\AgdaSpace{}%
\AgdaKeyword{import}\AgdaSpace{}%
\AgdaModule{Induction.WellFounded}\<%
\\
%
\>[4]\AgdaFunction{access}\AgdaSpace{}%
\AgdaSymbol{:}\AgdaSpace{}%
\AgdaSymbol{∀}\AgdaSpace{}%
\AgdaSymbol{\{}\AgdaBound{x}\AgdaSymbol{\}}\AgdaSpace{}%
\AgdaSymbol{→}\AgdaSpace{}%
\AgdaDatatype{Acc}\AgdaSpace{}%
\AgdaOperator{\AgdaFunction{\AgdaUnderscore{}<o\AgdaUnderscore{}}}\AgdaSpace{}%
\AgdaBound{x}\AgdaSpace{}%
\AgdaSymbol{→}\AgdaSpace{}%
\AgdaFunction{WfRec}\AgdaSpace{}%
\AgdaOperator{\AgdaFunction{\AgdaUnderscore{}<o\AgdaUnderscore{}}}\AgdaSpace{}%
\AgdaSymbol{(}\AgdaDatatype{Acc}\AgdaSpace{}%
\AgdaOperator{\AgdaFunction{\AgdaUnderscore{}<o\AgdaUnderscore{}}}\AgdaSymbol{)}\AgdaSpace{}%
\AgdaBound{x}\<%
\\
%
\>[4]\AgdaFunction{access}\AgdaSpace{}%
\AgdaSymbol{(}\AgdaInductiveConstructor{acc}\AgdaSpace{}%
\AgdaBound{r}\AgdaSymbol{)}\AgdaSpace{}%
\AgdaSymbol{=}\AgdaSpace{}%
\AgdaBound{r}\<%
\\
%
\\[\AgdaEmptyExtraSkip]%
%
\>[4]\AgdaComment{--\ https://cj-xu.github.io/agda/constructive-ordinals-in-hott/BrouwerTree.Code.Results.html\#3168}\<%
\\
%
\>[4]\AgdaComment{--\ TODO:\ proper\ credit}\<%
\\
%
\>[4]\AgdaFunction{smaller-accessible}\AgdaSpace{}%
\AgdaSymbol{:}\AgdaSpace{}%
\AgdaSymbol{(}\AgdaBound{x}\AgdaSpace{}%
\AgdaSymbol{:}\AgdaSpace{}%
\AgdaDatatype{Tree}\AgdaSymbol{)}\AgdaSpace{}%
\AgdaSymbol{→}\AgdaSpace{}%
\AgdaDatatype{Acc}\AgdaSpace{}%
\AgdaOperator{\AgdaFunction{\AgdaUnderscore{}<o\AgdaUnderscore{}}}\AgdaSpace{}%
\AgdaBound{x}\AgdaSpace{}%
\AgdaSymbol{→}\AgdaSpace{}%
\AgdaSymbol{∀}\AgdaSpace{}%
\AgdaBound{y}\AgdaSpace{}%
\AgdaSymbol{→}\AgdaSpace{}%
\AgdaBound{y}\AgdaSpace{}%
\AgdaOperator{\AgdaDatatype{≤}}\AgdaSpace{}%
\AgdaBound{x}\AgdaSpace{}%
\AgdaSymbol{→}\AgdaSpace{}%
\AgdaDatatype{Acc}\AgdaSpace{}%
\AgdaOperator{\AgdaFunction{\AgdaUnderscore{}<o\AgdaUnderscore{}}}\AgdaSpace{}%
\AgdaBound{y}\<%
\\
%
\>[4]\AgdaFunction{smaller-accessible}\AgdaSpace{}%
\AgdaBound{x}\AgdaSpace{}%
\AgdaBound{isAcc}\AgdaSpace{}%
\AgdaBound{y}\AgdaSpace{}%
\AgdaBound{x<y}\AgdaSpace{}%
\AgdaSymbol{=}\AgdaSpace{}%
\AgdaInductiveConstructor{acc}\AgdaSpace{}%
\AgdaSymbol{(λ}\AgdaSpace{}%
\AgdaBound{y'}\AgdaSpace{}%
\AgdaBound{y'<y}\AgdaSpace{}%
\AgdaSymbol{→}\AgdaSpace{}%
\AgdaFunction{access}\AgdaSpace{}%
\AgdaBound{isAcc}\AgdaSpace{}%
\AgdaBound{y'}\AgdaSpace{}%
\AgdaSymbol{(}\AgdaFunction{<∘≤-in-<}\AgdaSpace{}%
\AgdaBound{y'<y}\AgdaSpace{}%
\AgdaBound{x<y}\AgdaSymbol{))}\<%
\\
%
\\[\AgdaEmptyExtraSkip]%
%
\>[4]\AgdaComment{--\ https://cj-xu.github.io/agda/constructive-ordinals-in-hott/BrouwerTree.Code.Results.html\#3168}\<%
\\
%
\>[4]\AgdaComment{--\ TODO:\ proper\ credit}\<%
\\
%
\>[4]\AgdaFunction{ordWF}\AgdaSpace{}%
\AgdaSymbol{:}\AgdaSpace{}%
\AgdaFunction{WellFounded}\AgdaSpace{}%
\AgdaOperator{\AgdaFunction{\AgdaUnderscore{}<o\AgdaUnderscore{}}}\<%
\\
%
\>[4]\AgdaFunction{ordWF}\AgdaSpace{}%
\AgdaInductiveConstructor{Z}\AgdaSpace{}%
\AgdaSymbol{=}\AgdaSpace{}%
\AgdaInductiveConstructor{acc}\AgdaSpace{}%
\AgdaSymbol{λ}\AgdaSpace{}%
\AgdaSymbol{\AgdaUnderscore{}}\AgdaSpace{}%
\AgdaSymbol{()}\<%
\\
%
\>[4]\AgdaFunction{ordWF}\AgdaSpace{}%
\AgdaSymbol{(}\AgdaInductiveConstructor{↑}\AgdaSpace{}%
\AgdaBound{x}\AgdaSymbol{)}\AgdaSpace{}%
\AgdaSymbol{=}\AgdaSpace{}%
\AgdaInductiveConstructor{acc}\AgdaSpace{}%
\AgdaSymbol{(λ}\AgdaSpace{}%
\AgdaSymbol{\{}\AgdaSpace{}%
\AgdaBound{y}\AgdaSpace{}%
\AgdaSymbol{(}\AgdaInductiveConstructor{≤-sucMono}\AgdaSpace{}%
\AgdaBound{y≤x}\AgdaSymbol{)}\AgdaSpace{}%
\AgdaSymbol{→}\AgdaSpace{}%
\AgdaFunction{smaller-accessible}\AgdaSpace{}%
\AgdaBound{x}\AgdaSpace{}%
\AgdaSymbol{(}\AgdaFunction{ordWF}\AgdaSpace{}%
\AgdaBound{x}\AgdaSymbol{)}\AgdaSpace{}%
\AgdaBound{y}\AgdaSpace{}%
\AgdaBound{y≤x}\AgdaSymbol{\})}\<%
\\
%
\>[4]\AgdaFunction{ordWF}\AgdaSpace{}%
\AgdaSymbol{(}\AgdaInductiveConstructor{Lim}\AgdaSpace{}%
\AgdaBound{c}\AgdaSpace{}%
\AgdaBound{f}\AgdaSymbol{)}\AgdaSpace{}%
\AgdaSymbol{=}\AgdaSpace{}%
\AgdaInductiveConstructor{acc}\AgdaSpace{}%
\AgdaFunction{helper}\<%
\\
\>[4][@{}l@{\AgdaIndent{0}}]%
\>[6]\AgdaKeyword{where}\<%
\\
\>[6][@{}l@{\AgdaIndent{0}}]%
\>[8]\AgdaFunction{helper}\AgdaSpace{}%
\AgdaSymbol{:}\AgdaSpace{}%
\AgdaSymbol{(}\AgdaBound{y}\AgdaSpace{}%
\AgdaSymbol{:}\AgdaSpace{}%
\AgdaDatatype{Tree}\AgdaSymbol{)}\AgdaSpace{}%
\AgdaSymbol{→}\AgdaSpace{}%
\AgdaSymbol{(}\AgdaBound{y}\AgdaSpace{}%
\AgdaOperator{\AgdaFunction{<o}}\AgdaSpace{}%
\AgdaInductiveConstructor{Lim}\AgdaSpace{}%
\AgdaBound{c}\AgdaSpace{}%
\AgdaBound{f}\AgdaSymbol{)}\AgdaSpace{}%
\AgdaSymbol{→}\AgdaSpace{}%
\AgdaDatatype{Acc}\AgdaSpace{}%
\AgdaOperator{\AgdaFunction{\AgdaUnderscore{}<o\AgdaUnderscore{}}}\AgdaSpace{}%
\AgdaBound{y}\<%
\\
%
\>[8]\AgdaFunction{helper}\AgdaSpace{}%
\AgdaBound{y}\AgdaSpace{}%
\AgdaSymbol{(}\AgdaInductiveConstructor{≤-cocone}\AgdaSpace{}%
\AgdaDottedPattern{\AgdaSymbol{.}}\AgdaDottedPattern{\AgdaBound{f}}\AgdaSpace{}%
\AgdaBound{k}\AgdaSpace{}%
\AgdaBound{y<fk}\AgdaSymbol{)}\AgdaSpace{}%
\AgdaSymbol{=}\AgdaSpace{}%
\AgdaFunction{smaller-accessible}\AgdaSpace{}%
\AgdaSymbol{(}\AgdaBound{f}\AgdaSpace{}%
\AgdaBound{k}\AgdaSymbol{)}\AgdaSpace{}%
\AgdaSymbol{(}\AgdaFunction{ordWF}\AgdaSpace{}%
\AgdaSymbol{(}\AgdaBound{f}\AgdaSpace{}%
\AgdaBound{k}\AgdaSymbol{))}\AgdaSpace{}%
\AgdaBound{y}\AgdaSpace{}%
\AgdaSymbol{(}\AgdaFunction{<-in-≤}\AgdaSpace{}%
\AgdaBound{y<fk}\AgdaSymbol{)}\<%
\\
%
\\[\AgdaEmptyExtraSkip]%
%
\\[\AgdaEmptyExtraSkip]%
\>[0]\<%
\end{code}


\begin{code}%
\>[0][@{}l@{\AgdaIndent{1}}]%
\>[4]\AgdaFunction{limMax}\AgdaSpace{}%
\AgdaSymbol{:}\AgdaSpace{}%
\AgdaDatatype{Tree}\AgdaSpace{}%
\AgdaSymbol{→}\AgdaSpace{}%
\AgdaDatatype{Tree}\AgdaSpace{}%
\AgdaSymbol{→}\AgdaSpace{}%
\AgdaDatatype{Tree}\<%
\\
%
\>[4]\AgdaFunction{limMax}\AgdaSpace{}%
\AgdaBound{t1}\AgdaSpace{}%
\AgdaBound{t2}\AgdaSpace{}%
\AgdaSymbol{=}\AgdaSpace{}%
\AgdaInductiveConstructor{Lim}\AgdaSpace{}%
\AgdaBound{Cℕ}\AgdaSpace{}%
\AgdaSymbol{λ}\AgdaSpace{}%
\AgdaBound{k}\AgdaSpace{}%
\AgdaSymbol{→}\AgdaSpace{}%
\AgdaFunction{if0}\AgdaSpace{}%
\AgdaSymbol{(}\AgdaField{Iso.fun}\AgdaSpace{}%
\AgdaBound{CℕIso}\AgdaSpace{}%
\AgdaBound{k}\AgdaSymbol{)}\AgdaSpace{}%
\AgdaBound{t1}\AgdaSpace{}%
\AgdaBound{t2}\<%
\\
\>[4][@{}l@{\AgdaIndent{0}}]%
\>[6]\AgdaKeyword{where}\<%
\\
\>[6][@{}l@{\AgdaIndent{0}}]%
\>[8]\AgdaFunction{if0}\AgdaSpace{}%
\AgdaSymbol{:}\AgdaSpace{}%
\AgdaDatatype{ℕ}\AgdaSpace{}%
\AgdaSymbol{→}\AgdaSpace{}%
\AgdaDatatype{Tree}\AgdaSpace{}%
\AgdaSymbol{→}\AgdaSpace{}%
\AgdaDatatype{Tree}\AgdaSpace{}%
\AgdaSymbol{→}\AgdaSpace{}%
\AgdaDatatype{Tree}\<%
\\
%
\>[8]\AgdaFunction{if0}\AgdaSpace{}%
\AgdaInductiveConstructor{zero}\AgdaSpace{}%
\AgdaBound{z}\AgdaSpace{}%
\AgdaBound{s}\AgdaSpace{}%
\AgdaSymbol{=}\AgdaSpace{}%
\AgdaBound{z}\<%
\\
%
\>[8]\AgdaFunction{if0}\AgdaSpace{}%
\AgdaSymbol{(}\AgdaInductiveConstructor{suc}\AgdaSpace{}%
\AgdaBound{n}\AgdaSymbol{)}\AgdaSpace{}%
\AgdaBound{z}\AgdaSpace{}%
\AgdaBound{s}\AgdaSpace{}%
\AgdaSymbol{=}\AgdaSpace{}%
\AgdaBound{s}\<%
\end{code}


% \begin{code}


%     private
%       data IndMaxView : Tree → Tree → Set ℓ where
%         IndMaxZ-L : ∀ {t} → IndMaxView Z o
%         IndMaxZ-R : ∀ {t} → IndMaxView t Z
%         IndMaxLim-L : ∀ {o } {c : ℂ} {f : El c → Tree} → IndMaxView (Lim c f) o
%         IndMaxLim-R : ∀ {o } {c : ℂ} {f : El c → Tree}
%           → (∀   {c' : ℂ} {f' : El c' → Tree} → ¬ (o ≡ Lim  c' f'))
%           → IndMaxView t (Lim c f)
%         IndMaxLim-Suc : ∀  {t1 t2 } → IndMaxView (↑ t1) (↑ t2)

%       indMaxView : ∀ t1 t2 → IndMaxView t1 t2
%       indMaxView Z t2 = IndMaxZ-L
%       indMaxView (Lim c f) t2 = IndMaxLim-L
%       indMaxView (↑ t1) Z = IndMaxZ-R
%       indMaxView (↑ t1) (Lim c f) = IndMaxLim-R λ ()
%       indMaxView (↑ t1) (↑ t2) = IndMaxLim-Suc

%     abstract
%       indMax : Tree → Tree → Tree
%       indMax' : ∀ {t1 t2} → IndMaxView t1 t2 → Tree

%       indMax t1 t2 = indMax' (indMaxView t1 t2)

%       indMax' {.Z} {t2} IndMaxZ-L = t2
%       indMax' {t1} {.Z} IndMaxZ-R = t1
%       indMax' {(Lim c f)} {t2} IndMaxLim-L = Lim c λ x → indMax (f x) t2
%       indMax' {t1} {(Lim c f)} (IndMaxLim-R _) = Lim c (λ x → indMax t1 (f x))
%       indMax' {(↑ t1)} {(↑ t2)} IndMaxLim-Suc = ↑ (indMax t1 t2)

%       indMax-≤L : ∀ {t1 t2} → t1 ≤ indMax t1 t2
%       indMax-≤L {t1} {t2} with indMaxView t1 t2
%       ... | IndMaxZ-L = ≤-Z
%       ... | IndMaxZ-R = ≤-refl _
%       ... | IndMaxLim-L {f = f} = extLim f (λ x → indMax (f x) t2) (λ k → indMax-≤L)
%       ... | IndMaxLim-R {f = f} _ = underLim {!!} t1 (λ x → indMax t1 (f x)) (λ k → indMax-≤L)
%       ... | IndMaxLim-Suc = ≤-sucMono indMax-≤L


%       indMax-≤R : ∀ {t1 t2} → t2 ≤ indMax t1 t2
%       indMax-≤R {t1} {t2} with indMaxView t1 t2
%       ... | IndMaxZ-R = ≤-Z
%       ... | IndMaxZ-L = ≤-refl _
%       ... | IndMaxLim-R {f = f} _ = extLim f (λ x → indMax t1 (f x)) (λ k → indMax-≤R {t1 = t1} {f k})
%       ... | IndMaxLim-L {f = f} = underLim {!!} t2 (λ x → indMax (f x) t2) (λ k → indMax-≤R {t1 = f k} {t2 = t2})
%       ... | IndMaxLim-Suc {t1} {t2} = ≤-sucMono (indMax-≤R {t1 = t1} {t2 = t2})




%       indMax-monoR : ∀ {t1 t2 t2'} → t2 ≤ t2' → indMax t1 t2 ≤ indMax t1 t2'
%       indMax-monoR' : ∀ {t1 t2 t2'} → t2 <o t2' → indMax t1 t2 <o indMax (↑ t1) t2'

%       indMax-monoR {t1} {t2} {t2'} lt with indMaxView t1 t2 in eq1 | indMaxView t1 t2' in eq2
%       ... | IndMaxZ-L  | v2  = ≤-trans lt (≤-reflEq (cong indMax' eq2))
%       ... | IndMaxZ-R  | v2  = ≤-trans indMax-≤L (≤-reflEq (cong indMax' eq2))
%       ... | IndMaxLim-L {f = f1} |  IndMaxLim-L  = extLim _ _ λ k → indMax-monoR {t1 = f1 k} lt
%       indMax-monoR {t1} {(Lim _ f')} {.(Lim _ f)} (≤-cocone f k lt) | IndMaxLim-R neq  | IndMaxLim-R neq'
%         = ≤-limiting (λ x → indMax t1 (f' x)) (λ y → ≤-cocone (λ x → indMax t1 (f x)) k (indMax-monoR {t1 = t1} {t2 = f' y} (≤-trans (≤-cocone _ y (≤-refl _)) lt)))
%       indMax-monoR {t1} {.(Lim _ _)} {t2'} (≤-limiting f x₁) | IndMaxLim-R x  | v2  =
%         ≤-trans (≤-limiting (λ x₂ → indMax t1 (f x₂)) λ k → indMax-monoR {t1 = t1} (x₁ k)) (≤-reflEq (cong indMax' eq2))
%       indMax-monoR {(↑ t1)} {.(↑ _)} {.(↑ _)} (≤-sucMono lt) | IndMaxLim-Suc  | IndMaxLim-Suc  = ≤-sucMono (indMax-monoR {t1 = t1} lt)
%       indMax-monoR {(↑ t1)} {(↑ t2)} {(Lim _ f)} (≤-cocone f k lt) | IndMaxLim-Suc  | IndMaxLim-R x
%         = ≤-trans (indMax-monoR' {t1 = t1} {t2 = t2} {t2' = f k} lt) (≤-cocone (λ x₁ → indMax (↑ t1) (f x₁)) k (≤-refl _)) --indMax-monoR' {!lt!}

%       indMax-monoR' {t1} {t2} {t2'}  (≤-sucMono lt) = ≤-sucMono ( (indMax-monoR {t1 = t1} lt))
%       indMax-monoR' {t1} {t2} {.(Lim _ f)} (≤-cocone f k lt)
%         = ≤-cocone _ k (indMax-monoR' {t1 = t1} lt)


%       indMax-monoL : ∀ {t1 t1' t2} → t1 ≤ t1' → indMax t1 t2 ≤ indMax t1' t2
%       indMax-monoL' : ∀ {t1 t1' t2} → t1 <o t1' → indMax t1 t2 <o indMax t1' (↑ t2)
%       indMax-monoL {t1} {t1'} {t2} lt with indMaxView t1 t2 in eq1 | indMaxView t1' t2 in eq2
%       ... | IndMaxZ-L | v2 = ≤-trans (indMax-≤R {t1 = t1'}) (≤-reflEq (cong indMax' eq2))
%       ... | IndMaxZ-R | v2 = ≤-trans lt (≤-trans (indMax-≤L {t1 = t1'}) (≤-reflEq (cong indMax' eq2)))
%       indMax-monoL {.(Lim _ _)} {.(Lim _ f)} {t2} (≤-cocone f k lt) | IndMaxLim-L  | IndMaxLim-L
%         = ≤-cocone (λ x → indMax (f x) t2) k (indMax-monoL lt)
%       indMax-monoL {.(Lim _ _)} {t1'} {t2} (≤-limiting f lt) | IndMaxLim-L |  v2
%         = ≤-limiting (λ x₁ → indMax (f x₁) t2) λ k → ≤-trans (indMax-monoL (lt k)) (≤-reflEq (cong indMax' eq2))
%       indMax-monoL {.Z} {.Z} {.(Lim _ _)} ≤-Z | IndMaxLim-R neq  | IndMaxZ-L  = ≤-refl _
%       indMax-monoL  {.(Lim _ f)} {.Z} {.(Lim _ _)} (≤-limiting f x) | IndMaxLim-R neq | IndMaxZ-L
%         with () ← neq refl
%       indMax-monoL {t1} {.(Lim _ _)} {.(Lim _ _)} (≤-cocone _ k lt) | IndMaxLim-R {f = f} neq | IndMaxLim-L {f = f'}
%         = ≤-limiting (λ x → indMax t1 (f x)) (λ y → ≤-cocone (λ x → indMax (f' x) (Lim _ _)) k
%           (≤-trans (indMax-monoL lt) (indMax-monoR {t1 = f' k} (≤-cocone f y (≤-refl _)))))
%       ... | IndMaxLim-R neq | IndMaxLim-R {f = f} neq' = extLim (λ x → indMax t1 (f x)) (λ x → indMax t1' (f x)) (λ k → indMax-monoL lt)
%       indMax-monoL {.(↑ _)} {.(↑ _)} {.(↑ _)} (≤-sucMono lt) | IndMaxLim-Suc  | IndMaxLim-Suc
%         = ≤-sucMono (indMax-monoL lt)
%       indMax-monoL {.(↑ _)} {.(Lim _ f)} {.(↑ _)} (≤-cocone f k lt) | IndMaxLim-Suc  | IndMaxLim-L
%         = ≤-cocone (λ x → indMax (f x) (↑ _)) k (indMax-monoL' lt)

%       indMax-monoL' {t1} {t1'} {t2} lt with indMaxView t1 t2 in eq1 | indMaxView t1' t2 in eq2
%       indMax-monoL' {t1} {.(↑ _)} {t2} (≤-sucMono lt) | v1 | v2 = ≤-sucMono (≤-trans (≤-reflEq (cong indMax' (sym eq1))) (indMax-monoL lt))
%       indMax-monoL' {t1} {.(Lim _ f)} {t2} (≤-cocone f k lt) | v1 | v2
%         = ≤-cocone _ k (≤-trans (≤-sucMono (≤-reflEq (cong indMax' (sym eq1)))) (indMax-monoL' lt))


%       indMax-mono : ∀ {t1 t2 t1' t2'} → t1 ≤ t1' → t2 ≤ t2' → indMax t1 t2 ≤ indMax t1' t2'
%       indMax-mono {t1' = t1'} lt1 lt2 = ≤-trans (indMax-monoL lt1) (indMax-monoR {t1 = t1'} lt2)

%       indMax-strictMono : ∀ {t1 t2 t1' t2'} → t1 <o t1' → t2 <o t2' → indMax t1 t2 <o indMax t1' t2'
%       indMax-strictMono lt1 lt2 = indMax-mono lt1 lt2


%       indMax-sucMono : ∀ {t1 t2 t1' t2'} → indMax t1 t2 ≤ indMax t1' t2' → indMax t1 t2 <o indMax (↑ t1') (↑ t2')
%       indMax-sucMono lt = ≤-sucMono lt


%       -- indMax-Z : ∀ t → indMax t Z ≡ o
%       -- indMax-Z Z = refl
%       -- indMax-Z (↑ t) = refl
%       -- indMax-Z (Lim c f) = cong (Lim c) {!!} -- cong (Lim c) (funExt (λ x → indMax-Z (f x)))

%       indMax-Z : ∀ t → indMax t Z ≤ o
%       indMax-Z Z = ≤-Z
%       indMax-Z (↑ t) = ≤-refl (indMax (↑ t) Z)
%       indMax-Z (Lim c f) = extLim (λ x → indMax (f x) Z) f (λ k → indMax-Z (f k))

%       indMax-↑ : ∀ {t1 t2} → indMax (↑ t1) (↑ t2) ≡ ↑ (indMax t1 t2)
%       indMax-↑ = refl

%       indMax-≤Z : ∀ t → indMax t Z ≤ o
%       indMax-≤Z Z = ≤-refl _
%       indMax-≤Z (↑ t) = ≤-refl _
%       indMax-≤Z (Lim c f) = extLim _ _ (λ k → indMax-≤Z (f k))

%       -- indMax-oneL : ∀ {t} → indMax T1 (↑ t) ≤ ↑ t
%       -- indMax-oneL  = ≤-refl _

%       -- indMax-oneR : ∀ {t} → indMax (↑ t) T1 ≤ ↑ t
%       -- indMax-oneR {Z} = ≤-sucMono (≤-refl _)
%       -- indMax-oneR {↑ t} = ≤-sucMono (≤-refl _)
%       -- indMax-oneR {Lim c f} = ≤-sucMono (substPath (λ x → x ≤ Lim c f) (sym (indMax-Z (Lim c f))) (≤-refl (Lim c f))) -- rewrite ctop (indMax-Z (Lim c f))= ≤-refl _


%       indMax-limR : ∀   {c : ℂ} (f : El  c  → Tree) t → indMax t (Lim c f) ≤ Lim c (λ k → indMax t (f k))
%       indMax-limR f Z = ≤-refl _
%       indMax-limR f (↑ t) = extLim _ _ λ k → ≤-refl _
%       indMax-limR f (Lim c f₁) = ≤-limiting _ λ k → ≤-trans (indMax-limR f (f₁ k)) (extLim _ _ (λ k2 → indMax-monoL {t1 = f₁ k} {t1' = Lim c f₁} {t2 = f k2}  (≤-cocone _ k (≤-refl _))))


%       indMax-commut : ∀ t1 t2 → indMax t1 t2 ≤ indMax t2 t1
%       indMax-commut t1 t2 with indMaxView t1 t2
%       ... | IndMaxZ-L = indMax-≤L
%       ... | IndMaxZ-R = ≤-refl _
%       ... | IndMaxLim-R {f = f} x = extLim _ _ (λ k → indMax-commut t1 (f k))
%       ... | IndMaxLim-Suc {t1 = t1} {t2 = t2} = ≤-sucMono (indMax-commut t1 t2)
%       ... | IndMaxLim-L {c = c} {f = f} with indMaxView t2 t1
%       ... | IndMaxZ-L = extLim _ _ λ k → indMax-Z (f k)
%       ... | IndMaxLim-R x = extLim _ _ (λ k → indMax-commut (f k) t2)
%       ... | IndMaxLim-L {c = c2} {f = f2} =
%         ≤-trans (extLim _ _ λ k → indMax-limR f2 (f k))
%         (≤-trans (≤-limiting _ (λ k → ≤-limiting _ λ k2 → ≤-cocone _ k2 (≤-cocone _ k (≤-refl _))))
%         (≤-trans (≤-refl (Lim c2 λ k2 → Lim c λ k → indMax (f k) (f2 k2)))
%         (extLim _ _ (λ k2 → ≤-limiting _ λ k1 → ≤-trans (indMax-commut (f k1) (f2 k2)) (indMax-monoR {t1 = f2 k2} {t2 = f k1} {t2' = Lim c f} (≤-cocone _ k1 (≤-refl _)))))))


%       indMax-assocL : ∀ t1 t2 t3 → indMax t1 (indMax t2 t3) ≤ indMax (indMax t1 t2) t3
%       indMax-assocL t1 t2 t3 with indMaxView t2 t3 in eq23
%       ... | IndMaxZ-L = indMax-monoL {t1 = t1} {t1' = indMax t1 Z} {t2 = t3} indMax-≤L
%       ... | IndMaxZ-R = indMax-≤L
%       ... | m with indMaxView t1 t2
%       ... | IndMaxZ-L rewrite sym eq23 = ≤-refl _
%       ... | IndMaxZ-R rewrite sym eq23 = ≤-refl _
%       ... | IndMaxLim-R {f = f} x rewrite sym eq23 = ≤-trans (indMax-limR (λ x → indMax (f x) t3) t1) (extLim _ _ λ k → indMax-assocL t1 (f k) t3) -- f,indMax-limR f t1
%       indMax-assocL .(↑ _) .(↑ _) .Z | IndMaxZ-R  | IndMaxLim-Suc = ≤-refl _
%       indMax-assocL t1 t2 .(Lim _ _) | IndMaxLim-R {f = f} x   | IndMaxLim-Suc = extLim _ _ λ k → indMax-assocL t1 t2 (f k)
%       indMax-assocL (↑ t1) (↑ t2) (↑ t3) | IndMaxLim-Suc  | IndMaxLim-Suc = ≤-sucMono (indMax-assocL t1 t2 t3)
%       ... | IndMaxLim-L {f = f} rewrite sym eq23 = extLim _ _ λ k → indMax-assocL (f k) t2 t3



%       indMax-assocR : ∀ t1 t2 t3 →  indMax (indMax t1 t2) t3 ≤ indMax t1 (indMax t2 t3)
%       indMax-assocR t1 t2 t3 = ≤-trans (indMax-commut (indMax t1 t2) t3) (≤-trans (indMax-monoR {t1 = t3} (indMax-commut t1 t2))
%         (≤-trans (indMax-assocL t3 t2 t1) (≤-trans (indMax-commut (indMax t3 t2) t1) (indMax-monoR {t1 = t1} (indMax-commut t3 t2)))))


%       indMax-swap4 : ∀ {t1 t1' t2 t2'} → indMax (indMax t1 t1') (indMax t2 t2') ≤ indMax (indMax t1 t2) (indMax t1' t2')
%       indMax-swap4 {t1}{t1'}{t2 }{t2'} =
%         indMax-assocL (indMax t1 t1') t2 t2'
%         ≤⨟ indMax-monoL {t1 = indMax (indMax t1 t1') t2} {t2 = t2'}
%           (indMax-assocR t1 t1' t2 ≤⨟ indMax-monoR {t1 = t1} (indMax-commut t1' t2) ≤⨟ indMax-assocL t1 t2 t1')
%         ≤⨟ indMax-assocR (indMax t1 t2) t1' t2'

%       indMax-swap6 : ∀ {t1 t2 t3 t1' t2' t3'} → indMax (indMax t1 t1') (indMax (indMax t2 t2') (indMax t3 t3')) ≤ indMax (indMax t1 (indMax t2 t3)) (indMax t1' (indMax t2' t3'))
%       indMax-swap6 {t1} {t2} {t3} {t1'} {t2'} {t3'} =
%         indMax-monoR {t1 = indMax t1 t1'} (indMax-swap4 {t1 = t2} {t1' = t2'} {t2 = t3} {t2' = t3'})
%         ≤⨟ indMax-swap4 {t1 = t1} {t1' = t1'}

%       indMax-lim2L :
%         ∀
%         {c1 : ℂ}
%         (f1 : El  c1 → Tree)
%         {c2 : ℂ}
%         (f2 : El  c2 → Tree)
%         → Lim  c1 (λ k1 → Lim  c2 (λ k2 → indMax (f1 k1) (f2 k2))) ≤ indMax (Lim  c1 f1) (Lim  c2 f2)
%       indMax-lim2L f1 f2 = ≤-limiting  _ (λ k1 → ≤-limiting  _ λ k2 → indMax-mono (≤-cocone  f1 k1 (≤-refl _)) (≤-cocone  f2 k2 (≤-refl _)))

%       indMax-lim2R :
%         ∀
%         {c1 : ℂ}
%         (f1 : El  c1 → Tree)
%         {c2 : ℂ}
%         (f2 : El  c2 → Tree)
%         →  indMax (Lim  c1 f1) (Lim  c2 f2) ≤ Lim  c1 (λ k1 → Lim  c2 (λ k2 → indMax (f1 k1) (f2 k2)))
%       indMax-lim2R f1 f2 = extLim  _ _ (λ k1 → indMax-limR  _ (f1 k1))

%     --Attempt to have an idempotent version of indMax

%       nindMax : Tree → ℕ → Tree
%       nindMax t ℕ.zero = Z
%       nindMax t (ℕ.suc n) = indMax (nindMax t n) o

%       nindMax-mono : ∀ {t1 t2 } n → t1 ≤ t2 → nindMax t1 n ≤ nindMax t2 n
%       nindMax-mono ℕ.zero lt = ≤-Z
%       nindMax-mono {t1 = t1} {t2} (ℕ.suc n) lt = indMax-mono {t1 = nindMax t1 n} {t2 = t1} {t1' = nindMax t2 n} {t2' = t2} (nindMax-mono n lt) lt

%     --
%       indMax∞ : Tree → Tree
%       indMax∞ t = OℕLim (λ n → nindMax t n)


%       indMax-∞lt1 : ∀ t → indMax (indMax∞ o) t ≤ indMax∞ o
%       indMax-∞lt1 t = ≤-limiting  _ λ k → helper (Iso.fun CℕIso k)
%         where
%           helper : ∀ n → indMax (nindMax t n) t ≤ indMax∞ o
%           helper n = ≤-cocone  _ (Iso.inv CℕIso (ℕ.suc n)) (subst (λ sn → nindMax t (ℕ.suc n) ≤ nindMax t sn) (sym (Iso.rightInv CℕIso (suc n))) (≤-refl _))
%         -- helper (ℕ.suc n) = ≤-cocone  _ (CℕfromNat (ℕ.suc (ℕ.suc n))) (subst (λ sn → indMax (indMax (nindMax t n) o) t ≤ nindMax t sn) (sym (Cℕembed (ℕ.suc n)))
%         --   {!!})
%         --

%       -- nindMax-idem-absorb : ∀ t n → indMax t o ≤ t → nindMax t n ≤ o
%       -- nindMax-idem-absorb t ℕ.zero lt = ≤-Z
%       -- nindMax-idem-absorb t (ℕ.suc n) lt = indMax-monoL (nindMax-idem-absorb t n lt) ≤⨟ lt
%       -- indMax∞-idem-absorb : ∀ {t} → indMax t o ≤ t → indMax∞ t ≤ o
%       -- indMax∞-idem-absorb lt = ≤-limiting  (λ x → nindMax _ (CℕtoNat x)) (λ k → nindMax-idem-absorb _ (CℕtoNat k) lt)

%       indMax-∞ltn : ∀ n t → indMax (indMax∞ o) (nindMax t n) ≤ indMax∞ o
%       indMax-∞ltn ℕ.zero t = indMax-≤Z (indMax∞ o)
%       indMax-∞ltn (ℕ.suc n) t =
%         ≤-trans (indMax-monoR {t1 = indMax∞ o} (indMax-commut (nindMax t n) o))
%         (≤-trans (indMax-assocL (indMax∞ o) t (nindMax t n))
%         (≤-trans (indMax-monoL {t1 = indMax (indMax∞ o) o} {t2 = nindMax t n} (indMax-∞lt1 o)) (indMax-∞ltn n o)))

%       indMax∞-idem : ∀ t → indMax (indMax∞ o) (indMax∞ o) ≤ indMax∞ o
%       indMax∞-idem t = ≤-limiting  _ λ k → ≤-trans (indMax-commut (nindMax t (Iso.fun CℕIso k)) (indMax∞ o)) (indMax-∞ltn (Iso.fun CℕIso k) o)


%       indMax∞-self : ∀ t → t ≤ indMax∞ o
%       indMax∞-self t = ≤-cocone  _ (Iso.inv CℕIso 1) (subst (λ x → t ≤ nindMax t x) (sym (Iso.rightInv CℕIso 1)) (≤-refl _))

%       indMax∞-idem∞ : ∀ t → indMax t o ≤ indMax∞ o
%       indMax∞-idem∞ t = ≤-trans (indMax-mono (indMax∞-self o) (indMax∞-self o)) (indMax∞-idem o)

%       indMax∞-mono : ∀ {t1 t2} → t1 ≤ t2 → (indMax∞ t1) ≤ (indMax∞ t2)
%       indMax∞-mono lt = extLim  _ _ λ k → nindMax-mono (Iso.fun CℕIso k) lt



%       nindMax-≤ : ∀ {t} n → indMax t o ≤ t → nindMax t n ≤ o
%       nindMax-≤ ℕ.zero lt = ≤-Z
%       nindMax-≤ {o = o} (ℕ.suc n) lt = ≤-trans (indMax-monoL {t1 = nindMax t n} {t2 = o} (nindMax-≤ n lt)) lt

%       indMax∞-≤ : ∀ {t} → indMax t o ≤ t → indMax∞ t ≤ o
%       indMax∞-≤ lt = ≤-limiting  _ λ k → nindMax-≤ (Iso.fun CℕIso k) lt

%       -- Convenient helper for turing < with indMax∞ into < without
%       indMax<-∞ : ∀ {t1 t2 o} → indMax (indMax∞ (t1)) (indMax∞ t2) <o t → indMax t1 t2 <o o
%       indMax<-∞ lt = ≤∘<-in-< (indMax-mono (indMax∞-self _) (indMax∞-self _)) lt

%       indMax-<Ls : ∀ {t1 t2 t1' t2'} → indMax t1 t2 <o indMax (↑ (indMax t1 t1')) (↑ (indMax t2 t2'))
%       indMax-<Ls {t1} {t2} {t1'} {t2'} = indMax-sucMono {t1 = t1} {t2 = t2} {t1' = indMax t1 t1'} {t2' = indMax t2 t2'}
%         (indMax-mono {t1 = t1} {t2 = t2} (indMax-≤L) (indMax-≤L))

%       indMax∞-<Ls : ∀ {t1 t2 t1' t2'} → indMax t1 t2 <o indMax (↑ (indMax (indMax∞ t1) t1')) (↑ (indMax (indMax∞ t2) t2'))
%       indMax∞-<Ls {t1} {t2} {t1'} {t2'} =  <∘≤-in-< (indMax-<Ls {t1} {t2} {t1'} {t2'})
%         (indMax-mono {t1 = ↑ (indMax t1 t1')} {t2 = ↑ (indMax t2 t2')}
%           (≤-sucMono (indMax-monoL (indMax∞-self t1)))
%           (≤-sucMono (indMax-monoL (indMax∞-self t2))))


%       indMax∞-lub : ∀ {t1 t2 o} → t1 ≤ indMax∞ t → t2 ≤ indMax∞  t → indMax t1 t2 ≤ (indMax∞ o)
%       indMax∞-lub {t1 = t1} {t2 = t2} lt1 lt2 = indMax-mono {t1 = t1} {t2 = t2} lt1 lt2 ≤⨟ indMax∞-idem _

%       indMax∞-absorbL : ∀ {t1 t2 o} → t2 ≤ t1 → t1 ≤ indMax∞ t → indMax t1 t2 ≤ indMax∞ o
%       indMax∞-absorbL lt12 lt1 = indMax∞-lub lt1 (lt12 ≤⨟ lt1)

%       indMax∞-distL : ∀ {t1 t2} → indMax (indMax∞ t1) (indMax∞ t2) ≤ indMax∞ (indMax t1 t2)
%       indMax∞-distL {t1} {t2} =
%         indMax∞-lub {t1 = indMax∞ t1} {t2 = indMax∞ t2} (indMax∞-mono indMax-≤L) (indMax∞-mono (indMax-≤R {t1 = t1}))


%       indMax∞-distR : ∀ {t1 t2} → indMax∞ (indMax t1 t2) ≤ indMax (indMax∞ t1) (indMax∞ t2)
%       indMax∞-distR {t1} {t2} = ≤-limiting  _ λ k → helper {n = Iso.fun CℕIso k}
%         where
%         helper : ∀ {t1 t2 n} → nindMax (indMax t1 t2) n ≤ indMax (indMax∞ t1) (indMax∞ t2)
%         helper {t1} {t2} {ℕ.zero} = ≤-Z
%         helper {t1} {t2} {ℕ.suc n} =
%           indMax-monoL {t1 = nindMax (indMax t1 t2) n} (helper {t1 = t1} {t2} {n})
%           ≤⨟ indMax-swap4 {indMax∞ t1} {indMax∞ t2} {t1} {t2}
%           ≤⨟ indMax-mono {t1 = indMax (indMax∞ t1) t1} {t2 = indMax (indMax∞ t2) t2} {t1' = indMax∞ t1} {t2' = indMax∞ t2}
%             (indMax∞-lub {t1 = indMax∞ t1} (≤-refl _) (indMax∞-self _))
%             (indMax∞-lub {t1 = indMax∞ t2} (≤-refl _) (indMax∞-self _))


%       indMax∞-cocone : ∀  {c : ℂ} (f : El c → Tree) k →
%         f k ≤ indMax∞ (Lim  c f)
%       indMax∞-cocone f k =  indMax∞-self _ ≤⨟ indMax∞-mono (≤-cocone  _ k (≤-refl _))

%       -- indMax* : ∀ {n} → Vec Tree n → Tree
%       -- indMax* [] = Z
%       -- indMax* (x ∷ os) = indMax x (indMax* os)

%       -- indMax*-≤L : ∀ {n o} {os : Vec Tree n} → t ≤ indMax* (o ∷ os)
%       -- indMax*-≤L = indMax-≤L

%       -- indMax*-≤R : ∀ {n o} {os : Vec Tree n} → indMax* os ≤ indMax* (o ∷ os)
%       -- indMax*-≤R {o = o} = indMax-≤R {t1 = o}

%       -- indMax*-≤-n : ∀ {n} {os : Vec Tree n} (f : Fin n) → lookup f os ≤ indMax* os
%       -- indMax*-≤-n {os = t ∷ os} Fin.zero = indMax*-≤L {o = o} {os = os}
%       -- indMax*-≤-n {os = t ∷ os} (Fin.suc f) = indMax*-≤-n f ≤⨟ (indMax*-≤R {o = o} {os = os})

%       -- indMax*-swap : ∀ {n} {os1 os2 : Vec Tree n} → indMax* (zipWith indMax os1 os2) ≤ indMax (indMax* os1) (indMax* os2)
%       -- indMax*-swap {n = ℕ.zero} {[]} {[]} = ≤-Z
%       -- indMax*-swap {n = ℕ.suc n} {t1 ∷ os1} {t2 ∷ os2} = indMax-monoR {t1 = indMax t1 t2} (indMax*-swap {n = n}) ≤⨟ indMax-swap4 {t1 = t1} {t1' = t2} {t2 = indMax* os1} {t2' = indMax* os2}

%       -- indMax*-mono : ∀ {n} {os1 os2 : Vec Tree n} → foldr (λ (t1 , t2) rest → (t1 ≤ t2) × rest) Unit (zipWith _,_ os1 os2) → indMax* os1 ≤ indMax* os2
%       -- indMax*-mono {ℕ.zero} {[]} {[]} lt = ≤-Z
%       -- indMax*-mono {ℕ.suc n} {t1 ∷ os1} {t2 ∷ os2} (lt , rest) = indMax-mono {t1 = t1} {t1' = t2} lt (indMax*-mono {os1 = os1} {os2 = os2} rest)

%     -- orec : ∀  (P : Tree → Set ℓ)
%     --   → ((x : Tree) → (rec : (y : Tree) → (_ : ∥ y <o x ∥₁) → P y ) → P x)
%     --   → ∀ {t} → P o
%     -- orec P f = induction (λ x rec → f x rec) _
%     --   where open WFI (ordWFProp)


%     -- oPairRec : ∀  (P : Tree → Tree → Set ℓ)
%     --   → ((x1 x2 : Tree) → (rec : (y1 y2 : Tree) → (_ : (y1 , y2) <oPair (x1 , x2)) → P y1 y2 ) → P x1 x2)
%     --   → ∀ {t1 t2} → P t1 t2
%     -- oPairRec P f = induction (λ (x1 , x2) rec → f x1 x2 λ y1 y2 → rec (y1 , y2)) _
%     --   where open WFI (oPairWF)
% \end{code}





The usefulness of Brouwer trees is in defining well-founded recursion, but first we need on ordering on trees.

The four constructors order trees such that zero is the smallest tree, successor is monotone, and the limit of a function is both an upper bound on the image of that function, and is the least such upper bound.
The upper-bound and least constructors are written in a way to ensure that we can prove transitivity of our order, without resorting to taking the transitive closure. This will make it much easier to prove lemas by induction on on ordering derivation. The usual properties are easily recovered.

A strict order can be defined in terms of the successor function. This strict relation is a well quasi-order: it has no infinite descending chains, and hence
can be used as a decreasing metric
for recursive functions.

    TODO compare with cubical,
    TODO look up original trees

\subsubsection{Brouwer Trees}
\label{model:subsec:brouwer}
Unfortunately, it was not immediately apparent that any of the
``off-the-shelf'' formulations of constructive ordinals satisfied our critera,
so we built our own formulation. We use a refined version of Brouwer trees:
There is a zero ordinal, a successor operator, and a limit ordinal that is the least upper bound
of the image for a function from a code's type to ordinals.
We borrow the trick of taking the limits over types (or in our case, codes) from \citet{ionchyMasters},
since this lets us easily model the sizes of dependent functions and pairs.
The ordering on these trees is defined following \citet{KrausFX21}:
\begin{agda}
  data\ \_\le_o\_ : Ord -> Ord -> \sType{}\ where\nl
  \qquad {\le_o}Z : (o : Ord) -> OZ \le_o o  \nl
  \qquad {\le_o}sucMono : (o_1 : Ord) -> (o_2 : Ord) -> o_1 \le_o o_2 -> O{\uparrow}\  o_1 \le_o O{\uparrow}\  o_2  \nl
  \qquad {\le_o}cocone : (c : \bC\ \ell) -> (o : Ord) -> (f : El_{Approx}\ c -> Ord)
    -> (k : El_{Approx}\ c)
    \nl\qquad\qquad -> o \le_o f\ k  -> o \le_o OLim\ c\ f\nl
    \qquad {\le_o}limiting : (o : Ord) -> (c : \bC\ \ell) -> (f : El_{Approx}\ c -> Ord)
    \nl\qquad\qquad -> ((k : El_{Approx}\ c) -> f\ k \le_o o) -> OLim\ c\ f \le_o o\\\nl
    %
    o_1 <_o o_2 = O{\uparrow}\ o_1 \le_o o_2
  \end{agda}
  That is, zero is the smallest ordinal, the successor is monotone,
  and the limit is actually the least upper bound of the function's image.
Unlike \citet{KrausFX21}, we do not include transitivity as a rule, but we can prove
it as a theorem.
The maximum function on ordinals is defined as follows:
\begin{agda}
  max_o : Ord -> Ord -> Ord\nl
  max_o\ OZ\ o = o \nl
  max_o\ o\ OZ = o \nl
  max_o\ (O{\uparrow}\ o_1)\ (O{\uparrow}\ o_2) = O{\uparrow}\ (max_o\ o_1\ o_2)\nl
  max_o\ (OLim\ c\ f)\ o = OLim\ c\ (\lambda k \ldotp max_o\ (f\ k)\ o)\nl
  max_o\ o\ (OLim\ c\ f) = OLim\ c\ (\lambda k \ldotp max_o\ o\ (f\ k))
\end{agda}
Long but straightforward proofs show that $max_{o}$ is monotone
and computes and upper bound of its inputs.
It reduces when given $\s{O{\uparrow}}$ for both inputs, so it is strictly monotone.
However, we cannot prove that it is a least upper-bound.
The problem is that limits are not well-behaved with respect to the maximum.
We could instead construct the maximum using $\s{OLim}$, but this version
would not be strictly monotone.

\subsubsection{A Least Upper Bound}

We solve the problems with $\s{max_{o}}$ using a type of sizes, which include only the subset of
ordinals that are idempotent with respect to the maximum. We can then
define a type of sizes with the same interface as ordinals.
\begin{agda}
  Size : \sType{} \nl
  Size = (o : Ord) \times (max_o\ o\ o \le_o o)\\\nl
%
  \_\sansbigvee\_ : Size -> Size -> Size\nl
  s_1 \sansbigvee s_2 = (max_o\ (fst\ s_1)\ (fst\ s_2), \ldots)\\\nl
  %
  SZ : Size\nl
  SZ = (OZ , {\le_o}Z)\\\nl
  S{\uparrow} : Size -> Size\nl
  S{\uparrow}\ s =  (O{\uparrow}\ (fst\ s), {\le_s}sucMono\ (snd\ s))
\end{agda}
Critically, the sizes are closed under the maximum operation: if $\s{max_{o}\ o_{1}\ o_{1} \le_{o}\ o_{1}}$
and $\s{max_{o}\ o_{2}\ o_{2} \le_{o}\ o_{2}}$, then
$\s{max_{o}\ (max_{o}\ o_{1}\ o_{2})\ (max_{o}\ o_{1}\ o_{2}) \le (max_{o}\ o_{1}\ o_{2})}$.
% We omit the proof term, because it is long but boring.
Zero and a successor operation for sizes are easily implemented.
The difficulty is constructing a limit operator for sizes, since
the self-idempotent ordinals are not closed under $\s{OLim}$.
Our trick is to take the limit of maxing an ordinal with itself.
We assume we have a code $\s{C\bN}$ whose elements have an injection $\s{Cto\bN}$ into $\s{\bN}$.
The natural numbers can be defined as an inductive type, but in our Agda development we add it as an
extra code constructor.
Having numbers lets us take the maximum of an ordinal with itself infinitely many times, resulting in an ordinal
that is as large as the original but idempotent with respect to $\s{max_{o}}$.
\begin{agda}
  nmax : Ord -> \bN -> Ord \nl
  nmax\ o\ Z\ = OZ\nl
  nmax\ o\ (S\ n) = omax\ (nmax\ o\ n)\ o\\ \nl
  %
  max\infty : Ord -> Ord\nl
  max\infty\ o = OLim\ C\bN\ (\lambda k \ldotp nmax\ o\ (Cto\bN\ k)) \\ \nl
  %
  max\infty Idem : \{ o : Ord \} -> max_o\ (max\infty\ o)\ (max\infty\ o) \le_o (max\infty\ o)\\\nl
  %
  SLim : (c : \bC\ \ell) -> (El_{Approx}\ c -> Size) -> Size\nl
  SLim\ c\ f = (max\infty\ (OLim\ c\ (\lambda k \ldotp fst\ (f\ k))) ,\ max\infty Idem )
\end{agda}

Sizes satisfy all the same inequalities as raw ordinals,
listed in \cref{model:fig:size-order}.
The monotonicity of $\s\bigvee$ follows from the monotonicity of $\s{max_{o}}$,
and the idempotence  of $\s\bigvee$ follows by the definition of $\s{Size}$.
Monotonicity, idempotence, and transitivity of $\s{\le_{s}}$ together imply
that $\s\bigvee$ is a least upper bound,
and strict monotonicity follows from the strict monotonicity of $\s{max_{o}}$.
\begin{figure}
  \begin{agda}
    \_\le_s\_ : Size -> Size -> Size\nl
    s_1 \le_s s_2 = (fst\ s_1) \le_o (fst\ s_2)\\\nl
    %
    \_<_s\_ : Size -> Size -> Size\nl
    s_1 <_s s_2 = (S{\uparrow}\ s_1) \le_s s_2\\\nl
    %
    {\le_s}trans : (s_1 : Size) -> (s_2 : Size) -> (s_3 : Size) ->\nl
    \qquad (s_1 \le_s s_2) -> (s_2 \le_s s_3) -> (s_1 \le_s s_3)\nl
    {\le_s}Z : (s : Size) -> SZ \le_s s  \nl
    {\le_s}sucMono : (s_1 : Size) -> (s_2 : Size) -> s_1 \le_s s_2 -> S{\uparrow}\  s_1 \le_s S{\uparrow}\  s_2  \nl
    {\le_s}cocone : (c : \bC\ \ell) -> (s : Size) -> (f : El_{Approx}\ c -> Size)
    -> (k : El_{Approx}\ c)
    \nl\qquad -> s \le_s f\ k  -> s \le_s SLim\ c\ f\nl
    {\le_s}limiting : (s : Size) -> (c : \bC\ \ell) -> (f : El_{Approx}\ c -> Size)
    \nl\qquad -> ((k : El_{Approx}\ c) -> f\ k \le_s s) -> SLim\ c\ f \le_s s\\\nl
    %
    \sansbigvee\le : (s_1 : Size) -> (s_2 : Size) -> (s_1 \le_s s_1 \sansbigvee s_2) \times (s_2 \le_2 s_1 \sansbigvee s_2)\nl
    \sansbigvee mono : (s_1 : size) -> (s_2 : Size) -> (s'_1 : Size) -> (s'_2 : Size) \nl
    \qquad -> (s_1 \le_s s'_1) -> (s_2 \le_s s'_2) -> (s_1 \sansbigvee s_2) \le_s (s'_1 \sansbigvee s'_2)\nl
    \sansbigvee idem : (s : Size) -> (s \sansbigvee s) \le_s s\nl
    \sansbigvee lub : (s_1 : size) -> (s_2 : size) -> (s : Size) \nl
    \qquad -> (s_1 \le_s s) -> (s_2 \le_s s) -> (s_1 \sansbigvee s_2 \le_s s)
  \end{agda}
  \caption{Ordering on Sizes}
  \label{model:fig:size-order}
\end{figure}



\renewcommand{\emph}[1]{#1}

\bibliographystyle{ACM-Reference-Format}
\bibliography{myRefs}

% \clearpage
% \input{appendix}

\end{document}
\endinput
%%
%% End of file `sample-acmsmall.tex'.
