% !TEX root = main.tex
% This file provides examples of some useful macros for typesetting
% dissertations.  None of the macros defined here are necessary beyond
% for the template documentation, so feel free to change, remove, and add
% your own definitions.
%
% We recommend that you define macros to separate the semantics
% of the things you write from how they are presented.  For example,
% you'll see definitions below for a macro \file{}: by using
% \file{} consistently in the text, we can change how filenames
% are typeset simply by changing the definition of \file{} in
% this file.
%
%% The following is a directive for TeXShop to indicate the main file

%Treat thesis like appendex i.e. show things in gory detail

% \newcommand{\ifapx}[1]{#1}
% \newcommand{\ifnotapx}[1]{}
%Maybe show rule names
% \newcommand{\mname}[1]{\ifapx{#1}}


\newcommand{\onlydiss}[1]{#1}
% \newcommand{\thesisProp}{dissertation}

% Some useful macros for typesetting terms.
\newcommand{\file}[1]{\texttt{#1}}
\newcommand{\class}[1]{\texttt{#1}}
\newcommand{\latexpackage}[1]{\href{http://www.ctan.org/macros/latex/contrib/#1}{\texttt{#1}}}
\newcommand{\latexmiscpackage}[1]{\href{http://www.ctan.org/macros/latex/contrib/misc/#1.sty}{\texttt{#1}}}
\newcommand{\env}[1]{\texttt{#1}}
\newcommand{\BibTeX}{Bib\TeX}

% Define a command \doi{} to typeset a digital object identifier (DOI).
% Note: if the following definition raise an error, then you likely
% have an ancient version of url.sty.  Either find a more recent version
% (3.1 or later work fine) and simply copy it into this directory,  or
% comment out the following two lines and uncomment the third.
\DeclareUrlCommand\DOI{}
\newcommand{\doi}[1]{\href{http://dx.doi.org/#1}{\DOI{doi:#1}}}
%\newcommand{\doi}[1]{\href{http://dx.doi.org/#1}{doi:#1}}

% Useful macro to reference an online document with a hyperlink
% as well with the URL explicitly listed in a footnote
% #1: the URL
% #2: the anchoring text
\newcommand{\webref}[2]{\href{#1}{#2}\footnote{\url{#1}}}



% epigraph is a nice environment for typesetting quotations
\makeatletter
\newenvironment{epigraph}{%
  \begin{flushright}
  \begin{minipage}{\columnwidth-0.75in}
  \begin{flushright}
  \@ifundefined{singlespacing}{}{\singlespacing}%
    }{
  \end{flushright}
  \end{minipage}
  \end{flushright}}
\makeatother

% \FIXME{} is a useful macro for noting things needing to be changed.
% The following definition will also output a warning to the console
\newcommand{\FIXME}[1]{\typeout{**FIXME** #1}\textbf{[FIXME: #1]}}

% \newcommand\codeLine[1]{\texttt{#1}}
\newcommand\Int{\mathsf{Int}}
\newcommand\Bool{\mathsf{Bool}}


\newcommand{\jform}[1]{\fbox{#1}\hspace{\fill}\\}
\newcommand{\jformlow}[1]{\fbox{#1}\hspace{\fill}\vspace{-1em}\\}
% \newcommand{\synth}{\Rightarrow}
\newcommand{\psynth}[1]{{\,\mathbin{\Rightarrow_{#1}}}\,}
\newcommand{\prepsynth}[1]{{\,\mathbin{\Rightarrow*_{#1}}}\,}
\newcommand{\psynthstar}[1]{{\,\mathbin{\Rightarrow^{*}_{#1}}}\,}
% \newcommand{\check}{\Leftarrow}
\newcommand{\ssorts}{\sType{ }}
% \newcommand{\ulev}[1]{^{@#1}}
\newcommand{\ind}{\mathsf{ind}}
\newcommand{\smatch}[4]{\s{\operatorname{\ind}_{#1}(#2,#3,#4)}}
\newcommand{\smatchnoarg}[1]{\s{\operatorname{\ind}_{#1}}}
\newcommand{\rmatch}[4]{\r{\operatorname{\ind}_{#1}(#2,#3,#4)}}
\newcommand{\gmatch}[4]{\g{\operatorname{\ind}_{#1}(#2,#3,#4)}}
\newcommand{\gmatchNoarg}[1]{\g{\operatorname{\ind}_{#1}}}

\newcommand{\fix}[3]{\operatorname{\mathtt{fix}}#1 : #2 := #3}
\newcommand{\pars}{\operatorname{\mathbf{\color{black} Params}}}
\newcommand{\indices}{\operatorname{\mathbf{\color{black} Indices}}}
\newcommand{\ivals}[3]{\operatorname{\mathbf{\color{black} IValsFor}}(#1,#2,#3)}
\newcommand{\iargs}[4]{\operatorname{\mathbf{\color{black} IArgsFor_{#1}}}(#2,#3,#4)}
\newcommand{\args}{\operatorname{\mathbf{\color{black} Args}}}
\newcommand{\parsub}[1]{[#1]}
\newcommand{\hlev}[1]{_{#1}}
\newcommand{\ulev}[1]{\scalebox{0.7}{@\{#1\}}}


\newcommand{\iswf}{\s{{iswf}}}
\newcommand{\WF}{\s{{WF}}}
\newcommand{\field}{\s{\textit{field}}}
\newcommand{\sSigma}{{\color{BrickRed}\mathrm\Sigma}}
\newcommand{\sPi}{{\color{BrickRed}\mathrm\Pi}}
\newcommand{\uttrans}[1]{{\llbracket \g{#1} \rrbracket}}
\newcommand{\interpCode}[1]{\s{\llparenthesis #1\rrparenthesis_\Code}}
\newcommand{\interp}[1]{\s{\llparenthesis #1 \rrparenthesis}}

\newcommand{\sind}[2]{\s{\ind_{#1}(}#2\s{)}}
\newcommand{\gind}[2]{\g{\ind_{#1}(}#2\g{)}}
\newcommand{\selim}{\s{\mathsf{elim}}}
\newcommand{\gelim}{\g{\mathsf{elim}}}

\newcommand{\varAE}{{\!\! \textit{\AE} }}

\newcommand{\elabsto}{\rightarrowtriangle}
\newcommand{\echeck}[3]{#1 \elabsto #3 <= #2}
\newcommand{\esynth}[3]{#1 \elabsto #3 => #2}
\newcommand{\epsynth}[4]{#2 \elabsto #4 =>_{#1} #3}
\newcommand{\etype}[3]{#1 \elabsto #3 : \gType{}_{=>\g{#2}} }
\newcommand{\redsto}{\leadsto}
\newcommand{\ifapxCaption}[1]{\caption{#1}}

% END


\newcommand{\DCat}[1]{{\g{D^C\langle #1\rangle}}}
\newcommand{\DCatnog}[1]{{\g{D^C\langle} #1\g{\rangle}}}
% \newcommand{\seqSqube}[1]{\sqube_{{#1}}}
\newcommand{\sqube}{\sqsubseteq}
\newcommand{\squbr}{\sqsubseteq_{\mathsf{Surf}}}
% \newcommand{\squbo}{\sqsubseteq_{obs}}
\newcommand{\squbo}{\sqsubseteq^{\Vdash}}
\newcommand{\squbB}{\sqsubseteq_{\bB}}
\newcommand{\squbG}{\sqsubseteq^{Ctx}}
\newcommand{\squbstar}{\sqsubseteq^{C*}}
\newcommand{\sqeqstar}{\sqsupseteq\sqsubseteq^{C*}}
\newcommand{\squbs}{\sqsubseteq_{\alpha}}
% \newcommand{\squbstep}{\sqube_{\leadsto}}
\newcommand{\equiprecstep}{\sqsupseteq\sqube_{\leadsto}}
\newcommand{\qmt}[1]{\gqm_{\g{#1}}}
\newcommand{\qml}{\gqm_{\gType{\ell}}}
\newcommand{\qmat}[1]{\g{\gqm_{#1}}}
\newcommand{\attagl}[2]{\g{\langle#1\rangle_{\g\ell}#2}}


% \usepackage{fontspec}
% \usepackage{inconsolata}
% \usepackage{sansmath}
% % \DeclareMathAlphabet{\mathbbm}{U}{bbm}{m}{n}
% % \DeclareMathAlphabet{\mymathbb}{U}{bbold}{m}{n}
% \DeclareSymbolFont{mymathbb}{U}{bbold}{m}{n}
% \SetMathAlphabet{\mathsfsl}{sans}{\sansmathencoding}{\sfdefault}{m}{n}
% \SetMathAlphabet{\mathsf}{sans}{\sansmathencoding}{\sfdefault}{m}{n}
% \SetMathAlphabet{\mathrm}{sans}{\sansmathencoding}{\rmdefault}{m}{n}
% \SetMathAlphabet{\mathbf}{sans}{\sansmathencoding}{\rmdefault}{m}{n}
% % \DeclareMathAlphabet{\mathbb}{U}{bbold}{m}{n}
% % \SetMathAlphabet{\mathbb}{sans}{U}{bbold}{m}{n}
% % \SetMathAlphabet{\mathbb}{normal}{U}{bbold}{m}{n}
% % \SetMathAlphabet{\mathbb}{bold}{U}{bbold}{m}{n}
% % \SetSymbolFont{largesymbols}{sans}{OMX}{bbold}{m}{n}
% % \SetSymbolFont{numbers}{mymathbb}{OMX}{bbold}{m}{n}
% \SetSymbolFont{largesymbols}{sans}{OMX}{libertine}{m}{n}
% \DeclareMathDelimiter{(}{\mathopen} {operators}{"28}{largesymbols}{"00}
% \DeclareMathDelimiter{)}{\mathclose}{operators}{"29}{largesymbols}{"01}
% \DeclareMathSymbol{\sansbigvee}{1}{largesymbols}{'137}
% \DeclareMathSymbol{\mymathbbOne}{3}{mymathbb}{"31}
% \DeclareMathSymbol{\mymathbbOne}{3}{mymathbb}{"31}
% \DeclareMathSymbol{\mymathbbZero}{3}{mymathbb}{"30}
% \DeclareMathSymbol{\mymathbbN}{3}{mymathbb}{'116}
% \DeclareMathSymbol{\mymathbbB}{3}{mymathbb}{"42}
% \DeclareMathSymbol{\mymathbbC}{3}{mymathbb}{"43}
% \DeclareMathSymbol{\mymathbbColon}{3}{mymathbb}{'072}
% \DeclareMathSymbol{\mymathbbSemiColon}{3}{mymathbb}{'073}
%   \usepackage{fonttable}
% \usepackage{sansmath} %TODO: re-enable
\usepackage{scalerel}
\usepackage{float}


\usepackage[capitalise]{cleveref}

\newcommand{\oblset}[1]{\textsc{#1}}
\newcommand{\GType}{\oblset{GType}}
\newcommand{\SType}{\oblset{SType}}
\newcommand{\Nat}{\mathsf{Nat}}
\newcommand{\VVec}{\mathsf{Vec}}
%
\newcommand{\dom}{{\mathbf{\mathrm{dom}}}}
\newcommand{\cod}{{\mathbf{\mathrm{cod}}}}


\newcommand{\upCast}[2]{\g{\langle #2 \nwarrow #1 \rangle}}
\newcommand{\downCast}[2]{\g{\langle #2 \swarrow  #1 \rangle}}
\newcommand{\genCast}[2]{\g{\langle #2 \mathrlap{\,\nwarrow}\swarrow  #1 \rangle}}
% plainly styled cast
\newcommand{\pcast}[2]{\g\langle #2 \g{<=} #1 \g\rangle}

\newcommand{\T}[1]{{\color{black}\mathcal{T}\llbracket} \g{#1} {\color{black}\rrbracket}}
\newcommand{\E}[1]{{\color{black}\mathcal{E}\llbracket} \g{#1} {\color{black}\rrbracket}}
% \newcommand{\V}[1]{{\color{black}\mathcal{V}\llbracket} \g{#1} {\color{black}\rrbracket}}
% \usepackage[literate]{myidrislang}

\newcommand{\naive}{na\"ive\xspace} %TODO check if this is in all of them
\newcommand{\Naive}{Na\"ive\xspace} %TODO check if this is in all of them



% \newcommand{\TypeType}{{\mathbf{TypeTODO} } } %TODO get rid of


\newcommand{\ottdrule}[3]{ \inferrule[#3]{#1}{#2} }
\newcommand{\AllTerms}{\oblset{Snf}}
\newcommand{\AllGTerms}{\oblset{Gnf}}
\newcommand{\AllNeut}{\oblset{Sne}}
\newcommand{\AllGNeut}{\oblset{Gne}}
\newcommand{\pto}{\rightharpoonup}

\newcommand{\emptyspine}{\cdot}
\newcommand\leadsfrom{\reflectbox{$\leadsto$ } }
\newcommand{\Pow}{\mathcal{P}}
\newcommand\myepsilon{\r{\varepsilon}}


\newcommand{\J}{\mathbf{J}}
\newcommand{\Kax}{\mathbf{K}}

\newcommand{\defprec}{{\sqube_{\stepsto}}}
\newcommand{\defsuprec}{{\sqube^{\longleftarrow}_{\stepsto}}}
\newcommand{\defcst}{{{\cong}_{\stepsto}}}
\newcommand{\acst}{{{\cong}_{\alpha}}}

\newcommand{\genprec}{\Gbox{\defprec}}
\newcommand{\gensuprec}{\Gbox{\defsuprec}}
\newcommand{\gencst}{\Gbox{\defcst}}

\DeclareMathOperator*{\bigamp}{\mathlarger{\&}}
\newcommand{\gcomp}[1]{\mathbin{\g{\&_{#1}}}}
\newcommand{\gcompop}{\g{\&}}
\newcommand{\itercomp}{\seq{\bigamp}}

\newcommand{\germ}{\mathsf{\color{black} germ}}
\newcommand{\head}{\mathsf{\color{black} head}}



\newcommand{\cast}[2]{\g{\langle #2 <= #1 \rangle}}
\newcommand{\castnog}[2]{\g{\langle} #2 \g{<=} #1 \g{\rangle}}
\newcommand{\castenv}[2]{{\langle} #2 {<=} #1 {\rangle}}
% \newcommand{\rmeet}[3]{\g{#1 \sqcap_{#3} #2 }}
% \newcommand{\rmeetnog}[3]{{#1 \g{\sqcap}_{#3} #2}}
% \newcommand{\grefl}[3]{\g{refl_{#1 |- #2 \cong  #3}}}
\newcommand{\grefl}[3]{\g{refl(#1)_{|- #2 \cong  #3}}}
\newcommand{\greflnog}[3]{\g{refl}(#1)_{\g{|-} #2 \g{\cong} #3}}

% % \newcommand{\upCast}[2]{\g{\langle #2 \nwarrow #1 \rangle}}
% \newcommand{\downCast}[2]{\g{\langle #2 \swarrow  #1 \rangle}}
% \newcommand{\genCast}[2]{\g{\langle #2 \mathrlap{\,\nwarrow}\swarrow  #1 \rangle}}

% Theorems taken from ACMART
  \newtheorem{theorem}{Theorem}[section]
  \newtheorem{conjecture}[theorem]{Conjecture}
  \newtheorem{proposition}[theorem]{Proposition}
  \newtheorem{lemma}[theorem]{Lemma}
  \newtheorem{corollary}[theorem]{Corollary}
  \newtheorem{example}[theorem]{Example}
  \newtheorem{definition}[theorem]{Definition}


\renewcommand{\em}{\PackageError{main}{Remove Em}{Deprecated}}




\newcommand{\nl}{\\&}
\let\oldAE\AE
\let\oldae\ae
\renewcommand{\AE}{\textsf{\oldAE}}
\renewcommand{\ae}{\textsf{\oldae}}

\newcommand{\lob}{\textsf{l\"{o}b}}
\newcommand{\Lob}{{L\"{o}b}}
\newcommand{\sqm}{\s{\textbf{?}}\:\!\!}
\newcommand{\sqmTy}{\s{\sqm Ty}}


% \usepackage{agda}
\usepackage{newunicodechar}
\newunicodechar{⨟}{ \ensuremath{{ \fatsemi }}}
\newunicodechar{⋁}{ \ensuremath{{ \bigvee }}}
\newunicodechar{⊔}{ \ensuremath{\sqcup}}
% \newunicodechar{⟧}{ \ensuremath{\rrbracket} }
% \newunicodechar{≤}{ \ensuremath{\leq} }
% \newunicodechar{↝}{ \ensuremath{\arrowwaveright} }
% \newunicodechar{≟}{ \stackrel{?}{=} }
% \newunicodechar{⊑}{ \ensuremath{\sqsubseteq} }
% \newunicodechar{𝒟}{ \ensuremath{\mathscr{D}} }
% \newunicodechar{▹}{ \ensuremath{\triangleright} }
% \newunicodechar{⨟}{ \mymathbbSemiColon }
% \newunicodechar{≡}{ \ensuremath{\equiv} }
% \newunicodechar{𝟚}{ \ensuremath{\mathbb{2}} }
% \newunicodechar{◁}{ \triangleleft }
% \newunicodechar{ₚ}{_p}
% \newunicodechar{₃}{_3}
% \newunicodechar{ω}{ \ensuremath{\omega} }
% \newunicodechar{Δ}{ \ensuremath{\Delta} }
% \newunicodechar{̃}{^{\~}}
% \newunicodechar{λ}{ \ensuremath{\lambda} }
% \newunicodechar{∈}{ \ensuremath{\in} }
% \newunicodechar{θ}{ \ensuremath{\theta} }
% \newunicodechar{⇐}{ \ensuremath{\Leftarrow} }
% \newunicodechar{∨}{ \ensuremath{\vee} }
% \newunicodechar{Æ}{ {\AE} }
% \newunicodechar{β}{ \ensuremath{\beta} }
% \newunicodechar{⋯}{ \ensuremath{\cdots} }
% \newunicodechar{≔}{ \ensuremath{:=} }
% \newunicodechar{∧}{ \ensuremath{\wedge} }
% \newunicodechar{₂}{ _2 }
% \newunicodechar{⟨}{ \ensuremath{\langle} }
% \newunicodechar{𝟙}{ \ensuremath{\mathbb{1}} }
% \newunicodechar{ₛ}{ _s }
% \newunicodechar{⟦}{ \llbracket }
% \newunicodechar{▸}{ \ensuremath{\blacktriangleright} }
% \newunicodechar{₀}{ _0 }
% \newunicodechar{⁻}{ ^{-} }
% \newunicodechar{≢}{ \ensuremath{\not\equiv} }
% \newunicodechar{⇒}{ \ensuremath{\Rightarrow} }
% \newunicodechar{↦}{ \ensuremath{\mapsto} }
% \newunicodechar{∎}{ \ensuremath{\blacksquare} }
% \newunicodechar{ℕ}{ \ensuremath{\mathbb{N}} }
% \newunicodechar{Γ}{ \ensuremath{\Gamma} }
% \newunicodechar{ }{ ~ }
% \newunicodechar{⟩}{ \ensuremath{\rangle} }
% \newunicodechar{₄}{ _4 }
% \newunicodechar{∞}{ \ensuremath{\infty} }
% \newunicodechar{ℛ}{ \ensuremath{\mathscr{R}} }
% \newunicodechar{∷}{ \ensuremath{::} }
% \newunicodechar{⊥}{ \ensuremath{\perp} }
% \newunicodechar{ˡ}{ ^l }
% \newunicodechar{τ}{ \ensuremath{\tau} }
% \newunicodechar{⊛}{ \ensuremath{\circledast} }
% \newunicodechar{→}{ {\textrightarrow} }
% \newunicodechar{←}{ {\textleftarrow} }
% \newunicodechar{₁}{ _1 }
% \newunicodechar{∀}{ \ensuremath{\forall} }
% \newunicodechar{⦂}{ \mymathbbColon }
% \newunicodechar{⊎}{ \ensuremath{\uplus} }
% \newunicodechar{⊢}{ \ensuremath{\vdash} }
% \newunicodechar{¬}{ {\textlnot} }
% \newunicodechar{□}{ \ensuremath{\square} }
% \newunicodechar{◃}{ \ensuremath{\triangleleft} }
% \newunicodechar{δ}{ \ensuremath{\delta} }
% \newunicodechar{α}{ \ensuremath{\alpha} }
% \newunicodechar{≅}{ \ensuremath{\cong} }
% \newunicodechar{𝟘}{ \ensuremath{\mathbb{0}} }
% \newunicodechar{æ}{ {\ae} }
% \newunicodechar{ö}{ \"o }
% \newunicodechar{⟶}{ \ensuremath{\longrightarrow} }
% \newunicodechar{⦃}{ \{\{ }
% \newunicodechar{℧}{ {\textmho} }
% \newunicodechar{ℰ}{ \ensuremath{\mathscr{E}} }
% \newunicodechar{ℓ}{ \ensuremath{\ell} }
% \newunicodechar{Σ}{ \ensuremath{\Sigma} }
% \newunicodechar{μ}{ \ensuremath{\mu} }
% \newunicodechar{×}{ {\texttimes} }
% \newunicodechar{⁇}{ \ensuremath{\sqm} }
% \newunicodechar{Π}{ \ensuremath{\Pi} }
% \newunicodechar{⦄}{ \}\} }
% \newunicodechar{⊤}{ \ensuremath{\top} }
% \newunicodechar{∙}{ \ensuremath{\bullet} }
% \newunicodechar{𝒯}{ \ensuremath{\mathscr{T}} }
% \newunicodechar{ℂ}{ \ensuremath{\mathbb{C}} }
% \newunicodechar{φ}{ \ensuremath{\varphi} }
