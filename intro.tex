% !TEX root =  main.tex
\section{Introduction}

\subsection{Recursion and Dependent Types}
Dependently typed languages, such as Agda~\citep{TODO}, Coq~\citep{coqart},
Idris~\citep{TODO} and Lean~\citep{TODO}, bridge the gap between theorem
proving and programming.

Functions defined in dependently typed languages are typically required to be
\textit{total}: they must provably halt in all inputs. Since the halting problem
is undecidable, recursively-defined functions must be written in such a way that the type checker
can mechanically deduce termination.
Some functions only make recursive calls to structurally-smaller arguments,
so their termination is apparent to the compiler. However, some functions
cannot be easily expressed using structural recursion.
For such functions, the programmer must instead use \textit{well founded recursion}, showing that there is some ordering, with no infinitely-descending
chains, for which each recursive call is strictly smaller according to this ordering. For example, the typical quicksort algorithm is not structurally recursive, but can use well founded recursion on the length of the lists being sorted.

\subsection{Ordinals}

While numeric orderings work for first-order data, they are ill suited to recursion over
higher-order data structures, where some fields contain functions.

There are many formulations of ordinals in dependent type theory, each with their own advantages and disadvantages.


\subsection{Contributions}

This work defines \textit{strictly monotone Brouwer Trees}, henceforth SMB-trees,
a new presentation of ordinals that hit a sort of sweet-spot for defining functions by
well founded recursion. Specifically, SMB-trees:

\begin{itemize}
  \item are strictly ordered by a well founded relation;
  \item have a maximum operator which computes a least-upper bound;
  \item are \textit{strictly-monotone} with respect to the maximum: if $a < b$ and $c < d$, then $\max\ a\ c < \max\ b\ d$;
  \item can compute the limits of arbitrary sequences;
  \item are light in axiomatic requirements: they are defined without using axiom K,
        univalence, quotient types, or higher inductive types.
\end{itemize}


\subsection{Uses for SMB-trees}

\subsubsection{Well Founded Recursion}

Having a maximum operator for ordinals is particularly useful when traversing over multiple higher order
data structures in parallel, where neither argument takes priority over the other.
In such a case, a lexicographic ordering cannot be used.


As an example, consider a unification algorithm over some encoding of types, and suppose that $\alpha$-renaming or some other restriction prevents
structural recursion from being used.
To solve a unification problem $ \Sigma(x : A)\ldotp B = \Sigma(x : C)\ldotp D$
we must recursively
solve $A = C$ and $\forall x \ldotp B[x] = D[x]$.
However, the type of $x$ in the latter equation depends on the solution
to the first equation, which is bounded by the size of the maximum of the sizes
of both $A$ and $C$.
So for each recursive call to be on a smaller size,
the size of $a = c$ and $b = d$ must both be strictly smaller than
$(a,b)=(c,d)$. In a lexicographic ordering where the size of the left-hand size
dominates, we know that $a$ is strictly smaller than $(a,b)$, but we have no
guarantees that TODO.
Conversely, if we order unification problems by the size of the maximum of their
two sides.


This style of well founded induction was used to prove termination
in a syntactic model of gradual dependent types~\citep{TODO}. There, Brouwer trees
were used to establish termination of recursive procedures for
combining the type information in two imprecise types.
The decreasing metric was the maximum size of the codes for the types being combined. Brouwer trees' arbitrary limits were used to assign sizes
to dependent function and product types, and the strict monotonicity of the
maximum operator was essential for proving that recursive calls were on
strictly smaller arguments.

\subsubsection{Syntactic Models and Sized Types}

An alternate way view of our contribution is as a tool for modelling sized types~\citep{TODO}.
The implementation of sized types in Agda has been shown to be unsound~\citep{TODO}, due to the interaction
between propositional equality and the top size $\infty$ satisfying $\infty < \infty$.
\Citep{Chan2022} defines a dependently typed language with sized types that does not have a top size, proving it consistent
using a syntactic model based on Brouwer trees.

SMB-trees provide the capability to extend existing syntactic models to sized types
with a maximum operator.
This brings the capability of consistent sized types closder to feature parity with Agda,
which has a maximum operator for its sizes~\citep{TODO}, while still maintaining logical consistency.

\subsubsection{Algebraic Reasoning}
Another advantage of SMB-trees is that they allow Brouwer trees to
be interpreted using algebraic tools.
SMB-trees can be described as
In algebraic terminology, SMB-trees satisfy the following algebraic laws, up to the equivalence relation defined by $s \approx t := s \le t \le s $
\begin{itemize}
  \item Join-semlattice: the binary $\max$ is associative, commutative, and idempotent
  \item Bounded: there is a least tree $Z$ such that $\max\ t\ Z \approx t$
  \item Inflationary endomorphism: there is a successor operator $\up$
        such that ${\max\ (\up t)\ t \approx \up t}$
        and\\ ${\up (\max\ s\ t) = \max (\up s)\ (\up t)}$
\end{itemize}

\Citet{BEZEM20221} describe a polynomial time algorithm for solving equations in such an algebra,
and describe its usefulness for solving constraints involving universe levels
in dependent type checking. While equations involving limits of infinite sequences
are undecidable, the inflationary laws could be used to automatically discharge some equations involving sizes. This algebraic presentation is particularly
amenable to solving equations using free extensions of algebras~\citep{corbyn:proof-synthesis,allais2023frex}.

% Even without sizes integrated into types, SMB-trees are useful for creating syntactic models of dependently typed languages and defining terms by well founded recursion in these models.
% The ability to take the limit of arbitrary sequences of types makes it very easy to assign an ordinal size
% to encodings of dependent function or pair types: for $\Pi(x : A)\ldotp B$, if  $A$ has size $t_{A}$,
% and $B[x]$ has size $t_{B}[x]$ for each $x$, then the function type has size (using our notation from \cref{TODO})
% $\up (\max\ t_{A} (Lim\ A (\lambda x \ldotp B[x]))$:
% strictly larger than both the size of $A$ and the
% limit of size of $B$ for any $x : A$.
% The strict monotonicity property of SMB-trees let us compare function types:
% if $A_{1}$ is strictly smaller than $A_{2}$, and for every $x$ $B_{1}[x]$ is strictly smaller than $B_{2}[y]$ for some $y$, then strict monotonicity guarantees that $\Pi(x:A_{1})\ldotp B_{1}[x]$
% is strictly smaller than $$

\subsection{Implementation}

We have implemented SMB-trees in Agda 2.6.4.
Our library specifically avoids Agda-specific features
such as cubcal type theory or Axiom K, so we expect
that the library can be easily ported to other proof assistants.

This paper is written as a literate Agda document, and the definitions
given in the paper are valid Agda code.
Several definitions are presented with their body omitted due to
space restrictions. The full implementation can be found in the supplementary
materials section of this submission.
