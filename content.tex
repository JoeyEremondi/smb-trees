% !TEX root =  main.tex


% !TEX root =  main.tex
\begin{code}[hide]%
\>[0]\AgdaKeyword{module}\AgdaSpace{}%
\AgdaModule{SmallTree}\AgdaSpace{}%
\AgdaKeyword{where}\<%
\\
\>[0]\AgdaKeyword{open}\AgdaSpace{}%
\AgdaKeyword{import}\AgdaSpace{}%
\AgdaModule{Data.Nat}\<%
\end{code}


Brouwer trees  are a simple but elegant tool for proving termination of higher-order procedures.
Traditionally, they are defined as follows:
\begin{code}%
\>[0]\AgdaKeyword{data}\AgdaSpace{}%
\AgdaDatatype{SmallTree}\AgdaSpace{}%
\AgdaSymbol{:}\AgdaSpace{}%
\AgdaPrimitive{Set}\AgdaSpace{}%
\AgdaKeyword{where}\<%
\\
\>[0][@{}l@{\AgdaIndent{0}}]%
\>[4]\AgdaInductiveConstructor{Z}\AgdaSpace{}%
\AgdaSymbol{:}\AgdaSpace{}%
\AgdaDatatype{SmallTree}\<%
\\
%
\>[4]\AgdaInductiveConstructor{↑}\AgdaSpace{}%
\AgdaSymbol{:}\AgdaSpace{}%
\AgdaDatatype{SmallTree}\AgdaSpace{}%
\AgdaSymbol{→}\AgdaSpace{}%
\AgdaDatatype{SmallTree}\<%
\\
%
\>[4]\AgdaInductiveConstructor{Lim}\AgdaSpace{}%
\AgdaSymbol{:}\AgdaSpace{}%
\AgdaSymbol{(}\AgdaDatatype{ℕ}\AgdaSpace{}%
\AgdaSymbol{→}\AgdaSpace{}%
\AgdaDatatype{SmallTree}\AgdaSymbol{)}\AgdaSpace{}%
\AgdaSymbol{→}\AgdaSpace{}%
\AgdaDatatype{SmallTree}\<%
\end{code}

% !TEX root =  main.tex
\section{Brouwer Trees: An Introduction}
\begin{code}[hide]%
%
\>[2]\AgdaKeyword{open}\AgdaSpace{}%
\AgdaKeyword{import}\AgdaSpace{}%
\AgdaModule{Data.Nat}\AgdaSpace{}%
\AgdaKeyword{hiding}\AgdaSpace{}%
\AgdaSymbol{(}\AgdaOperator{\AgdaDatatype{\AgdaUnderscore{}≤\AgdaUnderscore{}}}\AgdaSpace{}%
\AgdaSymbol{;}\AgdaSpace{}%
\AgdaOperator{\AgdaFunction{\AgdaUnderscore{}<\AgdaUnderscore{}}}\AgdaSymbol{)}\<%
\\
%
\>[2]\AgdaKeyword{open}\AgdaSpace{}%
\AgdaKeyword{import}\AgdaSpace{}%
\AgdaModule{Relation.Binary.PropositionalEquality}\<%
\\
%
\>[2]\AgdaKeyword{open}\AgdaSpace{}%
\AgdaKeyword{import}\AgdaSpace{}%
\AgdaModule{Data.Product}\<%
\\
%
\>[2]\AgdaKeyword{open}\AgdaSpace{}%
\AgdaKeyword{import}\AgdaSpace{}%
\AgdaModule{Relation.Nullary}\<%
\\
%
\>[2]\AgdaKeyword{open}\AgdaSpace{}%
\AgdaKeyword{import}\AgdaSpace{}%
\AgdaModule{Iso}\<%
\\
\>[0]\<%
\end{code}

Under this definition, a Brouwer tree is either zero, the successor of another Brouwer tree, or the limit of a countable sequence of Brouwer trees. However, these are quite weak, in that they can only take the limit of countable sequences.
To represent the limits of uncountable sequences, we can paramterize our definition over some Universe \ala Tarski:

\begin{code}%
\>[0][@{}l@{\AgdaIndent{1}}]%
\>[2]\AgdaKeyword{module}\AgdaSpace{}%
\AgdaModule{RawTree}\AgdaSpace{}%
\AgdaSymbol{\{}\AgdaBound{ℓ}\AgdaSymbol{\}}\<%
\\
\>[2][@{}l@{\AgdaIndent{0}}]%
\>[4]\AgdaSymbol{(}\AgdaBound{ℂ}\AgdaSpace{}%
\AgdaSymbol{:}\AgdaSpace{}%
\AgdaPrimitive{Set}\AgdaSpace{}%
\AgdaBound{ℓ}\AgdaSymbol{)}\<%
\\
%
\>[4]\AgdaSymbol{(}\AgdaBound{El}\AgdaSpace{}%
\AgdaSymbol{:}\AgdaSpace{}%
\AgdaBound{ℂ}\AgdaSpace{}%
\AgdaSymbol{→}\AgdaSpace{}%
\AgdaPrimitive{Set}\AgdaSpace{}%
\AgdaBound{ℓ}\AgdaSymbol{)}\<%
\\
%
\>[4]\AgdaSymbol{(}\AgdaBound{Cℕ}\AgdaSpace{}%
\AgdaSymbol{:}\AgdaSpace{}%
\AgdaBound{ℂ}\AgdaSymbol{)}\AgdaSpace{}%
\AgdaSymbol{(}\AgdaBound{CℕIso}\AgdaSpace{}%
\AgdaSymbol{:}\AgdaSpace{}%
\AgdaRecord{Iso}\AgdaSpace{}%
\AgdaSymbol{(}\AgdaBound{El}\AgdaSpace{}%
\AgdaBound{Cℕ}\AgdaSymbol{)}\AgdaSpace{}%
\AgdaDatatype{ℕ}\AgdaSpace{}%
\AgdaSymbol{)}\AgdaSpace{}%
\AgdaKeyword{where}\<%
\end{code}

We then generalize limits to any function whose domain is the interpretation of some code.
\begin{code}%
%
\>[4]\AgdaKeyword{data}\AgdaSpace{}%
\AgdaDatatype{Tree}\AgdaSpace{}%
\AgdaSymbol{:}\AgdaSpace{}%
\AgdaPrimitive{Set}\AgdaSpace{}%
\AgdaBound{ℓ}\AgdaSpace{}%
\AgdaKeyword{where}\<%
\\
\>[4][@{}l@{\AgdaIndent{0}}]%
\>[6]\AgdaInductiveConstructor{Z}\AgdaSpace{}%
\AgdaSymbol{:}\AgdaSpace{}%
\AgdaDatatype{Tree}\<%
\\
%
\>[6]\AgdaInductiveConstructor{↑}\AgdaSpace{}%
\AgdaSymbol{:}\AgdaSpace{}%
\AgdaDatatype{Tree}\AgdaSpace{}%
\AgdaSymbol{→}\AgdaSpace{}%
\AgdaDatatype{Tree}\<%
\\
%
\>[6]\AgdaInductiveConstructor{Lim}\AgdaSpace{}%
\AgdaSymbol{:}\AgdaSpace{}%
\AgdaSymbol{∀}%
\>[15]\AgdaSymbol{(}\AgdaBound{c}\AgdaSpace{}%
\AgdaSymbol{:}\AgdaSpace{}%
\AgdaBound{ℂ}\AgdaSpace{}%
\AgdaSymbol{)}\AgdaSpace{}%
\AgdaSymbol{→}\AgdaSpace{}%
\AgdaSymbol{(}\AgdaBound{f}\AgdaSpace{}%
\AgdaSymbol{:}\AgdaSpace{}%
\AgdaBound{El}\AgdaSpace{}%
\AgdaBound{c}\AgdaSpace{}%
\AgdaSymbol{→}\AgdaSpace{}%
\AgdaDatatype{Tree}\AgdaSymbol{)}\AgdaSpace{}%
\AgdaSymbol{→}\AgdaSpace{}%
\AgdaDatatype{Tree}\<%
\end{code}

\begin{code}[hide]%
\>[0]\<%
\end{code}


Our module is paramterized over a universe level, a type $\bC$ of \textit{codes}, and an ``elements-of'' interpretation
function $\mathit{El}$, which computes the type represented by each code.
We require that there be a code whose interpretation is isomorphic to the natural numbers,
as this is essential to our construction in \cref{sec:TODO}.
Increasingly larger trees can be obtained by setting $\bC := \AgdaPrimitive{Set} \ \ell$ and
$\mathit{El} := \mathit{id}$ for increasing $\ell$.
However, by defining an inductive-recursive universe,
one can still capture limits over some non-countable types, since
 $\AgdaDatatype{Tree}$ is in $\AgdaPrimitive{Set}$ whenever $\bC$ is.

The small limit constructor can be recovered from the natural-number code
\begin{code}%
\>[0]\<%
\\
\>[0][@{}l@{\AgdaIndent{1}}]%
\>[4]\AgdaFunction{ℕLim}\AgdaSpace{}%
\AgdaSymbol{:}\AgdaSpace{}%
\AgdaSymbol{(}\AgdaDatatype{ℕ}\AgdaSpace{}%
\AgdaSymbol{→}\AgdaSpace{}%
\AgdaDatatype{Tree}\AgdaSymbol{)}\AgdaSpace{}%
\AgdaSymbol{→}\AgdaSpace{}%
\AgdaDatatype{Tree}\<%
\\
%
\>[4]\AgdaFunction{ℕLim}\AgdaSpace{}%
\AgdaBound{f}\AgdaSpace{}%
\AgdaSymbol{=}\AgdaSpace{}%
\AgdaInductiveConstructor{Lim}\AgdaSpace{}%
\AgdaBound{Cℕ}%
\>[21]\AgdaSymbol{(λ}\AgdaSpace{}%
\AgdaBound{cn}\AgdaSpace{}%
\AgdaSymbol{→}\AgdaSpace{}%
\AgdaBound{f}\AgdaSpace{}%
\AgdaSymbol{(}\AgdaField{Iso.fun}\AgdaSpace{}%
\AgdaBound{CℕIso}\AgdaSpace{}%
\AgdaBound{cn}\AgdaSymbol{))}\<%
\end{code}

Brouwer trees are a the quintessential example of a higher-order inductive type.%
\footnote{Not to be confused with Higher Inductive Types (HITs) from Homotopy Type Theory~\citep{hottbook}}:
Each tree is built using smaller trees or functions producing smaller trees, which is essentially
a way of storing a possibly infinite number of smaller trees.

\subsection{Ordering Trees}

Our ultimate goal is to have a well-founded ordering%
\footnote{Technically, this is a well-founded quasi-ordering because there are pairs of
  trees which are related by both $\leq$ and $\geq$, but which are not propositionally equal.},
so we define a relation to order Brouwer trees.

\begin{code}%
%
\>[4]\AgdaKeyword{data}\AgdaSpace{}%
\AgdaOperator{\AgdaDatatype{\AgdaUnderscore{}≤\AgdaUnderscore{}}}\AgdaSpace{}%
\AgdaSymbol{:}\AgdaSpace{}%
\AgdaDatatype{Tree}\AgdaSpace{}%
\AgdaSymbol{→}\AgdaSpace{}%
\AgdaDatatype{Tree}\AgdaSpace{}%
\AgdaSymbol{→}\AgdaSpace{}%
\AgdaPrimitive{Set}\AgdaSpace{}%
\AgdaBound{ℓ}\AgdaSpace{}%
\AgdaKeyword{where}\<%
\\
\>[4][@{}l@{\AgdaIndent{0}}]%
\>[6]\AgdaInductiveConstructor{≤-Z}\AgdaSpace{}%
\AgdaSymbol{:}\AgdaSpace{}%
\AgdaSymbol{∀}\AgdaSpace{}%
\AgdaSymbol{\{}\AgdaBound{t}\AgdaSymbol{\}}\AgdaSpace{}%
\AgdaSymbol{→}\AgdaSpace{}%
\AgdaInductiveConstructor{Z}\AgdaSpace{}%
\AgdaOperator{\AgdaDatatype{≤}}\AgdaSpace{}%
\AgdaBound{t}\<%
\\
%
\>[6]\AgdaInductiveConstructor{≤-sucMono}\AgdaSpace{}%
\AgdaSymbol{:}\AgdaSpace{}%
\AgdaSymbol{∀}\AgdaSpace{}%
\AgdaSymbol{\{}\AgdaBound{t1}\AgdaSpace{}%
\AgdaBound{t2}\AgdaSymbol{\}}\<%
\\
\>[6][@{}l@{\AgdaIndent{0}}]%
\>[8]\AgdaSymbol{→}\AgdaSpace{}%
\AgdaBound{t1}\AgdaSpace{}%
\AgdaOperator{\AgdaDatatype{≤}}\AgdaSpace{}%
\AgdaBound{t2}\<%
\\
%
\>[8]\AgdaSymbol{→}\AgdaSpace{}%
\AgdaInductiveConstructor{↑}\AgdaSpace{}%
\AgdaBound{t1}\AgdaSpace{}%
\AgdaOperator{\AgdaDatatype{≤}}\AgdaSpace{}%
\AgdaInductiveConstructor{↑}\AgdaSpace{}%
\AgdaBound{t2}\<%
\\
%
\>[6]\AgdaInductiveConstructor{≤-cocone}\AgdaSpace{}%
\AgdaSymbol{:}\AgdaSpace{}%
\AgdaSymbol{∀}%
\>[20]\AgdaSymbol{\{}\AgdaBound{t}\AgdaSymbol{\}}\AgdaSpace{}%
\AgdaSymbol{\{}\AgdaBound{c}\AgdaSpace{}%
\AgdaSymbol{:}\AgdaSpace{}%
\AgdaBound{ℂ}\AgdaSymbol{\}}\AgdaSpace{}%
\AgdaSymbol{(}\AgdaBound{f}\AgdaSpace{}%
\AgdaSymbol{:}\AgdaSpace{}%
\AgdaBound{El}\AgdaSpace{}%
\AgdaBound{c}%
\>[43]\AgdaSymbol{→}\AgdaSpace{}%
\AgdaDatatype{Tree}\AgdaSymbol{)}\AgdaSpace{}%
\AgdaSymbol{(}\AgdaBound{k}\AgdaSpace{}%
\AgdaSymbol{:}\AgdaSpace{}%
\AgdaBound{El}\AgdaSpace{}%
\AgdaBound{c}\AgdaSymbol{)}\<%
\\
\>[6][@{}l@{\AgdaIndent{0}}]%
\>[8]\AgdaSymbol{→}\AgdaSpace{}%
\AgdaBound{t}\AgdaSpace{}%
\AgdaOperator{\AgdaDatatype{≤}}\AgdaSpace{}%
\AgdaBound{f}\AgdaSpace{}%
\AgdaBound{k}\<%
\\
%
\>[8]\AgdaSymbol{→}\AgdaSpace{}%
\AgdaBound{t}\AgdaSpace{}%
\AgdaOperator{\AgdaDatatype{≤}}\AgdaSpace{}%
\AgdaInductiveConstructor{Lim}\AgdaSpace{}%
\AgdaBound{c}\AgdaSpace{}%
\AgdaBound{f}\<%
\\
%
\>[6]\AgdaInductiveConstructor{≤-limiting}\AgdaSpace{}%
\AgdaSymbol{:}\AgdaSpace{}%
\AgdaSymbol{∀}%
\>[23]\AgdaSymbol{\{}\AgdaBound{t}\AgdaSymbol{\}}\AgdaSpace{}%
\AgdaSymbol{\{}\AgdaBound{c}\AgdaSpace{}%
\AgdaSymbol{:}\AgdaSpace{}%
\AgdaBound{ℂ}\AgdaSymbol{\}}\<%
\\
\>[6][@{}l@{\AgdaIndent{0}}]%
\>[8]\AgdaSymbol{→}\AgdaSpace{}%
\AgdaSymbol{(}\AgdaBound{f}\AgdaSpace{}%
\AgdaSymbol{:}\AgdaSpace{}%
\AgdaBound{El}\AgdaSpace{}%
\AgdaBound{c}\AgdaSpace{}%
\AgdaSymbol{→}\AgdaSpace{}%
\AgdaDatatype{Tree}\AgdaSymbol{)}\<%
\\
%
\>[8]\AgdaSymbol{→}\AgdaSpace{}%
\AgdaSymbol{(∀}\AgdaSpace{}%
\AgdaBound{k}\AgdaSpace{}%
\AgdaSymbol{→}\AgdaSpace{}%
\AgdaBound{f}\AgdaSpace{}%
\AgdaBound{k}\AgdaSpace{}%
\AgdaOperator{\AgdaDatatype{≤}}\AgdaSpace{}%
\AgdaBound{t}\AgdaSymbol{)}\<%
\\
%
\>[8]\AgdaSymbol{→}\AgdaSpace{}%
\AgdaInductiveConstructor{Lim}\AgdaSpace{}%
\AgdaBound{c}\AgdaSpace{}%
\AgdaBound{f}\AgdaSpace{}%
\AgdaOperator{\AgdaDatatype{≤}}\AgdaSpace{}%
\AgdaBound{t}\<%
\\
\>[0]\<%
\end{code}

      This relation is reflexive:
\begin{code}%
\>[0][@{}l@{\AgdaIndent{1}}]%
\>[4]\AgdaFunction{≤-refl}\AgdaSpace{}%
\AgdaSymbol{:}\AgdaSpace{}%
\AgdaSymbol{∀}\AgdaSpace{}%
\AgdaBound{t}\AgdaSpace{}%
\AgdaSymbol{→}\AgdaSpace{}%
\AgdaBound{t}\AgdaSpace{}%
\AgdaOperator{\AgdaDatatype{≤}}\AgdaSpace{}%
\AgdaBound{t}\<%
\\
%
\>[4]\AgdaFunction{≤-refl}\AgdaSpace{}%
\AgdaInductiveConstructor{Z}\AgdaSpace{}%
\AgdaSymbol{=}\AgdaSpace{}%
\AgdaInductiveConstructor{≤-Z}\<%
\\
%
\>[4]\AgdaFunction{≤-refl}\AgdaSpace{}%
\AgdaSymbol{(}\AgdaInductiveConstructor{↑}\AgdaSpace{}%
\AgdaBound{t}\AgdaSymbol{)}\AgdaSpace{}%
\AgdaSymbol{=}\AgdaSpace{}%
\AgdaInductiveConstructor{≤-sucMono}\AgdaSpace{}%
\AgdaSymbol{(}\AgdaFunction{≤-refl}\AgdaSpace{}%
\AgdaBound{t}\AgdaSymbol{)}\<%
\\
%
\>[4]\AgdaFunction{≤-refl}\AgdaSpace{}%
\AgdaSymbol{(}\AgdaInductiveConstructor{Lim}\AgdaSpace{}%
\AgdaBound{c}\AgdaSpace{}%
\AgdaBound{f}\AgdaSymbol{)}\<%
\\
\>[4][@{}l@{\AgdaIndent{0}}]%
\>[6]\AgdaSymbol{=}\AgdaSpace{}%
\AgdaInductiveConstructor{≤-limiting}\AgdaSpace{}%
\AgdaBound{f}\AgdaSpace{}%
\AgdaSymbol{(λ}\AgdaSpace{}%
\AgdaBound{k}\AgdaSpace{}%
\AgdaSymbol{→}\AgdaSpace{}%
\AgdaInductiveConstructor{≤-cocone}\AgdaSpace{}%
\AgdaBound{f}\AgdaSpace{}%
\AgdaBound{k}\AgdaSpace{}%
\AgdaSymbol{(}\AgdaFunction{≤-refl}\AgdaSpace{}%
\AgdaSymbol{(}\AgdaBound{f}\AgdaSpace{}%
\AgdaBound{k}\AgdaSymbol{)))}\<%
\end{code}
\begin{code}[hide]%
%
\>[4]\AgdaFunction{≤-reflEq}\AgdaSpace{}%
\AgdaSymbol{:}\AgdaSpace{}%
\AgdaSymbol{∀}\AgdaSpace{}%
\AgdaSymbol{\{}\AgdaBound{t1}\AgdaSpace{}%
\AgdaBound{t2}\AgdaSymbol{\}}\AgdaSpace{}%
\AgdaSymbol{→}\AgdaSpace{}%
\AgdaBound{t1}\AgdaSpace{}%
\AgdaOperator{\AgdaDatatype{≡}}\AgdaSpace{}%
\AgdaBound{t2}\AgdaSpace{}%
\AgdaSymbol{→}\AgdaSpace{}%
\AgdaBound{t1}\AgdaSpace{}%
\AgdaOperator{\AgdaDatatype{≤}}\AgdaSpace{}%
\AgdaBound{t2}\<%
\\
%
\>[4]\AgdaFunction{≤-reflEq}\AgdaSpace{}%
\AgdaInductiveConstructor{refl}\AgdaSpace{}%
\AgdaSymbol{=}\AgdaSpace{}%
\AgdaFunction{≤-refl}\AgdaSpace{}%
\AgdaSymbol{\AgdaUnderscore{}}\<%
\end{code}

      Crucially, it is also transitive, making the relation a preorder.
We modify our the order relation from that of \citet{KRAUS2023113843}
so that transitivity can be proven constructively, rather than adding it as a constructor
for the relation. This allows us to prove well-foundedness of the relation without needing
quotient types or other advanced features.

\begin{code}%
%
\>[4]\AgdaFunction{≤-trans}\AgdaSpace{}%
\AgdaSymbol{:}\AgdaSpace{}%
\AgdaSymbol{∀}\AgdaSpace{}%
\AgdaSymbol{\{}\AgdaBound{t1}\AgdaSpace{}%
\AgdaBound{t2}\AgdaSpace{}%
\AgdaBound{t3}\AgdaSymbol{\}}\AgdaSpace{}%
\AgdaSymbol{→}\AgdaSpace{}%
\AgdaBound{t1}\AgdaSpace{}%
\AgdaOperator{\AgdaDatatype{≤}}\AgdaSpace{}%
\AgdaBound{t2}\AgdaSpace{}%
\AgdaSymbol{→}\AgdaSpace{}%
\AgdaBound{t2}\AgdaSpace{}%
\AgdaOperator{\AgdaDatatype{≤}}\AgdaSpace{}%
\AgdaBound{t3}\AgdaSpace{}%
\AgdaSymbol{→}\AgdaSpace{}%
\AgdaBound{t1}\AgdaSpace{}%
\AgdaOperator{\AgdaDatatype{≤}}\AgdaSpace{}%
\AgdaBound{t3}\<%
\\
%
\>[4]\AgdaFunction{≤-trans}\AgdaSpace{}%
\AgdaInductiveConstructor{≤-Z}\AgdaSpace{}%
\AgdaBound{p23}\AgdaSpace{}%
\AgdaSymbol{=}\AgdaSpace{}%
\AgdaInductiveConstructor{≤-Z}\<%
\\
%
\>[4]\AgdaFunction{≤-trans}\AgdaSpace{}%
\AgdaSymbol{(}\AgdaInductiveConstructor{≤-sucMono}\AgdaSpace{}%
\AgdaBound{p12}\AgdaSymbol{)}\AgdaSpace{}%
\AgdaSymbol{(}\AgdaInductiveConstructor{≤-sucMono}\AgdaSpace{}%
\AgdaBound{p23}\AgdaSymbol{)}\<%
\\
\>[4][@{}l@{\AgdaIndent{0}}]%
\>[6]\AgdaSymbol{=}\AgdaSpace{}%
\AgdaInductiveConstructor{≤-sucMono}\AgdaSpace{}%
\AgdaSymbol{(}\AgdaFunction{≤-trans}\AgdaSpace{}%
\AgdaBound{p12}\AgdaSpace{}%
\AgdaBound{p23}\AgdaSymbol{)}\<%
\\
%
\>[4]\AgdaCatchallClause{\AgdaFunction{≤-trans}}\AgdaSpace{}%
\AgdaCatchallClause{\AgdaBound{p12}}\AgdaSpace{}%
\AgdaCatchallClause{\AgdaSymbol{(}}\AgdaCatchallClause{\AgdaInductiveConstructor{≤-cocone}}\AgdaSpace{}%
\AgdaCatchallClause{\AgdaBound{f}}\AgdaSpace{}%
\AgdaCatchallClause{\AgdaBound{k}}\AgdaSpace{}%
\AgdaCatchallClause{\AgdaBound{p23}}\AgdaCatchallClause{\AgdaSymbol{)}}\<%
\\
\>[4][@{}l@{\AgdaIndent{0}}]%
\>[6]\AgdaSymbol{=}\AgdaSpace{}%
\AgdaInductiveConstructor{≤-cocone}\AgdaSpace{}%
\AgdaBound{f}\AgdaSpace{}%
\AgdaBound{k}\AgdaSpace{}%
\AgdaSymbol{(}\AgdaFunction{≤-trans}\AgdaSpace{}%
\AgdaBound{p12}\AgdaSpace{}%
\AgdaBound{p23}\AgdaSymbol{)}\<%
\\
%
\>[4]\AgdaCatchallClause{\AgdaFunction{≤-trans}}\AgdaSpace{}%
\AgdaCatchallClause{\AgdaSymbol{(}}\AgdaCatchallClause{\AgdaInductiveConstructor{≤-limiting}}\AgdaSpace{}%
\AgdaCatchallClause{\AgdaBound{f}}\AgdaSpace{}%
\AgdaCatchallClause{\AgdaBound{x}}\AgdaCatchallClause{\AgdaSymbol{)}}\AgdaSpace{}%
\AgdaCatchallClause{\AgdaBound{p23}}\<%
\\
\>[4][@{}l@{\AgdaIndent{0}}]%
\>[6]\AgdaSymbol{=}\AgdaSpace{}%
\AgdaInductiveConstructor{≤-limiting}\AgdaSpace{}%
\AgdaBound{f}\AgdaSpace{}%
\AgdaSymbol{(λ}\AgdaSpace{}%
\AgdaBound{k}\AgdaSpace{}%
\AgdaSymbol{→}\AgdaSpace{}%
\AgdaFunction{≤-trans}\AgdaSpace{}%
\AgdaSymbol{(}\AgdaBound{x}\AgdaSpace{}%
\AgdaBound{k}\AgdaSymbol{)}\AgdaSpace{}%
\AgdaBound{p23}\AgdaSymbol{)}\<%
\\
%
\>[4]\AgdaFunction{≤-trans}\AgdaSpace{}%
\AgdaSymbol{(}\AgdaInductiveConstructor{≤-cocone}\AgdaSpace{}%
\AgdaBound{f}\AgdaSpace{}%
\AgdaBound{k}\AgdaSpace{}%
\AgdaBound{p12}\AgdaSymbol{)}\AgdaSpace{}%
\AgdaSymbol{(}\AgdaInductiveConstructor{≤-limiting}\AgdaSpace{}%
\AgdaDottedPattern{\AgdaSymbol{.}}\AgdaDottedPattern{\AgdaBound{f}}\AgdaSpace{}%
\AgdaBound{x}\AgdaSymbol{)}\<%
\\
\>[4][@{}l@{\AgdaIndent{0}}]%
\>[6]\AgdaSymbol{=}\AgdaSpace{}%
\AgdaFunction{≤-trans}\AgdaSpace{}%
\AgdaBound{p12}\AgdaSpace{}%
\AgdaSymbol{(}\AgdaBound{x}\AgdaSpace{}%
\AgdaBound{k}\AgdaSymbol{)}\<%
\end{code}

\begin{code}[hide]%
%
\>[4]\AgdaFunction{extLim}\AgdaSpace{}%
\AgdaSymbol{:}\AgdaSpace{}%
\AgdaSymbol{∀}%
\>[17]\AgdaSymbol{\{}\AgdaBound{c}\AgdaSpace{}%
\AgdaSymbol{:}\AgdaSpace{}%
\AgdaBound{ℂ}\AgdaSymbol{\}}\<%
\\
\>[4][@{}l@{\AgdaIndent{0}}]%
\>[6]\AgdaSymbol{→}%
\>[9]\AgdaSymbol{(}\AgdaBound{f1}\AgdaSpace{}%
\AgdaBound{f2}\AgdaSpace{}%
\AgdaSymbol{:}\AgdaSpace{}%
\AgdaBound{El}\AgdaSpace{}%
\AgdaBound{c}\AgdaSpace{}%
\AgdaSymbol{→}\AgdaSpace{}%
\AgdaDatatype{Tree}\AgdaSymbol{)}\<%
\\
%
\>[6]\AgdaSymbol{→}\AgdaSpace{}%
\AgdaSymbol{(∀}\AgdaSpace{}%
\AgdaBound{k}\AgdaSpace{}%
\AgdaSymbol{→}\AgdaSpace{}%
\AgdaBound{f1}\AgdaSpace{}%
\AgdaBound{k}\AgdaSpace{}%
\AgdaOperator{\AgdaDatatype{≤}}\AgdaSpace{}%
\AgdaBound{f2}\AgdaSpace{}%
\AgdaBound{k}\AgdaSymbol{)}\<%
\\
%
\>[6]\AgdaSymbol{→}\AgdaSpace{}%
\AgdaInductiveConstructor{Lim}\AgdaSpace{}%
\AgdaBound{c}\AgdaSpace{}%
\AgdaBound{f1}\AgdaSpace{}%
\AgdaOperator{\AgdaDatatype{≤}}\AgdaSpace{}%
\AgdaInductiveConstructor{Lim}\AgdaSpace{}%
\AgdaBound{c}\AgdaSpace{}%
\AgdaBound{f2}\<%
\\
%
\>[4]\AgdaFunction{extLim}\AgdaSpace{}%
\AgdaSymbol{\{}\AgdaArgument{c}\AgdaSpace{}%
\AgdaSymbol{=}\AgdaSpace{}%
\AgdaBound{c}\AgdaSymbol{\}}\AgdaSpace{}%
\AgdaBound{f1}\AgdaSpace{}%
\AgdaBound{f2}\AgdaSpace{}%
\AgdaBound{all}\<%
\\
\>[4][@{}l@{\AgdaIndent{0}}]%
\>[6]\AgdaSymbol{=}\AgdaSpace{}%
\AgdaInductiveConstructor{≤-limiting}\AgdaSpace{}%
\AgdaBound{f1}\AgdaSpace{}%
\AgdaSymbol{(λ}\AgdaSpace{}%
\AgdaBound{k}\AgdaSpace{}%
\AgdaSymbol{→}\AgdaSpace{}%
\AgdaInductiveConstructor{≤-cocone}\AgdaSpace{}%
\AgdaBound{f2}\AgdaSpace{}%
\AgdaBound{k}\AgdaSpace{}%
\AgdaSymbol{(}\AgdaBound{all}\AgdaSpace{}%
\AgdaBound{k}\AgdaSymbol{))}\<%
\\
%
\\[\AgdaEmptyExtraSkip]%
%
\>[4]\AgdaKeyword{infixr}\AgdaSpace{}%
\AgdaNumber{10}\AgdaSpace{}%
\AgdaOperator{\AgdaFunction{\AgdaUnderscore{}≤⨟\AgdaUnderscore{}}}\<%
\end{code}

We create an infix version of transitivity for more readable construction of proofs:
\begin{code}%
%
\>[4]\AgdaOperator{\AgdaFunction{\AgdaUnderscore{}≤⨟\AgdaUnderscore{}}}\AgdaSpace{}%
\AgdaSymbol{:}%
\>[12]\AgdaSymbol{∀}\AgdaSpace{}%
\AgdaSymbol{\{}\AgdaBound{t1}\AgdaSpace{}%
\AgdaBound{t2}\AgdaSpace{}%
\AgdaBound{t3}\AgdaSymbol{\}}\AgdaSpace{}%
\AgdaSymbol{→}\AgdaSpace{}%
\AgdaBound{t1}\AgdaSpace{}%
\AgdaOperator{\AgdaDatatype{≤}}\AgdaSpace{}%
\AgdaBound{t2}\AgdaSpace{}%
\AgdaSymbol{→}\AgdaSpace{}%
\AgdaBound{t2}\AgdaSpace{}%
\AgdaOperator{\AgdaDatatype{≤}}\AgdaSpace{}%
\AgdaBound{t3}\AgdaSpace{}%
\AgdaSymbol{→}\AgdaSpace{}%
\AgdaBound{t1}\AgdaSpace{}%
\AgdaOperator{\AgdaDatatype{≤}}\AgdaSpace{}%
\AgdaBound{t3}\<%
\\
%
\>[4]\AgdaBound{lt1}\AgdaSpace{}%
\AgdaOperator{\AgdaFunction{≤⨟}}\AgdaSpace{}%
\AgdaBound{lt2}\AgdaSpace{}%
\AgdaSymbol{=}\AgdaSpace{}%
\AgdaFunction{≤-trans}\AgdaSpace{}%
\AgdaBound{lt1}\AgdaSpace{}%
\AgdaBound{lt2}\<%
\end{code}

\subsubsection{Strict Ordering}

We can define a strictly-less-than relation in terms of our less-than relation
and the successor constructor:
\begin{code}%
%
\>[4]\AgdaOperator{\AgdaFunction{\AgdaUnderscore{}<\AgdaUnderscore{}}}\AgdaSpace{}%
\AgdaSymbol{:}\AgdaSpace{}%
\AgdaDatatype{Tree}\AgdaSpace{}%
\AgdaSymbol{→}\AgdaSpace{}%
\AgdaDatatype{Tree}\AgdaSpace{}%
\AgdaSymbol{→}\AgdaSpace{}%
\AgdaPrimitive{Set}\AgdaSpace{}%
\AgdaBound{ℓ}\<%
\\
%
\>[4]\AgdaBound{t1}\AgdaSpace{}%
\AgdaOperator{\AgdaFunction{<}}\AgdaSpace{}%
\AgdaBound{t2}\AgdaSpace{}%
\AgdaSymbol{=}\AgdaSpace{}%
\AgdaInductiveConstructor{↑}\AgdaSpace{}%
\AgdaBound{t1}\AgdaSpace{}%
\AgdaOperator{\AgdaDatatype{≤}}\AgdaSpace{}%
\AgdaBound{t2}\<%
\end{code}

  That is, a $t_{1}$ is strictly smaller than $t_{2}$ if the tree one-size larger than $t_{1}$ is as small as $t_{2}$.
  This relation has the properties one expects of a strictly-less-than
  relation: it is a transitive  sub-relation of the less-than relation,
  and no tree is strictly smaller than zero.
  \je{TODO more?}

\begin{code}%
%
\>[4]\AgdaFunction{≤↑t}\AgdaSpace{}%
\AgdaSymbol{:}\AgdaSpace{}%
\AgdaSymbol{∀}\AgdaSpace{}%
\AgdaBound{t}\AgdaSpace{}%
\AgdaSymbol{→}\AgdaSpace{}%
\AgdaBound{t}\AgdaSpace{}%
\AgdaOperator{\AgdaDatatype{≤}}\AgdaSpace{}%
\AgdaInductiveConstructor{↑}\AgdaSpace{}%
\AgdaBound{t}\<%
\\
%
\>[4]\AgdaFunction{≤↑t}\AgdaSpace{}%
\AgdaInductiveConstructor{Z}\AgdaSpace{}%
\AgdaSymbol{=}\AgdaSpace{}%
\AgdaInductiveConstructor{≤-Z}\<%
\\
%
\>[4]\AgdaFunction{≤↑t}\AgdaSpace{}%
\AgdaSymbol{(}\AgdaInductiveConstructor{↑}\AgdaSpace{}%
\AgdaBound{t}\AgdaSymbol{)}\AgdaSpace{}%
\AgdaSymbol{=}\AgdaSpace{}%
\AgdaInductiveConstructor{≤-sucMono}\AgdaSpace{}%
\AgdaSymbol{(}\AgdaFunction{≤↑t}\AgdaSpace{}%
\AgdaBound{t}\AgdaSymbol{)}\<%
\\
%
\>[4]\AgdaFunction{≤↑t}\AgdaSpace{}%
\AgdaSymbol{(}\AgdaInductiveConstructor{Lim}\AgdaSpace{}%
\AgdaBound{c}\AgdaSpace{}%
\AgdaBound{f}\AgdaSymbol{)}\<%
\\
\>[4][@{}l@{\AgdaIndent{0}}]%
\>[6]\AgdaSymbol{=}%
\>[358I]\AgdaInductiveConstructor{≤-limiting}\AgdaSpace{}%
\AgdaBound{f}\AgdaSpace{}%
\AgdaSymbol{λ}\AgdaSpace{}%
\AgdaBound{k}\AgdaSpace{}%
\AgdaSymbol{→}\<%
\\
\>[.][@{}l@{}]\<[358I]%
\>[8]\AgdaSymbol{(}\AgdaFunction{≤↑t}\AgdaSpace{}%
\AgdaSymbol{(}\AgdaBound{f}\AgdaSpace{}%
\AgdaBound{k}\AgdaSymbol{))}\<%
\\
%
\>[8]\AgdaOperator{\AgdaFunction{≤⨟}}\AgdaSpace{}%
\AgdaSymbol{(}\AgdaInductiveConstructor{≤-sucMono}\AgdaSpace{}%
\AgdaSymbol{(}\AgdaInductiveConstructor{≤-cocone}\AgdaSpace{}%
\AgdaBound{f}\AgdaSpace{}%
\AgdaBound{k}\AgdaSpace{}%
\AgdaSymbol{(}\AgdaFunction{≤-refl}\AgdaSpace{}%
\AgdaSymbol{(}\AgdaBound{f}\AgdaSpace{}%
\AgdaBound{k}\AgdaSymbol{))))}\<%
\end{code}

\begin{code}%
%
\>[4]\AgdaFunction{<-in-≤}\AgdaSpace{}%
\AgdaSymbol{:}\AgdaSpace{}%
\AgdaSymbol{∀}\AgdaSpace{}%
\AgdaSymbol{\{}\AgdaBound{x}\AgdaSpace{}%
\AgdaBound{y}\AgdaSymbol{\}}\AgdaSpace{}%
\AgdaSymbol{→}\AgdaSpace{}%
\AgdaBound{x}\AgdaSpace{}%
\AgdaOperator{\AgdaFunction{<}}\AgdaSpace{}%
\AgdaBound{y}\AgdaSpace{}%
\AgdaSymbol{→}\AgdaSpace{}%
\AgdaBound{x}\AgdaSpace{}%
\AgdaOperator{\AgdaDatatype{≤}}\AgdaSpace{}%
\AgdaBound{y}\<%
\\
%
\>[4]\AgdaFunction{<-in-≤}\AgdaSpace{}%
\AgdaBound{pf}\AgdaSpace{}%
\AgdaSymbol{=}\AgdaSpace{}%
\AgdaFunction{≤-trans}\AgdaSpace{}%
\AgdaSymbol{(}\AgdaFunction{≤↑t}\AgdaSpace{}%
\AgdaSymbol{\AgdaUnderscore{})}\AgdaSpace{}%
\AgdaBound{pf}\<%
\\
%
\\[\AgdaEmptyExtraSkip]%
%
\>[4]\AgdaFunction{<∘≤-in-<}\AgdaSpace{}%
\AgdaSymbol{:}\AgdaSpace{}%
\AgdaSymbol{∀}\AgdaSpace{}%
\AgdaSymbol{\{}\AgdaBound{x}\AgdaSpace{}%
\AgdaBound{y}\AgdaSpace{}%
\AgdaBound{z}\AgdaSymbol{\}}\AgdaSpace{}%
\AgdaSymbol{→}\AgdaSpace{}%
\AgdaBound{x}\AgdaSpace{}%
\AgdaOperator{\AgdaFunction{<}}\AgdaSpace{}%
\AgdaBound{y}\AgdaSpace{}%
\AgdaSymbol{→}\AgdaSpace{}%
\AgdaBound{y}\AgdaSpace{}%
\AgdaOperator{\AgdaDatatype{≤}}\AgdaSpace{}%
\AgdaBound{z}\AgdaSpace{}%
\AgdaSymbol{→}\AgdaSpace{}%
\AgdaBound{x}\AgdaSpace{}%
\AgdaOperator{\AgdaFunction{<}}\AgdaSpace{}%
\AgdaBound{z}\<%
\\
%
\>[4]\AgdaFunction{<∘≤-in-<}\AgdaSpace{}%
\AgdaBound{x<y}\AgdaSpace{}%
\AgdaBound{y≤z}\AgdaSpace{}%
\AgdaSymbol{=}\AgdaSpace{}%
\AgdaFunction{≤-trans}\AgdaSpace{}%
\AgdaBound{x<y}\AgdaSpace{}%
\AgdaBound{y≤z}\<%
\\
%
\\[\AgdaEmptyExtraSkip]%
%
\>[4]\AgdaFunction{≤∘<-in-<}\AgdaSpace{}%
\AgdaSymbol{:}\AgdaSpace{}%
\AgdaSymbol{∀}\AgdaSpace{}%
\AgdaSymbol{\{}\AgdaBound{x}\AgdaSpace{}%
\AgdaBound{y}\AgdaSpace{}%
\AgdaBound{z}\AgdaSymbol{\}}\AgdaSpace{}%
\AgdaSymbol{→}\AgdaSpace{}%
\AgdaBound{x}\AgdaSpace{}%
\AgdaOperator{\AgdaDatatype{≤}}\AgdaSpace{}%
\AgdaBound{y}\AgdaSpace{}%
\AgdaSymbol{→}\AgdaSpace{}%
\AgdaBound{y}\AgdaSpace{}%
\AgdaOperator{\AgdaFunction{<}}\AgdaSpace{}%
\AgdaBound{z}\AgdaSpace{}%
\AgdaSymbol{→}\AgdaSpace{}%
\AgdaBound{x}\AgdaSpace{}%
\AgdaOperator{\AgdaFunction{<}}\AgdaSpace{}%
\AgdaBound{z}\<%
\\
%
\>[4]\AgdaFunction{≤∘<-in-<}\AgdaSpace{}%
\AgdaSymbol{\{}\AgdaBound{x}\AgdaSymbol{\}}\AgdaSpace{}%
\AgdaSymbol{\{}\AgdaBound{y}\AgdaSymbol{\}}\AgdaSpace{}%
\AgdaSymbol{\{}\AgdaBound{z}\AgdaSymbol{\}}\AgdaSpace{}%
\AgdaBound{x≤y}\AgdaSpace{}%
\AgdaBound{y<z}\AgdaSpace{}%
\AgdaSymbol{=}\AgdaSpace{}%
\AgdaFunction{≤-trans}\AgdaSpace{}%
\AgdaSymbol{(}\AgdaInductiveConstructor{≤-sucMono}\AgdaSpace{}%
\AgdaBound{x≤y}\AgdaSymbol{)}\AgdaSpace{}%
\AgdaBound{y<z}\<%
\\
%
\\[\AgdaEmptyExtraSkip]%
%
\>[4]\AgdaFunction{¬<Z}\AgdaSpace{}%
\AgdaSymbol{:}\AgdaSpace{}%
\AgdaSymbol{∀}\AgdaSpace{}%
\AgdaBound{t}\AgdaSpace{}%
\AgdaSymbol{→}\AgdaSpace{}%
\AgdaOperator{\AgdaFunction{¬}}\AgdaSymbol{(}\AgdaBound{t}\AgdaSpace{}%
\AgdaOperator{\AgdaFunction{<}}\AgdaSpace{}%
\AgdaInductiveConstructor{Z}\AgdaSymbol{)}\<%
\\
%
\>[4]\AgdaFunction{¬<Z}\AgdaSpace{}%
\AgdaBound{t}\AgdaSpace{}%
\AgdaSymbol{()}\<%
\end{code}


  \begin{code}[hide]%
\>[0]\<%
\end{code}






\begin{code}[hide]%
\>[0][@{}l@{\AgdaIndent{1}}]%
\>[4]\AgdaFunction{existsLim}\AgdaSpace{}%
\AgdaSymbol{:}\AgdaSpace{}%
\AgdaSymbol{∀}%
\>[19]\AgdaSymbol{\{}\AgdaBound{c1}\AgdaSpace{}%
\AgdaSymbol{:}\AgdaSpace{}%
\AgdaBound{ℂ}\AgdaSymbol{\}}\AgdaSpace{}%
\AgdaSymbol{\{}\AgdaBound{c2}\AgdaSpace{}%
\AgdaSymbol{:}\AgdaSpace{}%
\AgdaBound{ℂ}\AgdaSymbol{\}}\AgdaSpace{}%
\AgdaSymbol{→}%
\>[40]\AgdaSymbol{(}\AgdaBound{f1}\AgdaSpace{}%
\AgdaSymbol{:}\AgdaSpace{}%
\AgdaBound{El}\AgdaSpace{}%
\AgdaBound{c1}%
\>[53]\AgdaSymbol{→}\AgdaSpace{}%
\AgdaDatatype{Tree}\AgdaSymbol{)}\AgdaSpace{}%
\AgdaSymbol{(}\AgdaBound{f2}\AgdaSpace{}%
\AgdaSymbol{:}\AgdaSpace{}%
\AgdaBound{El}%
\>[71]\AgdaBound{c2}%
\>[75]\AgdaSymbol{→}\AgdaSpace{}%
\AgdaDatatype{Tree}\AgdaSymbol{)}\AgdaSpace{}%
\AgdaSymbol{→}\AgdaSpace{}%
\AgdaSymbol{(∀}\AgdaSpace{}%
\AgdaBound{k1}\AgdaSpace{}%
\AgdaSymbol{→}\AgdaSpace{}%
\AgdaFunction{Σ[}\AgdaSpace{}%
\AgdaBound{k2}\AgdaSpace{}%
\AgdaFunction{∈}\AgdaSpace{}%
\AgdaBound{El}%
\>[105]\AgdaBound{c2}\AgdaSpace{}%
\AgdaFunction{]}\AgdaSpace{}%
\AgdaBound{f1}\AgdaSpace{}%
\AgdaBound{k1}\AgdaSpace{}%
\AgdaOperator{\AgdaDatatype{≤}}\AgdaSpace{}%
\AgdaBound{f2}\AgdaSpace{}%
\AgdaBound{k2}\AgdaSymbol{)}\AgdaSpace{}%
\AgdaSymbol{→}\AgdaSpace{}%
\AgdaInductiveConstructor{Lim}%
\>[132]\AgdaBound{c1}\AgdaSpace{}%
\AgdaBound{f1}\AgdaSpace{}%
\AgdaOperator{\AgdaDatatype{≤}}\AgdaSpace{}%
\AgdaInductiveConstructor{Lim}%
\>[145]\AgdaBound{c2}\AgdaSpace{}%
\AgdaBound{f2}\<%
\\
%
\>[4]\AgdaFunction{existsLim}\AgdaSpace{}%
\AgdaSymbol{\{}\AgdaBound{æ1}\AgdaSymbol{\}}\AgdaSpace{}%
\AgdaSymbol{\{}\AgdaBound{æ2}\AgdaSymbol{\}}\AgdaSpace{}%
\AgdaBound{f1}\AgdaSpace{}%
\AgdaBound{f2}\AgdaSpace{}%
\AgdaBound{allex}\AgdaSpace{}%
\AgdaSymbol{=}\AgdaSpace{}%
\AgdaInductiveConstructor{≤-limiting}%
\>[50]\AgdaBound{f1}\AgdaSpace{}%
\AgdaSymbol{(λ}\AgdaSpace{}%
\AgdaBound{k}\AgdaSpace{}%
\AgdaSymbol{→}\AgdaSpace{}%
\AgdaInductiveConstructor{≤-cocone}\AgdaSpace{}%
\AgdaBound{f2}\AgdaSpace{}%
\AgdaSymbol{(}\AgdaField{proj₁}\AgdaSpace{}%
\AgdaSymbol{(}\AgdaBound{allex}\AgdaSpace{}%
\AgdaBound{k}\AgdaSymbol{))}\AgdaSpace{}%
\AgdaSymbol{(}\AgdaField{proj₂}\AgdaSpace{}%
\AgdaSymbol{(}\AgdaBound{allex}\AgdaSpace{}%
\AgdaBound{k}\AgdaSymbol{)))}\<%
\end{code}

\subsection{Well Founded Induction}

\begin{code}%
\>[0]\<%
\\
%
\>[4]\AgdaKeyword{open}\AgdaSpace{}%
\AgdaKeyword{import}\AgdaSpace{}%
\AgdaModule{Induction.WellFounded}\<%
\\
%
\>[4]\AgdaFunction{access}\AgdaSpace{}%
\AgdaSymbol{:}\AgdaSpace{}%
\AgdaSymbol{∀}\AgdaSpace{}%
\AgdaSymbol{\{}\AgdaBound{x}\AgdaSymbol{\}}\AgdaSpace{}%
\AgdaSymbol{→}\AgdaSpace{}%
\AgdaDatatype{Acc}\AgdaSpace{}%
\AgdaOperator{\AgdaFunction{\AgdaUnderscore{}<\AgdaUnderscore{}}}\AgdaSpace{}%
\AgdaBound{x}\AgdaSpace{}%
\AgdaSymbol{→}\AgdaSpace{}%
\AgdaFunction{WfRec}\AgdaSpace{}%
\AgdaOperator{\AgdaFunction{\AgdaUnderscore{}<\AgdaUnderscore{}}}\AgdaSpace{}%
\AgdaSymbol{(}\AgdaDatatype{Acc}\AgdaSpace{}%
\AgdaOperator{\AgdaFunction{\AgdaUnderscore{}<\AgdaUnderscore{}}}\AgdaSymbol{)}\AgdaSpace{}%
\AgdaBound{x}\<%
\\
%
\>[4]\AgdaFunction{access}\AgdaSpace{}%
\AgdaSymbol{(}\AgdaInductiveConstructor{acc}\AgdaSpace{}%
\AgdaBound{r}\AgdaSymbol{)}\AgdaSpace{}%
\AgdaSymbol{=}\AgdaSpace{}%
\AgdaBound{r}\<%
\\
%
\\[\AgdaEmptyExtraSkip]%
%
\>[4]\AgdaFunction{smaller-accessible}\AgdaSpace{}%
\AgdaSymbol{:}\AgdaSpace{}%
\AgdaSymbol{(}\AgdaBound{x}\AgdaSpace{}%
\AgdaSymbol{:}\AgdaSpace{}%
\AgdaDatatype{Tree}\AgdaSymbol{)}\<%
\\
\>[4][@{}l@{\AgdaIndent{0}}]%
\>[6]\AgdaSymbol{→}\AgdaSpace{}%
\AgdaDatatype{Acc}\AgdaSpace{}%
\AgdaOperator{\AgdaFunction{\AgdaUnderscore{}<\AgdaUnderscore{}}}\AgdaSpace{}%
\AgdaBound{x}\AgdaSpace{}%
\AgdaSymbol{→}\AgdaSpace{}%
\AgdaSymbol{∀}\AgdaSpace{}%
\AgdaBound{y}\AgdaSpace{}%
\AgdaSymbol{→}\AgdaSpace{}%
\AgdaBound{y}\AgdaSpace{}%
\AgdaOperator{\AgdaDatatype{≤}}\AgdaSpace{}%
\AgdaBound{x}\AgdaSpace{}%
\AgdaSymbol{→}\AgdaSpace{}%
\AgdaDatatype{Acc}\AgdaSpace{}%
\AgdaOperator{\AgdaFunction{\AgdaUnderscore{}<\AgdaUnderscore{}}}\AgdaSpace{}%
\AgdaBound{y}\<%
\\
%
\>[4]\AgdaFunction{smaller-accessible}\AgdaSpace{}%
\AgdaBound{x}\AgdaSpace{}%
\AgdaBound{isAcc}\AgdaSpace{}%
\AgdaBound{y}\AgdaSpace{}%
\AgdaBound{x<y}\<%
\\
\>[4][@{}l@{\AgdaIndent{0}}]%
\>[6]\AgdaSymbol{=}\AgdaSpace{}%
\AgdaInductiveConstructor{acc}\AgdaSpace{}%
\AgdaSymbol{(λ}\AgdaSpace{}%
\AgdaBound{y'}\AgdaSpace{}%
\AgdaBound{y'<y}\AgdaSpace{}%
\AgdaSymbol{→}\AgdaSpace{}%
\AgdaFunction{access}\AgdaSpace{}%
\AgdaBound{isAcc}\AgdaSpace{}%
\AgdaBound{y'}\AgdaSpace{}%
\AgdaSymbol{(}\AgdaFunction{<∘≤-in-<}\AgdaSpace{}%
\AgdaBound{y'<y}\AgdaSpace{}%
\AgdaBound{x<y}\AgdaSymbol{))}\<%
\\
%
\\[\AgdaEmptyExtraSkip]%
%
\>[4]\AgdaFunction{ordWF}\AgdaSpace{}%
\AgdaSymbol{:}\AgdaSpace{}%
\AgdaFunction{WellFounded}\AgdaSpace{}%
\AgdaOperator{\AgdaFunction{\AgdaUnderscore{}<\AgdaUnderscore{}}}\<%
\\
%
\>[4]\AgdaFunction{ordWF}\AgdaSpace{}%
\AgdaInductiveConstructor{Z}\AgdaSpace{}%
\AgdaSymbol{=}\AgdaSpace{}%
\AgdaInductiveConstructor{acc}\AgdaSpace{}%
\AgdaSymbol{λ}\AgdaSpace{}%
\AgdaBound{\AgdaUnderscore{}}\AgdaSpace{}%
\AgdaSymbol{()}\<%
\\
%
\>[4]\AgdaFunction{ordWF}\AgdaSpace{}%
\AgdaSymbol{(}\AgdaInductiveConstructor{↑}\AgdaSpace{}%
\AgdaBound{x}\AgdaSymbol{)}\<%
\\
\>[4][@{}l@{\AgdaIndent{0}}]%
\>[6]\AgdaSymbol{=}%
\>[566I]\AgdaInductiveConstructor{acc}\AgdaSpace{}%
\AgdaSymbol{(λ}\AgdaSpace{}%
\AgdaSymbol{\{}\AgdaSpace{}%
\AgdaBound{y}\AgdaSpace{}%
\AgdaSymbol{(}\AgdaInductiveConstructor{≤-sucMono}\AgdaSpace{}%
\AgdaBound{y≤x}\AgdaSymbol{)}\<%
\\
\>[.][@{}l@{}]\<[566I]%
\>[8]\AgdaSymbol{→}\AgdaSpace{}%
\AgdaFunction{smaller-accessible}\AgdaSpace{}%
\AgdaBound{x}\AgdaSpace{}%
\AgdaSymbol{(}\AgdaFunction{ordWF}\AgdaSpace{}%
\AgdaBound{x}\AgdaSymbol{)}\AgdaSpace{}%
\AgdaBound{y}\AgdaSpace{}%
\AgdaBound{y≤x}\AgdaSymbol{\})}\<%
\\
%
\>[4]\AgdaFunction{ordWF}\AgdaSpace{}%
\AgdaSymbol{(}\AgdaInductiveConstructor{Lim}\AgdaSpace{}%
\AgdaBound{c}\AgdaSpace{}%
\AgdaBound{f}\AgdaSymbol{)}\AgdaSpace{}%
\AgdaSymbol{=}\AgdaSpace{}%
\AgdaInductiveConstructor{acc}\AgdaSpace{}%
\AgdaFunction{helper}\<%
\\
\>[4][@{}l@{\AgdaIndent{0}}]%
\>[6]\AgdaKeyword{where}\<%
\\
\>[6][@{}l@{\AgdaIndent{0}}]%
\>[8]\AgdaFunction{helper}\AgdaSpace{}%
\AgdaSymbol{:}\AgdaSpace{}%
\AgdaSymbol{(}\AgdaBound{y}\AgdaSpace{}%
\AgdaSymbol{:}\AgdaSpace{}%
\AgdaDatatype{Tree}\AgdaSymbol{)}\AgdaSpace{}%
\AgdaSymbol{→}\AgdaSpace{}%
\AgdaSymbol{(}\AgdaBound{y}\AgdaSpace{}%
\AgdaOperator{\AgdaFunction{<}}\AgdaSpace{}%
\AgdaInductiveConstructor{Lim}\AgdaSpace{}%
\AgdaBound{c}\AgdaSpace{}%
\AgdaBound{f}\AgdaSymbol{)}\<%
\\
\>[8][@{}l@{\AgdaIndent{0}}]%
\>[10]\AgdaSymbol{→}\AgdaSpace{}%
\AgdaDatatype{Acc}\AgdaSpace{}%
\AgdaOperator{\AgdaFunction{\AgdaUnderscore{}<\AgdaUnderscore{}}}\AgdaSpace{}%
\AgdaBound{y}\<%
\\
%
\>[8]\AgdaFunction{helper}\AgdaSpace{}%
\AgdaBound{y}\AgdaSpace{}%
\AgdaSymbol{(}\AgdaInductiveConstructor{≤-cocone}\AgdaSpace{}%
\AgdaDottedPattern{\AgdaSymbol{.}}\AgdaDottedPattern{\AgdaBound{f}}\AgdaSpace{}%
\AgdaBound{k}\AgdaSpace{}%
\AgdaBound{y<fk}\AgdaSymbol{)}\<%
\\
\>[8][@{}l@{\AgdaIndent{0}}]%
\>[10]\AgdaSymbol{=}%
\>[602I]\AgdaFunction{smaller-accessible}\AgdaSpace{}%
\AgdaSymbol{(}\AgdaBound{f}\AgdaSpace{}%
\AgdaBound{k}\AgdaSymbol{)}\<%
\\
\>[.][@{}l@{}]\<[602I]%
\>[12]\AgdaSymbol{(}\AgdaFunction{ordWF}\AgdaSpace{}%
\AgdaSymbol{(}\AgdaBound{f}\AgdaSpace{}%
\AgdaBound{k}\AgdaSymbol{))}\AgdaSpace{}%
\AgdaBound{y}\AgdaSpace{}%
\AgdaSymbol{(}\AgdaFunction{<-in-≤}\AgdaSpace{}%
\AgdaBound{y<fk}\AgdaSymbol{)}\<%
\\
\>[0]\<%
\end{code}




\section{First Attempts at a Join}
  

\subsection{Limit-based Maximum}

Since the limit constructor finds the least upper bound
of the image of a function, it should be possible to define
the maximum of two trees as a special case of general limits.
Indeed, we can compute the maximum of $t_1$ and $t_2$ as the limit
of the function that produces $t_1$ when given $0$ and $t_2$ otherwise.

\begin{code}[hide]%
%
\>[2]\AgdaKeyword{open}\AgdaSpace{}%
\AgdaKeyword{import}\AgdaSpace{}%
\AgdaModule{Data.Nat}\AgdaSpace{}%
\AgdaKeyword{hiding}\AgdaSpace{}%
\AgdaSymbol{(}\AgdaOperator{\AgdaDatatype{\AgdaUnderscore{}≤\AgdaUnderscore{}}}\AgdaSpace{}%
\AgdaSymbol{;}\AgdaSpace{}%
\AgdaOperator{\AgdaFunction{\AgdaUnderscore{}<\AgdaUnderscore{}}}\AgdaSymbol{)}\<%
\\
%
\>[2]\AgdaKeyword{open}\AgdaSpace{}%
\AgdaKeyword{import}\AgdaSpace{}%
\AgdaModule{Relation.Binary.PropositionalEquality}\<%
\\
%
\>[2]\AgdaKeyword{open}\AgdaSpace{}%
\AgdaKeyword{import}\AgdaSpace{}%
\AgdaModule{Data.Product}\<%
\\
%
\>[2]\AgdaKeyword{open}\AgdaSpace{}%
\AgdaKeyword{import}\AgdaSpace{}%
\AgdaModule{Relation.Nullary}\<%
\\
%
\>[2]\AgdaKeyword{open}\AgdaSpace{}%
\AgdaKeyword{import}\AgdaSpace{}%
\AgdaModule{Iso}\<%
\\
%
\>[2]\AgdaKeyword{module}\AgdaSpace{}%
\AgdaModule{LimMax}\AgdaSpace{}%
\AgdaSymbol{\{}\AgdaBound{ℓ}\AgdaSymbol{\}}\<%
\\
\>[2][@{}l@{\AgdaIndent{0}}]%
\>[4]\AgdaSymbol{(}\AgdaBound{ℂ}\AgdaSpace{}%
\AgdaSymbol{:}\AgdaSpace{}%
\AgdaPrimitive{Set}\AgdaSpace{}%
\AgdaBound{ℓ}\AgdaSymbol{)}\<%
\\
%
\>[4]\AgdaSymbol{(}\AgdaBound{El}\AgdaSpace{}%
\AgdaSymbol{:}\AgdaSpace{}%
\AgdaBound{ℂ}\AgdaSpace{}%
\AgdaSymbol{→}\AgdaSpace{}%
\AgdaPrimitive{Set}\AgdaSpace{}%
\AgdaBound{ℓ}\AgdaSymbol{)}\<%
\\
%
\>[4]\AgdaSymbol{(}\AgdaBound{Cℕ}\AgdaSpace{}%
\AgdaSymbol{:}\AgdaSpace{}%
\AgdaBound{ℂ}\AgdaSymbol{)}\AgdaSpace{}%
\AgdaSymbol{(}\AgdaBound{CℕIso}\AgdaSpace{}%
\AgdaSymbol{:}\AgdaSpace{}%
\AgdaRecord{Iso}\AgdaSpace{}%
\AgdaSymbol{(}\AgdaBound{El}\AgdaSpace{}%
\AgdaBound{Cℕ}\AgdaSymbol{)}\AgdaSpace{}%
\AgdaDatatype{ℕ}\AgdaSpace{}%
\AgdaSymbol{)}\AgdaSpace{}%
\AgdaKeyword{where}\<%
\\
%
\>[4]\AgdaKeyword{open}\AgdaSpace{}%
\AgdaKeyword{import}\AgdaSpace{}%
\AgdaModule{RawTree}\AgdaSpace{}%
\AgdaBound{ℂ}\AgdaSpace{}%
\AgdaBound{El}\AgdaSpace{}%
\AgdaBound{Cℕ}\AgdaSpace{}%
\AgdaBound{CℕIso}\<%
\end{code}

\begin{code}%
%
\>[4]\AgdaFunction{limMax}\AgdaSpace{}%
\AgdaSymbol{:}\AgdaSpace{}%
\AgdaPostulate{Tree}\AgdaSpace{}%
\AgdaSymbol{→}\AgdaSpace{}%
\AgdaPostulate{Tree}\AgdaSpace{}%
\AgdaSymbol{→}\AgdaSpace{}%
\AgdaPostulate{Tree}\<%
\\
%
\>[4]\AgdaFunction{limMax}\AgdaSpace{}%
\AgdaBound{t1}\AgdaSpace{}%
\AgdaBound{t2}\AgdaSpace{}%
\AgdaSymbol{=}\AgdaSpace{}%
\AgdaPostulate{ℕLim}\AgdaSpace{}%
\AgdaSymbol{λ}\AgdaSpace{}%
\AgdaBound{n}\AgdaSpace{}%
\AgdaSymbol{→}\AgdaSpace{}%
\AgdaFunction{if0}\AgdaSpace{}%
\AgdaBound{n}\AgdaSpace{}%
\AgdaBound{t1}\AgdaSpace{}%
\AgdaBound{t2}\<%
\end{code}

This version of the maximum has several of the properties we want from a
maximum function: it is monotone, idempotent,
commutative, and is a true least-upper-bound of its inputs.

\begin{code}%
%
\>[4]\AgdaFunction{limMax≤L}\AgdaSpace{}%
\AgdaSymbol{:}\AgdaSpace{}%
\AgdaSymbol{∀}\AgdaSpace{}%
\AgdaSymbol{\{}\AgdaBound{t1}\AgdaSpace{}%
\AgdaBound{t2}\AgdaSymbol{\}}\AgdaSpace{}%
\AgdaSymbol{→}\AgdaSpace{}%
\AgdaBound{t1}\AgdaSpace{}%
\AgdaOperator{\AgdaPostulate{≤}}\AgdaSpace{}%
\AgdaFunction{limMax}\AgdaSpace{}%
\AgdaBound{t1}\AgdaSpace{}%
\AgdaBound{t2}\<%
\\
%
\>[4]\AgdaFunction{limMax≤L}\AgdaSpace{}%
\AgdaSymbol{\{}\AgdaBound{t1}\AgdaSymbol{\}}\AgdaSpace{}%
\AgdaSymbol{\{}\AgdaBound{t2}\AgdaSymbol{\}}\<%
\\
\>[4][@{}l@{\AgdaIndent{0}}]%
\>[8]\AgdaSymbol{=}%
\>[69I]\AgdaPostulate{≤-cocone}\AgdaSpace{}%
\AgdaSymbol{\AgdaUnderscore{}}\AgdaSpace{}%
\AgdaSymbol{(}\AgdaField{Iso.inv}\AgdaSpace{}%
\AgdaBound{CℕIso}\AgdaSpace{}%
\AgdaNumber{0}\AgdaSymbol{)}\<%
\\
\>[.][@{}l@{}]\<[69I]%
\>[10]\AgdaSymbol{(}\AgdaFunction{subst}\<%
\\
\>[10][@{}l@{\AgdaIndent{0}}]%
\>[12]\AgdaSymbol{(λ}\AgdaSpace{}%
\AgdaBound{x}\AgdaSpace{}%
\AgdaSymbol{→}\AgdaSpace{}%
\AgdaBound{t1}\AgdaSpace{}%
\AgdaOperator{\AgdaPostulate{≤}}\AgdaSpace{}%
\AgdaFunction{if0}\AgdaSpace{}%
\AgdaBound{x}\AgdaSpace{}%
\AgdaBound{t1}\AgdaSpace{}%
\AgdaBound{t2}\AgdaSymbol{)}\<%
\\
%
\>[12]\AgdaSymbol{(}\AgdaFunction{sym}\AgdaSpace{}%
\AgdaSymbol{(}\AgdaField{Iso.rightInv}\AgdaSpace{}%
\AgdaBound{CℕIso}\AgdaSpace{}%
\AgdaNumber{0}\AgdaSymbol{))}\<%
\\
%
\>[12]\AgdaSymbol{(}\AgdaPostulate{≤-refl}\AgdaSpace{}%
\AgdaBound{t1}\AgdaSymbol{))}\<%
\\
\>[0]\<%
\end{code}

\begin{code}%
\>[0][@{}l@{\AgdaIndent{1}}]%
\>[4]\AgdaFunction{limMax≤R}\AgdaSpace{}%
\AgdaSymbol{:}\AgdaSpace{}%
\AgdaSymbol{∀}\AgdaSpace{}%
\AgdaSymbol{\{}\AgdaBound{t1}\AgdaSpace{}%
\AgdaBound{t2}\AgdaSymbol{\}}\AgdaSpace{}%
\AgdaSymbol{→}\AgdaSpace{}%
\AgdaBound{t2}\AgdaSpace{}%
\AgdaOperator{\AgdaPostulate{≤}}\AgdaSpace{}%
\AgdaFunction{limMax}\AgdaSpace{}%
\AgdaBound{t1}\AgdaSpace{}%
\AgdaBound{t2}\<%
\\
%
\>[4]\AgdaFunction{limMax≤R}\AgdaSpace{}%
\AgdaSymbol{\{}\AgdaBound{t1}\AgdaSymbol{\}}\AgdaSpace{}%
\AgdaSymbol{\{}\AgdaBound{t2}\AgdaSymbol{\}}\<%
\\
\>[4][@{}l@{\AgdaIndent{0}}]%
\>[8]\AgdaSymbol{=}%
\>[98I]\AgdaPostulate{≤-cocone}\AgdaSpace{}%
\AgdaSymbol{\AgdaUnderscore{}}\AgdaSpace{}%
\AgdaSymbol{(}\AgdaField{Iso.inv}\AgdaSpace{}%
\AgdaBound{CℕIso}\AgdaSpace{}%
\AgdaNumber{1}\AgdaSymbol{)}\<%
\\
\>[.][@{}l@{}]\<[98I]%
\>[10]\AgdaSymbol{(}\AgdaFunction{subst}\<%
\\
\>[10][@{}l@{\AgdaIndent{0}}]%
\>[12]\AgdaSymbol{(λ}\AgdaSpace{}%
\AgdaBound{x}\AgdaSpace{}%
\AgdaSymbol{→}\AgdaSpace{}%
\AgdaBound{t2}\AgdaSpace{}%
\AgdaOperator{\AgdaPostulate{≤}}\AgdaSpace{}%
\AgdaFunction{if0}\AgdaSpace{}%
\AgdaBound{x}\AgdaSpace{}%
\AgdaBound{t1}\AgdaSpace{}%
\AgdaBound{t2}\AgdaSymbol{)}\<%
\\
%
\>[12]\AgdaSymbol{(}\AgdaFunction{sym}\AgdaSpace{}%
\AgdaSymbol{(}\AgdaField{Iso.rightInv}\AgdaSpace{}%
\AgdaBound{CℕIso}\AgdaSpace{}%
\AgdaNumber{1}\AgdaSymbol{))}\<%
\\
%
\>[12]\AgdaSymbol{(}\AgdaPostulate{≤-refl}\AgdaSpace{}%
\AgdaBound{t2}\AgdaSymbol{))}\<%
\\
\>[0]\<%
\end{code}

\begin{code}%
\>[0][@{}l@{\AgdaIndent{1}}]%
\>[4]\AgdaFunction{limMaxIdem}\AgdaSpace{}%
\AgdaSymbol{:}\AgdaSpace{}%
\AgdaSymbol{∀}\AgdaSpace{}%
\AgdaSymbol{\{}\AgdaBound{t}\AgdaSymbol{\}}\AgdaSpace{}%
\AgdaSymbol{→}\AgdaSpace{}%
\AgdaFunction{limMax}\AgdaSpace{}%
\AgdaBound{t}\AgdaSpace{}%
\AgdaBound{t}\AgdaSpace{}%
\AgdaOperator{\AgdaPostulate{≤}}\AgdaSpace{}%
\AgdaBound{t}\<%
\\
%
\>[4]\AgdaFunction{limMaxIdem}\AgdaSpace{}%
\AgdaSymbol{\{}\AgdaBound{t}\AgdaSymbol{\}}\AgdaSpace{}%
\AgdaSymbol{=}\AgdaSpace{}%
\AgdaPostulate{≤-limiting}\AgdaSpace{}%
\AgdaSymbol{\AgdaUnderscore{}}\AgdaSpace{}%
\AgdaFunction{helper}\<%
\\
\>[4][@{}l@{\AgdaIndent{0}}]%
\>[6]\AgdaKeyword{where}\<%
\\
\>[6][@{}l@{\AgdaIndent{0}}]%
\>[8]\AgdaFunction{helper}\AgdaSpace{}%
\AgdaSymbol{:}\AgdaSpace{}%
\AgdaSymbol{∀}\AgdaSpace{}%
\AgdaBound{k}\AgdaSpace{}%
\AgdaSymbol{→}\AgdaSpace{}%
\AgdaFunction{if0}\AgdaSpace{}%
\AgdaSymbol{(}\AgdaField{Iso.fun}\AgdaSpace{}%
\AgdaBound{CℕIso}\AgdaSpace{}%
\AgdaBound{k}\AgdaSymbol{)}\AgdaSpace{}%
\AgdaBound{t}\AgdaSpace{}%
\AgdaBound{t}\AgdaSpace{}%
\AgdaOperator{\AgdaPostulate{≤}}\AgdaSpace{}%
\AgdaBound{t}\<%
\\
%
\>[8]\AgdaFunction{helper}\AgdaSpace{}%
\AgdaBound{k}\AgdaSpace{}%
\AgdaKeyword{with}\AgdaSpace{}%
\AgdaField{Iso.fun}\AgdaSpace{}%
\AgdaBound{CℕIso}\AgdaSpace{}%
\AgdaBound{k}\<%
\\
%
\>[8]\AgdaSymbol{...}\AgdaSpace{}%
\AgdaSymbol{|}\AgdaSpace{}%
\AgdaInductiveConstructor{zero}\AgdaSpace{}%
\AgdaSymbol{=}\AgdaSpace{}%
\AgdaPostulate{≤-refl}\AgdaSpace{}%
\AgdaBound{t}\<%
\\
%
\>[8]\AgdaSymbol{...}\AgdaSpace{}%
\AgdaSymbol{|}\AgdaSpace{}%
\AgdaInductiveConstructor{suc}\AgdaSpace{}%
\AgdaBound{n}\AgdaSpace{}%
\AgdaSymbol{=}\AgdaSpace{}%
\AgdaPostulate{≤-refl}\AgdaSpace{}%
\AgdaBound{t}\<%
\end{code}

\begin{code}%
%
\>[4]\AgdaFunction{limMaxMono}\AgdaSpace{}%
\AgdaSymbol{:}\AgdaSpace{}%
\AgdaSymbol{∀}\AgdaSpace{}%
\AgdaSymbol{\{}\AgdaBound{t1}\AgdaSpace{}%
\AgdaBound{t2}\AgdaSpace{}%
\AgdaBound{t1'}\AgdaSpace{}%
\AgdaBound{t2'}\AgdaSymbol{\}}\<%
\\
\>[4][@{}l@{\AgdaIndent{0}}]%
\>[8]\AgdaSymbol{→}\AgdaSpace{}%
\AgdaBound{t1}\AgdaSpace{}%
\AgdaOperator{\AgdaPostulate{≤}}\AgdaSpace{}%
\AgdaBound{t1'}\AgdaSpace{}%
\AgdaSymbol{→}\AgdaSpace{}%
\AgdaBound{t2}\AgdaSpace{}%
\AgdaOperator{\AgdaPostulate{≤}}\AgdaSpace{}%
\AgdaBound{t2'}\<%
\\
%
\>[8]\AgdaSymbol{→}\AgdaSpace{}%
\AgdaFunction{limMax}\AgdaSpace{}%
\AgdaBound{t1}\AgdaSpace{}%
\AgdaBound{t2}\AgdaSpace{}%
\AgdaOperator{\AgdaPostulate{≤}}\AgdaSpace{}%
\AgdaFunction{limMax}\AgdaSpace{}%
\AgdaBound{t1'}\AgdaSpace{}%
\AgdaBound{t2'}\<%
\\
%
\>[4]\AgdaFunction{limMaxMono}\AgdaSpace{}%
\AgdaSymbol{\{}\AgdaBound{t1}\AgdaSymbol{\}}\AgdaSpace{}%
\AgdaSymbol{\{}\AgdaBound{t2}\AgdaSymbol{\}}\AgdaSpace{}%
\AgdaSymbol{\{}\AgdaBound{t1'}\AgdaSymbol{\}}\AgdaSpace{}%
\AgdaSymbol{\{}\AgdaBound{t2'}\AgdaSymbol{\}}\AgdaSpace{}%
\AgdaBound{lt1}\AgdaSpace{}%
\AgdaBound{lt2}\AgdaSpace{}%
\AgdaSymbol{=}\AgdaSpace{}%
\AgdaPostulate{extLim}\AgdaSpace{}%
\AgdaSymbol{\AgdaUnderscore{}}\AgdaSpace{}%
\AgdaSymbol{\AgdaUnderscore{}}\AgdaSpace{}%
\AgdaFunction{helper}\<%
\\
\>[4][@{}l@{\AgdaIndent{0}}]%
\>[6]\AgdaKeyword{where}\<%
\\
\>[6][@{}l@{\AgdaIndent{0}}]%
\>[8]\AgdaFunction{helper}\AgdaSpace{}%
\AgdaSymbol{:}\AgdaSpace{}%
\AgdaSymbol{∀}\AgdaSpace{}%
\AgdaBound{k}\AgdaSpace{}%
\AgdaSymbol{→}\<%
\\
\>[8][@{}l@{\AgdaIndent{0}}]%
\>[10]\AgdaFunction{if0}\AgdaSpace{}%
\AgdaSymbol{(}\AgdaField{Iso.fun}\AgdaSpace{}%
\AgdaBound{CℕIso}\AgdaSpace{}%
\AgdaBound{k}\AgdaSymbol{)}\AgdaSpace{}%
\AgdaBound{t1}\AgdaSpace{}%
\AgdaBound{t2}\<%
\\
\>[10][@{}l@{\AgdaIndent{0}}]%
\>[12]\AgdaOperator{\AgdaPostulate{≤}}\AgdaSpace{}%
\AgdaFunction{if0}\AgdaSpace{}%
\AgdaSymbol{(}\AgdaField{Iso.fun}\AgdaSpace{}%
\AgdaBound{CℕIso}\AgdaSpace{}%
\AgdaBound{k}\AgdaSymbol{)}\AgdaSpace{}%
\AgdaBound{t1'}\AgdaSpace{}%
\AgdaBound{t2'}\<%
\\
%
\>[8]\AgdaFunction{helper}\AgdaSpace{}%
\AgdaBound{k}\AgdaSpace{}%
\AgdaKeyword{with}\AgdaSpace{}%
\AgdaField{Iso.fun}\AgdaSpace{}%
\AgdaBound{CℕIso}\AgdaSpace{}%
\AgdaBound{k}\<%
\\
%
\>[8]\AgdaSymbol{...}\AgdaSpace{}%
\AgdaSymbol{|}\AgdaSpace{}%
\AgdaInductiveConstructor{zero}\AgdaSpace{}%
\AgdaSymbol{=}\AgdaSpace{}%
\AgdaBound{lt1}\<%
\\
%
\>[8]\AgdaSymbol{...}\AgdaSpace{}%
\AgdaSymbol{|}\AgdaSpace{}%
\AgdaInductiveConstructor{suc}\AgdaSpace{}%
\AgdaBound{n}\AgdaSpace{}%
\AgdaSymbol{=}\AgdaSpace{}%
\AgdaBound{lt2}\<%
\\
%
\\[\AgdaEmptyExtraSkip]%
%
\\[\AgdaEmptyExtraSkip]%
%
\>[4]\AgdaFunction{limMaxLUB}\AgdaSpace{}%
\AgdaSymbol{:}\AgdaSpace{}%
\AgdaSymbol{∀}\AgdaSpace{}%
\AgdaSymbol{\{}\AgdaBound{t1}\AgdaSpace{}%
\AgdaBound{t2}\AgdaSpace{}%
\AgdaBound{t}\AgdaSymbol{\}}\AgdaSpace{}%
\AgdaSymbol{→}\AgdaSpace{}%
\AgdaBound{t1}\AgdaSpace{}%
\AgdaOperator{\AgdaPostulate{≤}}\AgdaSpace{}%
\AgdaBound{t}\AgdaSpace{}%
\AgdaSymbol{→}\AgdaSpace{}%
\AgdaBound{t2}\AgdaSpace{}%
\AgdaOperator{\AgdaPostulate{≤}}\AgdaSpace{}%
\AgdaBound{t}\AgdaSpace{}%
\AgdaSymbol{→}\AgdaSpace{}%
\AgdaFunction{limMax}\AgdaSpace{}%
\AgdaBound{t1}\AgdaSpace{}%
\AgdaBound{t2}\AgdaSpace{}%
\AgdaOperator{\AgdaPostulate{≤}}\AgdaSpace{}%
\AgdaBound{t}\<%
\\
%
\>[4]\AgdaFunction{limMaxLUB}\AgdaSpace{}%
\AgdaBound{lt1}\AgdaSpace{}%
\AgdaBound{lt2}\AgdaSpace{}%
\AgdaSymbol{=}\AgdaSpace{}%
\AgdaFunction{limMaxMono}\AgdaSpace{}%
\AgdaBound{lt1}\AgdaSpace{}%
\AgdaBound{lt2}\AgdaSpace{}%
\AgdaOperator{\AgdaPostulate{≤⨟}}\AgdaSpace{}%
\AgdaFunction{limMaxIdem}\<%
\end{code}

  \begin{code}%
%
\>[4]\AgdaFunction{limMaxCommut}\AgdaSpace{}%
\AgdaSymbol{:}\AgdaSpace{}%
\AgdaSymbol{∀}\AgdaSpace{}%
\AgdaSymbol{\{}\AgdaBound{t1}\AgdaSpace{}%
\AgdaBound{t2}\AgdaSymbol{\}}\AgdaSpace{}%
\AgdaSymbol{→}\AgdaSpace{}%
\AgdaFunction{limMax}\AgdaSpace{}%
\AgdaBound{t1}\AgdaSpace{}%
\AgdaBound{t2}\AgdaSpace{}%
\AgdaOperator{\AgdaPostulate{≤}}\AgdaSpace{}%
\AgdaFunction{limMax}\AgdaSpace{}%
\AgdaBound{t2}\AgdaSpace{}%
\AgdaBound{t1}\<%
\\
%
\>[4]\AgdaFunction{limMaxCommut}\AgdaSpace{}%
\AgdaSymbol{=}\AgdaSpace{}%
\AgdaFunction{limMaxLUB}\AgdaSpace{}%
\AgdaFunction{limMax≤R}\AgdaSpace{}%
\AgdaFunction{limMax≤L}\<%
\end{code}

  \subsubsection{Limitation: Strict Monotonicity}

The one crucial property that this formulation lacks is that it is not
strictly monotone: we cannot deduce $\max\ t_1\ t_1 < \max\ t'_1 \ t'_2 $
from $t_1 < t'_1$ and $t_2 < t'_2$. This is because the only way to construct a
proof that $\up t \le \Lim\ c\ f$
is using the $\cocone$ constructor. So we would need to prove that
$\up (\max\ t_{1} \ t_{2}) \le t'_{1}$ or that
$\up (\max\ t_{1} \ t_{2}) \le t'_{2}$, which cannot be deduced from the
premises alone.
%
What we want is to have $\up \max\ (t_{1}) \ t_{2} \le \max (\up t_{1})\ (\up t_{2})$, so that strict monotonicity is a direct consequence of ordinary
monotonicity of the maximum. This is not possible when defining the constructor as a limit.

  
% !TEX root =  main.tex



\subsection{Recursive Maximum}


\begin{code}[hide]%
%
\>[2]\AgdaKeyword{open}\AgdaSpace{}%
\AgdaKeyword{import}\AgdaSpace{}%
\AgdaModule{Data.Nat}\AgdaSpace{}%
\AgdaKeyword{hiding}\AgdaSpace{}%
\AgdaSymbol{(}\AgdaOperator{\AgdaDatatype{\AgdaUnderscore{}≤\AgdaUnderscore{}}}\AgdaSpace{}%
\AgdaSymbol{;}\AgdaSpace{}%
\AgdaOperator{\AgdaFunction{\AgdaUnderscore{}<\AgdaUnderscore{}}}\AgdaSymbol{)}\<%
\\
%
\>[2]\AgdaKeyword{open}\AgdaSpace{}%
\AgdaKeyword{import}\AgdaSpace{}%
\AgdaModule{Relation.Binary.PropositionalEquality}\<%
\\
%
\>[2]\AgdaKeyword{open}\AgdaSpace{}%
\AgdaKeyword{import}\AgdaSpace{}%
\AgdaModule{Data.Product}\<%
\\
%
\>[2]\AgdaKeyword{open}\AgdaSpace{}%
\AgdaKeyword{import}\AgdaSpace{}%
\AgdaModule{Relation.Nullary}\<%
\\
%
\>[2]\AgdaKeyword{open}\AgdaSpace{}%
\AgdaKeyword{import}\AgdaSpace{}%
\AgdaModule{Iso}\<%
\\
%
\>[2]\AgdaKeyword{module}\AgdaSpace{}%
\AgdaModule{IndMax}\AgdaSpace{}%
\AgdaSymbol{\{}\AgdaBound{ℓ}\AgdaSymbol{\}}\<%
\\
\>[2][@{}l@{\AgdaIndent{0}}]%
\>[4]\AgdaSymbol{(}\AgdaBound{ℂ}\AgdaSpace{}%
\AgdaSymbol{:}\AgdaSpace{}%
\AgdaPrimitive{Set}\AgdaSpace{}%
\AgdaBound{ℓ}\AgdaSymbol{)}\<%
\\
%
\>[4]\AgdaSymbol{(}\AgdaBound{El}\AgdaSpace{}%
\AgdaSymbol{:}\AgdaSpace{}%
\AgdaBound{ℂ}\AgdaSpace{}%
\AgdaSymbol{→}\AgdaSpace{}%
\AgdaPrimitive{Set}\AgdaSpace{}%
\AgdaBound{ℓ}\AgdaSymbol{)}\<%
\\
%
\>[4]\AgdaSymbol{(}\AgdaBound{Cℕ}\AgdaSpace{}%
\AgdaSymbol{:}\AgdaSpace{}%
\AgdaBound{ℂ}\AgdaSymbol{)}\AgdaSpace{}%
\AgdaSymbol{(}\AgdaBound{CℕIso}\AgdaSpace{}%
\AgdaSymbol{:}\AgdaSpace{}%
\AgdaRecord{Iso}\AgdaSpace{}%
\AgdaSymbol{(}\AgdaBound{El}\AgdaSpace{}%
\AgdaBound{Cℕ}\AgdaSymbol{)}\AgdaSpace{}%
\AgdaDatatype{ℕ}\AgdaSpace{}%
\AgdaSymbol{)}\<%
\\
%
\>[4]\AgdaSymbol{(}\AgdaBound{default}\AgdaSpace{}%
\AgdaSymbol{:}\AgdaSpace{}%
\AgdaSymbol{(}\AgdaBound{c}\AgdaSpace{}%
\AgdaSymbol{:}\AgdaSpace{}%
\AgdaBound{ℂ}\AgdaSymbol{)}\AgdaSpace{}%
\AgdaSymbol{→}\AgdaSpace{}%
\AgdaBound{El}\AgdaSpace{}%
\AgdaBound{c}\AgdaSymbol{)}\AgdaSpace{}%
\AgdaKeyword{where}\<%
\\
%
\>[4]\AgdaKeyword{open}\AgdaSpace{}%
\AgdaKeyword{import}\AgdaSpace{}%
\AgdaModule{RawTree}\AgdaSpace{}%
\AgdaBound{ℂ}\AgdaSpace{}%
\AgdaBound{El}\AgdaSpace{}%
\AgdaBound{Cℕ}\AgdaSpace{}%
\AgdaBound{CℕIso}\<%
\end{code}

\begin{code}%
%
\>[4]\AgdaKeyword{private}\<%
\\
\>[4][@{}l@{\AgdaIndent{0}}]%
\>[8]\AgdaKeyword{data}\AgdaSpace{}%
\AgdaDatatype{IndMaxView}\AgdaSpace{}%
\AgdaSymbol{:}\AgdaSpace{}%
\AgdaPostulate{Tree}\AgdaSpace{}%
\AgdaSymbol{→}\AgdaSpace{}%
\AgdaPostulate{Tree}\AgdaSpace{}%
\AgdaSymbol{→}\AgdaSpace{}%
\AgdaPrimitive{Set}\AgdaSpace{}%
\AgdaBound{ℓ}\AgdaSpace{}%
\AgdaKeyword{where}\<%
\\
\>[8][@{}l@{\AgdaIndent{0}}]%
\>[10]\AgdaInductiveConstructor{IndMaxZ-L}\AgdaSpace{}%
\AgdaSymbol{:}\AgdaSpace{}%
\AgdaSymbol{∀}\AgdaSpace{}%
\AgdaSymbol{\{}\AgdaBound{t}\AgdaSymbol{\}}\AgdaSpace{}%
\AgdaSymbol{→}\AgdaSpace{}%
\AgdaDatatype{IndMaxView}\AgdaSpace{}%
\AgdaPostulate{Z}\AgdaSpace{}%
\AgdaBound{t}\<%
\\
%
\>[10]\AgdaInductiveConstructor{IndMaxZ-R}\AgdaSpace{}%
\AgdaSymbol{:}\AgdaSpace{}%
\AgdaSymbol{∀}\AgdaSpace{}%
\AgdaSymbol{\{}\AgdaBound{t}\AgdaSymbol{\}}\AgdaSpace{}%
\AgdaSymbol{→}\AgdaSpace{}%
\AgdaDatatype{IndMaxView}\AgdaSpace{}%
\AgdaBound{t}\AgdaSpace{}%
\AgdaPostulate{Z}\<%
\\
%
\>[10]\AgdaInductiveConstructor{IndMaxLim-L}\AgdaSpace{}%
\AgdaSymbol{:}\AgdaSpace{}%
\AgdaSymbol{∀}\AgdaSpace{}%
\AgdaSymbol{\{}\AgdaBound{t}\AgdaSymbol{\}}\AgdaSpace{}%
\AgdaSymbol{\{}\AgdaBound{c}\AgdaSpace{}%
\AgdaSymbol{:}\AgdaSpace{}%
\AgdaBound{ℂ}\AgdaSymbol{\}}\AgdaSpace{}%
\AgdaSymbol{\{}\AgdaBound{f}\AgdaSpace{}%
\AgdaSymbol{:}\AgdaSpace{}%
\AgdaBound{El}\AgdaSpace{}%
\AgdaBound{c}\AgdaSpace{}%
\AgdaSymbol{→}\AgdaSpace{}%
\AgdaPostulate{Tree}\AgdaSymbol{\}}\<%
\\
\>[10][@{}l@{\AgdaIndent{0}}]%
\>[12]\AgdaSymbol{→}\AgdaSpace{}%
\AgdaDatatype{IndMaxView}\AgdaSpace{}%
\AgdaSymbol{(}\AgdaPostulate{Lim}\AgdaSpace{}%
\AgdaBound{c}\AgdaSpace{}%
\AgdaBound{f}\AgdaSymbol{)}\AgdaSpace{}%
\AgdaBound{t}\<%
\\
%
\>[10]\AgdaInductiveConstructor{IndMaxLim-R}\AgdaSpace{}%
\AgdaSymbol{:}\AgdaSpace{}%
\AgdaSymbol{∀}\AgdaSpace{}%
\AgdaSymbol{\{}\AgdaBound{t}\AgdaSymbol{\}}\AgdaSpace{}%
\AgdaSymbol{\{}\AgdaBound{c}\AgdaSpace{}%
\AgdaSymbol{:}\AgdaSpace{}%
\AgdaBound{ℂ}\AgdaSymbol{\}}\AgdaSpace{}%
\AgdaSymbol{\{}\AgdaBound{f}\AgdaSpace{}%
\AgdaSymbol{:}\AgdaSpace{}%
\AgdaBound{El}\AgdaSpace{}%
\AgdaBound{c}\AgdaSpace{}%
\AgdaSymbol{→}\AgdaSpace{}%
\AgdaPostulate{Tree}\AgdaSymbol{\}}\<%
\\
\>[10][@{}l@{\AgdaIndent{0}}]%
\>[12]\AgdaSymbol{→}\AgdaSpace{}%
\AgdaSymbol{(∀}%
\>[19]\AgdaSymbol{\{}\AgdaBound{c'}\AgdaSpace{}%
\AgdaSymbol{:}\AgdaSpace{}%
\AgdaBound{ℂ}\AgdaSymbol{\}}\AgdaSpace{}%
\AgdaSymbol{\{}\AgdaBound{f'}\AgdaSpace{}%
\AgdaSymbol{:}\AgdaSpace{}%
\AgdaBound{El}\AgdaSpace{}%
\AgdaBound{c'}\AgdaSpace{}%
\AgdaSymbol{→}\AgdaSpace{}%
\AgdaPostulate{Tree}\AgdaSymbol{\}}\AgdaSpace{}%
\AgdaSymbol{→}\AgdaSpace{}%
\AgdaOperator{\AgdaFunction{¬}}\AgdaSpace{}%
\AgdaSymbol{(}\AgdaBound{t}\AgdaSpace{}%
\AgdaOperator{\AgdaDatatype{≡}}\AgdaSpace{}%
\AgdaPostulate{Lim}%
\>[62]\AgdaBound{c'}\AgdaSpace{}%
\AgdaBound{f'}\AgdaSymbol{))}\<%
\\
%
\>[12]\AgdaSymbol{→}\AgdaSpace{}%
\AgdaDatatype{IndMaxView}\AgdaSpace{}%
\AgdaBound{t}\AgdaSpace{}%
\AgdaSymbol{(}\AgdaPostulate{Lim}\AgdaSpace{}%
\AgdaBound{c}\AgdaSpace{}%
\AgdaBound{f}\AgdaSymbol{)}\<%
\\
%
\>[10]\AgdaInductiveConstructor{IndMaxLim-Suc}\AgdaSpace{}%
\AgdaSymbol{:}\AgdaSpace{}%
\AgdaSymbol{∀}%
\>[29]\AgdaSymbol{\{}\AgdaBound{t1}\AgdaSpace{}%
\AgdaBound{t2}\AgdaSpace{}%
\AgdaSymbol{\}}\AgdaSpace{}%
\AgdaSymbol{→}\AgdaSpace{}%
\AgdaDatatype{IndMaxView}\AgdaSpace{}%
\AgdaSymbol{(}\AgdaPostulate{↑}\AgdaSpace{}%
\AgdaBound{t1}\AgdaSymbol{)}\AgdaSpace{}%
\AgdaSymbol{(}\AgdaPostulate{↑}\AgdaSpace{}%
\AgdaBound{t2}\AgdaSymbol{)}\<%
\\
%
\>[4]\AgdaKeyword{opaque}\<%
\\
%
\\[\AgdaEmptyExtraSkip]%
\>[4][@{}l@{\AgdaIndent{0}}]%
\>[8]\AgdaFunction{indMaxView}\AgdaSpace{}%
\AgdaSymbol{:}\AgdaSpace{}%
\AgdaSymbol{∀}\AgdaSpace{}%
\AgdaBound{t1}\AgdaSpace{}%
\AgdaBound{t2}\AgdaSpace{}%
\AgdaSymbol{→}\AgdaSpace{}%
\AgdaDatatype{IndMaxView}\AgdaSpace{}%
\AgdaBound{t1}\AgdaSpace{}%
\AgdaBound{t2}\<%
\\
%
\>[8]\AgdaFunction{indMaxView}\AgdaSpace{}%
\AgdaInductiveConstructor{Z}\AgdaSpace{}%
\AgdaBound{t2}\AgdaSpace{}%
\AgdaSymbol{=}\AgdaSpace{}%
\AgdaInductiveConstructor{IndMaxZ-L}\<%
\\
%
\>[8]\AgdaFunction{indMaxView}\AgdaSpace{}%
\AgdaSymbol{(}\AgdaInductiveConstructor{Lim}\AgdaSpace{}%
\AgdaBound{c}\AgdaSpace{}%
\AgdaBound{f}\AgdaSymbol{)}\AgdaSpace{}%
\AgdaBound{t2}\AgdaSpace{}%
\AgdaSymbol{=}\AgdaSpace{}%
\AgdaInductiveConstructor{IndMaxLim-L}\<%
\\
%
\>[8]\AgdaFunction{indMaxView}\AgdaSpace{}%
\AgdaSymbol{(}\AgdaInductiveConstructor{↑}\AgdaSpace{}%
\AgdaBound{t1}\AgdaSymbol{)}\AgdaSpace{}%
\AgdaInductiveConstructor{Z}\AgdaSpace{}%
\AgdaSymbol{=}\AgdaSpace{}%
\AgdaInductiveConstructor{IndMaxZ-R}\<%
\\
%
\>[8]\AgdaFunction{indMaxView}\AgdaSpace{}%
\AgdaSymbol{(}\AgdaInductiveConstructor{↑}\AgdaSpace{}%
\AgdaBound{t1}\AgdaSymbol{)}\AgdaSpace{}%
\AgdaSymbol{(}\AgdaInductiveConstructor{Lim}\AgdaSpace{}%
\AgdaBound{c}\AgdaSpace{}%
\AgdaBound{f}\AgdaSymbol{)}\AgdaSpace{}%
\AgdaSymbol{=}\AgdaSpace{}%
\AgdaInductiveConstructor{IndMaxLim-R}\AgdaSpace{}%
\AgdaSymbol{λ}\AgdaSpace{}%
\AgdaSymbol{()}\<%
\\
%
\>[8]\AgdaFunction{indMaxView}\AgdaSpace{}%
\AgdaSymbol{(}\AgdaInductiveConstructor{↑}\AgdaSpace{}%
\AgdaBound{t1}\AgdaSymbol{)}\AgdaSpace{}%
\AgdaSymbol{(}\AgdaInductiveConstructor{↑}\AgdaSpace{}%
\AgdaBound{t2}\AgdaSymbol{)}\AgdaSpace{}%
\AgdaSymbol{=}\AgdaSpace{}%
\AgdaInductiveConstructor{IndMaxLim-Suc}\<%
\\
%
\\[\AgdaEmptyExtraSkip]%
%
\\[\AgdaEmptyExtraSkip]%
%
\>[8]\AgdaFunction{indMax}\AgdaSpace{}%
\AgdaSymbol{:}\AgdaSpace{}%
\AgdaPostulate{Tree}\AgdaSpace{}%
\AgdaSymbol{→}\AgdaSpace{}%
\AgdaPostulate{Tree}\AgdaSpace{}%
\AgdaSymbol{→}\AgdaSpace{}%
\AgdaPostulate{Tree}\<%
\\
%
\>[8]\AgdaFunction{indMax'}\AgdaSpace{}%
\AgdaSymbol{:}\AgdaSpace{}%
\AgdaSymbol{∀}\AgdaSpace{}%
\AgdaSymbol{\{}\AgdaBound{t1}\AgdaSpace{}%
\AgdaBound{t2}\AgdaSymbol{\}}\AgdaSpace{}%
\AgdaSymbol{→}\AgdaSpace{}%
\AgdaDatatype{IndMaxView}\AgdaSpace{}%
\AgdaBound{t1}\AgdaSpace{}%
\AgdaBound{t2}\AgdaSpace{}%
\AgdaSymbol{→}\AgdaSpace{}%
\AgdaPostulate{Tree}\<%
\\
%
\\[\AgdaEmptyExtraSkip]%
%
\>[8]\AgdaFunction{indMax}\AgdaSpace{}%
\AgdaBound{t1}\AgdaSpace{}%
\AgdaBound{t2}\AgdaSpace{}%
\AgdaSymbol{=}\AgdaSpace{}%
\AgdaFunction{indMax'}\AgdaSpace{}%
\AgdaSymbol{(}\AgdaFunction{indMaxView}\AgdaSpace{}%
\AgdaBound{t1}\AgdaSpace{}%
\AgdaBound{t2}\AgdaSymbol{)}\<%
\\
%
\\[\AgdaEmptyExtraSkip]%
%
\>[8]\AgdaFunction{indMax'}\AgdaSpace{}%
\AgdaSymbol{\{}\AgdaDottedPattern{\AgdaSymbol{.}}\AgdaDottedPattern{\AgdaPostulate{Z}}\AgdaSymbol{\}}\AgdaSpace{}%
\AgdaSymbol{\{}\AgdaBound{t2}\AgdaSymbol{\}}\AgdaSpace{}%
\AgdaInductiveConstructor{IndMaxZ-L}\AgdaSpace{}%
\AgdaSymbol{=}\AgdaSpace{}%
\AgdaBound{t2}\<%
\\
%
\>[8]\AgdaFunction{indMax'}\AgdaSpace{}%
\AgdaSymbol{\{}\AgdaBound{t1}\AgdaSymbol{\}}\AgdaSpace{}%
\AgdaSymbol{\{}\AgdaDottedPattern{\AgdaSymbol{.}}\AgdaDottedPattern{\AgdaPostulate{Z}}\AgdaSymbol{\}}\AgdaSpace{}%
\AgdaInductiveConstructor{IndMaxZ-R}\AgdaSpace{}%
\AgdaSymbol{=}\AgdaSpace{}%
\AgdaBound{t1}\<%
\\
%
\>[8]\AgdaFunction{indMax'}\AgdaSpace{}%
\AgdaSymbol{\{(}\AgdaInductiveConstructor{Lim}\AgdaSpace{}%
\AgdaBound{c}\AgdaSpace{}%
\AgdaBound{f}\AgdaSymbol{)\}}\AgdaSpace{}%
\AgdaSymbol{\{}\AgdaBound{t2}\AgdaSymbol{\}}\AgdaSpace{}%
\AgdaInductiveConstructor{IndMaxLim-L}\<%
\\
\>[8][@{}l@{\AgdaIndent{0}}]%
\>[12]\AgdaSymbol{=}\AgdaSpace{}%
\AgdaPostulate{Lim}\AgdaSpace{}%
\AgdaBound{c}\AgdaSpace{}%
\AgdaSymbol{λ}\AgdaSpace{}%
\AgdaBound{x}\AgdaSpace{}%
\AgdaSymbol{→}\AgdaSpace{}%
\AgdaFunction{indMax}\AgdaSpace{}%
\AgdaSymbol{(}\AgdaBound{f}\AgdaSpace{}%
\AgdaBound{x}\AgdaSymbol{)}\AgdaSpace{}%
\AgdaBound{t2}\<%
\\
%
\>[8]\AgdaFunction{indMax'}\AgdaSpace{}%
\AgdaSymbol{\{}\AgdaBound{t1}\AgdaSymbol{\}}\AgdaSpace{}%
\AgdaSymbol{\{(}\AgdaInductiveConstructor{Lim}\AgdaSpace{}%
\AgdaBound{c}\AgdaSpace{}%
\AgdaBound{f}\AgdaSymbol{)\}}\AgdaSpace{}%
\AgdaSymbol{(}\AgdaInductiveConstructor{IndMaxLim-R}\AgdaSpace{}%
\AgdaSymbol{\AgdaUnderscore{})}\<%
\\
\>[8][@{}l@{\AgdaIndent{0}}]%
\>[12]\AgdaSymbol{=}\AgdaSpace{}%
\AgdaPostulate{Lim}\AgdaSpace{}%
\AgdaBound{c}\AgdaSpace{}%
\AgdaSymbol{(λ}\AgdaSpace{}%
\AgdaBound{x}\AgdaSpace{}%
\AgdaSymbol{→}\AgdaSpace{}%
\AgdaFunction{indMax}\AgdaSpace{}%
\AgdaBound{t1}\AgdaSpace{}%
\AgdaSymbol{(}\AgdaBound{f}\AgdaSpace{}%
\AgdaBound{x}\AgdaSymbol{))}\<%
\\
%
\>[8]\AgdaFunction{indMax'}\AgdaSpace{}%
\AgdaSymbol{\{(}\AgdaInductiveConstructor{↑}\AgdaSpace{}%
\AgdaBound{t1}\AgdaSymbol{)\}}\AgdaSpace{}%
\AgdaSymbol{\{(}\AgdaInductiveConstructor{↑}\AgdaSpace{}%
\AgdaBound{t2}\AgdaSymbol{)\}}\AgdaSpace{}%
\AgdaInductiveConstructor{IndMaxLim-Suc}\AgdaSpace{}%
\AgdaSymbol{=}\AgdaSpace{}%
\AgdaPostulate{↑}\AgdaSpace{}%
\AgdaSymbol{(}\AgdaFunction{indMax}\AgdaSpace{}%
\AgdaBound{t1}\AgdaSpace{}%
\AgdaBound{t2}\AgdaSymbol{)}\<%
\\
%
\\[\AgdaEmptyExtraSkip]%
\>[0]\<%
\end{code}

  \begin{code}%
\>[0]\<%
\\
\>[0][@{}l@{\AgdaIndent{1}}]%
\>[4]\AgdaKeyword{module}\AgdaSpace{}%
\AgdaModule{IndMaxInhab}%
\>[24]\AgdaKeyword{where}\<%
\\
%
\\[\AgdaEmptyExtraSkip]%
\>[4][@{}l@{\AgdaIndent{0}}]%
\>[6]\AgdaFunction{underLim}\AgdaSpace{}%
\AgdaSymbol{:}\AgdaSpace{}%
\AgdaSymbol{∀}%
\>[21]\AgdaSymbol{\{}\AgdaBound{c}\AgdaSpace{}%
\AgdaSymbol{:}\AgdaSpace{}%
\AgdaBound{ℂ}\AgdaSymbol{\}}%
\>[30]\AgdaSymbol{\{}\AgdaBound{t}\AgdaSymbol{\}}\AgdaSpace{}%
\AgdaSymbol{→}%
\>[37]\AgdaSymbol{\{}\AgdaBound{f}\AgdaSpace{}%
\AgdaSymbol{:}\AgdaSpace{}%
\AgdaBound{El}\AgdaSpace{}%
\AgdaBound{c}\AgdaSpace{}%
\AgdaSymbol{→}\AgdaSpace{}%
\AgdaPostulate{Tree}\AgdaSymbol{\}}\AgdaSpace{}%
\AgdaSymbol{→}\AgdaSpace{}%
\AgdaSymbol{(∀}\AgdaSpace{}%
\AgdaBound{k}\AgdaSpace{}%
\AgdaSymbol{→}\AgdaSpace{}%
\AgdaBound{t}\AgdaSpace{}%
\AgdaOperator{\AgdaPostulate{≤}}\AgdaSpace{}%
\AgdaBound{f}\AgdaSpace{}%
\AgdaBound{k}\AgdaSymbol{)}\AgdaSpace{}%
\AgdaSymbol{→}\AgdaSpace{}%
\AgdaBound{t}\AgdaSpace{}%
\AgdaOperator{\AgdaPostulate{≤}}\AgdaSpace{}%
\AgdaPostulate{Lim}\AgdaSpace{}%
\AgdaBound{c}\AgdaSpace{}%
\AgdaBound{f}\<%
\\
%
\>[6]\AgdaFunction{underLim}\AgdaSpace{}%
\AgdaSymbol{\{}\AgdaArgument{c}\AgdaSpace{}%
\AgdaSymbol{=}\AgdaSpace{}%
\AgdaBound{c}\AgdaSymbol{\}}%
\>[24]\AgdaSymbol{\{}\AgdaBound{t}\AgdaSymbol{\}}\AgdaSpace{}%
\AgdaSymbol{\{}\AgdaBound{f}\AgdaSymbol{\}}\AgdaSpace{}%
\AgdaBound{all}\AgdaSpace{}%
\AgdaSymbol{=}\AgdaSpace{}%
\AgdaPostulate{≤-trans}\AgdaSpace{}%
\AgdaSymbol{(}\AgdaPostulate{≤-cocone}\AgdaSpace{}%
\AgdaSymbol{(λ}\AgdaSpace{}%
\AgdaBound{\AgdaUnderscore{}}\AgdaSpace{}%
\AgdaSymbol{→}\AgdaSpace{}%
\AgdaBound{t}\AgdaSymbol{)}\AgdaSpace{}%
\AgdaSymbol{(}\AgdaBound{default}\AgdaSpace{}%
\AgdaBound{c}\AgdaSymbol{)}\AgdaSpace{}%
\AgdaSymbol{(}\AgdaPostulate{≤-refl}\AgdaSpace{}%
\AgdaBound{t}\AgdaSymbol{))}\AgdaSpace{}%
\AgdaSymbol{(}\AgdaPostulate{≤-limiting}\AgdaSpace{}%
\AgdaSymbol{(λ}\AgdaSpace{}%
\AgdaBound{\AgdaUnderscore{}}\AgdaSpace{}%
\AgdaSymbol{→}\AgdaSpace{}%
\AgdaBound{t}\AgdaSymbol{)}\AgdaSpace{}%
\AgdaSymbol{(λ}\AgdaSpace{}%
\AgdaBound{k}\AgdaSpace{}%
\AgdaSymbol{→}\AgdaSpace{}%
\AgdaPostulate{≤-cocone}\AgdaSpace{}%
\AgdaBound{f}\AgdaSpace{}%
\AgdaBound{k}\AgdaSpace{}%
\AgdaSymbol{(}\AgdaBound{all}\AgdaSpace{}%
\AgdaBound{k}\AgdaSymbol{)))}\<%
\\
%
\\[\AgdaEmptyExtraSkip]%
%
\>[6]\AgdaKeyword{opaque}\<%
\\
\>[6][@{}l@{\AgdaIndent{0}}]%
\>[8]\AgdaKeyword{unfolding}\AgdaSpace{}%
\AgdaFunction{indMax}\AgdaSpace{}%
\AgdaFunction{indMax'}\<%
\\
%
\\[\AgdaEmptyExtraSkip]%
%
\>[8]\AgdaFunction{indMax-≤L}\AgdaSpace{}%
\AgdaSymbol{:}\AgdaSpace{}%
\AgdaSymbol{∀}\AgdaSpace{}%
\AgdaSymbol{\{}\AgdaBound{t1}\AgdaSpace{}%
\AgdaBound{t2}\AgdaSymbol{\}}\AgdaSpace{}%
\AgdaSymbol{→}\AgdaSpace{}%
\AgdaBound{t1}\AgdaSpace{}%
\AgdaOperator{\AgdaPostulate{≤}}\AgdaSpace{}%
\AgdaFunction{indMax}\AgdaSpace{}%
\AgdaBound{t1}\AgdaSpace{}%
\AgdaBound{t2}\<%
\\
%
\>[8]\AgdaFunction{indMax-≤L}\AgdaSpace{}%
\AgdaSymbol{\{}\AgdaBound{t1}\AgdaSymbol{\}}\AgdaSpace{}%
\AgdaSymbol{\{}\AgdaBound{t2}\AgdaSymbol{\}}\AgdaSpace{}%
\AgdaKeyword{with}\AgdaSpace{}%
\AgdaFunction{indMaxView}\AgdaSpace{}%
\AgdaBound{t1}\AgdaSpace{}%
\AgdaBound{t2}\<%
\\
%
\>[8]\AgdaSymbol{...}\AgdaSpace{}%
\AgdaSymbol{|}\AgdaSpace{}%
\AgdaInductiveConstructor{IndMaxZ-L}\AgdaSpace{}%
\AgdaSymbol{=}\AgdaSpace{}%
\AgdaPostulate{≤-Z}\<%
\\
%
\>[8]\AgdaSymbol{...}\AgdaSpace{}%
\AgdaSymbol{|}\AgdaSpace{}%
\AgdaInductiveConstructor{IndMaxZ-R}\AgdaSpace{}%
\AgdaSymbol{=}\AgdaSpace{}%
\AgdaPostulate{≤-refl}\AgdaSpace{}%
\AgdaSymbol{\AgdaUnderscore{}}\<%
\\
%
\>[8]\AgdaSymbol{...}\AgdaSpace{}%
\AgdaSymbol{|}\AgdaSpace{}%
\AgdaInductiveConstructor{IndMaxLim-L}\AgdaSpace{}%
\AgdaSymbol{\{}\AgdaArgument{f}\AgdaSpace{}%
\AgdaSymbol{=}\AgdaSpace{}%
\AgdaBound{f}\AgdaSymbol{\}}\AgdaSpace{}%
\AgdaSymbol{=}\AgdaSpace{}%
\AgdaPostulate{extLim}\AgdaSpace{}%
\AgdaBound{f}\AgdaSpace{}%
\AgdaSymbol{(λ}\AgdaSpace{}%
\AgdaBound{x}\AgdaSpace{}%
\AgdaSymbol{→}\AgdaSpace{}%
\AgdaFunction{indMax}\AgdaSpace{}%
\AgdaSymbol{(}\AgdaBound{f}\AgdaSpace{}%
\AgdaBound{x}\AgdaSymbol{)}\AgdaSpace{}%
\AgdaBound{t2}\AgdaSymbol{)}\AgdaSpace{}%
\AgdaSymbol{(λ}\AgdaSpace{}%
\AgdaBound{k}\AgdaSpace{}%
\AgdaSymbol{→}\AgdaSpace{}%
\AgdaFunction{indMax-≤L}\AgdaSymbol{)}\<%
\\
%
\>[8]\AgdaSymbol{...}\AgdaSpace{}%
\AgdaSymbol{|}\AgdaSpace{}%
\AgdaInductiveConstructor{IndMaxLim-R}\AgdaSpace{}%
\AgdaSymbol{\{}\AgdaArgument{f}\AgdaSpace{}%
\AgdaSymbol{=}\AgdaSpace{}%
\AgdaBound{f}\AgdaSymbol{\}}\AgdaSpace{}%
\AgdaSymbol{\AgdaUnderscore{}}\AgdaSpace{}%
\AgdaSymbol{=}\AgdaSpace{}%
\AgdaFunction{underLim}%
\>[48]\AgdaSymbol{λ}\AgdaSpace{}%
\AgdaBound{k}\AgdaSpace{}%
\AgdaSymbol{→}\AgdaSpace{}%
\AgdaFunction{indMax-≤L}\AgdaSpace{}%
\AgdaSymbol{\{}\AgdaArgument{t2}\AgdaSpace{}%
\AgdaSymbol{=}\AgdaSpace{}%
\AgdaBound{f}\AgdaSpace{}%
\AgdaBound{k}\AgdaSymbol{\}}\<%
\\
%
\>[8]\AgdaSymbol{...}\AgdaSpace{}%
\AgdaSymbol{|}\AgdaSpace{}%
\AgdaInductiveConstructor{IndMaxLim-Suc}\AgdaSpace{}%
\AgdaSymbol{=}\AgdaSpace{}%
\AgdaPostulate{≤-sucMono}\AgdaSpace{}%
\AgdaFunction{indMax-≤L}\<%
\\
%
\\[\AgdaEmptyExtraSkip]%
%
\\[\AgdaEmptyExtraSkip]%
%
\>[8]\AgdaFunction{indMax-≤R}\AgdaSpace{}%
\AgdaSymbol{:}\AgdaSpace{}%
\AgdaSymbol{∀}\AgdaSpace{}%
\AgdaSymbol{\{}\AgdaBound{t1}\AgdaSpace{}%
\AgdaBound{t2}\AgdaSymbol{\}}\AgdaSpace{}%
\AgdaSymbol{→}\AgdaSpace{}%
\AgdaBound{t2}\AgdaSpace{}%
\AgdaOperator{\AgdaPostulate{≤}}\AgdaSpace{}%
\AgdaFunction{indMax}\AgdaSpace{}%
\AgdaBound{t1}\AgdaSpace{}%
\AgdaBound{t2}\<%
\\
%
\>[8]\AgdaFunction{indMax-≤R}\AgdaSpace{}%
\AgdaSymbol{\{}\AgdaBound{t1}\AgdaSymbol{\}}\AgdaSpace{}%
\AgdaSymbol{\{}\AgdaBound{t2}\AgdaSymbol{\}}\AgdaSpace{}%
\AgdaKeyword{with}\AgdaSpace{}%
\AgdaFunction{indMaxView}\AgdaSpace{}%
\AgdaBound{t1}\AgdaSpace{}%
\AgdaBound{t2}\<%
\\
%
\>[8]\AgdaSymbol{...}\AgdaSpace{}%
\AgdaSymbol{|}\AgdaSpace{}%
\AgdaInductiveConstructor{IndMaxZ-R}\AgdaSpace{}%
\AgdaSymbol{=}\AgdaSpace{}%
\AgdaPostulate{≤-Z}\<%
\\
%
\>[8]\AgdaSymbol{...}\AgdaSpace{}%
\AgdaSymbol{|}\AgdaSpace{}%
\AgdaInductiveConstructor{IndMaxZ-L}\AgdaSpace{}%
\AgdaSymbol{=}\AgdaSpace{}%
\AgdaPostulate{≤-refl}\AgdaSpace{}%
\AgdaSymbol{\AgdaUnderscore{}}\<%
\\
%
\>[8]\AgdaSymbol{...}\AgdaSpace{}%
\AgdaSymbol{|}\AgdaSpace{}%
\AgdaInductiveConstructor{IndMaxLim-R}\AgdaSpace{}%
\AgdaSymbol{\{}\AgdaArgument{f}\AgdaSpace{}%
\AgdaSymbol{=}\AgdaSpace{}%
\AgdaBound{f}\AgdaSymbol{\}}\AgdaSpace{}%
\AgdaSymbol{\AgdaUnderscore{}}\AgdaSpace{}%
\AgdaSymbol{=}\AgdaSpace{}%
\AgdaPostulate{extLim}\AgdaSpace{}%
\AgdaBound{f}\AgdaSpace{}%
\AgdaSymbol{(λ}\AgdaSpace{}%
\AgdaBound{x}\AgdaSpace{}%
\AgdaSymbol{→}\AgdaSpace{}%
\AgdaFunction{indMax}\AgdaSpace{}%
\AgdaBound{t1}\AgdaSpace{}%
\AgdaSymbol{(}\AgdaBound{f}\AgdaSpace{}%
\AgdaBound{x}\AgdaSymbol{))}\AgdaSpace{}%
\AgdaSymbol{(λ}\AgdaSpace{}%
\AgdaBound{k}\AgdaSpace{}%
\AgdaSymbol{→}\AgdaSpace{}%
\AgdaFunction{indMax-≤R}\AgdaSpace{}%
\AgdaSymbol{\{}\AgdaArgument{t1}\AgdaSpace{}%
\AgdaSymbol{=}\AgdaSpace{}%
\AgdaBound{t1}\AgdaSymbol{\}}\AgdaSpace{}%
\AgdaSymbol{\{}\AgdaBound{f}\AgdaSpace{}%
\AgdaBound{k}\AgdaSymbol{\})}\<%
\\
%
\>[8]\AgdaSymbol{...}\AgdaSpace{}%
\AgdaSymbol{|}\AgdaSpace{}%
\AgdaInductiveConstructor{IndMaxLim-L}\AgdaSpace{}%
\AgdaSymbol{\{}\AgdaArgument{f}\AgdaSpace{}%
\AgdaSymbol{=}\AgdaSpace{}%
\AgdaBound{f}\AgdaSymbol{\}}\AgdaSpace{}%
\AgdaSymbol{=}\AgdaSpace{}%
\AgdaFunction{underLim}%
\>[46]\AgdaSymbol{λ}\AgdaSpace{}%
\AgdaBound{k}\AgdaSpace{}%
\AgdaSymbol{→}\AgdaSpace{}%
\AgdaFunction{indMax-≤R}\<%
\\
%
\>[8]\AgdaSymbol{...}\AgdaSpace{}%
\AgdaSymbol{|}\AgdaSpace{}%
\AgdaInductiveConstructor{IndMaxLim-Suc}\AgdaSpace{}%
\AgdaSymbol{\{}\AgdaBound{t1}\AgdaSymbol{\}}\AgdaSpace{}%
\AgdaSymbol{\{}\AgdaBound{t2}\AgdaSymbol{\}}\AgdaSpace{}%
\AgdaSymbol{=}\AgdaSpace{}%
\AgdaPostulate{≤-sucMono}\AgdaSpace{}%
\AgdaSymbol{(}\AgdaFunction{indMax-≤R}\AgdaSpace{}%
\AgdaSymbol{\{}\AgdaArgument{t1}\AgdaSpace{}%
\AgdaSymbol{=}\AgdaSpace{}%
\AgdaBound{t1}\AgdaSymbol{\}}\AgdaSpace{}%
\AgdaSymbol{\{}\AgdaArgument{t2}\AgdaSpace{}%
\AgdaSymbol{=}\AgdaSpace{}%
\AgdaBound{t2}\AgdaSymbol{\})}\<%
\\
%
\\[\AgdaEmptyExtraSkip]%
%
\\[\AgdaEmptyExtraSkip]%
%
\\[\AgdaEmptyExtraSkip]%
%
\\[\AgdaEmptyExtraSkip]%
%
\>[8]\AgdaFunction{indMax-monoR}\AgdaSpace{}%
\AgdaSymbol{:}\AgdaSpace{}%
\AgdaSymbol{∀}\AgdaSpace{}%
\AgdaSymbol{\{}\AgdaBound{t1}\AgdaSpace{}%
\AgdaBound{t2}\AgdaSpace{}%
\AgdaBound{t2'}\AgdaSymbol{\}}\AgdaSpace{}%
\AgdaSymbol{→}\AgdaSpace{}%
\AgdaBound{t2}\AgdaSpace{}%
\AgdaOperator{\AgdaPostulate{≤}}\AgdaSpace{}%
\AgdaBound{t2'}\AgdaSpace{}%
\AgdaSymbol{→}\AgdaSpace{}%
\AgdaFunction{indMax}\AgdaSpace{}%
\AgdaBound{t1}\AgdaSpace{}%
\AgdaBound{t2}\AgdaSpace{}%
\AgdaOperator{\AgdaPostulate{≤}}\AgdaSpace{}%
\AgdaFunction{indMax}\AgdaSpace{}%
\AgdaBound{t1}\AgdaSpace{}%
\AgdaBound{t2'}\<%
\\
%
\>[8]\AgdaFunction{indMax-monoR'}\AgdaSpace{}%
\AgdaSymbol{:}\AgdaSpace{}%
\AgdaSymbol{∀}\AgdaSpace{}%
\AgdaSymbol{\{}\AgdaBound{t1}\AgdaSpace{}%
\AgdaBound{t2}\AgdaSpace{}%
\AgdaBound{t2'}\AgdaSymbol{\}}\AgdaSpace{}%
\AgdaSymbol{→}\AgdaSpace{}%
\AgdaBound{t2}\AgdaSpace{}%
\AgdaOperator{\AgdaPostulate{<}}\AgdaSpace{}%
\AgdaBound{t2'}\AgdaSpace{}%
\AgdaSymbol{→}\AgdaSpace{}%
\AgdaFunction{indMax}\AgdaSpace{}%
\AgdaBound{t1}\AgdaSpace{}%
\AgdaBound{t2}\AgdaSpace{}%
\AgdaOperator{\AgdaPostulate{<}}\AgdaSpace{}%
\AgdaFunction{indMax}\AgdaSpace{}%
\AgdaSymbol{(}\AgdaPostulate{↑}\AgdaSpace{}%
\AgdaBound{t1}\AgdaSymbol{)}\AgdaSpace{}%
\AgdaBound{t2'}\<%
\\
%
\\[\AgdaEmptyExtraSkip]%
%
\>[8]\AgdaFunction{indMax-monoR}\AgdaSpace{}%
\AgdaSymbol{\{}\AgdaBound{t1}\AgdaSymbol{\}}\AgdaSpace{}%
\AgdaSymbol{\{}\AgdaBound{t2}\AgdaSymbol{\}}\AgdaSpace{}%
\AgdaSymbol{\{}\AgdaBound{t2'}\AgdaSymbol{\}}\AgdaSpace{}%
\AgdaBound{lt}\AgdaSpace{}%
\AgdaKeyword{with}\AgdaSpace{}%
\AgdaFunction{indMaxView}\AgdaSpace{}%
\AgdaBound{t1}\AgdaSpace{}%
\AgdaBound{t2}\AgdaSpace{}%
\AgdaKeyword{in}\AgdaSpace{}%
\AgdaArgument{eq1}\AgdaSpace{}%
\AgdaSymbol{|}\AgdaSpace{}%
\AgdaFunction{indMaxView}\AgdaSpace{}%
\AgdaBound{t1}\AgdaSpace{}%
\AgdaBound{t2'}\AgdaSpace{}%
\AgdaKeyword{in}\AgdaSpace{}%
\AgdaArgument{eq2}\<%
\\
%
\>[8]\AgdaSymbol{...}\AgdaSpace{}%
\AgdaSymbol{|}\AgdaSpace{}%
\AgdaInductiveConstructor{IndMaxZ-L}%
\>[25]\AgdaSymbol{|}\AgdaSpace{}%
\AgdaBound{v2}%
\>[31]\AgdaSymbol{=}\AgdaSpace{}%
\AgdaPostulate{≤-trans}\AgdaSpace{}%
\AgdaBound{lt}\AgdaSpace{}%
\AgdaSymbol{(}\AgdaPostulate{≤-reflEq}\AgdaSpace{}%
\AgdaSymbol{(}\AgdaFunction{cong}\AgdaSpace{}%
\AgdaFunction{indMax'}\AgdaSpace{}%
\AgdaBound{eq2}\AgdaSymbol{))}\<%
\\
%
\>[8]\AgdaSymbol{...}\AgdaSpace{}%
\AgdaSymbol{|}\AgdaSpace{}%
\AgdaInductiveConstructor{IndMaxZ-R}%
\>[25]\AgdaSymbol{|}\AgdaSpace{}%
\AgdaBound{v2}%
\>[31]\AgdaSymbol{=}\AgdaSpace{}%
\AgdaPostulate{≤-trans}\AgdaSpace{}%
\AgdaFunction{indMax-≤L}\AgdaSpace{}%
\AgdaSymbol{(}\AgdaPostulate{≤-reflEq}\AgdaSpace{}%
\AgdaSymbol{(}\AgdaFunction{cong}\AgdaSpace{}%
\AgdaFunction{indMax'}\AgdaSpace{}%
\AgdaBound{eq2}\AgdaSymbol{))}\<%
\\
%
\>[8]\AgdaSymbol{...}\AgdaSpace{}%
\AgdaSymbol{|}\AgdaSpace{}%
\AgdaInductiveConstructor{IndMaxLim-L}\AgdaSpace{}%
\AgdaSymbol{\{}\AgdaArgument{f}\AgdaSpace{}%
\AgdaSymbol{=}\AgdaSpace{}%
\AgdaBound{f1}\AgdaSymbol{\}}\AgdaSpace{}%
\AgdaSymbol{|}%
\>[38]\AgdaInductiveConstructor{IndMaxLim-L}%
\>[51]\AgdaSymbol{=}\AgdaSpace{}%
\AgdaPostulate{extLim}\AgdaSpace{}%
\AgdaSymbol{\AgdaUnderscore{}}\AgdaSpace{}%
\AgdaSymbol{\AgdaUnderscore{}}\AgdaSpace{}%
\AgdaSymbol{λ}\AgdaSpace{}%
\AgdaBound{k}\AgdaSpace{}%
\AgdaSymbol{→}\AgdaSpace{}%
\AgdaFunction{indMax-monoR}\AgdaSpace{}%
\AgdaSymbol{\{}\AgdaArgument{t1}\AgdaSpace{}%
\AgdaSymbol{=}\AgdaSpace{}%
\AgdaBound{f1}\AgdaSpace{}%
\AgdaBound{k}\AgdaSymbol{\}}\AgdaSpace{}%
\AgdaBound{lt}\<%
\\
%
\>[8]\AgdaFunction{indMax-monoR}\AgdaSpace{}%
\AgdaSymbol{\{}\AgdaBound{t1}\AgdaSymbol{\}}\AgdaSpace{}%
\AgdaSymbol{\{(}\AgdaInductiveConstructor{Lim}\AgdaSpace{}%
\AgdaSymbol{\AgdaUnderscore{}}\AgdaSpace{}%
\AgdaBound{f'}\AgdaSymbol{)\}}\AgdaSpace{}%
\AgdaSymbol{\{}\AgdaDottedPattern{\AgdaSymbol{.(}}\AgdaDottedPattern{\AgdaPostulate{Lim}}\AgdaSpace{}%
\AgdaDottedPattern{\AgdaSymbol{\AgdaUnderscore{}}}\AgdaSpace{}%
\AgdaDottedPattern{\AgdaBound{f}}\AgdaDottedPattern{\AgdaSymbol{)}}\AgdaSymbol{\}}\AgdaSpace{}%
\AgdaSymbol{(}\AgdaInductiveConstructor{≤-cocone}\AgdaSpace{}%
\AgdaBound{f}\AgdaSpace{}%
\AgdaBound{k}\AgdaSpace{}%
\AgdaBound{lt}\AgdaSymbol{)}\AgdaSpace{}%
\AgdaSymbol{|}\AgdaSpace{}%
\AgdaInductiveConstructor{IndMaxLim-R}\AgdaSpace{}%
\AgdaBound{neq}%
\>[89]\AgdaSymbol{|}\AgdaSpace{}%
\AgdaInductiveConstructor{IndMaxLim-R}\AgdaSpace{}%
\AgdaBound{neq'}\<%
\\
\>[8][@{}l@{\AgdaIndent{0}}]%
\>[12]\AgdaSymbol{=}\AgdaSpace{}%
\AgdaPostulate{≤-limiting}\AgdaSpace{}%
\AgdaSymbol{(λ}\AgdaSpace{}%
\AgdaBound{x}\AgdaSpace{}%
\AgdaSymbol{→}\AgdaSpace{}%
\AgdaFunction{indMax}\AgdaSpace{}%
\AgdaBound{t1}\AgdaSpace{}%
\AgdaSymbol{(}\AgdaBound{f'}\AgdaSpace{}%
\AgdaBound{x}\AgdaSymbol{))}\AgdaSpace{}%
\AgdaSymbol{(λ}\AgdaSpace{}%
\AgdaBound{y}\AgdaSpace{}%
\AgdaSymbol{→}\AgdaSpace{}%
\AgdaPostulate{≤-cocone}\AgdaSpace{}%
\AgdaSymbol{(λ}\AgdaSpace{}%
\AgdaBound{x}\AgdaSpace{}%
\AgdaSymbol{→}\AgdaSpace{}%
\AgdaFunction{indMax}\AgdaSpace{}%
\AgdaBound{t1}\AgdaSpace{}%
\AgdaSymbol{(}\AgdaBound{f}\AgdaSpace{}%
\AgdaBound{x}\AgdaSymbol{))}\AgdaSpace{}%
\AgdaBound{k}\AgdaSpace{}%
\AgdaSymbol{(}\AgdaFunction{indMax-monoR}\AgdaSpace{}%
\AgdaSymbol{\{}\AgdaArgument{t1}\AgdaSpace{}%
\AgdaSymbol{=}\AgdaSpace{}%
\AgdaBound{t1}\AgdaSymbol{\}}\AgdaSpace{}%
\AgdaSymbol{\{}\AgdaArgument{t2}\AgdaSpace{}%
\AgdaSymbol{=}\AgdaSpace{}%
\AgdaBound{f'}\AgdaSpace{}%
\AgdaBound{y}\AgdaSymbol{\}}\AgdaSpace{}%
\AgdaSymbol{(}\AgdaPostulate{≤-trans}\AgdaSpace{}%
\AgdaSymbol{(}\AgdaPostulate{≤-cocone}\AgdaSpace{}%
\AgdaSymbol{\AgdaUnderscore{}}\AgdaSpace{}%
\AgdaBound{y}\AgdaSpace{}%
\AgdaSymbol{(}\AgdaPostulate{≤-refl}\AgdaSpace{}%
\AgdaSymbol{\AgdaUnderscore{}))}\AgdaSpace{}%
\AgdaBound{lt}\AgdaSymbol{)))}\<%
\\
%
\>[8]\AgdaFunction{indMax-monoR}\AgdaSpace{}%
\AgdaSymbol{\{}\AgdaBound{t1}\AgdaSymbol{\}}\AgdaSpace{}%
\AgdaSymbol{\{}\AgdaDottedPattern{\AgdaSymbol{.(}}\AgdaDottedPattern{\AgdaPostulate{Lim}}\AgdaSpace{}%
\AgdaDottedPattern{\AgdaSymbol{\AgdaUnderscore{}}}\AgdaSpace{}%
\AgdaDottedPattern{\AgdaSymbol{\AgdaUnderscore{})}}\AgdaSymbol{\}}\AgdaSpace{}%
\AgdaSymbol{\{}\AgdaBound{t2'}\AgdaSymbol{\}}\AgdaSpace{}%
\AgdaSymbol{(}\AgdaInductiveConstructor{≤-limiting}\AgdaSpace{}%
\AgdaBound{f}\AgdaSpace{}%
\AgdaBound{x₁}\AgdaSymbol{)}\AgdaSpace{}%
\AgdaSymbol{|}\AgdaSpace{}%
\AgdaInductiveConstructor{IndMaxLim-R}\AgdaSpace{}%
\AgdaBound{x}%
\>[80]\AgdaSymbol{|}\AgdaSpace{}%
\AgdaBound{v2}%
\>[86]\AgdaSymbol{=}\<%
\\
\>[8][@{}l@{\AgdaIndent{0}}]%
\>[12]\AgdaPostulate{≤-trans}\AgdaSpace{}%
\AgdaSymbol{(}\AgdaPostulate{≤-limiting}\AgdaSpace{}%
\AgdaSymbol{(λ}\AgdaSpace{}%
\AgdaBound{x₂}\AgdaSpace{}%
\AgdaSymbol{→}\AgdaSpace{}%
\AgdaFunction{indMax}\AgdaSpace{}%
\AgdaBound{t1}\AgdaSpace{}%
\AgdaSymbol{(}\AgdaBound{f}\AgdaSpace{}%
\AgdaBound{x₂}\AgdaSymbol{))}\AgdaSpace{}%
\AgdaSymbol{λ}\AgdaSpace{}%
\AgdaBound{k}\AgdaSpace{}%
\AgdaSymbol{→}\AgdaSpace{}%
\AgdaFunction{indMax-monoR}\AgdaSpace{}%
\AgdaSymbol{\{}\AgdaArgument{t1}\AgdaSpace{}%
\AgdaSymbol{=}\AgdaSpace{}%
\AgdaBound{t1}\AgdaSymbol{\}}\AgdaSpace{}%
\AgdaSymbol{(}\AgdaBound{x₁}\AgdaSpace{}%
\AgdaBound{k}\AgdaSymbol{))}\AgdaSpace{}%
\AgdaSymbol{(}\AgdaPostulate{≤-reflEq}\AgdaSpace{}%
\AgdaSymbol{(}\AgdaFunction{cong}\AgdaSpace{}%
\AgdaFunction{indMax'}\AgdaSpace{}%
\AgdaBound{eq2}\AgdaSymbol{))}\<%
\\
%
\>[8]\AgdaFunction{indMax-monoR}\AgdaSpace{}%
\AgdaSymbol{\{(}\AgdaInductiveConstructor{↑}\AgdaSpace{}%
\AgdaBound{t1}\AgdaSymbol{)\}}\AgdaSpace{}%
\AgdaSymbol{\{}\AgdaDottedPattern{\AgdaSymbol{.(}}\AgdaDottedPattern{\AgdaPostulate{↑}}\AgdaSpace{}%
\AgdaDottedPattern{\AgdaSymbol{\AgdaUnderscore{})}}\AgdaSymbol{\}}\AgdaSpace{}%
\AgdaSymbol{\{}\AgdaDottedPattern{\AgdaSymbol{.(}}\AgdaDottedPattern{\AgdaPostulate{↑}}\AgdaSpace{}%
\AgdaDottedPattern{\AgdaSymbol{\AgdaUnderscore{})}}\AgdaSymbol{\}}\AgdaSpace{}%
\AgdaSymbol{(}\AgdaInductiveConstructor{≤-sucMono}\AgdaSpace{}%
\AgdaBound{lt}\AgdaSymbol{)}\AgdaSpace{}%
\AgdaSymbol{|}\AgdaSpace{}%
\AgdaInductiveConstructor{IndMaxLim-Suc}%
\>[80]\AgdaSymbol{|}\AgdaSpace{}%
\AgdaInductiveConstructor{IndMaxLim-Suc}%
\>[97]\AgdaSymbol{=}\AgdaSpace{}%
\AgdaPostulate{≤-sucMono}\AgdaSpace{}%
\AgdaSymbol{(}\AgdaFunction{indMax-monoR}\AgdaSpace{}%
\AgdaSymbol{\{}\AgdaArgument{t1}\AgdaSpace{}%
\AgdaSymbol{=}\AgdaSpace{}%
\AgdaBound{t1}\AgdaSymbol{\}}\AgdaSpace{}%
\AgdaBound{lt}\AgdaSymbol{)}\<%
\\
%
\>[8]\AgdaFunction{indMax-monoR}\AgdaSpace{}%
\AgdaSymbol{\{(}\AgdaInductiveConstructor{↑}\AgdaSpace{}%
\AgdaBound{t1}\AgdaSymbol{)\}}\AgdaSpace{}%
\AgdaSymbol{\{(}\AgdaInductiveConstructor{↑}\AgdaSpace{}%
\AgdaBound{t2}\AgdaSymbol{)\}}\AgdaSpace{}%
\AgdaSymbol{\{(}\AgdaInductiveConstructor{Lim}\AgdaSpace{}%
\AgdaSymbol{\AgdaUnderscore{}}\AgdaSpace{}%
\AgdaBound{f}\AgdaSymbol{)\}}\AgdaSpace{}%
\AgdaSymbol{(}\AgdaInductiveConstructor{≤-cocone}\AgdaSpace{}%
\AgdaBound{f}\AgdaSpace{}%
\AgdaBound{k}\AgdaSpace{}%
\AgdaBound{lt}\AgdaSymbol{)}\AgdaSpace{}%
\AgdaSymbol{|}\AgdaSpace{}%
\AgdaInductiveConstructor{IndMaxLim-Suc}%
\>[86]\AgdaSymbol{|}\AgdaSpace{}%
\AgdaInductiveConstructor{IndMaxLim-R}\AgdaSpace{}%
\AgdaBound{x}\<%
\\
\>[8][@{}l@{\AgdaIndent{0}}]%
\>[12]\AgdaSymbol{=}\AgdaSpace{}%
\AgdaPostulate{≤-trans}\AgdaSpace{}%
\AgdaSymbol{(}\AgdaFunction{indMax-monoR'}\AgdaSpace{}%
\AgdaSymbol{\{}\AgdaArgument{t1}\AgdaSpace{}%
\AgdaSymbol{=}\AgdaSpace{}%
\AgdaBound{t1}\AgdaSymbol{\}}\AgdaSpace{}%
\AgdaSymbol{\{}\AgdaArgument{t2}\AgdaSpace{}%
\AgdaSymbol{=}\AgdaSpace{}%
\AgdaBound{t2}\AgdaSymbol{\}}\AgdaSpace{}%
\AgdaSymbol{\{}\AgdaArgument{t2'}\AgdaSpace{}%
\AgdaSymbol{=}\AgdaSpace{}%
\AgdaBound{f}\AgdaSpace{}%
\AgdaBound{k}\AgdaSymbol{\}}\AgdaSpace{}%
\AgdaBound{lt}\AgdaSymbol{)}\AgdaSpace{}%
\AgdaSymbol{(}\AgdaPostulate{≤-cocone}\AgdaSpace{}%
\AgdaSymbol{(λ}\AgdaSpace{}%
\AgdaBound{x₁}\AgdaSpace{}%
\AgdaSymbol{→}\AgdaSpace{}%
\AgdaFunction{indMax}\AgdaSpace{}%
\AgdaSymbol{(}\AgdaPostulate{↑}\AgdaSpace{}%
\AgdaBound{t1}\AgdaSymbol{)}\AgdaSpace{}%
\AgdaSymbol{(}\AgdaBound{f}\AgdaSpace{}%
\AgdaBound{x₁}\AgdaSymbol{))}\AgdaSpace{}%
\AgdaBound{k}\AgdaSpace{}%
\AgdaSymbol{(}\AgdaPostulate{≤-refl}\AgdaSpace{}%
\AgdaSymbol{\AgdaUnderscore{}))}\AgdaSpace{}%
\AgdaComment{--indMax-monoR'\ \{!lt!\}}\<%
\\
%
\\[\AgdaEmptyExtraSkip]%
%
\>[8]\AgdaFunction{indMax-monoR'}\AgdaSpace{}%
\AgdaSymbol{\{}\AgdaBound{t1}\AgdaSymbol{\}}\AgdaSpace{}%
\AgdaSymbol{\{}\AgdaBound{t2}\AgdaSymbol{\}}\AgdaSpace{}%
\AgdaSymbol{\{}\AgdaBound{t2'}\AgdaSymbol{\}}%
\>[39]\AgdaSymbol{(}\AgdaInductiveConstructor{≤-sucMono}\AgdaSpace{}%
\AgdaBound{lt}\AgdaSymbol{)}\AgdaSpace{}%
\AgdaSymbol{=}\AgdaSpace{}%
\AgdaPostulate{≤-sucMono}\AgdaSpace{}%
\AgdaSymbol{(}\AgdaSpace{}%
\AgdaSymbol{(}\AgdaFunction{indMax-monoR}\AgdaSpace{}%
\AgdaSymbol{\{}\AgdaArgument{t1}\AgdaSpace{}%
\AgdaSymbol{=}\AgdaSpace{}%
\AgdaBound{t1}\AgdaSymbol{\}}\AgdaSpace{}%
\AgdaBound{lt}\AgdaSymbol{))}\<%
\\
%
\>[8]\AgdaFunction{indMax-monoR'}\AgdaSpace{}%
\AgdaSymbol{\{}\AgdaBound{t1}\AgdaSymbol{\}}\AgdaSpace{}%
\AgdaSymbol{\{}\AgdaBound{t2}\AgdaSymbol{\}}\AgdaSpace{}%
\AgdaSymbol{\{}\AgdaDottedPattern{\AgdaSymbol{.(}}\AgdaDottedPattern{\AgdaPostulate{Lim}}\AgdaSpace{}%
\AgdaDottedPattern{\AgdaSymbol{\AgdaUnderscore{}}}\AgdaSpace{}%
\AgdaDottedPattern{\AgdaBound{f}}\AgdaDottedPattern{\AgdaSymbol{)}}\AgdaSymbol{\}}\AgdaSpace{}%
\AgdaSymbol{(}\AgdaInductiveConstructor{≤-cocone}\AgdaSpace{}%
\AgdaBound{f}\AgdaSpace{}%
\AgdaBound{k}\AgdaSpace{}%
\AgdaBound{lt}\AgdaSymbol{)}\<%
\\
\>[8][@{}l@{\AgdaIndent{0}}]%
\>[12]\AgdaSymbol{=}\AgdaSpace{}%
\AgdaPostulate{≤-cocone}\AgdaSpace{}%
\AgdaSymbol{\AgdaUnderscore{}}\AgdaSpace{}%
\AgdaBound{k}\AgdaSpace{}%
\AgdaSymbol{(}\AgdaFunction{indMax-monoR'}\AgdaSpace{}%
\AgdaSymbol{\{}\AgdaArgument{t1}\AgdaSpace{}%
\AgdaSymbol{=}\AgdaSpace{}%
\AgdaBound{t1}\AgdaSymbol{\}}\AgdaSpace{}%
\AgdaBound{lt}\AgdaSymbol{)}\<%
\\
%
\\[\AgdaEmptyExtraSkip]%
%
\\[\AgdaEmptyExtraSkip]%
%
\>[8]\AgdaFunction{indMax-monoL}\AgdaSpace{}%
\AgdaSymbol{:}\AgdaSpace{}%
\AgdaSymbol{∀}\AgdaSpace{}%
\AgdaSymbol{\{}\AgdaBound{t1}\AgdaSpace{}%
\AgdaBound{t1'}\AgdaSpace{}%
\AgdaBound{t2}\AgdaSymbol{\}}\AgdaSpace{}%
\AgdaSymbol{→}\AgdaSpace{}%
\AgdaBound{t1}\AgdaSpace{}%
\AgdaOperator{\AgdaPostulate{≤}}\AgdaSpace{}%
\AgdaBound{t1'}\AgdaSpace{}%
\AgdaSymbol{→}\AgdaSpace{}%
\AgdaFunction{indMax}\AgdaSpace{}%
\AgdaBound{t1}\AgdaSpace{}%
\AgdaBound{t2}\AgdaSpace{}%
\AgdaOperator{\AgdaPostulate{≤}}\AgdaSpace{}%
\AgdaFunction{indMax}\AgdaSpace{}%
\AgdaBound{t1'}\AgdaSpace{}%
\AgdaBound{t2}\<%
\\
%
\>[8]\AgdaFunction{indMax-monoL'}\AgdaSpace{}%
\AgdaSymbol{:}\AgdaSpace{}%
\AgdaSymbol{∀}\AgdaSpace{}%
\AgdaSymbol{\{}\AgdaBound{t1}\AgdaSpace{}%
\AgdaBound{t1'}\AgdaSpace{}%
\AgdaBound{t2}\AgdaSymbol{\}}\AgdaSpace{}%
\AgdaSymbol{→}\AgdaSpace{}%
\AgdaBound{t1}\AgdaSpace{}%
\AgdaOperator{\AgdaPostulate{<}}\AgdaSpace{}%
\AgdaBound{t1'}\AgdaSpace{}%
\AgdaSymbol{→}\AgdaSpace{}%
\AgdaFunction{indMax}\AgdaSpace{}%
\AgdaBound{t1}\AgdaSpace{}%
\AgdaBound{t2}\AgdaSpace{}%
\AgdaOperator{\AgdaPostulate{<}}\AgdaSpace{}%
\AgdaFunction{indMax}\AgdaSpace{}%
\AgdaBound{t1'}\AgdaSpace{}%
\AgdaSymbol{(}\AgdaPostulate{↑}\AgdaSpace{}%
\AgdaBound{t2}\AgdaSymbol{)}\<%
\\
%
\>[8]\AgdaFunction{indMax-monoL}\AgdaSpace{}%
\AgdaSymbol{\{}\AgdaBound{t1}\AgdaSymbol{\}}\AgdaSpace{}%
\AgdaSymbol{\{}\AgdaBound{t1'}\AgdaSymbol{\}}\AgdaSpace{}%
\AgdaSymbol{\{}\AgdaBound{t2}\AgdaSymbol{\}}\AgdaSpace{}%
\AgdaBound{lt}\AgdaSpace{}%
\AgdaKeyword{with}\AgdaSpace{}%
\AgdaFunction{indMaxView}\AgdaSpace{}%
\AgdaBound{t1}\AgdaSpace{}%
\AgdaBound{t2}\AgdaSpace{}%
\AgdaKeyword{in}\AgdaSpace{}%
\AgdaArgument{eq1}\AgdaSpace{}%
\AgdaSymbol{|}\AgdaSpace{}%
\AgdaFunction{indMaxView}\AgdaSpace{}%
\AgdaBound{t1'}\AgdaSpace{}%
\AgdaBound{t2}\AgdaSpace{}%
\AgdaKeyword{in}\AgdaSpace{}%
\AgdaArgument{eq2}\<%
\\
%
\>[8]\AgdaSymbol{...}\AgdaSpace{}%
\AgdaSymbol{|}\AgdaSpace{}%
\AgdaInductiveConstructor{IndMaxZ-L}\AgdaSpace{}%
\AgdaSymbol{|}\AgdaSpace{}%
\AgdaBound{v2}\AgdaSpace{}%
\AgdaSymbol{=}\AgdaSpace{}%
\AgdaPostulate{≤-trans}\AgdaSpace{}%
\AgdaSymbol{(}\AgdaFunction{indMax-≤R}\AgdaSpace{}%
\AgdaSymbol{\{}\AgdaArgument{t1}\AgdaSpace{}%
\AgdaSymbol{=}\AgdaSpace{}%
\AgdaBound{t1'}\AgdaSymbol{\})}\AgdaSpace{}%
\AgdaSymbol{(}\AgdaPostulate{≤-reflEq}\AgdaSpace{}%
\AgdaSymbol{(}\AgdaFunction{cong}\AgdaSpace{}%
\AgdaFunction{indMax'}\AgdaSpace{}%
\AgdaBound{eq2}\AgdaSymbol{))}\<%
\\
%
\>[8]\AgdaSymbol{...}\AgdaSpace{}%
\AgdaSymbol{|}\AgdaSpace{}%
\AgdaInductiveConstructor{IndMaxZ-R}\AgdaSpace{}%
\AgdaSymbol{|}\AgdaSpace{}%
\AgdaBound{v2}\AgdaSpace{}%
\AgdaSymbol{=}\AgdaSpace{}%
\AgdaPostulate{≤-trans}\AgdaSpace{}%
\AgdaBound{lt}\AgdaSpace{}%
\AgdaSymbol{(}\AgdaPostulate{≤-trans}\AgdaSpace{}%
\AgdaSymbol{(}\AgdaFunction{indMax-≤L}\AgdaSpace{}%
\AgdaSymbol{\{}\AgdaArgument{t1}\AgdaSpace{}%
\AgdaSymbol{=}\AgdaSpace{}%
\AgdaBound{t1'}\AgdaSymbol{\})}\AgdaSpace{}%
\AgdaSymbol{(}\AgdaPostulate{≤-reflEq}\AgdaSpace{}%
\AgdaSymbol{(}\AgdaFunction{cong}\AgdaSpace{}%
\AgdaFunction{indMax'}\AgdaSpace{}%
\AgdaBound{eq2}\AgdaSymbol{)))}\<%
\\
%
\>[8]\AgdaFunction{indMax-monoL}\AgdaSpace{}%
\AgdaSymbol{\{}\AgdaDottedPattern{\AgdaSymbol{.(}}\AgdaDottedPattern{\AgdaPostulate{Lim}}\AgdaSpace{}%
\AgdaDottedPattern{\AgdaSymbol{\AgdaUnderscore{}}}\AgdaSpace{}%
\AgdaDottedPattern{\AgdaSymbol{\AgdaUnderscore{})}}\AgdaSymbol{\}}\AgdaSpace{}%
\AgdaSymbol{\{}\AgdaDottedPattern{\AgdaSymbol{.(}}\AgdaDottedPattern{\AgdaPostulate{Lim}}\AgdaSpace{}%
\AgdaDottedPattern{\AgdaSymbol{\AgdaUnderscore{}}}\AgdaSpace{}%
\AgdaDottedPattern{\AgdaBound{f}}\AgdaDottedPattern{\AgdaSymbol{)}}\AgdaSymbol{\}}\AgdaSpace{}%
\AgdaSymbol{\{}\AgdaBound{t2}\AgdaSymbol{\}}\AgdaSpace{}%
\AgdaSymbol{(}\AgdaInductiveConstructor{≤-cocone}\AgdaSpace{}%
\AgdaBound{f}\AgdaSpace{}%
\AgdaBound{k}\AgdaSpace{}%
\AgdaBound{lt}\AgdaSymbol{)}\AgdaSpace{}%
\AgdaSymbol{|}\AgdaSpace{}%
\AgdaInductiveConstructor{IndMaxLim-L}%
\>[85]\AgdaSymbol{|}\AgdaSpace{}%
\AgdaInductiveConstructor{IndMaxLim-L}\<%
\\
\>[8][@{}l@{\AgdaIndent{0}}]%
\>[12]\AgdaSymbol{=}\AgdaSpace{}%
\AgdaPostulate{≤-cocone}\AgdaSpace{}%
\AgdaSymbol{(λ}\AgdaSpace{}%
\AgdaBound{x}\AgdaSpace{}%
\AgdaSymbol{→}\AgdaSpace{}%
\AgdaFunction{indMax}\AgdaSpace{}%
\AgdaSymbol{(}\AgdaBound{f}\AgdaSpace{}%
\AgdaBound{x}\AgdaSymbol{)}\AgdaSpace{}%
\AgdaBound{t2}\AgdaSymbol{)}\AgdaSpace{}%
\AgdaBound{k}\AgdaSpace{}%
\AgdaSymbol{(}\AgdaFunction{indMax-monoL}\AgdaSpace{}%
\AgdaBound{lt}\AgdaSymbol{)}\<%
\\
%
\>[8]\AgdaFunction{indMax-monoL}\AgdaSpace{}%
\AgdaSymbol{\{}\AgdaDottedPattern{\AgdaSymbol{.(}}\AgdaDottedPattern{\AgdaPostulate{Lim}}\AgdaSpace{}%
\AgdaDottedPattern{\AgdaSymbol{\AgdaUnderscore{}}}\AgdaSpace{}%
\AgdaDottedPattern{\AgdaSymbol{\AgdaUnderscore{})}}\AgdaSymbol{\}}\AgdaSpace{}%
\AgdaSymbol{\{}\AgdaBound{t1'}\AgdaSymbol{\}}\AgdaSpace{}%
\AgdaSymbol{\{}\AgdaBound{t2}\AgdaSymbol{\}}\AgdaSpace{}%
\AgdaSymbol{(}\AgdaInductiveConstructor{≤-limiting}\AgdaSpace{}%
\AgdaBound{f}\AgdaSpace{}%
\AgdaBound{lt}\AgdaSymbol{)}\AgdaSpace{}%
\AgdaSymbol{|}\AgdaSpace{}%
\AgdaInductiveConstructor{IndMaxLim-L}\AgdaSpace{}%
\AgdaSymbol{|}%
\>[80]\AgdaBound{v2}\<%
\\
\>[8][@{}l@{\AgdaIndent{0}}]%
\>[12]\AgdaSymbol{=}\AgdaSpace{}%
\AgdaPostulate{≤-limiting}\AgdaSpace{}%
\AgdaSymbol{(λ}\AgdaSpace{}%
\AgdaBound{x₁}\AgdaSpace{}%
\AgdaSymbol{→}\AgdaSpace{}%
\AgdaFunction{indMax}\AgdaSpace{}%
\AgdaSymbol{(}\AgdaBound{f}\AgdaSpace{}%
\AgdaBound{x₁}\AgdaSymbol{)}\AgdaSpace{}%
\AgdaBound{t2}\AgdaSymbol{)}\AgdaSpace{}%
\AgdaSymbol{λ}\AgdaSpace{}%
\AgdaBound{k}\AgdaSpace{}%
\AgdaSymbol{→}\AgdaSpace{}%
\AgdaPostulate{≤-trans}\AgdaSpace{}%
\AgdaSymbol{(}\AgdaFunction{indMax-monoL}\AgdaSpace{}%
\AgdaSymbol{(}\AgdaBound{lt}\AgdaSpace{}%
\AgdaBound{k}\AgdaSymbol{))}\AgdaSpace{}%
\AgdaSymbol{(}\AgdaPostulate{≤-reflEq}\AgdaSpace{}%
\AgdaSymbol{(}\AgdaFunction{cong}\AgdaSpace{}%
\AgdaFunction{indMax'}\AgdaSpace{}%
\AgdaBound{eq2}\AgdaSymbol{))}\<%
\\
%
\>[8]\AgdaFunction{indMax-monoL}\AgdaSpace{}%
\AgdaSymbol{\{}\AgdaDottedPattern{\AgdaSymbol{.}}\AgdaDottedPattern{\AgdaPostulate{Z}}\AgdaSymbol{\}}\AgdaSpace{}%
\AgdaSymbol{\{}\AgdaDottedPattern{\AgdaSymbol{.}}\AgdaDottedPattern{\AgdaPostulate{Z}}\AgdaSymbol{\}}\AgdaSpace{}%
\AgdaSymbol{\{}\AgdaDottedPattern{\AgdaSymbol{.(}}\AgdaDottedPattern{\AgdaPostulate{Lim}}\AgdaSpace{}%
\AgdaDottedPattern{\AgdaSymbol{\AgdaUnderscore{}}}\AgdaSpace{}%
\AgdaDottedPattern{\AgdaSymbol{\AgdaUnderscore{})}}\AgdaSymbol{\}}\AgdaSpace{}%
\AgdaInductiveConstructor{≤-Z}\AgdaSpace{}%
\AgdaSymbol{|}\AgdaSpace{}%
\AgdaInductiveConstructor{IndMaxLim-R}\AgdaSpace{}%
\AgdaBound{neq}%
\>[67]\AgdaSymbol{|}\AgdaSpace{}%
\AgdaInductiveConstructor{IndMaxZ-L}%
\>[80]\AgdaSymbol{=}\AgdaSpace{}%
\AgdaPostulate{≤-refl}\AgdaSpace{}%
\AgdaSymbol{\AgdaUnderscore{}}\<%
\\
%
\>[8]\AgdaFunction{indMax-monoL}%
\>[22]\AgdaSymbol{\{}\AgdaDottedPattern{\AgdaSymbol{.(}}\AgdaDottedPattern{\AgdaPostulate{Lim}}\AgdaSpace{}%
\AgdaDottedPattern{\AgdaSymbol{\AgdaUnderscore{}}}\AgdaSpace{}%
\AgdaDottedPattern{\AgdaBound{f}}\AgdaDottedPattern{\AgdaSymbol{)}}\AgdaSymbol{\}}\AgdaSpace{}%
\AgdaSymbol{\{}\AgdaDottedPattern{\AgdaSymbol{.}}\AgdaDottedPattern{\AgdaPostulate{Z}}\AgdaSymbol{\}}\AgdaSpace{}%
\AgdaSymbol{\{}\AgdaDottedPattern{\AgdaSymbol{.(}}\AgdaDottedPattern{\AgdaPostulate{Lim}}\AgdaSpace{}%
\AgdaDottedPattern{\AgdaSymbol{\AgdaUnderscore{}}}\AgdaSpace{}%
\AgdaDottedPattern{\AgdaSymbol{\AgdaUnderscore{})}}\AgdaSymbol{\}}\AgdaSpace{}%
\AgdaSymbol{(}\AgdaInductiveConstructor{≤-limiting}\AgdaSpace{}%
\AgdaBound{f}\AgdaSpace{}%
\AgdaBound{x}\AgdaSymbol{)}\AgdaSpace{}%
\AgdaSymbol{|}\AgdaSpace{}%
\AgdaInductiveConstructor{IndMaxLim-R}\AgdaSpace{}%
\AgdaBound{neq}\AgdaSpace{}%
\AgdaSymbol{|}\AgdaSpace{}%
\AgdaInductiveConstructor{IndMaxZ-L}\<%
\\
\>[8][@{}l@{\AgdaIndent{0}}]%
\>[12]\AgdaKeyword{with}\AgdaSpace{}%
\AgdaSymbol{()}\ ←\ \AgdaBound{neq}\AgdaSpace{}%
\AgdaInductiveConstructor{refl}\<%
\\
%
\>[8]\AgdaFunction{indMax-monoL}\AgdaSpace{}%
\AgdaSymbol{\{}\AgdaBound{t1}\AgdaSymbol{\}}\AgdaSpace{}%
\AgdaSymbol{\{}\AgdaDottedPattern{\AgdaSymbol{.(}}\AgdaDottedPattern{\AgdaPostulate{Lim}}\AgdaSpace{}%
\AgdaDottedPattern{\AgdaSymbol{\AgdaUnderscore{}}}\AgdaSpace{}%
\AgdaDottedPattern{\AgdaSymbol{\AgdaUnderscore{})}}\AgdaSymbol{\}}\AgdaSpace{}%
\AgdaSymbol{\{}\AgdaDottedPattern{\AgdaSymbol{.(}}\AgdaDottedPattern{\AgdaPostulate{Lim}}\AgdaSpace{}%
\AgdaDottedPattern{\AgdaSymbol{\AgdaUnderscore{}}}\AgdaSpace{}%
\AgdaDottedPattern{\AgdaSymbol{\AgdaUnderscore{})}}\AgdaSymbol{\}}\AgdaSpace{}%
\AgdaSymbol{(}\AgdaInductiveConstructor{≤-cocone}\AgdaSpace{}%
\AgdaSymbol{\AgdaUnderscore{}}\AgdaSpace{}%
\AgdaBound{k}\AgdaSpace{}%
\AgdaBound{lt}\AgdaSymbol{)}\AgdaSpace{}%
\AgdaSymbol{|}\AgdaSpace{}%
\AgdaInductiveConstructor{IndMaxLim-R}\AgdaSpace{}%
\AgdaSymbol{\{}\AgdaArgument{f}\AgdaSpace{}%
\AgdaSymbol{=}\AgdaSpace{}%
\AgdaBound{f}\AgdaSymbol{\}}\AgdaSpace{}%
\AgdaBound{neq}\AgdaSpace{}%
\AgdaSymbol{|}\AgdaSpace{}%
\AgdaInductiveConstructor{IndMaxLim-L}\AgdaSpace{}%
\AgdaSymbol{\{}\AgdaArgument{f}\AgdaSpace{}%
\AgdaSymbol{=}\AgdaSpace{}%
\AgdaBound{f'}\AgdaSymbol{\}}\<%
\\
\>[8][@{}l@{\AgdaIndent{0}}]%
\>[12]\AgdaSymbol{=}\AgdaSpace{}%
\AgdaPostulate{≤-limiting}\AgdaSpace{}%
\AgdaSymbol{(λ}\AgdaSpace{}%
\AgdaBound{x}\AgdaSpace{}%
\AgdaSymbol{→}\AgdaSpace{}%
\AgdaFunction{indMax}\AgdaSpace{}%
\AgdaBound{t1}\AgdaSpace{}%
\AgdaSymbol{(}\AgdaBound{f}\AgdaSpace{}%
\AgdaBound{x}\AgdaSymbol{))}\AgdaSpace{}%
\AgdaSymbol{(λ}\AgdaSpace{}%
\AgdaBound{y}\AgdaSpace{}%
\AgdaSymbol{→}\AgdaSpace{}%
\AgdaPostulate{≤-cocone}\AgdaSpace{}%
\AgdaSymbol{(λ}\AgdaSpace{}%
\AgdaBound{x}\AgdaSpace{}%
\AgdaSymbol{→}\AgdaSpace{}%
\AgdaFunction{indMax}\AgdaSpace{}%
\AgdaSymbol{(}\AgdaBound{f'}\AgdaSpace{}%
\AgdaBound{x}\AgdaSymbol{)}\AgdaSpace{}%
\AgdaSymbol{(}\AgdaPostulate{Lim}\AgdaSpace{}%
\AgdaSymbol{\AgdaUnderscore{}}\AgdaSpace{}%
\AgdaSymbol{\AgdaUnderscore{}))}\AgdaSpace{}%
\AgdaBound{k}\<%
\\
%
\>[12]\AgdaSymbol{(}\AgdaPostulate{≤-trans}\AgdaSpace{}%
\AgdaSymbol{(}\AgdaFunction{indMax-monoL}\AgdaSpace{}%
\AgdaBound{lt}\AgdaSymbol{)}\AgdaSpace{}%
\AgdaSymbol{(}\AgdaFunction{indMax-monoR}\AgdaSpace{}%
\AgdaSymbol{\{}\AgdaArgument{t1}\AgdaSpace{}%
\AgdaSymbol{=}\AgdaSpace{}%
\AgdaBound{f'}\AgdaSpace{}%
\AgdaBound{k}\AgdaSymbol{\}}\AgdaSpace{}%
\AgdaSymbol{(}\AgdaPostulate{≤-cocone}\AgdaSpace{}%
\AgdaBound{f}\AgdaSpace{}%
\AgdaBound{y}\AgdaSpace{}%
\AgdaSymbol{(}\AgdaPostulate{≤-refl}\AgdaSpace{}%
\AgdaSymbol{\AgdaUnderscore{})))))}\<%
\\
%
\>[8]\AgdaSymbol{...}\AgdaSpace{}%
\AgdaSymbol{|}\AgdaSpace{}%
\AgdaInductiveConstructor{IndMaxLim-R}\AgdaSpace{}%
\AgdaBound{neq}\AgdaSpace{}%
\AgdaSymbol{|}\AgdaSpace{}%
\AgdaInductiveConstructor{IndMaxLim-R}\AgdaSpace{}%
\AgdaSymbol{\{}\AgdaArgument{f}\AgdaSpace{}%
\AgdaSymbol{=}\AgdaSpace{}%
\AgdaBound{f}\AgdaSymbol{\}}\AgdaSpace{}%
\AgdaBound{neq'}\AgdaSpace{}%
\AgdaSymbol{=}\AgdaSpace{}%
\AgdaPostulate{extLim}\AgdaSpace{}%
\AgdaSymbol{(λ}\AgdaSpace{}%
\AgdaBound{x}\AgdaSpace{}%
\AgdaSymbol{→}\AgdaSpace{}%
\AgdaFunction{indMax}\AgdaSpace{}%
\AgdaBound{t1}\AgdaSpace{}%
\AgdaSymbol{(}\AgdaBound{f}\AgdaSpace{}%
\AgdaBound{x}\AgdaSymbol{))}\AgdaSpace{}%
\AgdaSymbol{(λ}\AgdaSpace{}%
\AgdaBound{x}\AgdaSpace{}%
\AgdaSymbol{→}\AgdaSpace{}%
\AgdaFunction{indMax}\AgdaSpace{}%
\AgdaBound{t1'}\AgdaSpace{}%
\AgdaSymbol{(}\AgdaBound{f}\AgdaSpace{}%
\AgdaBound{x}\AgdaSymbol{))}\AgdaSpace{}%
\AgdaSymbol{(λ}\AgdaSpace{}%
\AgdaBound{k}\AgdaSpace{}%
\AgdaSymbol{→}\AgdaSpace{}%
\AgdaFunction{indMax-monoL}\AgdaSpace{}%
\AgdaBound{lt}\AgdaSymbol{)}\<%
\\
%
\>[8]\AgdaFunction{indMax-monoL}\AgdaSpace{}%
\AgdaSymbol{\{}\AgdaDottedPattern{\AgdaSymbol{.(}}\AgdaDottedPattern{\AgdaPostulate{↑}}\AgdaSpace{}%
\AgdaDottedPattern{\AgdaSymbol{\AgdaUnderscore{})}}\AgdaSymbol{\}}\AgdaSpace{}%
\AgdaSymbol{\{}\AgdaDottedPattern{\AgdaSymbol{.(}}\AgdaDottedPattern{\AgdaPostulate{↑}}\AgdaSpace{}%
\AgdaDottedPattern{\AgdaSymbol{\AgdaUnderscore{})}}\AgdaSymbol{\}}\AgdaSpace{}%
\AgdaSymbol{\{}\AgdaDottedPattern{\AgdaSymbol{.(}}\AgdaDottedPattern{\AgdaPostulate{↑}}\AgdaSpace{}%
\AgdaDottedPattern{\AgdaSymbol{\AgdaUnderscore{})}}\AgdaSymbol{\}}\AgdaSpace{}%
\AgdaSymbol{(}\AgdaInductiveConstructor{≤-sucMono}\AgdaSpace{}%
\AgdaBound{lt}\AgdaSymbol{)}\AgdaSpace{}%
\AgdaSymbol{|}\AgdaSpace{}%
\AgdaInductiveConstructor{IndMaxLim-Suc}%
\>[80]\AgdaSymbol{|}\AgdaSpace{}%
\AgdaInductiveConstructor{IndMaxLim-Suc}\<%
\\
\>[8][@{}l@{\AgdaIndent{0}}]%
\>[12]\AgdaSymbol{=}\AgdaSpace{}%
\AgdaPostulate{≤-sucMono}\AgdaSpace{}%
\AgdaSymbol{(}\AgdaFunction{indMax-monoL}\AgdaSpace{}%
\AgdaBound{lt}\AgdaSymbol{)}\<%
\\
%
\>[8]\AgdaFunction{indMax-monoL}\AgdaSpace{}%
\AgdaSymbol{\{}\AgdaDottedPattern{\AgdaSymbol{.(}}\AgdaDottedPattern{\AgdaPostulate{↑}}\AgdaSpace{}%
\AgdaDottedPattern{\AgdaSymbol{\AgdaUnderscore{})}}\AgdaSymbol{\}}\AgdaSpace{}%
\AgdaSymbol{\{}\AgdaDottedPattern{\AgdaSymbol{.(}}\AgdaDottedPattern{\AgdaPostulate{Lim}}\AgdaSpace{}%
\AgdaDottedPattern{\AgdaSymbol{\AgdaUnderscore{}}}\AgdaSpace{}%
\AgdaDottedPattern{\AgdaBound{f}}\AgdaDottedPattern{\AgdaSymbol{)}}\AgdaSymbol{\}}\AgdaSpace{}%
\AgdaSymbol{\{}\AgdaDottedPattern{\AgdaSymbol{.(}}\AgdaDottedPattern{\AgdaPostulate{↑}}\AgdaSpace{}%
\AgdaDottedPattern{\AgdaSymbol{\AgdaUnderscore{})}}\AgdaSymbol{\}}\AgdaSpace{}%
\AgdaSymbol{(}\AgdaInductiveConstructor{≤-cocone}\AgdaSpace{}%
\AgdaBound{f}\AgdaSpace{}%
\AgdaBound{k}\AgdaSpace{}%
\AgdaBound{lt}\AgdaSymbol{)}\AgdaSpace{}%
\AgdaSymbol{|}\AgdaSpace{}%
\AgdaInductiveConstructor{IndMaxLim-Suc}%
\>[87]\AgdaSymbol{|}\AgdaSpace{}%
\AgdaInductiveConstructor{IndMaxLim-L}\<%
\\
\>[8][@{}l@{\AgdaIndent{0}}]%
\>[12]\AgdaSymbol{=}\AgdaSpace{}%
\AgdaPostulate{≤-cocone}\AgdaSpace{}%
\AgdaSymbol{(λ}\AgdaSpace{}%
\AgdaBound{x}\AgdaSpace{}%
\AgdaSymbol{→}\AgdaSpace{}%
\AgdaFunction{indMax}\AgdaSpace{}%
\AgdaSymbol{(}\AgdaBound{f}\AgdaSpace{}%
\AgdaBound{x}\AgdaSymbol{)}\AgdaSpace{}%
\AgdaSymbol{(}\AgdaPostulate{↑}\AgdaSpace{}%
\AgdaSymbol{\AgdaUnderscore{}))}\AgdaSpace{}%
\AgdaBound{k}\AgdaSpace{}%
\AgdaSymbol{(}\AgdaFunction{indMax-monoL'}\AgdaSpace{}%
\AgdaBound{lt}\AgdaSymbol{)}\<%
\\
%
\\[\AgdaEmptyExtraSkip]%
%
\>[8]\AgdaFunction{indMax-monoL'}\AgdaSpace{}%
\AgdaSymbol{\{}\AgdaBound{t1}\AgdaSymbol{\}}\AgdaSpace{}%
\AgdaSymbol{\{}\AgdaBound{t1'}\AgdaSymbol{\}}\AgdaSpace{}%
\AgdaSymbol{\{}\AgdaBound{t2}\AgdaSymbol{\}}\AgdaSpace{}%
\AgdaBound{lt}\AgdaSpace{}%
\AgdaKeyword{with}\AgdaSpace{}%
\AgdaFunction{indMaxView}\AgdaSpace{}%
\AgdaBound{t1}\AgdaSpace{}%
\AgdaBound{t2}\AgdaSpace{}%
\AgdaKeyword{in}\AgdaSpace{}%
\AgdaArgument{eq1}\AgdaSpace{}%
\AgdaSymbol{|}\AgdaSpace{}%
\AgdaFunction{indMaxView}\AgdaSpace{}%
\AgdaBound{t1'}\AgdaSpace{}%
\AgdaBound{t2}\AgdaSpace{}%
\AgdaKeyword{in}\AgdaSpace{}%
\AgdaArgument{eq2}\<%
\\
%
\>[8]\AgdaFunction{indMax-monoL'}\AgdaSpace{}%
\AgdaSymbol{\{}\AgdaBound{t1}\AgdaSymbol{\}}\AgdaSpace{}%
\AgdaSymbol{\{}\AgdaDottedPattern{\AgdaSymbol{.(}}\AgdaDottedPattern{\AgdaPostulate{↑}}\AgdaSpace{}%
\AgdaDottedPattern{\AgdaSymbol{\AgdaUnderscore{})}}\AgdaSymbol{\}}\AgdaSpace{}%
\AgdaSymbol{\{}\AgdaBound{t2}\AgdaSymbol{\}}\AgdaSpace{}%
\AgdaSymbol{(}\AgdaInductiveConstructor{≤-sucMono}\AgdaSpace{}%
\AgdaBound{lt}\AgdaSymbol{)}\AgdaSpace{}%
\AgdaSymbol{|}\AgdaSpace{}%
\AgdaBound{v1}\AgdaSpace{}%
\AgdaSymbol{|}\AgdaSpace{}%
\AgdaBound{v2}\AgdaSpace{}%
\AgdaSymbol{=}\AgdaSpace{}%
\AgdaPostulate{≤-sucMono}\AgdaSpace{}%
\AgdaSymbol{(}\AgdaPostulate{≤-trans}\AgdaSpace{}%
\AgdaSymbol{(}\AgdaPostulate{≤-reflEq}\AgdaSpace{}%
\AgdaSymbol{(}\AgdaFunction{cong}\AgdaSpace{}%
\AgdaFunction{indMax'}\AgdaSpace{}%
\AgdaSymbol{(}\AgdaFunction{sym}\AgdaSpace{}%
\AgdaBound{eq1}\AgdaSymbol{)))}\AgdaSpace{}%
\AgdaSymbol{(}\AgdaFunction{indMax-monoL}\AgdaSpace{}%
\AgdaBound{lt}\AgdaSymbol{))}\<%
\\
%
\>[8]\AgdaFunction{indMax-monoL'}\AgdaSpace{}%
\AgdaSymbol{\{}\AgdaBound{t1}\AgdaSymbol{\}}\AgdaSpace{}%
\AgdaSymbol{\{}\AgdaDottedPattern{\AgdaSymbol{.(}}\AgdaDottedPattern{\AgdaPostulate{Lim}}\AgdaSpace{}%
\AgdaDottedPattern{\AgdaSymbol{\AgdaUnderscore{}}}\AgdaSpace{}%
\AgdaDottedPattern{\AgdaBound{f}}\AgdaDottedPattern{\AgdaSymbol{)}}\AgdaSymbol{\}}\AgdaSpace{}%
\AgdaSymbol{\{}\AgdaBound{t2}\AgdaSymbol{\}}\AgdaSpace{}%
\AgdaSymbol{(}\AgdaInductiveConstructor{≤-cocone}\AgdaSpace{}%
\AgdaBound{f}\AgdaSpace{}%
\AgdaBound{k}\AgdaSpace{}%
\AgdaBound{lt}\AgdaSymbol{)}\AgdaSpace{}%
\AgdaSymbol{|}\AgdaSpace{}%
\AgdaBound{v1}\AgdaSpace{}%
\AgdaSymbol{|}\AgdaSpace{}%
\AgdaBound{v2}\<%
\\
\>[8][@{}l@{\AgdaIndent{0}}]%
\>[12]\AgdaSymbol{=}\AgdaSpace{}%
\AgdaPostulate{≤-cocone}\AgdaSpace{}%
\AgdaSymbol{\AgdaUnderscore{}}\AgdaSpace{}%
\AgdaBound{k}\AgdaSpace{}%
\AgdaSymbol{(}\AgdaPostulate{≤-trans}\AgdaSpace{}%
\AgdaSymbol{(}\AgdaPostulate{≤-sucMono}\AgdaSpace{}%
\AgdaSymbol{(}\AgdaPostulate{≤-reflEq}\AgdaSpace{}%
\AgdaSymbol{(}\AgdaFunction{cong}\AgdaSpace{}%
\AgdaFunction{indMax'}\AgdaSpace{}%
\AgdaSymbol{(}\AgdaFunction{sym}\AgdaSpace{}%
\AgdaBound{eq1}\AgdaSymbol{))))}\AgdaSpace{}%
\AgdaSymbol{(}\AgdaFunction{indMax-monoL'}\AgdaSpace{}%
\AgdaBound{lt}\AgdaSymbol{))}\<%
\end{code}


\subsubsection{Limitation: Idempotence}




\begin{code}%
\>[0]\<%
\\
%
\>[8]\AgdaFunction{indMax-mono}\AgdaSpace{}%
\AgdaSymbol{:}\AgdaSpace{}%
\AgdaSymbol{∀}\AgdaSpace{}%
\AgdaSymbol{\{}\AgdaBound{t1}\AgdaSpace{}%
\AgdaBound{t2}\AgdaSpace{}%
\AgdaBound{t1'}\AgdaSpace{}%
\AgdaBound{t2'}\AgdaSymbol{\}}\AgdaSpace{}%
\AgdaSymbol{→}\AgdaSpace{}%
\AgdaBound{t1}\AgdaSpace{}%
\AgdaOperator{\AgdaPostulate{≤}}\AgdaSpace{}%
\AgdaBound{t1'}\AgdaSpace{}%
\AgdaSymbol{→}\AgdaSpace{}%
\AgdaBound{t2}\AgdaSpace{}%
\AgdaOperator{\AgdaPostulate{≤}}\AgdaSpace{}%
\AgdaBound{t2'}\AgdaSpace{}%
\AgdaSymbol{→}\AgdaSpace{}%
\AgdaFunction{indMax}\AgdaSpace{}%
\AgdaBound{t1}\AgdaSpace{}%
\AgdaBound{t2}\AgdaSpace{}%
\AgdaOperator{\AgdaPostulate{≤}}\AgdaSpace{}%
\AgdaFunction{indMax}\AgdaSpace{}%
\AgdaBound{t1'}\AgdaSpace{}%
\AgdaBound{t2'}\<%
\\
%
\>[8]\AgdaFunction{indMax-mono}\AgdaSpace{}%
\AgdaSymbol{\{}\AgdaArgument{t1'}\AgdaSpace{}%
\AgdaSymbol{=}\AgdaSpace{}%
\AgdaBound{t1'}\AgdaSymbol{\}}\AgdaSpace{}%
\AgdaBound{lt1}\AgdaSpace{}%
\AgdaBound{lt2}\AgdaSpace{}%
\AgdaSymbol{=}\AgdaSpace{}%
\AgdaPostulate{≤-trans}\AgdaSpace{}%
\AgdaSymbol{(}\AgdaFunction{indMax-monoL}\AgdaSpace{}%
\AgdaBound{lt1}\AgdaSymbol{)}\AgdaSpace{}%
\AgdaSymbol{(}\AgdaFunction{indMax-monoR}\AgdaSpace{}%
\AgdaSymbol{\{}\AgdaArgument{t1}\AgdaSpace{}%
\AgdaSymbol{=}\AgdaSpace{}%
\AgdaBound{t1'}\AgdaSymbol{\}}\AgdaSpace{}%
\AgdaBound{lt2}\AgdaSymbol{)}\<%
\\
%
\\[\AgdaEmptyExtraSkip]%
%
\>[8]\AgdaFunction{indMax-strictMono}\AgdaSpace{}%
\AgdaSymbol{:}\AgdaSpace{}%
\AgdaSymbol{∀}\AgdaSpace{}%
\AgdaSymbol{\{}\AgdaBound{t1}\AgdaSpace{}%
\AgdaBound{t2}\AgdaSpace{}%
\AgdaBound{t1'}\AgdaSpace{}%
\AgdaBound{t2'}\AgdaSymbol{\}}\AgdaSpace{}%
\AgdaSymbol{→}\AgdaSpace{}%
\AgdaBound{t1}\AgdaSpace{}%
\AgdaOperator{\AgdaPostulate{<}}\AgdaSpace{}%
\AgdaBound{t1'}\AgdaSpace{}%
\AgdaSymbol{→}\AgdaSpace{}%
\AgdaBound{t2}\AgdaSpace{}%
\AgdaOperator{\AgdaPostulate{<}}\AgdaSpace{}%
\AgdaBound{t2'}\AgdaSpace{}%
\AgdaSymbol{→}\AgdaSpace{}%
\AgdaFunction{indMax}\AgdaSpace{}%
\AgdaBound{t1}\AgdaSpace{}%
\AgdaBound{t2}\AgdaSpace{}%
\AgdaOperator{\AgdaPostulate{<}}\AgdaSpace{}%
\AgdaFunction{indMax}\AgdaSpace{}%
\AgdaBound{t1'}\AgdaSpace{}%
\AgdaBound{t2'}\<%
\\
%
\>[8]\AgdaFunction{indMax-strictMono}\AgdaSpace{}%
\AgdaBound{lt1}\AgdaSpace{}%
\AgdaBound{lt2}\AgdaSpace{}%
\AgdaSymbol{=}\AgdaSpace{}%
\AgdaFunction{indMax-mono}\AgdaSpace{}%
\AgdaBound{lt1}\AgdaSpace{}%
\AgdaBound{lt2}\<%
\\
%
\\[\AgdaEmptyExtraSkip]%
%
\\[\AgdaEmptyExtraSkip]%
%
\>[8]\AgdaFunction{indMax-sucMono}\AgdaSpace{}%
\AgdaSymbol{:}\AgdaSpace{}%
\AgdaSymbol{∀}\AgdaSpace{}%
\AgdaSymbol{\{}\AgdaBound{t1}\AgdaSpace{}%
\AgdaBound{t2}\AgdaSpace{}%
\AgdaBound{t1'}\AgdaSpace{}%
\AgdaBound{t2'}\AgdaSymbol{\}}\AgdaSpace{}%
\AgdaSymbol{→}\AgdaSpace{}%
\AgdaFunction{indMax}\AgdaSpace{}%
\AgdaBound{t1}\AgdaSpace{}%
\AgdaBound{t2}\AgdaSpace{}%
\AgdaOperator{\AgdaPostulate{≤}}\AgdaSpace{}%
\AgdaFunction{indMax}\AgdaSpace{}%
\AgdaBound{t1'}\AgdaSpace{}%
\AgdaBound{t2'}\AgdaSpace{}%
\AgdaSymbol{→}\AgdaSpace{}%
\AgdaFunction{indMax}\AgdaSpace{}%
\AgdaBound{t1}\AgdaSpace{}%
\AgdaBound{t2}\AgdaSpace{}%
\AgdaOperator{\AgdaPostulate{<}}\AgdaSpace{}%
\AgdaFunction{indMax}\AgdaSpace{}%
\AgdaSymbol{(}\AgdaPostulate{↑}\AgdaSpace{}%
\AgdaBound{t1'}\AgdaSymbol{)}\AgdaSpace{}%
\AgdaSymbol{(}\AgdaPostulate{↑}\AgdaSpace{}%
\AgdaBound{t2'}\AgdaSymbol{)}\<%
\\
%
\>[8]\AgdaFunction{indMax-sucMono}\AgdaSpace{}%
\AgdaBound{lt}\AgdaSpace{}%
\AgdaSymbol{=}\AgdaSpace{}%
\AgdaPostulate{≤-sucMono}\AgdaSpace{}%
\AgdaBound{lt}\<%
\\
%
\\[\AgdaEmptyExtraSkip]%
%
\\[\AgdaEmptyExtraSkip]%
%
\>[6]\AgdaComment{--\ indMax-Z\ :\ ∀\ t\ →\ indMax\ t\ Z\ ≡\ o}\<%
\\
%
\>[6]\AgdaComment{--\ indMax-Z\ Z\ =\ refl}\<%
\\
%
\>[6]\AgdaComment{--\ indMax-Z\ (↑\ t)\ =\ refl}\<%
\\
%
\>[6]\AgdaComment{--\ indMax-Z\ (Lim\ c\ f)\ =\ cong\ (Lim\ c)\ \{!!\}\ --\ cong\ (Lim\ c)\ (funExt\ (λ\ x\ →\ indMax-Z\ (f\ x)))}\<%
\\
%
\\[\AgdaEmptyExtraSkip]%
\>[6][@{}l@{\AgdaIndent{0}}]%
\>[8]\AgdaFunction{indMax-Z}\AgdaSpace{}%
\AgdaSymbol{:}\AgdaSpace{}%
\AgdaSymbol{∀}\AgdaSpace{}%
\AgdaBound{t}\AgdaSpace{}%
\AgdaSymbol{→}\AgdaSpace{}%
\AgdaFunction{indMax}\AgdaSpace{}%
\AgdaBound{t}\AgdaSpace{}%
\AgdaPostulate{Z}\AgdaSpace{}%
\AgdaOperator{\AgdaPostulate{≤}}\AgdaSpace{}%
\AgdaBound{t}\<%
\\
%
\>[8]\AgdaFunction{indMax-Z}\AgdaSpace{}%
\AgdaInductiveConstructor{Z}\AgdaSpace{}%
\AgdaSymbol{=}\AgdaSpace{}%
\AgdaPostulate{≤-Z}\<%
\\
%
\>[8]\AgdaFunction{indMax-Z}\AgdaSpace{}%
\AgdaSymbol{(}\AgdaInductiveConstructor{↑}\AgdaSpace{}%
\AgdaBound{t}\AgdaSymbol{)}\AgdaSpace{}%
\AgdaSymbol{=}\AgdaSpace{}%
\AgdaPostulate{≤-refl}\AgdaSpace{}%
\AgdaSymbol{(}\AgdaFunction{indMax}\AgdaSpace{}%
\AgdaSymbol{(}\AgdaPostulate{↑}\AgdaSpace{}%
\AgdaBound{t}\AgdaSymbol{)}\AgdaSpace{}%
\AgdaPostulate{Z}\AgdaSymbol{)}\<%
\\
%
\>[8]\AgdaFunction{indMax-Z}\AgdaSpace{}%
\AgdaSymbol{(}\AgdaInductiveConstructor{Lim}\AgdaSpace{}%
\AgdaBound{c}\AgdaSpace{}%
\AgdaBound{f}\AgdaSymbol{)}\AgdaSpace{}%
\AgdaSymbol{=}\AgdaSpace{}%
\AgdaPostulate{extLim}\AgdaSpace{}%
\AgdaSymbol{(λ}\AgdaSpace{}%
\AgdaBound{x}\AgdaSpace{}%
\AgdaSymbol{→}\AgdaSpace{}%
\AgdaFunction{indMax}\AgdaSpace{}%
\AgdaSymbol{(}\AgdaBound{f}\AgdaSpace{}%
\AgdaBound{x}\AgdaSymbol{)}\AgdaSpace{}%
\AgdaPostulate{Z}\AgdaSymbol{)}\AgdaSpace{}%
\AgdaBound{f}\AgdaSpace{}%
\AgdaSymbol{(λ}\AgdaSpace{}%
\AgdaBound{k}\AgdaSpace{}%
\AgdaSymbol{→}\AgdaSpace{}%
\AgdaFunction{indMax-Z}\AgdaSpace{}%
\AgdaSymbol{(}\AgdaBound{f}\AgdaSpace{}%
\AgdaBound{k}\AgdaSymbol{))}\<%
\\
%
\\[\AgdaEmptyExtraSkip]%
%
\>[8]\AgdaFunction{indMax-↑}\AgdaSpace{}%
\AgdaSymbol{:}\AgdaSpace{}%
\AgdaSymbol{∀}\AgdaSpace{}%
\AgdaSymbol{\{}\AgdaBound{t1}\AgdaSpace{}%
\AgdaBound{t2}\AgdaSymbol{\}}\AgdaSpace{}%
\AgdaSymbol{→}\AgdaSpace{}%
\AgdaFunction{indMax}\AgdaSpace{}%
\AgdaSymbol{(}\AgdaPostulate{↑}\AgdaSpace{}%
\AgdaBound{t1}\AgdaSymbol{)}\AgdaSpace{}%
\AgdaSymbol{(}\AgdaPostulate{↑}\AgdaSpace{}%
\AgdaBound{t2}\AgdaSymbol{)}\AgdaSpace{}%
\AgdaOperator{\AgdaDatatype{≡}}\AgdaSpace{}%
\AgdaPostulate{↑}\AgdaSpace{}%
\AgdaSymbol{(}\AgdaFunction{indMax}\AgdaSpace{}%
\AgdaBound{t1}\AgdaSpace{}%
\AgdaBound{t2}\AgdaSymbol{)}\<%
\\
%
\>[8]\AgdaFunction{indMax-↑}\AgdaSpace{}%
\AgdaSymbol{=}\AgdaSpace{}%
\AgdaInductiveConstructor{refl}\<%
\\
%
\\[\AgdaEmptyExtraSkip]%
%
\>[8]\AgdaFunction{indMax-≤Z}\AgdaSpace{}%
\AgdaSymbol{:}\AgdaSpace{}%
\AgdaSymbol{∀}\AgdaSpace{}%
\AgdaBound{t}\AgdaSpace{}%
\AgdaSymbol{→}\AgdaSpace{}%
\AgdaFunction{indMax}\AgdaSpace{}%
\AgdaBound{t}\AgdaSpace{}%
\AgdaPostulate{Z}\AgdaSpace{}%
\AgdaOperator{\AgdaPostulate{≤}}\AgdaSpace{}%
\AgdaBound{t}\<%
\\
%
\>[8]\AgdaFunction{indMax-≤Z}\AgdaSpace{}%
\AgdaInductiveConstructor{Z}\AgdaSpace{}%
\AgdaSymbol{=}\AgdaSpace{}%
\AgdaPostulate{≤-refl}\AgdaSpace{}%
\AgdaSymbol{\AgdaUnderscore{}}\<%
\\
%
\>[8]\AgdaFunction{indMax-≤Z}\AgdaSpace{}%
\AgdaSymbol{(}\AgdaInductiveConstructor{↑}\AgdaSpace{}%
\AgdaBound{t}\AgdaSymbol{)}\AgdaSpace{}%
\AgdaSymbol{=}\AgdaSpace{}%
\AgdaPostulate{≤-refl}\AgdaSpace{}%
\AgdaSymbol{\AgdaUnderscore{}}\<%
\\
%
\>[8]\AgdaFunction{indMax-≤Z}\AgdaSpace{}%
\AgdaSymbol{(}\AgdaInductiveConstructor{Lim}\AgdaSpace{}%
\AgdaBound{c}\AgdaSpace{}%
\AgdaBound{f}\AgdaSymbol{)}\AgdaSpace{}%
\AgdaSymbol{=}\AgdaSpace{}%
\AgdaPostulate{extLim}\AgdaSpace{}%
\AgdaSymbol{\AgdaUnderscore{}}\AgdaSpace{}%
\AgdaSymbol{\AgdaUnderscore{}}\AgdaSpace{}%
\AgdaSymbol{(λ}\AgdaSpace{}%
\AgdaBound{k}\AgdaSpace{}%
\AgdaSymbol{→}\AgdaSpace{}%
\AgdaFunction{indMax-≤Z}\AgdaSpace{}%
\AgdaSymbol{(}\AgdaBound{f}\AgdaSpace{}%
\AgdaBound{k}\AgdaSymbol{))}\<%
\\
%
\\[\AgdaEmptyExtraSkip]%
%
\>[6]\AgdaComment{--\ indMax-oneL\ :\ ∀\ \{t\}\ →\ indMax\ T1\ (↑\ t)\ ≤\ ↑\ t}\<%
\\
%
\>[6]\AgdaComment{--\ indMax-oneL\ \ =\ ≤-refl\ \AgdaUnderscore{}}\<%
\\
%
\\[\AgdaEmptyExtraSkip]%
%
\>[6]\AgdaComment{--\ indMax-oneR\ :\ ∀\ \{t\}\ →\ indMax\ (↑\ t)\ T1\ ≤\ ↑\ t}\<%
\\
%
\>[6]\AgdaComment{--\ indMax-oneR\ \{Z\}\ =\ ≤-sucMono\ (≤-refl\ \AgdaUnderscore{})}\<%
\\
%
\>[6]\AgdaComment{--\ indMax-oneR\ \{↑\ t\}\ =\ ≤-sucMono\ (≤-refl\ \AgdaUnderscore{})}\<%
\\
%
\>[6]\AgdaComment{--\ indMax-oneR\ \{Lim\ c\ f\}\ =\ ≤-sucMono\ (substPath\ (λ\ x\ →\ x\ ≤\ Lim\ c\ f)\ (sym\ (indMax-Z\ (Lim\ c\ f)))\ (≤-refl\ (Lim\ c\ f)))\ --\ rewrite\ ctop\ (indMax-Z\ (Lim\ c\ f))=\ ≤-refl\ \AgdaUnderscore{}}\<%
\\
%
\\[\AgdaEmptyExtraSkip]%
%
\\[\AgdaEmptyExtraSkip]%
\>[6][@{}l@{\AgdaIndent{0}}]%
\>[8]\AgdaFunction{indMax-limR}\AgdaSpace{}%
\AgdaSymbol{:}\AgdaSpace{}%
\AgdaSymbol{∀}%
\>[26]\AgdaSymbol{\{}\AgdaBound{c}\AgdaSpace{}%
\AgdaSymbol{:}\AgdaSpace{}%
\AgdaBound{ℂ}\AgdaSymbol{\}}\AgdaSpace{}%
\AgdaSymbol{(}\AgdaBound{f}\AgdaSpace{}%
\AgdaSymbol{:}\AgdaSpace{}%
\AgdaBound{El}%
\>[43]\AgdaBound{c}%
\>[46]\AgdaSymbol{→}\AgdaSpace{}%
\AgdaPostulate{Tree}\AgdaSymbol{)}\AgdaSpace{}%
\AgdaBound{t}\AgdaSpace{}%
\AgdaSymbol{→}\AgdaSpace{}%
\AgdaFunction{indMax}\AgdaSpace{}%
\AgdaBound{t}\AgdaSpace{}%
\AgdaSymbol{(}\AgdaPostulate{Lim}\AgdaSpace{}%
\AgdaBound{c}\AgdaSpace{}%
\AgdaBound{f}\AgdaSymbol{)}\AgdaSpace{}%
\AgdaOperator{\AgdaPostulate{≤}}\AgdaSpace{}%
\AgdaPostulate{Lim}\AgdaSpace{}%
\AgdaBound{c}\AgdaSpace{}%
\AgdaSymbol{(λ}\AgdaSpace{}%
\AgdaBound{k}\AgdaSpace{}%
\AgdaSymbol{→}\AgdaSpace{}%
\AgdaFunction{indMax}\AgdaSpace{}%
\AgdaBound{t}\AgdaSpace{}%
\AgdaSymbol{(}\AgdaBound{f}\AgdaSpace{}%
\AgdaBound{k}\AgdaSymbol{))}\<%
\\
%
\>[8]\AgdaFunction{indMax-limR}\AgdaSpace{}%
\AgdaBound{f}\AgdaSpace{}%
\AgdaInductiveConstructor{Z}\AgdaSpace{}%
\AgdaSymbol{=}\AgdaSpace{}%
\AgdaPostulate{≤-refl}\AgdaSpace{}%
\AgdaSymbol{\AgdaUnderscore{}}\<%
\\
%
\>[8]\AgdaFunction{indMax-limR}\AgdaSpace{}%
\AgdaBound{f}\AgdaSpace{}%
\AgdaSymbol{(}\AgdaInductiveConstructor{↑}\AgdaSpace{}%
\AgdaBound{t}\AgdaSymbol{)}\AgdaSpace{}%
\AgdaSymbol{=}\AgdaSpace{}%
\AgdaPostulate{extLim}\AgdaSpace{}%
\AgdaSymbol{\AgdaUnderscore{}}\AgdaSpace{}%
\AgdaSymbol{\AgdaUnderscore{}}\AgdaSpace{}%
\AgdaSymbol{λ}\AgdaSpace{}%
\AgdaBound{k}\AgdaSpace{}%
\AgdaSymbol{→}\AgdaSpace{}%
\AgdaPostulate{≤-refl}\AgdaSpace{}%
\AgdaSymbol{\AgdaUnderscore{}}\<%
\\
%
\>[8]\AgdaFunction{indMax-limR}\AgdaSpace{}%
\AgdaBound{f}\AgdaSpace{}%
\AgdaSymbol{(}\AgdaInductiveConstructor{Lim}\AgdaSpace{}%
\AgdaBound{c}\AgdaSpace{}%
\AgdaBound{f₁}\AgdaSymbol{)}\AgdaSpace{}%
\AgdaSymbol{=}\AgdaSpace{}%
\AgdaPostulate{≤-limiting}\AgdaSpace{}%
\AgdaSymbol{\AgdaUnderscore{}}\AgdaSpace{}%
\AgdaSymbol{λ}\AgdaSpace{}%
\AgdaBound{k}\AgdaSpace{}%
\AgdaSymbol{→}\AgdaSpace{}%
\AgdaPostulate{≤-trans}\AgdaSpace{}%
\AgdaSymbol{(}\AgdaFunction{indMax-limR}\AgdaSpace{}%
\AgdaBound{f}\AgdaSpace{}%
\AgdaSymbol{(}\AgdaBound{f₁}\AgdaSpace{}%
\AgdaBound{k}\AgdaSymbol{))}\AgdaSpace{}%
\AgdaSymbol{(}\AgdaPostulate{extLim}\AgdaSpace{}%
\AgdaSymbol{\AgdaUnderscore{}}\AgdaSpace{}%
\AgdaSymbol{\AgdaUnderscore{}}\AgdaSpace{}%
\AgdaSymbol{(λ}\AgdaSpace{}%
\AgdaBound{k2}\AgdaSpace{}%
\AgdaSymbol{→}\AgdaSpace{}%
\AgdaFunction{indMax-monoL}\AgdaSpace{}%
\AgdaSymbol{\{}\AgdaArgument{t1}\AgdaSpace{}%
\AgdaSymbol{=}\AgdaSpace{}%
\AgdaBound{f₁}\AgdaSpace{}%
\AgdaBound{k}\AgdaSymbol{\}}\AgdaSpace{}%
\AgdaSymbol{\{}\AgdaArgument{t1'}\AgdaSpace{}%
\AgdaSymbol{=}\AgdaSpace{}%
\AgdaPostulate{Lim}\AgdaSpace{}%
\AgdaBound{c}\AgdaSpace{}%
\AgdaBound{f₁}\AgdaSymbol{\}}\AgdaSpace{}%
\AgdaSymbol{\{}\AgdaArgument{t2}\AgdaSpace{}%
\AgdaSymbol{=}\AgdaSpace{}%
\AgdaBound{f}\AgdaSpace{}%
\AgdaBound{k2}\AgdaSymbol{\}}%
\>[160]\AgdaSymbol{(}\AgdaPostulate{≤-cocone}\AgdaSpace{}%
\AgdaSymbol{\AgdaUnderscore{}}\AgdaSpace{}%
\AgdaBound{k}\AgdaSpace{}%
\AgdaSymbol{(}\AgdaPostulate{≤-refl}\AgdaSpace{}%
\AgdaSymbol{\AgdaUnderscore{}))))}\<%
\\
%
\\[\AgdaEmptyExtraSkip]%
%
\\[\AgdaEmptyExtraSkip]%
%
\>[8]\AgdaFunction{indMax-commut}\AgdaSpace{}%
\AgdaSymbol{:}\AgdaSpace{}%
\AgdaSymbol{∀}\AgdaSpace{}%
\AgdaBound{t1}\AgdaSpace{}%
\AgdaBound{t2}\AgdaSpace{}%
\AgdaSymbol{→}\AgdaSpace{}%
\AgdaFunction{indMax}\AgdaSpace{}%
\AgdaBound{t1}\AgdaSpace{}%
\AgdaBound{t2}\AgdaSpace{}%
\AgdaOperator{\AgdaPostulate{≤}}\AgdaSpace{}%
\AgdaFunction{indMax}\AgdaSpace{}%
\AgdaBound{t2}\AgdaSpace{}%
\AgdaBound{t1}\<%
\\
%
\>[8]\AgdaFunction{indMax-commut}\AgdaSpace{}%
\AgdaBound{t1}\AgdaSpace{}%
\AgdaBound{t2}\AgdaSpace{}%
\AgdaKeyword{with}\AgdaSpace{}%
\AgdaFunction{indMaxView}\AgdaSpace{}%
\AgdaBound{t1}\AgdaSpace{}%
\AgdaBound{t2}\<%
\\
%
\>[8]\AgdaSymbol{...}\AgdaSpace{}%
\AgdaSymbol{|}\AgdaSpace{}%
\AgdaInductiveConstructor{IndMaxZ-L}\AgdaSpace{}%
\AgdaSymbol{=}\AgdaSpace{}%
\AgdaFunction{indMax-≤L}\<%
\\
%
\>[8]\AgdaSymbol{...}\AgdaSpace{}%
\AgdaSymbol{|}\AgdaSpace{}%
\AgdaInductiveConstructor{IndMaxZ-R}\AgdaSpace{}%
\AgdaSymbol{=}\AgdaSpace{}%
\AgdaPostulate{≤-refl}\AgdaSpace{}%
\AgdaSymbol{\AgdaUnderscore{}}\<%
\\
%
\>[8]\AgdaSymbol{...}\AgdaSpace{}%
\AgdaSymbol{|}\AgdaSpace{}%
\AgdaInductiveConstructor{IndMaxLim-R}\AgdaSpace{}%
\AgdaSymbol{\{}\AgdaArgument{f}\AgdaSpace{}%
\AgdaSymbol{=}\AgdaSpace{}%
\AgdaBound{f}\AgdaSymbol{\}}\AgdaSpace{}%
\AgdaBound{x}\AgdaSpace{}%
\AgdaSymbol{=}\AgdaSpace{}%
\AgdaPostulate{extLim}\AgdaSpace{}%
\AgdaSymbol{\AgdaUnderscore{}}\AgdaSpace{}%
\AgdaSymbol{\AgdaUnderscore{}}\AgdaSpace{}%
\AgdaSymbol{(λ}\AgdaSpace{}%
\AgdaBound{k}\AgdaSpace{}%
\AgdaSymbol{→}\AgdaSpace{}%
\AgdaFunction{indMax-commut}\AgdaSpace{}%
\AgdaBound{t1}\AgdaSpace{}%
\AgdaSymbol{(}\AgdaBound{f}\AgdaSpace{}%
\AgdaBound{k}\AgdaSymbol{))}\<%
\\
%
\>[8]\AgdaSymbol{...}\AgdaSpace{}%
\AgdaSymbol{|}\AgdaSpace{}%
\AgdaInductiveConstructor{IndMaxLim-Suc}\AgdaSpace{}%
\AgdaSymbol{\{}\AgdaArgument{t1}\AgdaSpace{}%
\AgdaSymbol{=}\AgdaSpace{}%
\AgdaBound{t1}\AgdaSymbol{\}}\AgdaSpace{}%
\AgdaSymbol{\{}\AgdaArgument{t2}\AgdaSpace{}%
\AgdaSymbol{=}\AgdaSpace{}%
\AgdaBound{t2}\AgdaSymbol{\}}\AgdaSpace{}%
\AgdaSymbol{=}\AgdaSpace{}%
\AgdaPostulate{≤-sucMono}\AgdaSpace{}%
\AgdaSymbol{(}\AgdaFunction{indMax-commut}\AgdaSpace{}%
\AgdaBound{t1}\AgdaSpace{}%
\AgdaBound{t2}\AgdaSymbol{)}\<%
\\
%
\>[8]\AgdaSymbol{...}\AgdaSpace{}%
\AgdaSymbol{|}\AgdaSpace{}%
\AgdaInductiveConstructor{IndMaxLim-L}\AgdaSpace{}%
\AgdaSymbol{\{}\AgdaArgument{c}\AgdaSpace{}%
\AgdaSymbol{=}\AgdaSpace{}%
\AgdaBound{c}\AgdaSymbol{\}}\AgdaSpace{}%
\AgdaSymbol{\{}\AgdaArgument{f}\AgdaSpace{}%
\AgdaSymbol{=}\AgdaSpace{}%
\AgdaBound{f}\AgdaSymbol{\}}\AgdaSpace{}%
\AgdaKeyword{with}\AgdaSpace{}%
\AgdaFunction{indMaxView}\AgdaSpace{}%
\AgdaBound{t2}\AgdaSpace{}%
\AgdaBound{t1}\<%
\\
%
\>[8]\AgdaSymbol{...}\AgdaSpace{}%
\AgdaSymbol{|}\AgdaSpace{}%
\AgdaInductiveConstructor{IndMaxZ-L}\AgdaSpace{}%
\AgdaSymbol{=}\AgdaSpace{}%
\AgdaPostulate{extLim}\AgdaSpace{}%
\AgdaSymbol{\AgdaUnderscore{}}\AgdaSpace{}%
\AgdaSymbol{\AgdaUnderscore{}}\AgdaSpace{}%
\AgdaSymbol{λ}\AgdaSpace{}%
\AgdaBound{k}\AgdaSpace{}%
\AgdaSymbol{→}\AgdaSpace{}%
\AgdaFunction{indMax-Z}\AgdaSpace{}%
\AgdaSymbol{(}\AgdaBound{f}\AgdaSpace{}%
\AgdaBound{k}\AgdaSymbol{)}\<%
\\
%
\>[8]\AgdaSymbol{...}\AgdaSpace{}%
\AgdaSymbol{|}\AgdaSpace{}%
\AgdaInductiveConstructor{IndMaxLim-R}\AgdaSpace{}%
\AgdaBound{x}\AgdaSpace{}%
\AgdaSymbol{=}\AgdaSpace{}%
\AgdaPostulate{extLim}\AgdaSpace{}%
\AgdaSymbol{\AgdaUnderscore{}}\AgdaSpace{}%
\AgdaSymbol{\AgdaUnderscore{}}\AgdaSpace{}%
\AgdaSymbol{(λ}\AgdaSpace{}%
\AgdaBound{k}\AgdaSpace{}%
\AgdaSymbol{→}\AgdaSpace{}%
\AgdaFunction{indMax-commut}\AgdaSpace{}%
\AgdaSymbol{(}\AgdaBound{f}\AgdaSpace{}%
\AgdaBound{k}\AgdaSymbol{)}\AgdaSpace{}%
\AgdaBound{t2}\AgdaSymbol{)}\<%
\\
%
\>[8]\AgdaSymbol{...}%
\>[1396I]\AgdaSymbol{|}\AgdaSpace{}%
\AgdaInductiveConstructor{IndMaxLim-L}\AgdaSpace{}%
\AgdaSymbol{\{}\AgdaArgument{c}\AgdaSpace{}%
\AgdaSymbol{=}\AgdaSpace{}%
\AgdaBound{c2}\AgdaSymbol{\}}\AgdaSpace{}%
\AgdaSymbol{\{}\AgdaArgument{f}\AgdaSpace{}%
\AgdaSymbol{=}\AgdaSpace{}%
\AgdaBound{f2}\AgdaSymbol{\}}\AgdaSpace{}%
\AgdaSymbol{=}\<%
\\
\>[.][@{}l@{}]\<[1396I]%
\>[12]\AgdaPostulate{≤-trans}\AgdaSpace{}%
\AgdaSymbol{(}\AgdaPostulate{extLim}\AgdaSpace{}%
\AgdaSymbol{\AgdaUnderscore{}}\AgdaSpace{}%
\AgdaSymbol{\AgdaUnderscore{}}\AgdaSpace{}%
\AgdaSymbol{λ}\AgdaSpace{}%
\AgdaBound{k}\AgdaSpace{}%
\AgdaSymbol{→}\AgdaSpace{}%
\AgdaFunction{indMax-limR}\AgdaSpace{}%
\AgdaBound{f2}\AgdaSpace{}%
\AgdaSymbol{(}\AgdaBound{f}\AgdaSpace{}%
\AgdaBound{k}\AgdaSymbol{))}\<%
\\
%
\>[12]\AgdaSymbol{(}\AgdaPostulate{≤-trans}\AgdaSpace{}%
\AgdaSymbol{(}\AgdaPostulate{≤-limiting}\AgdaSpace{}%
\AgdaSymbol{\AgdaUnderscore{}}\AgdaSpace{}%
\AgdaSymbol{(λ}\AgdaSpace{}%
\AgdaBound{k}\AgdaSpace{}%
\AgdaSymbol{→}\AgdaSpace{}%
\AgdaPostulate{≤-limiting}\AgdaSpace{}%
\AgdaSymbol{\AgdaUnderscore{}}\AgdaSpace{}%
\AgdaSymbol{λ}\AgdaSpace{}%
\AgdaBound{k2}\AgdaSpace{}%
\AgdaSymbol{→}\AgdaSpace{}%
\AgdaPostulate{≤-cocone}\AgdaSpace{}%
\AgdaSymbol{\AgdaUnderscore{}}\AgdaSpace{}%
\AgdaBound{k2}\AgdaSpace{}%
\AgdaSymbol{(}\AgdaPostulate{≤-cocone}\AgdaSpace{}%
\AgdaSymbol{\AgdaUnderscore{}}\AgdaSpace{}%
\AgdaBound{k}\AgdaSpace{}%
\AgdaSymbol{(}\AgdaPostulate{≤-refl}\AgdaSpace{}%
\AgdaSymbol{\AgdaUnderscore{}))))}\<%
\\
%
\>[12]\AgdaSymbol{(}\AgdaPostulate{≤-trans}\AgdaSpace{}%
\AgdaSymbol{(}\AgdaPostulate{≤-refl}\AgdaSpace{}%
\AgdaSymbol{(}\AgdaPostulate{Lim}\AgdaSpace{}%
\AgdaBound{c2}\AgdaSpace{}%
\AgdaSymbol{λ}\AgdaSpace{}%
\AgdaBound{k2}\AgdaSpace{}%
\AgdaSymbol{→}\AgdaSpace{}%
\AgdaPostulate{Lim}\AgdaSpace{}%
\AgdaBound{c}\AgdaSpace{}%
\AgdaSymbol{λ}\AgdaSpace{}%
\AgdaBound{k}\AgdaSpace{}%
\AgdaSymbol{→}\AgdaSpace{}%
\AgdaFunction{indMax}\AgdaSpace{}%
\AgdaSymbol{(}\AgdaBound{f}\AgdaSpace{}%
\AgdaBound{k}\AgdaSymbol{)}\AgdaSpace{}%
\AgdaSymbol{(}\AgdaBound{f2}\AgdaSpace{}%
\AgdaBound{k2}\AgdaSymbol{)))}\<%
\\
%
\>[12]\AgdaSymbol{(}\AgdaPostulate{extLim}\AgdaSpace{}%
\AgdaSymbol{\AgdaUnderscore{}}\AgdaSpace{}%
\AgdaSymbol{\AgdaUnderscore{}}\AgdaSpace{}%
\AgdaSymbol{(λ}\AgdaSpace{}%
\AgdaBound{k2}\AgdaSpace{}%
\AgdaSymbol{→}\AgdaSpace{}%
\AgdaPostulate{≤-limiting}\AgdaSpace{}%
\AgdaSymbol{\AgdaUnderscore{}}\AgdaSpace{}%
\AgdaSymbol{λ}\AgdaSpace{}%
\AgdaBound{k1}\AgdaSpace{}%
\AgdaSymbol{→}\AgdaSpace{}%
\AgdaPostulate{≤-trans}\AgdaSpace{}%
\AgdaSymbol{(}\AgdaFunction{indMax-commut}\AgdaSpace{}%
\AgdaSymbol{(}\AgdaBound{f}\AgdaSpace{}%
\AgdaBound{k1}\AgdaSymbol{)}\AgdaSpace{}%
\AgdaSymbol{(}\AgdaBound{f2}\AgdaSpace{}%
\AgdaBound{k2}\AgdaSymbol{))}\AgdaSpace{}%
\AgdaSymbol{(}\AgdaFunction{indMax-monoR}\AgdaSpace{}%
\AgdaSymbol{\{}\AgdaArgument{t1}\AgdaSpace{}%
\AgdaSymbol{=}\AgdaSpace{}%
\AgdaBound{f2}\AgdaSpace{}%
\AgdaBound{k2}\AgdaSymbol{\}}\AgdaSpace{}%
\AgdaSymbol{\{}\AgdaArgument{t2}\AgdaSpace{}%
\AgdaSymbol{=}\AgdaSpace{}%
\AgdaBound{f}\AgdaSpace{}%
\AgdaBound{k1}\AgdaSymbol{\}}\AgdaSpace{}%
\AgdaSymbol{\{}\AgdaArgument{t2'}\AgdaSpace{}%
\AgdaSymbol{=}\AgdaSpace{}%
\AgdaPostulate{Lim}\AgdaSpace{}%
\AgdaBound{c}\AgdaSpace{}%
\AgdaBound{f}\AgdaSymbol{\}}\AgdaSpace{}%
\AgdaSymbol{(}\AgdaPostulate{≤-cocone}\AgdaSpace{}%
\AgdaSymbol{\AgdaUnderscore{}}\AgdaSpace{}%
\AgdaBound{k1}\AgdaSpace{}%
\AgdaSymbol{(}\AgdaPostulate{≤-refl}\AgdaSpace{}%
\AgdaSymbol{\AgdaUnderscore{})))))))}\<%
\\
%
\\[\AgdaEmptyExtraSkip]%
%
\\[\AgdaEmptyExtraSkip]%
%
\>[8]\AgdaFunction{indMax-assocL}\AgdaSpace{}%
\AgdaSymbol{:}\AgdaSpace{}%
\AgdaSymbol{∀}\AgdaSpace{}%
\AgdaBound{t1}\AgdaSpace{}%
\AgdaBound{t2}\AgdaSpace{}%
\AgdaBound{t3}\AgdaSpace{}%
\AgdaSymbol{→}\AgdaSpace{}%
\AgdaFunction{indMax}\AgdaSpace{}%
\AgdaBound{t1}\AgdaSpace{}%
\AgdaSymbol{(}\AgdaFunction{indMax}\AgdaSpace{}%
\AgdaBound{t2}\AgdaSpace{}%
\AgdaBound{t3}\AgdaSymbol{)}\AgdaSpace{}%
\AgdaOperator{\AgdaPostulate{≤}}\AgdaSpace{}%
\AgdaFunction{indMax}\AgdaSpace{}%
\AgdaSymbol{(}\AgdaFunction{indMax}\AgdaSpace{}%
\AgdaBound{t1}\AgdaSpace{}%
\AgdaBound{t2}\AgdaSymbol{)}\AgdaSpace{}%
\AgdaBound{t3}\<%
\\
%
\>[8]\AgdaFunction{indMax-assocL}\AgdaSpace{}%
\AgdaBound{t1}\AgdaSpace{}%
\AgdaBound{t2}\AgdaSpace{}%
\AgdaBound{t3}\AgdaSpace{}%
\AgdaKeyword{with}\AgdaSpace{}%
\AgdaFunction{indMaxView}\AgdaSpace{}%
\AgdaBound{t2}\AgdaSpace{}%
\AgdaBound{t3}\AgdaSpace{}%
\AgdaKeyword{in}\AgdaSpace{}%
\AgdaArgument{eq23}\<%
\\
%
\>[8]\AgdaSymbol{...}\AgdaSpace{}%
\AgdaSymbol{|}\AgdaSpace{}%
\AgdaInductiveConstructor{IndMaxZ-L}\AgdaSpace{}%
\AgdaSymbol{=}\AgdaSpace{}%
\AgdaFunction{indMax-monoL}\AgdaSpace{}%
\AgdaSymbol{\{}\AgdaArgument{t1}\AgdaSpace{}%
\AgdaSymbol{=}\AgdaSpace{}%
\AgdaBound{t1}\AgdaSymbol{\}}\AgdaSpace{}%
\AgdaSymbol{\{}\AgdaArgument{t1'}\AgdaSpace{}%
\AgdaSymbol{=}\AgdaSpace{}%
\AgdaFunction{indMax}\AgdaSpace{}%
\AgdaBound{t1}\AgdaSpace{}%
\AgdaPostulate{Z}\AgdaSymbol{\}}\AgdaSpace{}%
\AgdaSymbol{\{}\AgdaArgument{t2}\AgdaSpace{}%
\AgdaSymbol{=}\AgdaSpace{}%
\AgdaBound{t3}\AgdaSymbol{\}}\AgdaSpace{}%
\AgdaFunction{indMax-≤L}\<%
\\
%
\>[8]\AgdaSymbol{...}\AgdaSpace{}%
\AgdaSymbol{|}\AgdaSpace{}%
\AgdaInductiveConstructor{IndMaxZ-R}\AgdaSpace{}%
\AgdaSymbol{=}\AgdaSpace{}%
\AgdaFunction{indMax-≤L}\<%
\\
%
\>[8]\AgdaSymbol{...}\AgdaSpace{}%
\AgdaSymbol{|}\AgdaSpace{}%
\AgdaBound{m}\AgdaSpace{}%
\AgdaKeyword{with}\AgdaSpace{}%
\AgdaFunction{indMaxView}\AgdaSpace{}%
\AgdaBound{t1}\AgdaSpace{}%
\AgdaBound{t2}\<%
\\
%
\>[8]\AgdaSymbol{...}\AgdaSpace{}%
\AgdaSymbol{|}\AgdaSpace{}%
\AgdaInductiveConstructor{IndMaxZ-L}\AgdaSpace{}%
\AgdaKeyword{rewrite}\AgdaSpace{}%
\AgdaFunction{sym}\AgdaSpace{}%
\AgdaBound{eq23}\AgdaSpace{}%
\AgdaSymbol{=}\AgdaSpace{}%
\AgdaPostulate{≤-refl}\AgdaSpace{}%
\AgdaSymbol{\AgdaUnderscore{}}\<%
\\
%
\>[8]\AgdaSymbol{...}\AgdaSpace{}%
\AgdaSymbol{|}\AgdaSpace{}%
\AgdaInductiveConstructor{IndMaxZ-R}\AgdaSpace{}%
\AgdaKeyword{rewrite}\AgdaSpace{}%
\AgdaFunction{sym}\AgdaSpace{}%
\AgdaBound{eq23}\AgdaSpace{}%
\AgdaSymbol{=}\AgdaSpace{}%
\AgdaPostulate{≤-refl}\AgdaSpace{}%
\AgdaSymbol{\AgdaUnderscore{}}\<%
\\
%
\>[8]\AgdaSymbol{...}\AgdaSpace{}%
\AgdaSymbol{|}\AgdaSpace{}%
\AgdaInductiveConstructor{IndMaxLim-R}\AgdaSpace{}%
\AgdaSymbol{\{}\AgdaArgument{f}\AgdaSpace{}%
\AgdaSymbol{=}\AgdaSpace{}%
\AgdaBound{f}\AgdaSymbol{\}}\AgdaSpace{}%
\AgdaBound{x}\AgdaSpace{}%
\AgdaKeyword{rewrite}\AgdaSpace{}%
\AgdaFunction{sym}\AgdaSpace{}%
\AgdaBound{eq23}\AgdaSpace{}%
\AgdaSymbol{=}\AgdaSpace{}%
\AgdaPostulate{≤-trans}\AgdaSpace{}%
\AgdaSymbol{(}\AgdaFunction{indMax-limR}\AgdaSpace{}%
\AgdaSymbol{(λ}\AgdaSpace{}%
\AgdaBound{x}\AgdaSpace{}%
\AgdaSymbol{→}\AgdaSpace{}%
\AgdaFunction{indMax}\AgdaSpace{}%
\AgdaSymbol{(}\AgdaBound{f}\AgdaSpace{}%
\AgdaBound{x}\AgdaSymbol{)}\AgdaSpace{}%
\AgdaBound{t3}\AgdaSymbol{)}\AgdaSpace{}%
\AgdaBound{t1}\AgdaSymbol{)}\AgdaSpace{}%
\AgdaSymbol{(}\AgdaPostulate{extLim}\AgdaSpace{}%
\AgdaSymbol{\AgdaUnderscore{}}\AgdaSpace{}%
\AgdaSymbol{\AgdaUnderscore{}}\AgdaSpace{}%
\AgdaSymbol{λ}\AgdaSpace{}%
\AgdaBound{k}\AgdaSpace{}%
\AgdaSymbol{→}\AgdaSpace{}%
\AgdaFunction{indMax-assocL}\AgdaSpace{}%
\AgdaBound{t1}\AgdaSpace{}%
\AgdaSymbol{(}\AgdaBound{f}\AgdaSpace{}%
\AgdaBound{k}\AgdaSymbol{)}\AgdaSpace{}%
\AgdaBound{t3}\AgdaSymbol{)}\AgdaSpace{}%
\AgdaComment{--\ f,indMax-limR\ f\ t1}\<%
\\
%
\>[8]\AgdaFunction{indMax-assocL}\AgdaSpace{}%
\AgdaDottedPattern{\AgdaSymbol{.(}}\AgdaDottedPattern{\AgdaPostulate{↑}}\AgdaSpace{}%
\AgdaDottedPattern{\AgdaSymbol{\AgdaUnderscore{})}}\AgdaSpace{}%
\AgdaDottedPattern{\AgdaSymbol{.(}}\AgdaDottedPattern{\AgdaPostulate{↑}}\AgdaSpace{}%
\AgdaDottedPattern{\AgdaSymbol{\AgdaUnderscore{})}}\AgdaSpace{}%
\AgdaDottedPattern{\AgdaSymbol{.}}\AgdaDottedPattern{\AgdaPostulate{Z}}\AgdaSpace{}%
\AgdaSymbol{|}\AgdaSpace{}%
\AgdaInductiveConstructor{IndMaxZ-R}%
\>[52]\AgdaSymbol{|}\AgdaSpace{}%
\AgdaInductiveConstructor{IndMaxLim-Suc}\AgdaSpace{}%
\AgdaSymbol{=}\AgdaSpace{}%
\AgdaPostulate{≤-refl}\AgdaSpace{}%
\AgdaSymbol{\AgdaUnderscore{}}\<%
\\
%
\>[8]\AgdaFunction{indMax-assocL}\AgdaSpace{}%
\AgdaBound{t1}\AgdaSpace{}%
\AgdaBound{t2}\AgdaSpace{}%
\AgdaDottedPattern{\AgdaSymbol{.(}}\AgdaDottedPattern{\AgdaPostulate{Lim}}\AgdaSpace{}%
\AgdaDottedPattern{\AgdaSymbol{\AgdaUnderscore{}}}\AgdaSpace{}%
\AgdaDottedPattern{\AgdaSymbol{\AgdaUnderscore{})}}\AgdaSpace{}%
\AgdaSymbol{|}\AgdaSpace{}%
\AgdaInductiveConstructor{IndMaxLim-R}\AgdaSpace{}%
\AgdaSymbol{\{}\AgdaArgument{f}\AgdaSpace{}%
\AgdaSymbol{=}\AgdaSpace{}%
\AgdaBound{f}\AgdaSymbol{\}}\AgdaSpace{}%
\AgdaBound{x}%
\>[65]\AgdaSymbol{|}\AgdaSpace{}%
\AgdaInductiveConstructor{IndMaxLim-Suc}\AgdaSpace{}%
\AgdaSymbol{=}\AgdaSpace{}%
\AgdaPostulate{extLim}\AgdaSpace{}%
\AgdaSymbol{\AgdaUnderscore{}}\AgdaSpace{}%
\AgdaSymbol{\AgdaUnderscore{}}\AgdaSpace{}%
\AgdaSymbol{λ}\AgdaSpace{}%
\AgdaBound{k}\AgdaSpace{}%
\AgdaSymbol{→}\AgdaSpace{}%
\AgdaFunction{indMax-assocL}\AgdaSpace{}%
\AgdaBound{t1}\AgdaSpace{}%
\AgdaBound{t2}\AgdaSpace{}%
\AgdaSymbol{(}\AgdaBound{f}\AgdaSpace{}%
\AgdaBound{k}\AgdaSymbol{)}\<%
\\
%
\>[8]\AgdaFunction{indMax-assocL}\AgdaSpace{}%
\AgdaSymbol{(}\AgdaInductiveConstructor{↑}\AgdaSpace{}%
\AgdaBound{t1}\AgdaSymbol{)}\AgdaSpace{}%
\AgdaSymbol{(}\AgdaInductiveConstructor{↑}\AgdaSpace{}%
\AgdaBound{t2}\AgdaSymbol{)}\AgdaSpace{}%
\AgdaSymbol{(}\AgdaInductiveConstructor{↑}\AgdaSpace{}%
\AgdaBound{t3}\AgdaSymbol{)}\AgdaSpace{}%
\AgdaSymbol{|}\AgdaSpace{}%
\AgdaInductiveConstructor{IndMaxLim-Suc}%
\>[60]\AgdaSymbol{|}\AgdaSpace{}%
\AgdaInductiveConstructor{IndMaxLim-Suc}\AgdaSpace{}%
\AgdaSymbol{=}\AgdaSpace{}%
\AgdaPostulate{≤-sucMono}\AgdaSpace{}%
\AgdaSymbol{(}\AgdaFunction{indMax-assocL}\AgdaSpace{}%
\AgdaBound{t1}\AgdaSpace{}%
\AgdaBound{t2}\AgdaSpace{}%
\AgdaBound{t3}\AgdaSymbol{)}\<%
\\
%
\>[8]\AgdaSymbol{...}\AgdaSpace{}%
\AgdaSymbol{|}\AgdaSpace{}%
\AgdaInductiveConstructor{IndMaxLim-L}\AgdaSpace{}%
\AgdaSymbol{\{}\AgdaArgument{f}\AgdaSpace{}%
\AgdaSymbol{=}\AgdaSpace{}%
\AgdaBound{f}\AgdaSymbol{\}}\AgdaSpace{}%
\AgdaKeyword{rewrite}\AgdaSpace{}%
\AgdaFunction{sym}\AgdaSpace{}%
\AgdaBound{eq23}\AgdaSpace{}%
\AgdaSymbol{=}\AgdaSpace{}%
\AgdaPostulate{extLim}\AgdaSpace{}%
\AgdaSymbol{\AgdaUnderscore{}}\AgdaSpace{}%
\AgdaSymbol{\AgdaUnderscore{}}\AgdaSpace{}%
\AgdaSymbol{λ}\AgdaSpace{}%
\AgdaBound{k}\AgdaSpace{}%
\AgdaSymbol{→}\AgdaSpace{}%
\AgdaFunction{indMax-assocL}\AgdaSpace{}%
\AgdaSymbol{(}\AgdaBound{f}\AgdaSpace{}%
\AgdaBound{k}\AgdaSymbol{)}\AgdaSpace{}%
\AgdaBound{t2}\AgdaSpace{}%
\AgdaBound{t3}\<%
\\
%
\\[\AgdaEmptyExtraSkip]%
%
\\[\AgdaEmptyExtraSkip]%
%
\\[\AgdaEmptyExtraSkip]%
%
\>[8]\AgdaFunction{indMax-assocR}\AgdaSpace{}%
\AgdaSymbol{:}\AgdaSpace{}%
\AgdaSymbol{∀}\AgdaSpace{}%
\AgdaBound{t1}\AgdaSpace{}%
\AgdaBound{t2}\AgdaSpace{}%
\AgdaBound{t3}\AgdaSpace{}%
\AgdaSymbol{→}%
\>[38]\AgdaFunction{indMax}\AgdaSpace{}%
\AgdaSymbol{(}\AgdaFunction{indMax}\AgdaSpace{}%
\AgdaBound{t1}\AgdaSpace{}%
\AgdaBound{t2}\AgdaSymbol{)}\AgdaSpace{}%
\AgdaBound{t3}\AgdaSpace{}%
\AgdaOperator{\AgdaPostulate{≤}}\AgdaSpace{}%
\AgdaFunction{indMax}\AgdaSpace{}%
\AgdaBound{t1}\AgdaSpace{}%
\AgdaSymbol{(}\AgdaFunction{indMax}\AgdaSpace{}%
\AgdaBound{t2}\AgdaSpace{}%
\AgdaBound{t3}\AgdaSymbol{)}\<%
\\
%
\>[8]\AgdaFunction{indMax-assocR}\AgdaSpace{}%
\AgdaBound{t1}\AgdaSpace{}%
\AgdaBound{t2}\AgdaSpace{}%
\AgdaBound{t3}\AgdaSpace{}%
\AgdaSymbol{=}\AgdaSpace{}%
\AgdaPostulate{≤-trans}\AgdaSpace{}%
\AgdaSymbol{(}\AgdaFunction{indMax-commut}\AgdaSpace{}%
\AgdaSymbol{(}\AgdaFunction{indMax}\AgdaSpace{}%
\AgdaBound{t1}\AgdaSpace{}%
\AgdaBound{t2}\AgdaSymbol{)}\AgdaSpace{}%
\AgdaBound{t3}\AgdaSymbol{)}\AgdaSpace{}%
\AgdaSymbol{(}\AgdaPostulate{≤-trans}\AgdaSpace{}%
\AgdaSymbol{(}\AgdaFunction{indMax-monoR}\AgdaSpace{}%
\AgdaSymbol{\{}\AgdaArgument{t1}\AgdaSpace{}%
\AgdaSymbol{=}\AgdaSpace{}%
\AgdaBound{t3}\AgdaSymbol{\}}\AgdaSpace{}%
\AgdaSymbol{(}\AgdaFunction{indMax-commut}\AgdaSpace{}%
\AgdaBound{t1}\AgdaSpace{}%
\AgdaBound{t2}\AgdaSymbol{))}\<%
\\
\>[8][@{}l@{\AgdaIndent{0}}]%
\>[12]\AgdaSymbol{(}\AgdaPostulate{≤-trans}\AgdaSpace{}%
\AgdaSymbol{(}\AgdaFunction{indMax-assocL}\AgdaSpace{}%
\AgdaBound{t3}\AgdaSpace{}%
\AgdaBound{t2}\AgdaSpace{}%
\AgdaBound{t1}\AgdaSymbol{)}\AgdaSpace{}%
\AgdaSymbol{(}\AgdaPostulate{≤-trans}\AgdaSpace{}%
\AgdaSymbol{(}\AgdaFunction{indMax-commut}\AgdaSpace{}%
\AgdaSymbol{(}\AgdaFunction{indMax}\AgdaSpace{}%
\AgdaBound{t3}\AgdaSpace{}%
\AgdaBound{t2}\AgdaSymbol{)}\AgdaSpace{}%
\AgdaBound{t1}\AgdaSymbol{)}\AgdaSpace{}%
\AgdaSymbol{(}\AgdaFunction{indMax-monoR}\AgdaSpace{}%
\AgdaSymbol{\{}\AgdaArgument{t1}\AgdaSpace{}%
\AgdaSymbol{=}\AgdaSpace{}%
\AgdaBound{t1}\AgdaSymbol{\}}\AgdaSpace{}%
\AgdaSymbol{(}\AgdaFunction{indMax-commut}\AgdaSpace{}%
\AgdaBound{t3}\AgdaSpace{}%
\AgdaBound{t2}\AgdaSymbol{)))))}\<%
\\
%
\\[\AgdaEmptyExtraSkip]%
%
\\[\AgdaEmptyExtraSkip]%
%
\>[8]\AgdaFunction{indMax-swap4}\AgdaSpace{}%
\AgdaSymbol{:}\AgdaSpace{}%
\AgdaSymbol{∀}\AgdaSpace{}%
\AgdaSymbol{\{}\AgdaBound{t1}\AgdaSpace{}%
\AgdaBound{t1'}\AgdaSpace{}%
\AgdaBound{t2}\AgdaSpace{}%
\AgdaBound{t2'}\AgdaSymbol{\}}\AgdaSpace{}%
\AgdaSymbol{→}\AgdaSpace{}%
\AgdaFunction{indMax}\AgdaSpace{}%
\AgdaSymbol{(}\AgdaFunction{indMax}\AgdaSpace{}%
\AgdaBound{t1}\AgdaSpace{}%
\AgdaBound{t1'}\AgdaSymbol{)}\AgdaSpace{}%
\AgdaSymbol{(}\AgdaFunction{indMax}\AgdaSpace{}%
\AgdaBound{t2}\AgdaSpace{}%
\AgdaBound{t2'}\AgdaSymbol{)}\AgdaSpace{}%
\AgdaOperator{\AgdaPostulate{≤}}\AgdaSpace{}%
\AgdaFunction{indMax}\AgdaSpace{}%
\AgdaSymbol{(}\AgdaFunction{indMax}\AgdaSpace{}%
\AgdaBound{t1}\AgdaSpace{}%
\AgdaBound{t2}\AgdaSymbol{)}\AgdaSpace{}%
\AgdaSymbol{(}\AgdaFunction{indMax}\AgdaSpace{}%
\AgdaBound{t1'}\AgdaSpace{}%
\AgdaBound{t2'}\AgdaSymbol{)}\<%
\\
%
\>[8]\AgdaFunction{indMax-swap4}\AgdaSpace{}%
\AgdaSymbol{\{}\AgdaBound{t1}\AgdaSymbol{\}\{}\AgdaBound{t1'}\AgdaSymbol{\}\{}\AgdaBound{t2}\AgdaSpace{}%
\AgdaSymbol{\}\{}\AgdaBound{t2'}\AgdaSymbol{\}}\AgdaSpace{}%
\AgdaSymbol{=}\<%
\\
\>[8][@{}l@{\AgdaIndent{0}}]%
\>[12]\AgdaFunction{indMax-assocL}\AgdaSpace{}%
\AgdaSymbol{(}\AgdaFunction{indMax}\AgdaSpace{}%
\AgdaBound{t1}\AgdaSpace{}%
\AgdaBound{t1'}\AgdaSymbol{)}\AgdaSpace{}%
\AgdaBound{t2}\AgdaSpace{}%
\AgdaBound{t2'}\<%
\\
%
\>[12]\AgdaOperator{\AgdaPostulate{≤⨟}}\AgdaSpace{}%
\AgdaFunction{indMax-monoL}\AgdaSpace{}%
\AgdaSymbol{\{}\AgdaArgument{t1}\AgdaSpace{}%
\AgdaSymbol{=}\AgdaSpace{}%
\AgdaFunction{indMax}\AgdaSpace{}%
\AgdaSymbol{(}\AgdaFunction{indMax}\AgdaSpace{}%
\AgdaBound{t1}\AgdaSpace{}%
\AgdaBound{t1'}\AgdaSymbol{)}\AgdaSpace{}%
\AgdaBound{t2}\AgdaSymbol{\}}\AgdaSpace{}%
\AgdaSymbol{\{}\AgdaArgument{t2}\AgdaSpace{}%
\AgdaSymbol{=}\AgdaSpace{}%
\AgdaBound{t2'}\AgdaSymbol{\}}\<%
\\
%
\>[12]\AgdaSymbol{(}\AgdaFunction{indMax-assocR}\AgdaSpace{}%
\AgdaBound{t1}\AgdaSpace{}%
\AgdaBound{t1'}\AgdaSpace{}%
\AgdaBound{t2}\AgdaSpace{}%
\AgdaOperator{\AgdaPostulate{≤⨟}}\AgdaSpace{}%
\AgdaFunction{indMax-monoR}\AgdaSpace{}%
\AgdaSymbol{\{}\AgdaArgument{t1}\AgdaSpace{}%
\AgdaSymbol{=}\AgdaSpace{}%
\AgdaBound{t1}\AgdaSymbol{\}}\AgdaSpace{}%
\AgdaSymbol{(}\AgdaFunction{indMax-commut}\AgdaSpace{}%
\AgdaBound{t1'}\AgdaSpace{}%
\AgdaBound{t2}\AgdaSymbol{)}\AgdaSpace{}%
\AgdaOperator{\AgdaPostulate{≤⨟}}\AgdaSpace{}%
\AgdaFunction{indMax-assocL}\AgdaSpace{}%
\AgdaBound{t1}\AgdaSpace{}%
\AgdaBound{t2}\AgdaSpace{}%
\AgdaBound{t1'}\AgdaSymbol{)}\<%
\\
%
\>[12]\AgdaOperator{\AgdaPostulate{≤⨟}}\AgdaSpace{}%
\AgdaFunction{indMax-assocR}\AgdaSpace{}%
\AgdaSymbol{(}\AgdaFunction{indMax}\AgdaSpace{}%
\AgdaBound{t1}\AgdaSpace{}%
\AgdaBound{t2}\AgdaSymbol{)}\AgdaSpace{}%
\AgdaBound{t1'}\AgdaSpace{}%
\AgdaBound{t2'}\<%
\\
%
\\[\AgdaEmptyExtraSkip]%
%
\>[8]\AgdaFunction{indMax-swap6}\AgdaSpace{}%
\AgdaSymbol{:}\AgdaSpace{}%
\AgdaSymbol{∀}\AgdaSpace{}%
\AgdaSymbol{\{}\AgdaBound{t1}\AgdaSpace{}%
\AgdaBound{t2}\AgdaSpace{}%
\AgdaBound{t3}\AgdaSpace{}%
\AgdaBound{t1'}\AgdaSpace{}%
\AgdaBound{t2'}\AgdaSpace{}%
\AgdaBound{t3'}\AgdaSymbol{\}}\AgdaSpace{}%
\AgdaSymbol{→}\AgdaSpace{}%
\AgdaFunction{indMax}\AgdaSpace{}%
\AgdaSymbol{(}\AgdaFunction{indMax}\AgdaSpace{}%
\AgdaBound{t1}\AgdaSpace{}%
\AgdaBound{t1'}\AgdaSymbol{)}\AgdaSpace{}%
\AgdaSymbol{(}\AgdaFunction{indMax}\AgdaSpace{}%
\AgdaSymbol{(}\AgdaFunction{indMax}\AgdaSpace{}%
\AgdaBound{t2}\AgdaSpace{}%
\AgdaBound{t2'}\AgdaSymbol{)}\AgdaSpace{}%
\AgdaSymbol{(}\AgdaFunction{indMax}\AgdaSpace{}%
\AgdaBound{t3}\AgdaSpace{}%
\AgdaBound{t3'}\AgdaSymbol{))}\AgdaSpace{}%
\AgdaOperator{\AgdaPostulate{≤}}\AgdaSpace{}%
\AgdaFunction{indMax}\AgdaSpace{}%
\AgdaSymbol{(}\AgdaFunction{indMax}\AgdaSpace{}%
\AgdaBound{t1}\AgdaSpace{}%
\AgdaSymbol{(}\AgdaFunction{indMax}\AgdaSpace{}%
\AgdaBound{t2}\AgdaSpace{}%
\AgdaBound{t3}\AgdaSymbol{))}\AgdaSpace{}%
\AgdaSymbol{(}\AgdaFunction{indMax}\AgdaSpace{}%
\AgdaBound{t1'}\AgdaSpace{}%
\AgdaSymbol{(}\AgdaFunction{indMax}\AgdaSpace{}%
\AgdaBound{t2'}\AgdaSpace{}%
\AgdaBound{t3'}\AgdaSymbol{))}\<%
\\
%
\>[8]\AgdaFunction{indMax-swap6}\AgdaSpace{}%
\AgdaSymbol{\{}\AgdaBound{t1}\AgdaSymbol{\}}\AgdaSpace{}%
\AgdaSymbol{\{}\AgdaBound{t2}\AgdaSymbol{\}}\AgdaSpace{}%
\AgdaSymbol{\{}\AgdaBound{t3}\AgdaSymbol{\}}\AgdaSpace{}%
\AgdaSymbol{\{}\AgdaBound{t1'}\AgdaSymbol{\}}\AgdaSpace{}%
\AgdaSymbol{\{}\AgdaBound{t2'}\AgdaSymbol{\}}\AgdaSpace{}%
\AgdaSymbol{\{}\AgdaBound{t3'}\AgdaSymbol{\}}\AgdaSpace{}%
\AgdaSymbol{=}\<%
\\
\>[8][@{}l@{\AgdaIndent{0}}]%
\>[12]\AgdaFunction{indMax-monoR}\AgdaSpace{}%
\AgdaSymbol{\{}\AgdaArgument{t1}\AgdaSpace{}%
\AgdaSymbol{=}\AgdaSpace{}%
\AgdaFunction{indMax}\AgdaSpace{}%
\AgdaBound{t1}\AgdaSpace{}%
\AgdaBound{t1'}\AgdaSymbol{\}}\AgdaSpace{}%
\AgdaSymbol{(}\AgdaFunction{indMax-swap4}\AgdaSpace{}%
\AgdaSymbol{\{}\AgdaArgument{t1}\AgdaSpace{}%
\AgdaSymbol{=}\AgdaSpace{}%
\AgdaBound{t2}\AgdaSymbol{\}}\AgdaSpace{}%
\AgdaSymbol{\{}\AgdaArgument{t1'}\AgdaSpace{}%
\AgdaSymbol{=}\AgdaSpace{}%
\AgdaBound{t2'}\AgdaSymbol{\}}\AgdaSpace{}%
\AgdaSymbol{\{}\AgdaArgument{t2}\AgdaSpace{}%
\AgdaSymbol{=}\AgdaSpace{}%
\AgdaBound{t3}\AgdaSymbol{\}}\AgdaSpace{}%
\AgdaSymbol{\{}\AgdaArgument{t2'}\AgdaSpace{}%
\AgdaSymbol{=}\AgdaSpace{}%
\AgdaBound{t3'}\AgdaSymbol{\})}\<%
\\
%
\>[12]\AgdaOperator{\AgdaPostulate{≤⨟}}\AgdaSpace{}%
\AgdaFunction{indMax-swap4}\AgdaSpace{}%
\AgdaSymbol{\{}\AgdaArgument{t1}\AgdaSpace{}%
\AgdaSymbol{=}\AgdaSpace{}%
\AgdaBound{t1}\AgdaSymbol{\}}\AgdaSpace{}%
\AgdaSymbol{\{}\AgdaArgument{t1'}\AgdaSpace{}%
\AgdaSymbol{=}\AgdaSpace{}%
\AgdaBound{t1'}\AgdaSymbol{\}}\<%
\\
%
\\[\AgdaEmptyExtraSkip]%
%
\>[8]\AgdaFunction{indMax-lim2L}\AgdaSpace{}%
\AgdaSymbol{:}\<%
\\
\>[8][@{}l@{\AgdaIndent{0}}]%
\>[12]\AgdaSymbol{∀}\<%
\\
%
\>[12]\AgdaSymbol{\{}\AgdaBound{c1}\AgdaSpace{}%
\AgdaSymbol{:}\AgdaSpace{}%
\AgdaBound{ℂ}\AgdaSymbol{\}}\<%
\\
%
\>[12]\AgdaSymbol{(}\AgdaBound{f1}\AgdaSpace{}%
\AgdaSymbol{:}\AgdaSpace{}%
\AgdaBound{El}%
\>[22]\AgdaBound{c1}\AgdaSpace{}%
\AgdaSymbol{→}\AgdaSpace{}%
\AgdaPostulate{Tree}\AgdaSymbol{)}\<%
\\
%
\>[12]\AgdaSymbol{\{}\AgdaBound{c2}\AgdaSpace{}%
\AgdaSymbol{:}\AgdaSpace{}%
\AgdaBound{ℂ}\AgdaSymbol{\}}\<%
\\
%
\>[12]\AgdaSymbol{(}\AgdaBound{f2}\AgdaSpace{}%
\AgdaSymbol{:}\AgdaSpace{}%
\AgdaBound{El}%
\>[22]\AgdaBound{c2}\AgdaSpace{}%
\AgdaSymbol{→}\AgdaSpace{}%
\AgdaPostulate{Tree}\AgdaSymbol{)}\<%
\\
%
\>[12]\AgdaSymbol{→}\AgdaSpace{}%
\AgdaPostulate{Lim}%
\>[19]\AgdaBound{c1}\AgdaSpace{}%
\AgdaSymbol{(λ}\AgdaSpace{}%
\AgdaBound{k1}\AgdaSpace{}%
\AgdaSymbol{→}\AgdaSpace{}%
\AgdaPostulate{Lim}%
\>[35]\AgdaBound{c2}\AgdaSpace{}%
\AgdaSymbol{(λ}\AgdaSpace{}%
\AgdaBound{k2}\AgdaSpace{}%
\AgdaSymbol{→}\AgdaSpace{}%
\AgdaFunction{indMax}\AgdaSpace{}%
\AgdaSymbol{(}\AgdaBound{f1}\AgdaSpace{}%
\AgdaBound{k1}\AgdaSymbol{)}\AgdaSpace{}%
\AgdaSymbol{(}\AgdaBound{f2}\AgdaSpace{}%
\AgdaBound{k2}\AgdaSymbol{)))}\AgdaSpace{}%
\AgdaOperator{\AgdaPostulate{≤}}\AgdaSpace{}%
\AgdaFunction{indMax}\AgdaSpace{}%
\AgdaSymbol{(}\AgdaPostulate{Lim}%
\>[86]\AgdaBound{c1}\AgdaSpace{}%
\AgdaBound{f1}\AgdaSymbol{)}\AgdaSpace{}%
\AgdaSymbol{(}\AgdaPostulate{Lim}%
\>[99]\AgdaBound{c2}\AgdaSpace{}%
\AgdaBound{f2}\AgdaSymbol{)}\<%
\\
%
\>[8]\AgdaFunction{indMax-lim2L}\AgdaSpace{}%
\AgdaBound{f1}\AgdaSpace{}%
\AgdaBound{f2}\AgdaSpace{}%
\AgdaSymbol{=}\AgdaSpace{}%
\AgdaPostulate{≤-limiting}%
\>[41]\AgdaSymbol{\AgdaUnderscore{}}\AgdaSpace{}%
\AgdaSymbol{(λ}\AgdaSpace{}%
\AgdaBound{k1}\AgdaSpace{}%
\AgdaSymbol{→}\AgdaSpace{}%
\AgdaPostulate{≤-limiting}%
\>[63]\AgdaSymbol{\AgdaUnderscore{}}\AgdaSpace{}%
\AgdaSymbol{λ}\AgdaSpace{}%
\AgdaBound{k2}\AgdaSpace{}%
\AgdaSymbol{→}\AgdaSpace{}%
\AgdaFunction{indMax-mono}\AgdaSpace{}%
\AgdaSymbol{(}\AgdaPostulate{≤-cocone}%
\>[95]\AgdaBound{f1}\AgdaSpace{}%
\AgdaBound{k1}\AgdaSpace{}%
\AgdaSymbol{(}\AgdaPostulate{≤-refl}\AgdaSpace{}%
\AgdaSymbol{\AgdaUnderscore{}))}\AgdaSpace{}%
\AgdaSymbol{(}\AgdaPostulate{≤-cocone}%
\>[124]\AgdaBound{f2}\AgdaSpace{}%
\AgdaBound{k2}\AgdaSpace{}%
\AgdaSymbol{(}\AgdaPostulate{≤-refl}\AgdaSpace{}%
\AgdaSymbol{\AgdaUnderscore{})))}\<%
\\
%
\\[\AgdaEmptyExtraSkip]%
%
\>[8]\AgdaFunction{indMax-lim2R}\AgdaSpace{}%
\AgdaSymbol{:}\<%
\\
\>[8][@{}l@{\AgdaIndent{0}}]%
\>[12]\AgdaSymbol{∀}\<%
\\
%
\>[12]\AgdaSymbol{\{}\AgdaBound{c1}\AgdaSpace{}%
\AgdaSymbol{:}\AgdaSpace{}%
\AgdaBound{ℂ}\AgdaSymbol{\}}\<%
\\
%
\>[12]\AgdaSymbol{(}\AgdaBound{f1}\AgdaSpace{}%
\AgdaSymbol{:}\AgdaSpace{}%
\AgdaBound{El}%
\>[22]\AgdaBound{c1}\AgdaSpace{}%
\AgdaSymbol{→}\AgdaSpace{}%
\AgdaPostulate{Tree}\AgdaSymbol{)}\<%
\\
%
\>[12]\AgdaSymbol{\{}\AgdaBound{c2}\AgdaSpace{}%
\AgdaSymbol{:}\AgdaSpace{}%
\AgdaBound{ℂ}\AgdaSymbol{\}}\<%
\\
%
\>[12]\AgdaSymbol{(}\AgdaBound{f2}\AgdaSpace{}%
\AgdaSymbol{:}\AgdaSpace{}%
\AgdaBound{El}%
\>[22]\AgdaBound{c2}\AgdaSpace{}%
\AgdaSymbol{→}\AgdaSpace{}%
\AgdaPostulate{Tree}\AgdaSymbol{)}\<%
\\
%
\>[12]\AgdaSymbol{→}%
\>[15]\AgdaFunction{indMax}\AgdaSpace{}%
\AgdaSymbol{(}\AgdaPostulate{Lim}%
\>[28]\AgdaBound{c1}\AgdaSpace{}%
\AgdaBound{f1}\AgdaSymbol{)}\AgdaSpace{}%
\AgdaSymbol{(}\AgdaPostulate{Lim}%
\>[41]\AgdaBound{c2}\AgdaSpace{}%
\AgdaBound{f2}\AgdaSymbol{)}\AgdaSpace{}%
\AgdaOperator{\AgdaPostulate{≤}}\AgdaSpace{}%
\AgdaPostulate{Lim}%
\>[55]\AgdaBound{c1}\AgdaSpace{}%
\AgdaSymbol{(λ}\AgdaSpace{}%
\AgdaBound{k1}\AgdaSpace{}%
\AgdaSymbol{→}\AgdaSpace{}%
\AgdaPostulate{Lim}%
\>[71]\AgdaBound{c2}\AgdaSpace{}%
\AgdaSymbol{(λ}\AgdaSpace{}%
\AgdaBound{k2}\AgdaSpace{}%
\AgdaSymbol{→}\AgdaSpace{}%
\AgdaFunction{indMax}\AgdaSpace{}%
\AgdaSymbol{(}\AgdaBound{f1}\AgdaSpace{}%
\AgdaBound{k1}\AgdaSymbol{)}\AgdaSpace{}%
\AgdaSymbol{(}\AgdaBound{f2}\AgdaSpace{}%
\AgdaBound{k2}\AgdaSymbol{)))}\<%
\\
%
\>[8]\AgdaFunction{indMax-lim2R}\AgdaSpace{}%
\AgdaBound{f1}\AgdaSpace{}%
\AgdaBound{f2}\AgdaSpace{}%
\AgdaSymbol{=}\AgdaSpace{}%
\AgdaPostulate{extLim}%
\>[37]\AgdaSymbol{\AgdaUnderscore{}}\AgdaSpace{}%
\AgdaSymbol{\AgdaUnderscore{}}\AgdaSpace{}%
\AgdaSymbol{(λ}\AgdaSpace{}%
\AgdaBound{k1}\AgdaSpace{}%
\AgdaSymbol{→}\AgdaSpace{}%
\AgdaFunction{indMax-limR}%
\>[62]\AgdaSymbol{\AgdaUnderscore{}}\AgdaSpace{}%
\AgdaSymbol{(}\AgdaBound{f1}\AgdaSpace{}%
\AgdaBound{k1}\AgdaSymbol{))}\<%
\\
%
\\[\AgdaEmptyExtraSkip]%
%
\\[\AgdaEmptyExtraSkip]%
\>[0]\<%
\end{code}



% !TEX root =  main.tex

\subsection{Strictly Monotone Brouwer Trees}

Now that we have identified a substantial class of well behaved Brouwer trees,
we want to define a new type containing only those trees.
These are SMB-trees: strictly monotone Brouwer trees.
In this section, we will define them, and show how
they can be given a similar interface to Brouwer trees.

\begin{code}[hide]%
\>[0]\AgdaKeyword{open}\AgdaSpace{}%
\AgdaKeyword{import}\AgdaSpace{}%
\AgdaModule{Data.Nat}\AgdaSpace{}%
\AgdaKeyword{hiding}\AgdaSpace{}%
\AgdaSymbol{(}\AgdaOperator{\AgdaDatatype{\AgdaUnderscore{}≤\AgdaUnderscore{}}}\AgdaSpace{}%
\AgdaSymbol{;}\AgdaSpace{}%
\AgdaOperator{\AgdaFunction{\AgdaUnderscore{}<\AgdaUnderscore{}}}\AgdaSymbol{)}\<%
\\
\>[0]\AgdaKeyword{open}\AgdaSpace{}%
\AgdaKeyword{import}\AgdaSpace{}%
\AgdaModule{Relation.Binary.PropositionalEquality}\<%
\\
\>[0]\AgdaKeyword{open}\AgdaSpace{}%
\AgdaKeyword{import}\AgdaSpace{}%
\AgdaModule{Data.Product}\<%
\\
\>[0]\AgdaKeyword{open}\AgdaSpace{}%
\AgdaKeyword{import}\AgdaSpace{}%
\AgdaModule{Data.Maybe}\<%
\\
\>[0]\AgdaKeyword{open}\AgdaSpace{}%
\AgdaKeyword{import}\AgdaSpace{}%
\AgdaModule{Relation.Nullary}\<%
\\
\>[0]\AgdaKeyword{open}\AgdaSpace{}%
\AgdaKeyword{import}\AgdaSpace{}%
\AgdaModule{Iso}\<%
\end{code}

To begin, we declare a new Agda module, with the same parameters
we have been working with thus far: a type of codes, interpretations of those codes into types,
and a code whose interpretation is isomorphic to $\bN$.
\begin{code}%
\>[0]\AgdaKeyword{module}\AgdaSpace{}%
\AgdaModule{Idem}\AgdaSpace{}%
\AgdaSymbol{\{}\AgdaBound{ℓ}\AgdaSymbol{\}}\<%
\\
\>[0][@{}l@{\AgdaIndent{0}}]%
\>[4]\AgdaSymbol{(}\AgdaBound{ℂ}\AgdaSpace{}%
\AgdaSymbol{:}\AgdaSpace{}%
\AgdaPrimitive{Set}\AgdaSpace{}%
\AgdaBound{ℓ}\AgdaSymbol{)}\<%
\\
%
\>[4]\AgdaSymbol{(}\AgdaBound{El}\AgdaSpace{}%
\AgdaSymbol{:}\AgdaSpace{}%
\AgdaBound{ℂ}\AgdaSpace{}%
\AgdaSymbol{→}\AgdaSpace{}%
\AgdaPrimitive{Set}\AgdaSpace{}%
\AgdaBound{ℓ}\AgdaSymbol{)}\<%
\\
%
\>[4]\AgdaSymbol{(}\AgdaBound{Cℕ}\AgdaSpace{}%
\AgdaSymbol{:}\AgdaSpace{}%
\AgdaBound{ℂ}\AgdaSymbol{)}\AgdaSpace{}%
\AgdaSymbol{(}\AgdaBound{CℕIso}\AgdaSpace{}%
\AgdaSymbol{:}\AgdaSpace{}%
\AgdaRecord{Iso}\AgdaSpace{}%
\AgdaSymbol{(}\AgdaBound{El}\AgdaSpace{}%
\AgdaBound{Cℕ}\AgdaSymbol{)}\AgdaSpace{}%
\AgdaDatatype{ℕ}\AgdaSpace{}%
\AgdaSymbol{)}\AgdaSpace{}%
\AgdaKeyword{where}\<%
\end{code}

We import all of our definitions so far, using the ``Brouwer" prefix to distinguish
them from the trees and ordering we are about to define.
Critically, we do not instantiate these with the same interpretation function.
Instead, we interpret each code wrapped in $\AgdaDatatype{Maybe}$.
This ensures that we always take Brouwer limits over non-empty sets,
an assumption that was critical for the definitions of \cref{TODO}.
However, we place no such restriction on SMB-trees.
\begin{code}%
\>[0]\AgdaKeyword{import}\AgdaSpace{}%
\AgdaModule{Brouwer}\AgdaSpace{}%
\AgdaBound{ℂ}\AgdaSpace{}%
\AgdaSymbol{(λ}\AgdaSpace{}%
\AgdaBound{c}\AgdaSpace{}%
\AgdaSymbol{→}\AgdaSpace{}%
\AgdaDatatype{Maybe}\AgdaSpace{}%
\AgdaSymbol{(}\AgdaBound{El}\AgdaSpace{}%
\AgdaBound{c}\AgdaSymbol{))}\AgdaSpace{}%
\AgdaBound{Cℕ}\AgdaSpace{}%
\AgdaSymbol{(}\AgdaFunction{maybeNatIso}\AgdaSpace{}%
\AgdaBound{CℕIso}\AgdaSymbol{)}\AgdaSpace{}%
\AgdaSymbol{as}\AgdaSpace{}%
\AgdaModule{Brouwer}\<%
\end{code}


\begin{code}[hide]%
\>[0]\AgdaKeyword{open}\AgdaSpace{}%
\AgdaKeyword{import}\AgdaSpace{}%
\AgdaModule{IndMax}\AgdaSpace{}%
\AgdaBound{ℂ}\AgdaSpace{}%
\AgdaSymbol{(λ}\AgdaSpace{}%
\AgdaBound{c}\AgdaSpace{}%
\AgdaSymbol{→}\AgdaSpace{}%
\AgdaDatatype{Maybe}\AgdaSpace{}%
\AgdaSymbol{(}\AgdaBound{El}\AgdaSpace{}%
\AgdaBound{c}\AgdaSymbol{))}\AgdaSpace{}%
\AgdaBound{Cℕ}\AgdaSpace{}%
\AgdaSymbol{(}\AgdaFunction{maybeNatIso}\AgdaSpace{}%
\AgdaBound{CℕIso}\AgdaSymbol{)}\AgdaSpace{}%
\AgdaSymbol{(λ}\AgdaSpace{}%
\AgdaBound{c}\AgdaSpace{}%
\AgdaSymbol{→}\AgdaSpace{}%
\AgdaInductiveConstructor{nothing}\AgdaSymbol{)}\<%
\\
\>[0]\AgdaKeyword{open}\AgdaSpace{}%
\AgdaKeyword{import}\AgdaSpace{}%
\AgdaModule{InfinityMax}\AgdaSpace{}%
\AgdaBound{ℂ}\AgdaSpace{}%
\AgdaSymbol{(λ}\AgdaSpace{}%
\AgdaBound{c}\AgdaSpace{}%
\AgdaSymbol{→}\AgdaSpace{}%
\AgdaDatatype{Maybe}\AgdaSpace{}%
\AgdaSymbol{(}\AgdaBound{El}\AgdaSpace{}%
\AgdaBound{c}\AgdaSymbol{))}\AgdaSpace{}%
\AgdaBound{Cℕ}\AgdaSpace{}%
\AgdaSymbol{(}\AgdaFunction{maybeNatIso}\AgdaSpace{}%
\AgdaBound{CℕIso}\AgdaSymbol{)}\AgdaSpace{}%
\AgdaSymbol{(λ}\AgdaSpace{}%
\AgdaBound{c}\AgdaSpace{}%
\AgdaSymbol{→}\AgdaSpace{}%
\AgdaInductiveConstructor{nothing}\AgdaSymbol{)}\<%
\\
\>[0]\AgdaKeyword{infixr}\AgdaSpace{}%
\AgdaNumber{10}\AgdaSpace{}%
\AgdaOperator{\AgdaFunction{\AgdaUnderscore{}≤⨟\AgdaUnderscore{}}}\<%
\end{code}


\subsubsection{Refining Brouwer Trees}

We define SMB-trees as a dependent record,
containing an underlying Brouwer tree, and a proof
that $\indMax$ is idempotent on this tree.

\begin{code}%
\>[0]\AgdaKeyword{record}\AgdaSpace{}%
\AgdaRecord{SMBTree}\AgdaSpace{}%
\AgdaSymbol{:}\AgdaSpace{}%
\AgdaPrimitive{Set}\AgdaSpace{}%
\AgdaBound{ℓ}\AgdaSpace{}%
\AgdaKeyword{where}\<%
\\
\>[0][@{}l@{\AgdaIndent{0}}]%
\>[2]\AgdaKeyword{constructor}\AgdaSpace{}%
\AgdaInductiveConstructor{MkTree}\<%
\\
%
\>[2]\AgdaKeyword{field}\<%
\\
\>[2][@{}l@{\AgdaIndent{0}}]%
\>[4]\AgdaField{rawTree}\AgdaSpace{}%
\AgdaSymbol{:}\AgdaSpace{}%
\AgdaPostulate{Brouwer.Tree}\<%
\\
%
\>[4]\AgdaField{isIdem}\AgdaSpace{}%
\AgdaSymbol{:}\AgdaSpace{}%
\AgdaSymbol{(}\AgdaPostulate{indMax}\AgdaSpace{}%
\AgdaField{rawTree}\AgdaSpace{}%
\AgdaField{rawTree}\AgdaSymbol{)}\AgdaSpace{}%
\AgdaOperator{\AgdaPostulate{Brouwer.≤}}\AgdaSpace{}%
\AgdaField{rawTree}\<%
\\
\>[0]\AgdaKeyword{open}\AgdaSpace{}%
\AgdaModule{SMBTree}\<%
\end{code}
%

We can then define so-called ``smart-constructors'' corresponding to each of the constructors
for Brouwer-trees: zero, successor, and limit.
Zero and successor directly correspond to the Brouwer tree zero and successor.
Their proofs of idempotence are trivial from the properties of Brouwer $\le$.
\begin{code}%
\>[0]\AgdaKeyword{opaque}\<%
\\
\>[0][@{}l@{\AgdaIndent{0}}]%
\>[2]\AgdaKeyword{unfolding}\AgdaSpace{}%
\AgdaPostulate{indMax}\<%
\\
%
\\[\AgdaEmptyExtraSkip]%
%
\>[2]\AgdaFunction{Z}\AgdaSpace{}%
\AgdaSymbol{:}\AgdaSpace{}%
\AgdaRecord{SMBTree}\<%
\\
%
\>[2]\AgdaFunction{Z}\AgdaSpace{}%
\AgdaSymbol{=}\AgdaSpace{}%
\AgdaInductiveConstructor{MkTree}\AgdaSpace{}%
\AgdaPostulate{Brouwer.Z}\AgdaSpace{}%
\AgdaPostulate{Brouwer.≤-Z}\<%
\\
%
\\[\AgdaEmptyExtraSkip]%
%
\>[2]\AgdaFunction{↑}\AgdaSpace{}%
\AgdaSymbol{:}\AgdaSpace{}%
\AgdaRecord{SMBTree}\AgdaSpace{}%
\AgdaSymbol{→}\AgdaSpace{}%
\AgdaRecord{SMBTree}\<%
\\
%
\>[2]\AgdaFunction{↑}\AgdaSpace{}%
\AgdaSymbol{(}\AgdaInductiveConstructor{MkTree}\AgdaSpace{}%
\AgdaBound{t}\AgdaSpace{}%
\AgdaBound{pf}\AgdaSymbol{)}\AgdaSpace{}%
\AgdaSymbol{=}\AgdaSpace{}%
\AgdaInductiveConstructor{MkTree}\AgdaSpace{}%
\AgdaSymbol{(}\AgdaPostulate{Brouwer.↑}\AgdaSpace{}%
\AgdaBound{t}\AgdaSymbol{)}\AgdaSpace{}%
\AgdaSymbol{(}\AgdaPostulate{Brouwer.≤-sucMono}\AgdaSpace{}%
\AgdaBound{pf}\AgdaSymbol{)}\<%
\end{code}

However, constructing the limit of a sequence of SMB-trees is not so easy.
Since we instantiated $\AgdaBound{El}$ to wrap its result in $\AgdaDatatype{Maybe}$,
we need to handle $\AgdaDatatypeConstructor{nothing}$ for each limit,
but we can use $\AgdaFunction{Z}$ as a default value, since adding it to any sequence
does not change the least upper bound.
More challenging is how, as we saw in \cref{TODO}, Brouwer trees do not have $\indMax\ (\Lim\ c\ f)\ (\Lim\ c\ f) \le \Lim\ c\ f$, so we cannot directly produce a proof of idempotence.

Our key insight is to define limits of SMB-trees using $\maxInf$ on the underlying trees:
for any function producing SMB-trees, we take the limit of the underlying trees,
then $\indMax$ that result with itself an infinite numer of times.
The idempotence proof is then the property of $\maxInf$ that we proved in \cref{TODO}.
\begin{code}%
%
\>[2]\AgdaFunction{Lim}\AgdaSpace{}%
\AgdaSymbol{:}\AgdaSpace{}%
\AgdaSymbol{∀}%
\>[12]\AgdaSymbol{(}\AgdaBound{c}\AgdaSpace{}%
\AgdaSymbol{:}\AgdaSpace{}%
\AgdaBound{ℂ}\AgdaSymbol{)}\AgdaSpace{}%
\AgdaSymbol{→}\AgdaSpace{}%
\AgdaSymbol{(}\AgdaBound{f}\AgdaSpace{}%
\AgdaSymbol{:}\AgdaSpace{}%
\AgdaBound{El}\AgdaSpace{}%
\AgdaBound{c}\AgdaSpace{}%
\AgdaSymbol{→}\AgdaSpace{}%
\AgdaRecord{SMBTree}\AgdaSymbol{)}\AgdaSpace{}%
\AgdaSymbol{→}\AgdaSpace{}%
\AgdaRecord{SMBTree}\<%
\\
%
\>[2]\AgdaFunction{Lim}\AgdaSpace{}%
\AgdaBound{c}\AgdaSpace{}%
\AgdaBound{f}\AgdaSpace{}%
\AgdaSymbol{=}\<%
\\
\>[2][@{}l@{\AgdaIndent{0}}]%
\>[4]\AgdaInductiveConstructor{MkTree}\<%
\\
%
\>[4]\AgdaSymbol{(}\AgdaPostulate{indMax∞}\AgdaSpace{}%
\AgdaSymbol{(}\AgdaPostulate{Brouwer.Lim}\AgdaSpace{}%
\AgdaBound{c}\AgdaSpace{}%
\AgdaSymbol{(}\AgdaFunction{maybe′}\AgdaSpace{}%
\AgdaSymbol{(λ}\AgdaSpace{}%
\AgdaBound{x}\AgdaSpace{}%
\AgdaSymbol{→}\AgdaSpace{}%
\AgdaField{rawTree}\AgdaSpace{}%
\AgdaSymbol{(}\AgdaBound{f}\AgdaSpace{}%
\AgdaBound{x}\AgdaSymbol{))}\AgdaSpace{}%
\AgdaPostulate{Brouwer.Z}\AgdaSymbol{)))}\<%
\\
%
\>[4]\AgdaSymbol{(}\AgdaPostulate{indMax∞-idem}\AgdaSpace{}%
\AgdaSymbol{\AgdaUnderscore{})}\<%
\end{code}



\subsubsection{Ordering SMB-trees}

SMB-trees are ordered by the order on their underlying Brouwer trees:
%
\begin{code}%
\>[0]\AgdaKeyword{record}\AgdaSpace{}%
\AgdaOperator{\AgdaRecord{\AgdaUnderscore{}≤\AgdaUnderscore{}}}\AgdaSpace{}%
\AgdaSymbol{(}\AgdaBound{t1}\AgdaSpace{}%
\AgdaBound{t2}\AgdaSpace{}%
\AgdaSymbol{:}\AgdaSpace{}%
\AgdaRecord{SMBTree}\AgdaSymbol{)}\AgdaSpace{}%
\AgdaSymbol{:}\AgdaSpace{}%
\AgdaPrimitive{Set}\AgdaSpace{}%
\AgdaBound{ℓ}\AgdaSpace{}%
\AgdaKeyword{where}\<%
\\
\>[0][@{}l@{\AgdaIndent{0}}]%
\>[2]\AgdaKeyword{constructor}\AgdaSpace{}%
\AgdaInductiveConstructor{mk≤}\<%
\\
%
\>[2]\AgdaKeyword{inductive}\<%
\\
%
\>[2]\AgdaKeyword{field}\<%
\\
\>[2][@{}l@{\AgdaIndent{0}}]%
\>[4]\AgdaField{get≤}\AgdaSpace{}%
\AgdaSymbol{:}\AgdaSpace{}%
\AgdaSymbol{(}\AgdaField{rawTree}\AgdaSpace{}%
\AgdaBound{t1}\AgdaSymbol{)}\AgdaSpace{}%
\AgdaOperator{\AgdaPostulate{Brouwer.≤}}\AgdaSpace{}%
\AgdaSymbol{(}\AgdaField{rawTree}\AgdaSpace{}%
\AgdaBound{t2}\AgdaSymbol{)}\<%
\\
\>[0]\AgdaKeyword{open}\AgdaSpace{}%
\AgdaOperator{\AgdaModule{\AgdaUnderscore{}≤\AgdaUnderscore{}}}\<%
\\
\>[0]\<%
\end{code}
%
Having a successor function allows us to define a strict ording on SMB-trees.
\begin{code}%
\>[0]\AgdaOperator{\AgdaFunction{\AgdaUnderscore{}<\AgdaUnderscore{}}}\AgdaSpace{}%
\AgdaSymbol{:}\AgdaSpace{}%
\AgdaRecord{SMBTree}\AgdaSpace{}%
\AgdaSymbol{→}\AgdaSpace{}%
\AgdaRecord{SMBTree}\AgdaSpace{}%
\AgdaSymbol{→}\AgdaSpace{}%
\AgdaPrimitive{Set}\AgdaSpace{}%
\AgdaBound{ℓ}\<%
\\
\>[0]\AgdaOperator{\AgdaFunction{\AgdaUnderscore{}<\AgdaUnderscore{}}}\AgdaSpace{}%
\AgdaBound{t1}\AgdaSpace{}%
\AgdaBound{t2}\AgdaSpace{}%
\AgdaSymbol{=}\AgdaSpace{}%
\AgdaSymbol{(}\AgdaFunction{↑}\AgdaSpace{}%
\AgdaBound{t1}\AgdaSymbol{)}\AgdaSpace{}%
\AgdaOperator{\AgdaRecord{≤}}\AgdaSpace{}%
\AgdaBound{t2}\<%
\end{code}

The next step is to prove that our SMB-tree constructors satisfy the same
inequalities as Brouwer trees. Since SMB-trees are ordered by their underlying
Brouwer trees, most properties can be directly lifted from  Brouwer trees
to SMB-trees.

\begin{code}%
\>[0]\AgdaKeyword{opaque}\<%
\\
\>[0][@{}l@{\AgdaIndent{0}}]%
\>[2]\AgdaKeyword{unfolding}\AgdaSpace{}%
\AgdaFunction{Z}\AgdaSpace{}%
\AgdaFunction{↑}\<%
\\
%
\>[2]\AgdaFunction{≤↑}\AgdaSpace{}%
\AgdaSymbol{:}\AgdaSpace{}%
\AgdaSymbol{∀}\AgdaSpace{}%
\AgdaBound{t}\AgdaSpace{}%
\AgdaSymbol{→}\AgdaSpace{}%
\AgdaBound{t}\AgdaSpace{}%
\AgdaOperator{\AgdaRecord{≤}}\AgdaSpace{}%
\AgdaFunction{↑}\AgdaSpace{}%
\AgdaBound{t}\<%
\\
%
\>[2]\AgdaFunction{≤↑}\AgdaSpace{}%
\AgdaBound{t}\AgdaSpace{}%
\AgdaSymbol{=}%
\>[10]\AgdaInductiveConstructor{mk≤}\AgdaSpace{}%
\AgdaSymbol{(}\AgdaPostulate{Brouwer.≤↑t}\AgdaSpace{}%
\AgdaSymbol{\AgdaUnderscore{})}\<%
\\
%
\\[\AgdaEmptyExtraSkip]%
%
\>[2]\AgdaOperator{\AgdaFunction{\AgdaUnderscore{}≤⨟\AgdaUnderscore{}}}\AgdaSpace{}%
\AgdaSymbol{:}\AgdaSpace{}%
\AgdaSymbol{∀}\AgdaSpace{}%
\AgdaSymbol{\{}\AgdaBound{t1}\AgdaSpace{}%
\AgdaBound{t2}\AgdaSpace{}%
\AgdaBound{t3}\AgdaSymbol{\}}\AgdaSpace{}%
\AgdaSymbol{→}\AgdaSpace{}%
\AgdaBound{t1}\AgdaSpace{}%
\AgdaOperator{\AgdaRecord{≤}}\AgdaSpace{}%
\AgdaBound{t2}\AgdaSpace{}%
\AgdaSymbol{→}\AgdaSpace{}%
\AgdaBound{t2}\AgdaSpace{}%
\AgdaOperator{\AgdaRecord{≤}}\AgdaSpace{}%
\AgdaBound{t3}\AgdaSpace{}%
\AgdaSymbol{→}\AgdaSpace{}%
\AgdaBound{t1}\AgdaSpace{}%
\AgdaOperator{\AgdaRecord{≤}}\AgdaSpace{}%
\AgdaBound{t3}\<%
\\
%
\>[2]\AgdaOperator{\AgdaFunction{\AgdaUnderscore{}≤⨟\AgdaUnderscore{}}}\AgdaSpace{}%
\AgdaSymbol{(}\AgdaInductiveConstructor{mk≤}\AgdaSpace{}%
\AgdaBound{lt1}\AgdaSymbol{)}\AgdaSpace{}%
\AgdaSymbol{(}\AgdaInductiveConstructor{mk≤}\AgdaSpace{}%
\AgdaBound{lt2}\AgdaSymbol{)}\AgdaSpace{}%
\AgdaSymbol{=}\AgdaSpace{}%
\AgdaInductiveConstructor{mk≤}\AgdaSpace{}%
\AgdaSymbol{(}\AgdaPostulate{Brouwer.≤-trans}\AgdaSpace{}%
\AgdaBound{lt1}\AgdaSpace{}%
\AgdaBound{lt2}\AgdaSymbol{)}\<%
\\
%
\\[\AgdaEmptyExtraSkip]%
%
\>[2]\AgdaFunction{≤-refl}\AgdaSpace{}%
\AgdaSymbol{:}\AgdaSpace{}%
\AgdaSymbol{∀}\AgdaSpace{}%
\AgdaSymbol{\{}\AgdaBound{t}\AgdaSymbol{\}}\AgdaSpace{}%
\AgdaSymbol{→}\AgdaSpace{}%
\AgdaBound{t}\AgdaSpace{}%
\AgdaOperator{\AgdaRecord{≤}}\AgdaSpace{}%
\AgdaBound{t}\<%
\\
%
\>[2]\AgdaFunction{≤-refl}\AgdaSpace{}%
\AgdaSymbol{=}%
\>[12]\AgdaInductiveConstructor{mk≤}\AgdaSpace{}%
\AgdaSymbol{(}\AgdaPostulate{Brouwer.≤-refl}\AgdaSpace{}%
\AgdaSymbol{\AgdaUnderscore{})}\<%
\end{code}

The constructors for $\le$ each have a counterpart for SMB-trees.
For zero and successor, these are trivially lifted.
\begin{code}%
%
\>[2]\AgdaFunction{≤-Z}\AgdaSpace{}%
\AgdaSymbol{:}\AgdaSpace{}%
\AgdaSymbol{∀}\AgdaSpace{}%
\AgdaSymbol{\{}\AgdaBound{t}\AgdaSymbol{\}}\AgdaSpace{}%
\AgdaSymbol{→}\AgdaSpace{}%
\AgdaFunction{Z}\AgdaSpace{}%
\AgdaOperator{\AgdaRecord{≤}}\AgdaSpace{}%
\AgdaBound{t}\<%
\\
%
\>[2]\AgdaFunction{≤-Z}\AgdaSpace{}%
\AgdaSymbol{=}%
\>[9]\AgdaInductiveConstructor{mk≤}\AgdaSpace{}%
\AgdaPostulate{Brouwer.≤-Z}\<%
\\
%
\\[\AgdaEmptyExtraSkip]%
%
\>[2]\AgdaFunction{≤-sucMono}\AgdaSpace{}%
\AgdaSymbol{:}\AgdaSpace{}%
\AgdaSymbol{∀}\AgdaSpace{}%
\AgdaSymbol{\{}\AgdaBound{t1}\AgdaSpace{}%
\AgdaBound{t2}\AgdaSymbol{\}}\AgdaSpace{}%
\AgdaSymbol{→}\AgdaSpace{}%
\AgdaBound{t1}\AgdaSpace{}%
\AgdaOperator{\AgdaRecord{≤}}\AgdaSpace{}%
\AgdaBound{t2}\AgdaSpace{}%
\AgdaSymbol{→}\AgdaSpace{}%
\AgdaFunction{↑}\AgdaSpace{}%
\AgdaBound{t1}\AgdaSpace{}%
\AgdaOperator{\AgdaRecord{≤}}\AgdaSpace{}%
\AgdaFunction{↑}\AgdaSpace{}%
\AgdaBound{t2}\<%
\\
%
\>[2]\AgdaFunction{≤-sucMono}\AgdaSpace{}%
\AgdaSymbol{(}\AgdaInductiveConstructor{mk≤}\AgdaSpace{}%
\AgdaBound{lt}\AgdaSymbol{)}\AgdaSpace{}%
\AgdaSymbol{=}\AgdaSpace{}%
\AgdaInductiveConstructor{mk≤}\AgdaSpace{}%
\AgdaSymbol{(}\AgdaPostulate{Brouwer.≤-sucMono}\AgdaSpace{}%
\AgdaBound{lt}\AgdaSymbol{)}\<%
\end{code}
  The constructors for ordering limits require more attention.
  To show that an SMB-tree limit is an upper-bound, we use the fact
  that the underlying limit was an upper bound, and the fact that $\maxInf$ is an upper bound,
  since the SMB-tree $\AgdaFunction{Lim}$ wraps its result in $\maxInf$.
  Note that, since we already have transitivity for our new $\le$,
  we can simply show that $f\ k$ is less than the limit of $f$,
  avoiding the more complicated form of $\cocone$.
\begin{code}%
%
\>[2]\AgdaFunction{≤-limUpperBound}\AgdaSpace{}%
\AgdaSymbol{:}\AgdaSpace{}%
\AgdaSymbol{∀}%
\>[24]\AgdaSymbol{\{}\AgdaBound{c}\AgdaSpace{}%
\AgdaSymbol{:}\AgdaSpace{}%
\AgdaBound{ℂ}\AgdaSymbol{\}}\AgdaSpace{}%
\AgdaSymbol{→}\AgdaSpace{}%
\AgdaSymbol{\{}\AgdaBound{f}\AgdaSpace{}%
\AgdaSymbol{:}\AgdaSpace{}%
\AgdaBound{El}\AgdaSpace{}%
\AgdaBound{c}\AgdaSpace{}%
\AgdaSymbol{→}\AgdaSpace{}%
\AgdaRecord{SMBTree}\AgdaSymbol{\}}\<%
\\
\>[2][@{}l@{\AgdaIndent{0}}]%
\>[4]\AgdaSymbol{→}\AgdaSpace{}%
\AgdaSymbol{∀}\AgdaSpace{}%
\AgdaBound{k}\AgdaSpace{}%
\AgdaSymbol{→}\AgdaSpace{}%
\AgdaBound{f}\AgdaSpace{}%
\AgdaBound{k}\AgdaSpace{}%
\AgdaOperator{\AgdaRecord{≤}}\AgdaSpace{}%
\AgdaFunction{Lim}\AgdaSpace{}%
\AgdaBound{c}\AgdaSpace{}%
\AgdaBound{f}\<%
\\
%
\>[2]\AgdaFunction{≤-limUpperBound}\AgdaSpace{}%
\AgdaSymbol{\{}\AgdaArgument{c}\AgdaSpace{}%
\AgdaSymbol{=}\AgdaSpace{}%
\AgdaBound{c}\AgdaSymbol{\}}\AgdaSpace{}%
\AgdaSymbol{\{}\AgdaArgument{f}\AgdaSpace{}%
\AgdaSymbol{=}\AgdaSpace{}%
\AgdaBound{f}\AgdaSymbol{\}}\AgdaSpace{}%
\AgdaBound{k}\<%
\\
\>[2][@{}l@{\AgdaIndent{0}}]%
\>[4]\AgdaSymbol{=}\AgdaSpace{}%
\AgdaInductiveConstructor{mk≤}%
\>[283I]\AgdaSymbol{(}\AgdaPostulate{Brouwer.≤-cocone}\AgdaSpace{}%
\AgdaSymbol{\AgdaUnderscore{}}\AgdaSpace{}%
\AgdaSymbol{(}\AgdaInductiveConstructor{just}\AgdaSpace{}%
\AgdaBound{k}\AgdaSymbol{)}\AgdaSpace{}%
\AgdaSymbol{(}\AgdaPostulate{Brouwer.≤-refl}\AgdaSpace{}%
\AgdaSymbol{\AgdaUnderscore{})}\<%
\\
\>[283I][@{}l@{\AgdaIndent{0}}]%
\>[11]\AgdaOperator{\AgdaPostulate{Brouwer.≤⨟}}\AgdaSpace{}%
\AgdaPostulate{indMax∞-self}\AgdaSpace{}%
\AgdaSymbol{(}\AgdaPostulate{Brouwer.Lim}\AgdaSpace{}%
\AgdaBound{c}\AgdaSpace{}%
\AgdaSymbol{\AgdaUnderscore{}))}\<%
\end{code}

Finally, we need to show that the SMT-tree limit is less than all other upper bounds.
Suppose $t : \AgdaDatatype{SMBTRee}$ is an upper bound for $f$,
and $t_u$ is the underlying tree for $t$, and $f_u$ computes the underlying trees for $f$.
Then $\limiting$ gives that the underlying tree for $t$ is an upper bound for the trees underlying the image of $f$.
However, the SMB-tree limit wraps its result in $\maxInf$.
The monotonicity of $\maxInf$ then gives that $\indMax (\Lim\ c\ f_u)$ is less than $\maxInf\ t'$.
In \cref{TODO}, we showed that $\maxInf$ had no effect on Brouwer trees that $\indMax$ was idempotent on.
This is exactly what the  $\AgdaField{isIdem}$ field of SMB-trees contains! So we have $\maxInf\ t' \le\ t'$,
and transitivity gives our result.
\begin{code}%
%
\>[2]\AgdaFunction{≤-limLeast}\AgdaSpace{}%
\AgdaSymbol{:}\AgdaSpace{}%
\AgdaSymbol{∀}%
\>[19]\AgdaSymbol{\{}\AgdaBound{c}\AgdaSpace{}%
\AgdaSymbol{:}\AgdaSpace{}%
\AgdaBound{ℂ}\AgdaSymbol{\}}\AgdaSpace{}%
\AgdaSymbol{→}\AgdaSpace{}%
\AgdaSymbol{\{}\AgdaBound{f}\AgdaSpace{}%
\AgdaSymbol{:}\AgdaSpace{}%
\AgdaBound{El}\AgdaSpace{}%
\AgdaBound{c}\AgdaSpace{}%
\AgdaSymbol{→}\AgdaSpace{}%
\AgdaRecord{SMBTree}\AgdaSymbol{\}}\<%
\\
\>[2][@{}l@{\AgdaIndent{0}}]%
\>[4]\AgdaSymbol{→}\AgdaSpace{}%
\AgdaSymbol{\{}\AgdaBound{t}\AgdaSpace{}%
\AgdaSymbol{:}\AgdaSpace{}%
\AgdaRecord{SMBTree}\AgdaSymbol{\}}\<%
\\
%
\>[4]\AgdaSymbol{→}\AgdaSpace{}%
\AgdaSymbol{(∀}\AgdaSpace{}%
\AgdaBound{k}\AgdaSpace{}%
\AgdaSymbol{→}\AgdaSpace{}%
\AgdaBound{f}\AgdaSpace{}%
\AgdaBound{k}\AgdaSpace{}%
\AgdaOperator{\AgdaRecord{≤}}\AgdaSpace{}%
\AgdaBound{t}\AgdaSymbol{)}\AgdaSpace{}%
\AgdaSymbol{→}\AgdaSpace{}%
\AgdaFunction{Lim}\AgdaSpace{}%
\AgdaBound{c}\AgdaSpace{}%
\AgdaBound{f}\AgdaSpace{}%
\AgdaOperator{\AgdaRecord{≤}}\AgdaSpace{}%
\AgdaBound{t}\<%
\\
%
\>[2]\AgdaFunction{≤-limLeast}\AgdaSpace{}%
\AgdaSymbol{\{}\AgdaArgument{f}\AgdaSpace{}%
\AgdaSymbol{=}\AgdaSpace{}%
\AgdaBound{f}\AgdaSymbol{\}}\AgdaSpace{}%
\AgdaSymbol{\{}\AgdaArgument{t}\AgdaSpace{}%
\AgdaSymbol{=}\AgdaSpace{}%
\AgdaInductiveConstructor{MkTree}\AgdaSpace{}%
\AgdaBound{t}\AgdaSpace{}%
\AgdaBound{idem}\AgdaSymbol{\}}\AgdaSpace{}%
\AgdaBound{lt}\<%
\\
\>[2][@{}l@{\AgdaIndent{0}}]%
\>[4]\AgdaSymbol{=}%
\>[329I]\AgdaInductiveConstructor{mk≤}\AgdaSpace{}%
\AgdaSymbol{(}\<%
\\
\>[.][@{}l@{}]\<[329I]%
\>[6]\AgdaPostulate{indMax∞-mono}\<%
\\
\>[6][@{}l@{\AgdaIndent{0}}]%
\>[8]\AgdaSymbol{(}\AgdaPostulate{Brouwer.≤-limiting}\AgdaSpace{}%
\AgdaSymbol{\AgdaUnderscore{}}\<%
\\
\>[8][@{}l@{\AgdaIndent{0}}]%
\>[10]\AgdaSymbol{(}\AgdaFunction{maybe}\AgdaSpace{}%
\AgdaSymbol{(λ}\AgdaSpace{}%
\AgdaBound{k}\AgdaSpace{}%
\AgdaSymbol{→}\AgdaSpace{}%
\AgdaField{get≤}\AgdaSpace{}%
\AgdaSymbol{(}\AgdaBound{lt}\AgdaSpace{}%
\AgdaBound{k}\AgdaSymbol{))}\AgdaSpace{}%
\AgdaPostulate{Brouwer.≤-Z}\AgdaSymbol{))}\<%
\\
%
\>[6]\AgdaOperator{\AgdaPostulate{Brouwer.≤⨟}}\AgdaSpace{}%
\AgdaSymbol{(}\AgdaPostulate{indMax∞-≤}\AgdaSpace{}%
\AgdaBound{idem}\AgdaSymbol{)}\AgdaSpace{}%
\AgdaSymbol{)}\<%
\end{code}


\subsubsection{The Join for SMB-trees}
Our whole reason for defining SMB-trees was to define a well-behaved maximum operator,
and we finally have the tools to do so.
We can define the join in terms of $\indMax$ on the underlying trees.
The proof that the $\indMax$ is idempotent on the result follows from
associativity, commutativity, and monotonicity of $\indMax$.
\begin{code}%
\>[0]\AgdaKeyword{opaque}\<%
\\
\>[0][@{}l@{\AgdaIndent{0}}]%
\>[2]\AgdaKeyword{unfolding}\AgdaSpace{}%
\AgdaPostulate{indMax}\AgdaSpace{}%
\AgdaFunction{Z}\AgdaSpace{}%
\AgdaFunction{↑}\AgdaSpace{}%
\AgdaPostulate{indMaxView}\<%
\\
%
\>[2]\AgdaFunction{max}\AgdaSpace{}%
\AgdaSymbol{:}\AgdaSpace{}%
\AgdaRecord{SMBTree}\AgdaSpace{}%
\AgdaSymbol{→}\AgdaSpace{}%
\AgdaRecord{SMBTree}\AgdaSpace{}%
\AgdaSymbol{→}\AgdaSpace{}%
\AgdaRecord{SMBTree}\<%
\\
%
\>[2]\AgdaFunction{max}\AgdaSpace{}%
\AgdaBound{t1}\AgdaSpace{}%
\AgdaBound{t2}\AgdaSpace{}%
\AgdaSymbol{=}\<%
\\
\>[2][@{}l@{\AgdaIndent{0}}]%
\>[4]\AgdaInductiveConstructor{MkTree}\<%
\\
\>[4][@{}l@{\AgdaIndent{0}}]%
\>[6]\AgdaSymbol{(}\AgdaPostulate{indMax}\AgdaSpace{}%
\AgdaSymbol{(}\AgdaField{rawTree}\AgdaSpace{}%
\AgdaBound{t1}\AgdaSymbol{)}\AgdaSpace{}%
\AgdaSymbol{(}\AgdaField{rawTree}\AgdaSpace{}%
\AgdaBound{t2}\AgdaSymbol{))}\<%
\\
%
\>[6]\AgdaSymbol{(}\AgdaPostulate{indMax-swap4}\<%
\\
\>[6][@{}l@{\AgdaIndent{0}}]%
\>[8]\AgdaOperator{\AgdaPostulate{Brouwer.≤⨟}}\AgdaSpace{}%
\AgdaPostulate{indMax-mono}\AgdaSpace{}%
\AgdaSymbol{(}\AgdaField{isIdem}\AgdaSpace{}%
\AgdaBound{t1}\AgdaSymbol{)}\AgdaSpace{}%
\AgdaSymbol{(}\AgdaField{isIdem}\AgdaSpace{}%
\AgdaBound{t2}\AgdaSymbol{))}\<%
\end{code}

For Brouwer trees, $\indMax$ had all the properties we wanted
except for idempotence. All of these can be lifted directly to
SMB-trees:

\begin{code}%
%
\>[2]\AgdaFunction{max-≤L}\AgdaSpace{}%
\AgdaSymbol{:}\AgdaSpace{}%
\AgdaSymbol{∀}\AgdaSpace{}%
\AgdaSymbol{\{}\AgdaBound{t1}\AgdaSpace{}%
\AgdaBound{t2}\AgdaSymbol{\}}\AgdaSpace{}%
\AgdaSymbol{→}\AgdaSpace{}%
\AgdaBound{t1}\AgdaSpace{}%
\AgdaOperator{\AgdaRecord{≤}}\AgdaSpace{}%
\AgdaFunction{max}\AgdaSpace{}%
\AgdaBound{t1}\AgdaSpace{}%
\AgdaBound{t2}\<%
\\
%
\\[\AgdaEmptyExtraSkip]%
%
\>[2]\AgdaFunction{max-≤R}\AgdaSpace{}%
\AgdaSymbol{:}\AgdaSpace{}%
\AgdaSymbol{∀}\AgdaSpace{}%
\AgdaSymbol{\{}\AgdaBound{t1}\AgdaSpace{}%
\AgdaBound{t2}\AgdaSymbol{\}}\AgdaSpace{}%
\AgdaSymbol{→}\AgdaSpace{}%
\AgdaBound{t2}\AgdaSpace{}%
\AgdaOperator{\AgdaRecord{≤}}\AgdaSpace{}%
\AgdaFunction{max}\AgdaSpace{}%
\AgdaBound{t1}\AgdaSpace{}%
\AgdaBound{t2}\<%
\\
%
\\[\AgdaEmptyExtraSkip]%
%
\>[2]\AgdaFunction{max-mono}\AgdaSpace{}%
\AgdaSymbol{:}\AgdaSpace{}%
\AgdaSymbol{∀}\AgdaSpace{}%
\AgdaSymbol{\{}\AgdaBound{t1}\AgdaSpace{}%
\AgdaBound{t1'}\AgdaSpace{}%
\AgdaBound{t2}\AgdaSpace{}%
\AgdaBound{t2'}\AgdaSymbol{\}}\AgdaSpace{}%
\AgdaSymbol{→}\AgdaSpace{}%
\AgdaBound{t1}\AgdaSpace{}%
\AgdaOperator{\AgdaRecord{≤}}\AgdaSpace{}%
\AgdaBound{t1'}\AgdaSpace{}%
\AgdaSymbol{→}\AgdaSpace{}%
\AgdaBound{t2}\AgdaSpace{}%
\AgdaOperator{\AgdaRecord{≤}}\AgdaSpace{}%
\AgdaBound{t2'}\AgdaSpace{}%
\AgdaSymbol{→}\<%
\\
\>[2][@{}l@{\AgdaIndent{0}}]%
\>[4]\AgdaFunction{max}\AgdaSpace{}%
\AgdaBound{t1}\AgdaSpace{}%
\AgdaBound{t2}\AgdaSpace{}%
\AgdaOperator{\AgdaRecord{≤}}\AgdaSpace{}%
\AgdaFunction{max}\AgdaSpace{}%
\AgdaBound{t1'}\AgdaSpace{}%
\AgdaBound{t2'}\<%
\\
%
\\[\AgdaEmptyExtraSkip]%
%
\>[2]\AgdaFunction{max-idem≤}\AgdaSpace{}%
\AgdaSymbol{:}\AgdaSpace{}%
\AgdaSymbol{∀}\AgdaSpace{}%
\AgdaSymbol{\{}\AgdaBound{t}\AgdaSymbol{\}}\AgdaSpace{}%
\AgdaSymbol{→}\AgdaSpace{}%
\AgdaBound{t}\AgdaSpace{}%
\AgdaOperator{\AgdaRecord{≤}}\AgdaSpace{}%
\AgdaFunction{max}\AgdaSpace{}%
\AgdaBound{t}\AgdaSpace{}%
\AgdaBound{t}\<%
\\
%
\\[\AgdaEmptyExtraSkip]%
%
\>[2]\AgdaFunction{max-commut}\AgdaSpace{}%
\AgdaSymbol{:}\AgdaSpace{}%
\AgdaSymbol{∀}\AgdaSpace{}%
\AgdaBound{t1}\AgdaSpace{}%
\AgdaBound{t2}\AgdaSpace{}%
\AgdaSymbol{→}\AgdaSpace{}%
\AgdaFunction{max}\AgdaSpace{}%
\AgdaBound{t1}\AgdaSpace{}%
\AgdaBound{t2}\AgdaSpace{}%
\AgdaOperator{\AgdaRecord{≤}}\AgdaSpace{}%
\AgdaFunction{max}\AgdaSpace{}%
\AgdaBound{t2}\AgdaSpace{}%
\AgdaBound{t1}\<%
\\
%
\\[\AgdaEmptyExtraSkip]%
%
\>[2]\AgdaFunction{max-assocL}\AgdaSpace{}%
\AgdaSymbol{:}\AgdaSpace{}%
\AgdaSymbol{∀}\AgdaSpace{}%
\AgdaBound{t1}\AgdaSpace{}%
\AgdaBound{t2}\AgdaSpace{}%
\AgdaBound{t3}\AgdaSpace{}%
\AgdaSymbol{→}\AgdaSpace{}%
\AgdaFunction{max}\AgdaSpace{}%
\AgdaBound{t1}\AgdaSpace{}%
\AgdaSymbol{(}\AgdaFunction{max}\AgdaSpace{}%
\AgdaBound{t2}\AgdaSpace{}%
\AgdaBound{t3}\AgdaSymbol{)}\AgdaSpace{}%
\AgdaOperator{\AgdaRecord{≤}}\AgdaSpace{}%
\AgdaFunction{max}\AgdaSpace{}%
\AgdaSymbol{(}\AgdaFunction{max}\AgdaSpace{}%
\AgdaBound{t1}\AgdaSpace{}%
\AgdaBound{t2}\AgdaSymbol{)}\AgdaSpace{}%
\AgdaBound{t3}\<%
\\
%
\\[\AgdaEmptyExtraSkip]%
%
\>[2]\AgdaFunction{max-assocR}\AgdaSpace{}%
\AgdaSymbol{:}\AgdaSpace{}%
\AgdaSymbol{∀}\AgdaSpace{}%
\AgdaBound{t1}\AgdaSpace{}%
\AgdaBound{t2}\AgdaSpace{}%
\AgdaBound{t3}\AgdaSpace{}%
\AgdaSymbol{→}%
\>[29]\AgdaFunction{max}\AgdaSpace{}%
\AgdaSymbol{(}\AgdaFunction{max}\AgdaSpace{}%
\AgdaBound{t1}\AgdaSpace{}%
\AgdaBound{t2}\AgdaSymbol{)}\AgdaSpace{}%
\AgdaBound{t3}\AgdaSpace{}%
\AgdaOperator{\AgdaRecord{≤}}\AgdaSpace{}%
\AgdaFunction{max}\AgdaSpace{}%
\AgdaBound{t1}\AgdaSpace{}%
\AgdaSymbol{(}\AgdaFunction{max}\AgdaSpace{}%
\AgdaBound{t2}\AgdaSpace{}%
\AgdaBound{t3}\AgdaSymbol{)}\<%
\end{code}

In particular, $\AgdaFunction{max}$ is strictly monotone, and distributes over
the successor:

\begin{code}%
%
\>[2]\AgdaFunction{max-strictMono}\AgdaSpace{}%
\AgdaSymbol{:}\AgdaSpace{}%
\AgdaSymbol{∀}\AgdaSpace{}%
\AgdaSymbol{\{}\AgdaBound{t1}\AgdaSpace{}%
\AgdaBound{t1'}\AgdaSpace{}%
\AgdaBound{t2}\AgdaSpace{}%
\AgdaBound{t2'}\AgdaSpace{}%
\AgdaSymbol{:}\AgdaSpace{}%
\AgdaRecord{SMBTree}\AgdaSymbol{\}}\<%
\\
\>[2][@{}l@{\AgdaIndent{0}}]%
\>[4]\AgdaSymbol{→}\AgdaSpace{}%
\AgdaBound{t1}\AgdaSpace{}%
\AgdaOperator{\AgdaFunction{<}}\AgdaSpace{}%
\AgdaBound{t1'}\AgdaSpace{}%
\AgdaSymbol{→}\AgdaSpace{}%
\AgdaBound{t2}\AgdaSpace{}%
\AgdaOperator{\AgdaFunction{<}}\AgdaSpace{}%
\AgdaBound{t2'}\AgdaSpace{}%
\AgdaSymbol{→}\AgdaSpace{}%
\AgdaFunction{max}\AgdaSpace{}%
\AgdaBound{t1}\AgdaSpace{}%
\AgdaBound{t2}\AgdaSpace{}%
\AgdaOperator{\AgdaFunction{<}}\AgdaSpace{}%
\AgdaFunction{max}\AgdaSpace{}%
\AgdaBound{t1'}\AgdaSpace{}%
\AgdaBound{t2'}\<%
\\
%
\\[\AgdaEmptyExtraSkip]%
%
\>[2]\AgdaFunction{max-sucMono}\AgdaSpace{}%
\AgdaSymbol{:}\AgdaSpace{}%
\AgdaSymbol{∀}\AgdaSpace{}%
\AgdaSymbol{\{}\AgdaBound{t1}\AgdaSpace{}%
\AgdaBound{t2}\AgdaSpace{}%
\AgdaBound{t1'}\AgdaSpace{}%
\AgdaBound{t2'}\AgdaSpace{}%
\AgdaSymbol{:}\AgdaSpace{}%
\AgdaRecord{SMBTree}\AgdaSymbol{\}}\<%
\\
\>[2][@{}l@{\AgdaIndent{0}}]%
\>[4]\AgdaSymbol{→}\AgdaSpace{}%
\AgdaFunction{max}\AgdaSpace{}%
\AgdaBound{t1}\AgdaSpace{}%
\AgdaBound{t2}\AgdaSpace{}%
\AgdaOperator{\AgdaRecord{≤}}\AgdaSpace{}%
\AgdaFunction{max}\AgdaSpace{}%
\AgdaBound{t1'}\AgdaSpace{}%
\AgdaBound{t2'}\AgdaSpace{}%
\AgdaSymbol{→}\AgdaSpace{}%
\AgdaFunction{max}\AgdaSpace{}%
\AgdaBound{t1}\AgdaSpace{}%
\AgdaBound{t2}\AgdaSpace{}%
\AgdaOperator{\AgdaFunction{<}}\AgdaSpace{}%
\AgdaFunction{max}\AgdaSpace{}%
\AgdaSymbol{(}\AgdaFunction{↑}\AgdaSpace{}%
\AgdaBound{t1'}\AgdaSymbol{)}\AgdaSpace{}%
\AgdaSymbol{(}\AgdaFunction{↑}\AgdaSpace{}%
\AgdaBound{t2'}\AgdaSymbol{)}\<%
\end{code}

However, because we restricted SMB-trees to only contain Brouwer trees that
$\indMax$ is idempotent on, we can prove that $\AgdaFunction{Max}$ is
idempotent for SMB-trees:

\begin{code}%
%
\>[2]\AgdaFunction{max-idem}\AgdaSpace{}%
\AgdaSymbol{:}\AgdaSpace{}%
\AgdaSymbol{∀}\AgdaSpace{}%
\AgdaSymbol{\{}\AgdaBound{t}\AgdaSpace{}%
\AgdaSymbol{:}\AgdaSpace{}%
\AgdaRecord{SMBTree}\AgdaSymbol{\}}\AgdaSpace{}%
\AgdaSymbol{→}\AgdaSpace{}%
\AgdaFunction{max}\AgdaSpace{}%
\AgdaBound{t}\AgdaSpace{}%
\AgdaBound{t}\AgdaSpace{}%
\AgdaOperator{\AgdaRecord{≤}}\AgdaSpace{}%
\AgdaBound{t}\<%
\\
%
\>[2]\AgdaFunction{max-idem}\AgdaSpace{}%
\AgdaSymbol{\{}\AgdaArgument{t}\AgdaSpace{}%
\AgdaSymbol{=}\AgdaSpace{}%
\AgdaInductiveConstructor{MkTree}\AgdaSpace{}%
\AgdaBound{t}\AgdaSpace{}%
\AgdaBound{pf}\AgdaSymbol{\}}\AgdaSpace{}%
\AgdaSymbol{=}\AgdaSpace{}%
\AgdaInductiveConstructor{mk≤}\AgdaSpace{}%
\AgdaBound{pf}\<%
\end{code}

\begin{code}[hide]%
\>[0]\<%
\\
%
\>[2]\AgdaFunction{≤-extLim}\AgdaSpace{}%
\AgdaSymbol{:}\AgdaSpace{}%
\AgdaSymbol{∀}%
\>[16]\AgdaSymbol{\{}\AgdaBound{c}\AgdaSpace{}%
\AgdaSymbol{:}\AgdaSpace{}%
\AgdaBound{ℂ}\AgdaSymbol{\}}\AgdaSpace{}%
\AgdaSymbol{→}\AgdaSpace{}%
\AgdaSymbol{\{}\AgdaBound{f1}\AgdaSpace{}%
\AgdaBound{f2}\AgdaSpace{}%
\AgdaSymbol{:}\AgdaSpace{}%
\AgdaBound{El}\AgdaSpace{}%
\AgdaBound{c}\AgdaSpace{}%
\AgdaSymbol{→}\AgdaSpace{}%
\AgdaRecord{SMBTree}\AgdaSymbol{\}}\<%
\\
\>[2][@{}l@{\AgdaIndent{0}}]%
\>[4]\AgdaSymbol{→}\AgdaSpace{}%
\AgdaSymbol{(∀}\AgdaSpace{}%
\AgdaBound{k}\AgdaSpace{}%
\AgdaSymbol{→}\AgdaSpace{}%
\AgdaBound{f1}\AgdaSpace{}%
\AgdaBound{k}\AgdaSpace{}%
\AgdaOperator{\AgdaRecord{≤}}\AgdaSpace{}%
\AgdaBound{f2}\AgdaSpace{}%
\AgdaBound{k}\AgdaSymbol{)}\<%
\\
%
\>[4]\AgdaSymbol{→}\AgdaSpace{}%
\AgdaFunction{Lim}\AgdaSpace{}%
\AgdaBound{c}\AgdaSpace{}%
\AgdaBound{f1}\AgdaSpace{}%
\AgdaOperator{\AgdaRecord{≤}}\AgdaSpace{}%
\AgdaFunction{Lim}\AgdaSpace{}%
\AgdaBound{c}\AgdaSpace{}%
\AgdaBound{f2}\<%
\\
%
\>[2]\AgdaFunction{≤-extLim}\AgdaSpace{}%
\AgdaBound{lt}\AgdaSpace{}%
\AgdaSymbol{=}\AgdaSpace{}%
\AgdaFunction{≤-limLeast}\AgdaSpace{}%
\AgdaSymbol{(λ}\AgdaSpace{}%
\AgdaBound{k}\AgdaSpace{}%
\AgdaSymbol{→}\AgdaSpace{}%
\AgdaBound{lt}\AgdaSpace{}%
\AgdaBound{k}\AgdaSpace{}%
\AgdaOperator{\AgdaFunction{≤⨟}}\AgdaSpace{}%
\AgdaFunction{≤-limUpperBound}\AgdaSpace{}%
\AgdaBound{k}\AgdaSymbol{)}\<%
\\
%
\\[\AgdaEmptyExtraSkip]%
%
\>[2]\AgdaFunction{≤-extExists}\AgdaSpace{}%
\AgdaSymbol{:}\AgdaSpace{}%
\AgdaSymbol{∀}%
\>[19]\AgdaSymbol{\{}\AgdaBound{c1}\AgdaSpace{}%
\AgdaBound{c2}\AgdaSpace{}%
\AgdaSymbol{:}\AgdaSpace{}%
\AgdaBound{ℂ}\AgdaSymbol{\}}\AgdaSpace{}%
\AgdaSymbol{→}\AgdaSpace{}%
\AgdaSymbol{\{}\AgdaBound{f1}\AgdaSpace{}%
\AgdaSymbol{:}\AgdaSpace{}%
\AgdaBound{El}\AgdaSpace{}%
\AgdaBound{c1}\AgdaSpace{}%
\AgdaSymbol{→}\AgdaSpace{}%
\AgdaRecord{SMBTree}\AgdaSymbol{\}}\AgdaSpace{}%
\AgdaSymbol{\{}\AgdaBound{f2}\AgdaSpace{}%
\AgdaSymbol{:}\AgdaSpace{}%
\AgdaBound{El}\AgdaSpace{}%
\AgdaBound{c2}\AgdaSpace{}%
\AgdaSymbol{→}\AgdaSpace{}%
\AgdaRecord{SMBTree}\AgdaSymbol{\}}\<%
\\
\>[2][@{}l@{\AgdaIndent{0}}]%
\>[4]\AgdaSymbol{→}\AgdaSpace{}%
\AgdaSymbol{(∀}\AgdaSpace{}%
\AgdaBound{k1}\AgdaSpace{}%
\AgdaSymbol{→}\AgdaSpace{}%
\AgdaFunction{Σ[}\AgdaSpace{}%
\AgdaBound{k2}\AgdaSpace{}%
\AgdaFunction{∈}\AgdaSpace{}%
\AgdaBound{El}\AgdaSpace{}%
\AgdaBound{c2}\AgdaSpace{}%
\AgdaFunction{]}\AgdaSpace{}%
\AgdaBound{f1}\AgdaSpace{}%
\AgdaBound{k1}\AgdaSpace{}%
\AgdaOperator{\AgdaRecord{≤}}\AgdaSpace{}%
\AgdaBound{f2}\AgdaSpace{}%
\AgdaBound{k2}\AgdaSymbol{)}\<%
\\
%
\>[4]\AgdaSymbol{→}\AgdaSpace{}%
\AgdaFunction{Lim}\AgdaSpace{}%
\AgdaBound{c1}\AgdaSpace{}%
\AgdaBound{f1}\AgdaSpace{}%
\AgdaOperator{\AgdaRecord{≤}}\AgdaSpace{}%
\AgdaFunction{Lim}\AgdaSpace{}%
\AgdaBound{c2}\AgdaSpace{}%
\AgdaBound{f2}\<%
\\
%
\>[2]\AgdaFunction{≤-extExists}\AgdaSpace{}%
\AgdaSymbol{\{}\AgdaArgument{f1}\AgdaSpace{}%
\AgdaSymbol{=}\AgdaSpace{}%
\AgdaBound{f1}\AgdaSymbol{\}}\AgdaSpace{}%
\AgdaSymbol{\{}\AgdaBound{f2}\AgdaSymbol{\}}\AgdaSpace{}%
\AgdaBound{lt}\AgdaSpace{}%
\AgdaSymbol{=}\AgdaSpace{}%
\AgdaFunction{≤-limLeast}\AgdaSpace{}%
\AgdaSymbol{(λ}\AgdaSpace{}%
\AgdaBound{k1}\AgdaSpace{}%
\AgdaSymbol{→}\AgdaSpace{}%
\AgdaField{proj₂}\AgdaSpace{}%
\AgdaSymbol{(}\AgdaBound{lt}\AgdaSpace{}%
\AgdaBound{k1}\AgdaSymbol{)}\AgdaSpace{}%
\AgdaOperator{\AgdaFunction{≤⨟}}\AgdaSpace{}%
\AgdaFunction{≤-limUpperBound}\AgdaSpace{}%
\AgdaSymbol{(}\AgdaField{proj₁}\AgdaSpace{}%
\AgdaSymbol{(}\AgdaBound{lt}\AgdaSpace{}%
\AgdaBound{k1}\AgdaSymbol{)))}\<%
\\
%
\\[\AgdaEmptyExtraSkip]%
%
\>[2]\AgdaFunction{¬Z<↑}\AgdaSpace{}%
\AgdaSymbol{:}\AgdaSpace{}%
\AgdaSymbol{∀}%
\>[12]\AgdaBound{t}\AgdaSpace{}%
\AgdaSymbol{→}\AgdaSpace{}%
\AgdaOperator{\AgdaFunction{¬}}\AgdaSpace{}%
\AgdaSymbol{((}\AgdaFunction{↑}\AgdaSpace{}%
\AgdaBound{t}\AgdaSymbol{)}\AgdaSpace{}%
\AgdaOperator{\AgdaRecord{≤}}\AgdaSpace{}%
\AgdaFunction{Z}\AgdaSymbol{)}\<%
\\
%
\>[2]\AgdaFunction{¬Z<↑}\AgdaSpace{}%
\AgdaBound{t}\AgdaSpace{}%
\AgdaBound{pf}\AgdaSpace{}%
\AgdaSymbol{=}\AgdaSpace{}%
\AgdaPostulate{Brouwer.¬<Z}\AgdaSpace{}%
\AgdaSymbol{(}\AgdaField{rawTree}\AgdaSpace{}%
\AgdaBound{t}\AgdaSymbol{)}\AgdaSpace{}%
\AgdaSymbol{(}\AgdaField{get≤}\AgdaSpace{}%
\AgdaBound{pf}\AgdaSymbol{)}\<%
\\
%
\\[\AgdaEmptyExtraSkip]%
%
\>[2]\AgdaFunction{max-≤L}\AgdaSpace{}%
\AgdaSymbol{=}\AgdaSpace{}%
\AgdaInductiveConstructor{mk≤}\AgdaSpace{}%
\AgdaPostulate{indMax-≤L}\<%
\\
%
\\[\AgdaEmptyExtraSkip]%
%
\>[2]\AgdaFunction{max-≤R}\AgdaSpace{}%
\AgdaSymbol{=}%
\>[12]\AgdaInductiveConstructor{mk≤}\AgdaSpace{}%
\AgdaPostulate{indMax-≤R}\<%
\\
%
\\[\AgdaEmptyExtraSkip]%
%
\>[2]\AgdaFunction{max-mono}\AgdaSpace{}%
\AgdaBound{lt1}\AgdaSpace{}%
\AgdaBound{lt2}\AgdaSpace{}%
\AgdaSymbol{=}\AgdaSpace{}%
\AgdaInductiveConstructor{mk≤}\AgdaSpace{}%
\AgdaSymbol{(}\AgdaPostulate{indMax-mono}\AgdaSpace{}%
\AgdaSymbol{(}\AgdaField{get≤}\AgdaSpace{}%
\AgdaBound{lt1}\AgdaSymbol{)}\AgdaSpace{}%
\AgdaSymbol{(}\AgdaField{get≤}\AgdaSpace{}%
\AgdaBound{lt2}\AgdaSymbol{))}\<%
\\
%
\\[\AgdaEmptyExtraSkip]%
%
\>[2]\AgdaFunction{max-monoR}\AgdaSpace{}%
\AgdaSymbol{:}\AgdaSpace{}%
\AgdaSymbol{∀}\AgdaSpace{}%
\AgdaSymbol{\{}\AgdaBound{t1}\AgdaSpace{}%
\AgdaBound{t2}\AgdaSpace{}%
\AgdaBound{t2'}\AgdaSymbol{\}}\AgdaSpace{}%
\AgdaSymbol{→}\AgdaSpace{}%
\AgdaBound{t2}\AgdaSpace{}%
\AgdaOperator{\AgdaRecord{≤}}\AgdaSpace{}%
\AgdaBound{t2'}\AgdaSpace{}%
\AgdaSymbol{→}\AgdaSpace{}%
\AgdaFunction{max}\AgdaSpace{}%
\AgdaBound{t1}\AgdaSpace{}%
\AgdaBound{t2}\AgdaSpace{}%
\AgdaOperator{\AgdaRecord{≤}}\AgdaSpace{}%
\AgdaFunction{max}\AgdaSpace{}%
\AgdaBound{t1}\AgdaSpace{}%
\AgdaBound{t2'}\<%
\\
%
\>[2]\AgdaFunction{max-monoR}\AgdaSpace{}%
\AgdaSymbol{\{}\AgdaBound{t1}\AgdaSymbol{\}}\AgdaSpace{}%
\AgdaSymbol{\{}\AgdaBound{t2}\AgdaSymbol{\}}\AgdaSpace{}%
\AgdaSymbol{\{}\AgdaBound{t2'}\AgdaSymbol{\}}\AgdaSpace{}%
\AgdaBound{lt}\AgdaSpace{}%
\AgdaSymbol{=}\AgdaSpace{}%
\AgdaFunction{max-mono}\AgdaSpace{}%
\AgdaSymbol{\{}\AgdaArgument{t1}\AgdaSpace{}%
\AgdaSymbol{=}\AgdaSpace{}%
\AgdaBound{t1}\AgdaSymbol{\}}\AgdaSpace{}%
\AgdaSymbol{\{}\AgdaArgument{t1'}\AgdaSpace{}%
\AgdaSymbol{=}\AgdaSpace{}%
\AgdaBound{t1}\AgdaSymbol{\}}\AgdaSpace{}%
\AgdaSymbol{\{}\AgdaArgument{t2}\AgdaSpace{}%
\AgdaSymbol{=}\AgdaSpace{}%
\AgdaBound{t2}\AgdaSymbol{\}}\AgdaSpace{}%
\AgdaSymbol{\{}\AgdaArgument{t2'}\AgdaSpace{}%
\AgdaSymbol{=}\AgdaSpace{}%
\AgdaBound{t2'}\AgdaSymbol{\}}\AgdaSpace{}%
\AgdaSymbol{(}\AgdaFunction{≤-refl}\AgdaSpace{}%
\AgdaSymbol{\{}\AgdaBound{t1}\AgdaSymbol{\})}\AgdaSpace{}%
\AgdaBound{lt}\<%
\\
%
\\[\AgdaEmptyExtraSkip]%
%
\>[2]\AgdaFunction{max-monoL}\AgdaSpace{}%
\AgdaSymbol{:}\AgdaSpace{}%
\AgdaSymbol{∀}\AgdaSpace{}%
\AgdaSymbol{\{}\AgdaBound{t1}\AgdaSpace{}%
\AgdaBound{t1'}\AgdaSpace{}%
\AgdaBound{t2}\AgdaSymbol{\}}\AgdaSpace{}%
\AgdaSymbol{→}\AgdaSpace{}%
\AgdaBound{t1}\AgdaSpace{}%
\AgdaOperator{\AgdaRecord{≤}}\AgdaSpace{}%
\AgdaBound{t1'}\AgdaSpace{}%
\AgdaSymbol{→}\AgdaSpace{}%
\AgdaFunction{max}\AgdaSpace{}%
\AgdaBound{t1}\AgdaSpace{}%
\AgdaBound{t2}\AgdaSpace{}%
\AgdaOperator{\AgdaRecord{≤}}\AgdaSpace{}%
\AgdaFunction{max}\AgdaSpace{}%
\AgdaBound{t1'}\AgdaSpace{}%
\AgdaBound{t2}\<%
\\
%
\>[2]\AgdaFunction{max-monoL}\AgdaSpace{}%
\AgdaSymbol{\{}\AgdaBound{t1}\AgdaSymbol{\}}\AgdaSpace{}%
\AgdaSymbol{\{}\AgdaBound{t1'}\AgdaSymbol{\}}\AgdaSpace{}%
\AgdaSymbol{\{}\AgdaBound{t2}\AgdaSymbol{\}}\AgdaSpace{}%
\AgdaBound{lt}\AgdaSpace{}%
\AgdaSymbol{=}\AgdaSpace{}%
\AgdaFunction{max-mono}\AgdaSpace{}%
\AgdaSymbol{\{}\AgdaBound{t1}\AgdaSymbol{\}}\AgdaSpace{}%
\AgdaSymbol{\{}\AgdaBound{t1'}\AgdaSymbol{\}}\AgdaSpace{}%
\AgdaSymbol{\{}\AgdaBound{t2}\AgdaSymbol{\}}\AgdaSpace{}%
\AgdaSymbol{\{}\AgdaBound{t2}\AgdaSymbol{\}}\AgdaSpace{}%
\AgdaBound{lt}\AgdaSpace{}%
\AgdaSymbol{(}\AgdaFunction{≤-refl}\AgdaSpace{}%
\AgdaSymbol{\{}\AgdaBound{t2}\AgdaSymbol{\})}\<%
\\
%
\\[\AgdaEmptyExtraSkip]%
%
\\[\AgdaEmptyExtraSkip]%
%
\>[2]\AgdaFunction{max-idem≤}\AgdaSpace{}%
\AgdaSymbol{\{}\AgdaArgument{t}\AgdaSpace{}%
\AgdaSymbol{=}\AgdaSpace{}%
\AgdaInductiveConstructor{MkTree}\AgdaSpace{}%
\AgdaBound{t}\AgdaSpace{}%
\AgdaBound{pf}\AgdaSymbol{\}}\AgdaSpace{}%
\AgdaSymbol{=}\AgdaSpace{}%
\AgdaFunction{max-≤L}\<%
\\
%
\\[\AgdaEmptyExtraSkip]%
%
\\[\AgdaEmptyExtraSkip]%
%
\>[2]\AgdaFunction{max-commut}\AgdaSpace{}%
\AgdaBound{t1}\AgdaSpace{}%
\AgdaBound{t2}\AgdaSpace{}%
\AgdaSymbol{=}%
\>[22]\AgdaInductiveConstructor{mk≤}\AgdaSpace{}%
\AgdaSymbol{(}\AgdaPostulate{indMax-commut}\AgdaSpace{}%
\AgdaSymbol{(}\AgdaField{rawTree}\AgdaSpace{}%
\AgdaBound{t1}\AgdaSymbol{)}\AgdaSpace{}%
\AgdaSymbol{(}\AgdaField{rawTree}\AgdaSpace{}%
\AgdaBound{t2}\AgdaSymbol{))}\<%
\\
%
\\[\AgdaEmptyExtraSkip]%
%
\>[2]\AgdaFunction{max-assocL}\AgdaSpace{}%
\AgdaBound{t1}\AgdaSpace{}%
\AgdaBound{t2}\AgdaSpace{}%
\AgdaBound{t3}\AgdaSpace{}%
\AgdaSymbol{=}\AgdaSpace{}%
\AgdaInductiveConstructor{mk≤}\AgdaSpace{}%
\AgdaSymbol{(}\AgdaPostulate{indMax-assocL}\AgdaSpace{}%
\AgdaSymbol{\AgdaUnderscore{}}\AgdaSpace{}%
\AgdaSymbol{\AgdaUnderscore{}}\AgdaSpace{}%
\AgdaSymbol{\AgdaUnderscore{})}\<%
\\
%
\\[\AgdaEmptyExtraSkip]%
%
\>[2]\AgdaFunction{max-assocR}\AgdaSpace{}%
\AgdaBound{t1}\AgdaSpace{}%
\AgdaBound{t2}\AgdaSpace{}%
\AgdaBound{t3}\AgdaSpace{}%
\AgdaSymbol{=}\AgdaSpace{}%
\AgdaInductiveConstructor{mk≤}\AgdaSpace{}%
\AgdaSymbol{(}\AgdaPostulate{indMax-assocR}\AgdaSpace{}%
\AgdaSymbol{\AgdaUnderscore{}}\AgdaSpace{}%
\AgdaSymbol{\AgdaUnderscore{}}\AgdaSpace{}%
\AgdaSymbol{\AgdaUnderscore{})}\<%
\\
%
\\[\AgdaEmptyExtraSkip]%
%
\>[2]\AgdaFunction{max-swap4}\AgdaSpace{}%
\AgdaSymbol{:}\AgdaSpace{}%
\AgdaSymbol{∀}\AgdaSpace{}%
\AgdaSymbol{\{}\AgdaBound{t1}\AgdaSpace{}%
\AgdaBound{t1'}\AgdaSpace{}%
\AgdaBound{t2}\AgdaSpace{}%
\AgdaBound{t2'}\AgdaSymbol{\}}\AgdaSpace{}%
\AgdaSymbol{→}\AgdaSpace{}%
\AgdaFunction{max}\AgdaSpace{}%
\AgdaSymbol{(}\AgdaFunction{max}\AgdaSpace{}%
\AgdaBound{t1}\AgdaSpace{}%
\AgdaBound{t1'}\AgdaSymbol{)}\AgdaSpace{}%
\AgdaSymbol{(}\AgdaFunction{max}\AgdaSpace{}%
\AgdaBound{t2}\AgdaSpace{}%
\AgdaBound{t2'}\AgdaSymbol{)}\AgdaSpace{}%
\AgdaOperator{\AgdaRecord{≤}}\AgdaSpace{}%
\AgdaFunction{max}\AgdaSpace{}%
\AgdaSymbol{(}\AgdaFunction{max}\AgdaSpace{}%
\AgdaBound{t1}\AgdaSpace{}%
\AgdaBound{t2}\AgdaSymbol{)}\AgdaSpace{}%
\AgdaSymbol{(}\AgdaFunction{max}\AgdaSpace{}%
\AgdaBound{t1'}\AgdaSpace{}%
\AgdaBound{t2'}\AgdaSymbol{)}\<%
\\
%
\>[2]\AgdaFunction{max-swap4}\AgdaSpace{}%
\AgdaSymbol{=}%
\>[15]\AgdaInductiveConstructor{mk≤}\AgdaSpace{}%
\AgdaPostulate{indMax-swap4}\<%
\\
%
\\[\AgdaEmptyExtraSkip]%
%
\>[2]\AgdaFunction{max-strictMono}\AgdaSpace{}%
\AgdaBound{lt1}\AgdaSpace{}%
\AgdaBound{lt2}\AgdaSpace{}%
\AgdaSymbol{=}\AgdaSpace{}%
\AgdaInductiveConstructor{mk≤}\AgdaSpace{}%
\AgdaSymbol{(}\AgdaPostulate{indMax-strictMono}\AgdaSpace{}%
\AgdaSymbol{(}\AgdaField{get≤}\AgdaSpace{}%
\AgdaBound{lt1}\AgdaSymbol{)}\AgdaSpace{}%
\AgdaSymbol{(}\AgdaField{get≤}\AgdaSpace{}%
\AgdaBound{lt2}\AgdaSymbol{))}\<%
\\
%
\\[\AgdaEmptyExtraSkip]%
%
\>[2]\AgdaFunction{max-sucMono}\AgdaSpace{}%
\AgdaBound{lt}\AgdaSpace{}%
\AgdaSymbol{=}%
\>[20]\AgdaInductiveConstructor{mk≤}\AgdaSpace{}%
\AgdaSymbol{(}\AgdaPostulate{indMax-sucMono}\AgdaSpace{}%
\AgdaSymbol{(}\AgdaField{get≤}\AgdaSpace{}%
\AgdaBound{lt}\AgdaSymbol{))}\<%
\\
\>[0]\<%
\end{code}

These together are enough to prove that our maximum is
the least of all upper bounds.
\begin{code}%
\>[0][@{}l@{\AgdaIndent{1}}]%
\>[2]\AgdaFunction{max-LUB}\AgdaSpace{}%
\AgdaSymbol{:}\AgdaSpace{}%
\AgdaSymbol{∀}\AgdaSpace{}%
\AgdaSymbol{\{}\AgdaBound{t1}\AgdaSpace{}%
\AgdaBound{t2}\AgdaSpace{}%
\AgdaBound{t}\AgdaSymbol{\}}\AgdaSpace{}%
\AgdaSymbol{→}\AgdaSpace{}%
\AgdaBound{t1}\AgdaSpace{}%
\AgdaOperator{\AgdaRecord{≤}}\AgdaSpace{}%
\AgdaBound{t}\AgdaSpace{}%
\AgdaSymbol{→}\AgdaSpace{}%
\AgdaBound{t2}\AgdaSpace{}%
\AgdaOperator{\AgdaRecord{≤}}\AgdaSpace{}%
\AgdaBound{t}\AgdaSpace{}%
\AgdaSymbol{→}\AgdaSpace{}%
\AgdaFunction{max}\AgdaSpace{}%
\AgdaBound{t1}\AgdaSpace{}%
\AgdaBound{t2}\AgdaSpace{}%
\AgdaOperator{\AgdaRecord{≤}}\AgdaSpace{}%
\AgdaBound{t}\<%
\\
%
\>[2]\AgdaFunction{max-LUB}\AgdaSpace{}%
\AgdaBound{lt1}\AgdaSpace{}%
\AgdaBound{lt2}\AgdaSpace{}%
\AgdaSymbol{=}\AgdaSpace{}%
\AgdaFunction{max-mono}\AgdaSpace{}%
\AgdaBound{lt1}\AgdaSpace{}%
\AgdaBound{lt2}\AgdaSpace{}%
\AgdaOperator{\AgdaFunction{≤⨟}}\AgdaSpace{}%
\AgdaFunction{max-idem}\<%
\end{code}

  Perhaps surprisingly, this means that an SMB-tree version of $\limMax$
  is equivalent to $\AgdaFunction{max}$, since they are both the least upper bound:
  \begin{code}%
\>[0]\<%
\\
\>[0]\AgdaFunction{ℕLim}\AgdaSpace{}%
\AgdaSymbol{:}\AgdaSpace{}%
\AgdaSymbol{(}\AgdaDatatype{ℕ}\AgdaSpace{}%
\AgdaSymbol{→}\AgdaSpace{}%
\AgdaRecord{SMBTree}\AgdaSymbol{)}\AgdaSpace{}%
\AgdaSymbol{→}\AgdaSpace{}%
\AgdaRecord{SMBTree}\<%
\\
\>[0]\AgdaFunction{ℕLim}\AgdaSpace{}%
\AgdaBound{f}\AgdaSpace{}%
\AgdaSymbol{=}\AgdaSpace{}%
\AgdaFunction{Lim}\AgdaSpace{}%
\AgdaBound{Cℕ}%
\>[17]\AgdaSymbol{(λ}\AgdaSpace{}%
\AgdaBound{cn}\AgdaSpace{}%
\AgdaSymbol{→}\AgdaSpace{}%
\AgdaBound{f}\AgdaSpace{}%
\AgdaSymbol{(}\AgdaField{Iso.fun}\AgdaSpace{}%
\AgdaBound{CℕIso}\AgdaSpace{}%
\AgdaBound{cn}\AgdaSymbol{))}\<%
\\
%
\\[\AgdaEmptyExtraSkip]%
\>[0]\AgdaFunction{max'}\AgdaSpace{}%
\AgdaSymbol{:}\AgdaSpace{}%
\AgdaRecord{SMBTree}\AgdaSpace{}%
\AgdaSymbol{→}\AgdaSpace{}%
\AgdaRecord{SMBTree}\AgdaSpace{}%
\AgdaSymbol{→}\AgdaSpace{}%
\AgdaRecord{SMBTree}\<%
\\
\>[0]\AgdaFunction{max'}\AgdaSpace{}%
\AgdaBound{t1}\AgdaSpace{}%
\AgdaBound{t2}\AgdaSpace{}%
\AgdaSymbol{=}\AgdaSpace{}%
\AgdaFunction{ℕLim}\AgdaSpace{}%
\AgdaSymbol{(λ}\AgdaSpace{}%
\AgdaBound{n}\AgdaSpace{}%
\AgdaSymbol{→}\AgdaSpace{}%
\AgdaFunction{if0}\AgdaSpace{}%
\AgdaBound{n}\AgdaSpace{}%
\AgdaBound{t1}\AgdaSpace{}%
\AgdaBound{t2}\AgdaSymbol{)}\<%
\\
%
\\[\AgdaEmptyExtraSkip]%
\>[0]\AgdaFunction{max'-≤L}\AgdaSpace{}%
\AgdaSymbol{:}\AgdaSpace{}%
\AgdaSymbol{∀}\AgdaSpace{}%
\AgdaSymbol{\{}\AgdaBound{t1}\AgdaSpace{}%
\AgdaBound{t2}\AgdaSymbol{\}}\AgdaSpace{}%
\AgdaSymbol{→}\AgdaSpace{}%
\AgdaBound{t1}\AgdaSpace{}%
\AgdaOperator{\AgdaRecord{≤}}\AgdaSpace{}%
\AgdaFunction{max'}\AgdaSpace{}%
\AgdaBound{t1}\AgdaSpace{}%
\AgdaBound{t2}\<%
\\
%
\\[\AgdaEmptyExtraSkip]%
\>[0]\AgdaFunction{max'-≤R}\AgdaSpace{}%
\AgdaSymbol{:}\AgdaSpace{}%
\AgdaSymbol{∀}\AgdaSpace{}%
\AgdaSymbol{\{}\AgdaBound{t1}\AgdaSpace{}%
\AgdaBound{t2}\AgdaSymbol{\}}\AgdaSpace{}%
\AgdaSymbol{→}\AgdaSpace{}%
\AgdaBound{t2}\AgdaSpace{}%
\AgdaOperator{\AgdaRecord{≤}}\AgdaSpace{}%
\AgdaFunction{max'}\AgdaSpace{}%
\AgdaBound{t1}\AgdaSpace{}%
\AgdaBound{t2}\<%
\\
%
\\[\AgdaEmptyExtraSkip]%
\>[0]\AgdaFunction{max'-LUB}\AgdaSpace{}%
\AgdaSymbol{:}\AgdaSpace{}%
\AgdaSymbol{∀}\AgdaSpace{}%
\AgdaSymbol{\{}\AgdaBound{t1}\AgdaSpace{}%
\AgdaBound{t2}\AgdaSpace{}%
\AgdaBound{t}\AgdaSymbol{\}}\AgdaSpace{}%
\AgdaSymbol{→}\AgdaSpace{}%
\AgdaBound{t1}\AgdaSpace{}%
\AgdaOperator{\AgdaRecord{≤}}\AgdaSpace{}%
\AgdaBound{t}\AgdaSpace{}%
\AgdaSymbol{→}\AgdaSpace{}%
\AgdaBound{t2}\AgdaSpace{}%
\AgdaOperator{\AgdaRecord{≤}}\AgdaSpace{}%
\AgdaBound{t}\AgdaSpace{}%
\AgdaSymbol{→}\AgdaSpace{}%
\AgdaFunction{max'}\AgdaSpace{}%
\AgdaBound{t1}\AgdaSpace{}%
\AgdaBound{t2}\AgdaSpace{}%
\AgdaOperator{\AgdaRecord{≤}}\AgdaSpace{}%
\AgdaBound{t}\<%
\\
%
\\[\AgdaEmptyExtraSkip]%
\>[0]\AgdaFunction{max≤max'}\AgdaSpace{}%
\AgdaSymbol{:}\AgdaSpace{}%
\AgdaSymbol{∀}\AgdaSpace{}%
\AgdaSymbol{\{}\AgdaBound{t1}\AgdaSpace{}%
\AgdaBound{t2}\AgdaSymbol{\}}\AgdaSpace{}%
\AgdaSymbol{→}\AgdaSpace{}%
\AgdaFunction{max}\AgdaSpace{}%
\AgdaBound{t1}\AgdaSpace{}%
\AgdaBound{t2}\AgdaSpace{}%
\AgdaOperator{\AgdaRecord{≤}}\AgdaSpace{}%
\AgdaFunction{max'}\AgdaSpace{}%
\AgdaBound{t1}\AgdaSpace{}%
\AgdaBound{t2}\<%
\\
\>[0]\AgdaFunction{max≤max'}\AgdaSpace{}%
\AgdaSymbol{=}\AgdaSpace{}%
\AgdaFunction{max-LUB}\AgdaSpace{}%
\AgdaFunction{max'-≤L}\AgdaSpace{}%
\AgdaFunction{max'-≤R}\<%
\\
%
\\[\AgdaEmptyExtraSkip]%
\>[0]\AgdaFunction{max'≤max}\AgdaSpace{}%
\AgdaSymbol{:}\AgdaSpace{}%
\AgdaSymbol{∀}\AgdaSpace{}%
\AgdaSymbol{\{}\AgdaBound{t1}\AgdaSpace{}%
\AgdaBound{t2}\AgdaSymbol{\}}\AgdaSpace{}%
\AgdaSymbol{→}\AgdaSpace{}%
\AgdaFunction{max'}\AgdaSpace{}%
\AgdaBound{t1}\AgdaSpace{}%
\AgdaBound{t2}\AgdaSpace{}%
\AgdaOperator{\AgdaRecord{≤}}\AgdaSpace{}%
\AgdaFunction{max}\AgdaSpace{}%
\AgdaBound{t1}\AgdaSpace{}%
\AgdaBound{t2}\<%
\\
\>[0]\AgdaFunction{max'≤max}\AgdaSpace{}%
\AgdaSymbol{=}\AgdaSpace{}%
\AgdaFunction{max'-LUB}\AgdaSpace{}%
\AgdaFunction{max-≤L}\AgdaSpace{}%
\AgdaFunction{max-≤R}\<%
\end{code}


\begin{code}[hide]%
\>[0]\<%
\\
%
\\[\AgdaEmptyExtraSkip]%
%
\\[\AgdaEmptyExtraSkip]%
%
\\[\AgdaEmptyExtraSkip]%
\>[0]\AgdaFunction{max'-≤L}\AgdaSpace{}%
\AgdaSymbol{\{}\AgdaBound{t1}\AgdaSymbol{\}}\AgdaSpace{}%
\AgdaSymbol{\{}\AgdaBound{t2}\AgdaSymbol{\}}\<%
\\
\>[0][@{}l@{\AgdaIndent{0}}]%
\>[4]\AgdaSymbol{=}%
\>[923I]\AgdaFunction{subst}\AgdaSpace{}%
\AgdaSymbol{(λ}\AgdaSpace{}%
\AgdaBound{x}\AgdaSpace{}%
\AgdaSymbol{→}\AgdaSpace{}%
\AgdaBound{t1}\AgdaSpace{}%
\AgdaOperator{\AgdaRecord{≤}}\AgdaSpace{}%
\AgdaFunction{if0}\AgdaSpace{}%
\AgdaBound{x}\AgdaSpace{}%
\AgdaBound{t1}\AgdaSpace{}%
\AgdaBound{t2}\AgdaSymbol{)}\AgdaSpace{}%
\AgdaSymbol{(}\AgdaFunction{sym}\AgdaSpace{}%
\AgdaSymbol{(}\AgdaField{Iso.rightInv}\AgdaSpace{}%
\AgdaBound{CℕIso}\AgdaSpace{}%
\AgdaNumber{0}\AgdaSymbol{))}\AgdaSpace{}%
\AgdaFunction{≤-refl}\AgdaSpace{}%
\AgdaOperator{\AgdaFunction{≤⨟}}\<%
\\
\>[.][@{}l@{}]\<[923I]%
\>[6]\AgdaFunction{≤-limUpperBound}%
\>[23]\AgdaSymbol{(}\AgdaField{Iso.inv}\AgdaSpace{}%
\AgdaBound{CℕIso}\AgdaSpace{}%
\AgdaNumber{0}\AgdaSymbol{)}\<%
\\
%
\\[\AgdaEmptyExtraSkip]%
\>[0]\AgdaFunction{max'-≤R}\AgdaSpace{}%
\AgdaSymbol{\{}\AgdaBound{t1}\AgdaSymbol{\}}\AgdaSpace{}%
\AgdaSymbol{\{}\AgdaBound{t2}\AgdaSymbol{\}}\<%
\\
\>[0][@{}l@{\AgdaIndent{0}}]%
\>[4]\AgdaSymbol{=}%
\>[943I]\AgdaFunction{subst}\AgdaSpace{}%
\AgdaSymbol{(λ}\AgdaSpace{}%
\AgdaBound{x}\AgdaSpace{}%
\AgdaSymbol{→}\AgdaSpace{}%
\AgdaBound{t2}\AgdaSpace{}%
\AgdaOperator{\AgdaRecord{≤}}\AgdaSpace{}%
\AgdaFunction{if0}\AgdaSpace{}%
\AgdaBound{x}\AgdaSpace{}%
\AgdaBound{t1}\AgdaSpace{}%
\AgdaBound{t2}\AgdaSymbol{)}\AgdaSpace{}%
\AgdaSymbol{(}\AgdaFunction{sym}\AgdaSpace{}%
\AgdaSymbol{(}\AgdaField{Iso.rightInv}\AgdaSpace{}%
\AgdaBound{CℕIso}\AgdaSpace{}%
\AgdaNumber{1}\AgdaSymbol{))}\AgdaSpace{}%
\AgdaFunction{≤-refl}\AgdaSpace{}%
\AgdaOperator{\AgdaFunction{≤⨟}}\<%
\\
\>[.][@{}l@{}]\<[943I]%
\>[6]\AgdaFunction{≤-limUpperBound}%
\>[23]\AgdaSymbol{(}\AgdaField{Iso.inv}\AgdaSpace{}%
\AgdaBound{CℕIso}\AgdaSpace{}%
\AgdaNumber{1}\AgdaSymbol{)}\<%
\\
%
\\[\AgdaEmptyExtraSkip]%
%
\\[\AgdaEmptyExtraSkip]%
\>[0]\AgdaFunction{max'-Idem}\AgdaSpace{}%
\AgdaSymbol{:}\AgdaSpace{}%
\AgdaSymbol{∀}\AgdaSpace{}%
\AgdaSymbol{\{}\AgdaBound{t}\AgdaSymbol{\}}\AgdaSpace{}%
\AgdaSymbol{→}\AgdaSpace{}%
\AgdaFunction{max'}\AgdaSpace{}%
\AgdaBound{t}\AgdaSpace{}%
\AgdaBound{t}\AgdaSpace{}%
\AgdaOperator{\AgdaRecord{≤}}\AgdaSpace{}%
\AgdaBound{t}\<%
\\
\>[0]\AgdaFunction{max'-Idem}\AgdaSpace{}%
\AgdaSymbol{\{}\AgdaBound{t}\AgdaSymbol{\}}\AgdaSpace{}%
\AgdaSymbol{=}\AgdaSpace{}%
\AgdaFunction{≤-limLeast}%
\>[28]\AgdaFunction{helper}\<%
\\
\>[0][@{}l@{\AgdaIndent{0}}]%
\>[4]\AgdaKeyword{where}\<%
\\
%
\>[4]\AgdaFunction{helper}\AgdaSpace{}%
\AgdaSymbol{:}\AgdaSpace{}%
\AgdaSymbol{∀}\AgdaSpace{}%
\AgdaBound{k}\AgdaSpace{}%
\AgdaSymbol{→}\AgdaSpace{}%
\AgdaFunction{if0}\AgdaSpace{}%
\AgdaSymbol{(}\AgdaField{Iso.fun}\AgdaSpace{}%
\AgdaBound{CℕIso}\AgdaSpace{}%
\AgdaBound{k}\AgdaSymbol{)}\AgdaSpace{}%
\AgdaBound{t}\AgdaSpace{}%
\AgdaBound{t}\AgdaSpace{}%
\AgdaOperator{\AgdaRecord{≤}}\AgdaSpace{}%
\AgdaBound{t}\<%
\\
%
\>[4]\AgdaFunction{helper}\AgdaSpace{}%
\AgdaBound{k}\AgdaSpace{}%
\AgdaKeyword{with}\AgdaSpace{}%
\AgdaField{Iso.fun}\AgdaSpace{}%
\AgdaBound{CℕIso}\AgdaSpace{}%
\AgdaBound{k}\<%
\\
%
\>[4]\AgdaSymbol{...}\AgdaSpace{}%
\AgdaSymbol{|}\AgdaSpace{}%
\AgdaInductiveConstructor{zero}\AgdaSpace{}%
\AgdaSymbol{=}\AgdaSpace{}%
\AgdaFunction{≤-refl}\<%
\\
%
\>[4]\AgdaSymbol{...}\AgdaSpace{}%
\AgdaSymbol{|}\AgdaSpace{}%
\AgdaInductiveConstructor{suc}\AgdaSpace{}%
\AgdaBound{n}\AgdaSpace{}%
\AgdaSymbol{=}\AgdaSpace{}%
\AgdaFunction{≤-refl}\<%
\\
%
\\[\AgdaEmptyExtraSkip]%
\>[0]\AgdaFunction{max'-Mono}\AgdaSpace{}%
\AgdaSymbol{:}\AgdaSpace{}%
\AgdaSymbol{∀}\AgdaSpace{}%
\AgdaSymbol{\{}\AgdaBound{t1}\AgdaSpace{}%
\AgdaBound{t2}\AgdaSpace{}%
\AgdaBound{t1'}\AgdaSpace{}%
\AgdaBound{t2'}\AgdaSymbol{\}}\<%
\\
\>[0][@{}l@{\AgdaIndent{0}}]%
\>[4]\AgdaSymbol{→}\AgdaSpace{}%
\AgdaBound{t1}\AgdaSpace{}%
\AgdaOperator{\AgdaRecord{≤}}\AgdaSpace{}%
\AgdaBound{t1'}\AgdaSpace{}%
\AgdaSymbol{→}\AgdaSpace{}%
\AgdaBound{t2}\AgdaSpace{}%
\AgdaOperator{\AgdaRecord{≤}}\AgdaSpace{}%
\AgdaBound{t2'}\<%
\\
%
\>[4]\AgdaSymbol{→}\AgdaSpace{}%
\AgdaFunction{max'}\AgdaSpace{}%
\AgdaBound{t1}\AgdaSpace{}%
\AgdaBound{t2}\AgdaSpace{}%
\AgdaOperator{\AgdaRecord{≤}}\AgdaSpace{}%
\AgdaFunction{max'}\AgdaSpace{}%
\AgdaBound{t1'}\AgdaSpace{}%
\AgdaBound{t2'}\<%
\\
\>[0]\AgdaFunction{max'-Mono}\AgdaSpace{}%
\AgdaSymbol{\{}\AgdaBound{t1}\AgdaSymbol{\}}\AgdaSpace{}%
\AgdaSymbol{\{}\AgdaBound{t2}\AgdaSymbol{\}}\AgdaSpace{}%
\AgdaSymbol{\{}\AgdaBound{t1'}\AgdaSymbol{\}}\AgdaSpace{}%
\AgdaSymbol{\{}\AgdaBound{t2'}\AgdaSymbol{\}}\AgdaSpace{}%
\AgdaBound{lt1}\AgdaSpace{}%
\AgdaBound{lt2}\AgdaSpace{}%
\AgdaSymbol{=}\AgdaSpace{}%
\AgdaFunction{≤-extLim}%
\>[52]\AgdaFunction{helper}\<%
\\
\>[0][@{}l@{\AgdaIndent{0}}]%
\>[4]\AgdaKeyword{where}\<%
\\
%
\>[4]\AgdaFunction{helper}\AgdaSpace{}%
\AgdaSymbol{:}\AgdaSpace{}%
\AgdaSymbol{∀}\AgdaSpace{}%
\AgdaBound{k}\AgdaSpace{}%
\AgdaSymbol{→}\AgdaSpace{}%
\AgdaFunction{if0}\AgdaSpace{}%
\AgdaSymbol{(}\AgdaField{Iso.fun}\AgdaSpace{}%
\AgdaBound{CℕIso}\AgdaSpace{}%
\AgdaBound{k}\AgdaSymbol{)}\AgdaSpace{}%
\AgdaBound{t1}\AgdaSpace{}%
\AgdaBound{t2}\AgdaSpace{}%
\AgdaOperator{\AgdaRecord{≤}}\AgdaSpace{}%
\AgdaFunction{if0}\AgdaSpace{}%
\AgdaSymbol{(}\AgdaField{Iso.fun}\AgdaSpace{}%
\AgdaBound{CℕIso}\AgdaSpace{}%
\AgdaBound{k}\AgdaSymbol{)}\AgdaSpace{}%
\AgdaBound{t1'}\AgdaSpace{}%
\AgdaBound{t2'}\<%
\\
%
\>[4]\AgdaFunction{helper}\AgdaSpace{}%
\AgdaBound{k}\AgdaSpace{}%
\AgdaKeyword{with}\AgdaSpace{}%
\AgdaField{Iso.fun}\AgdaSpace{}%
\AgdaBound{CℕIso}\AgdaSpace{}%
\AgdaBound{k}\<%
\\
%
\>[4]\AgdaSymbol{...}\AgdaSpace{}%
\AgdaSymbol{|}\AgdaSpace{}%
\AgdaInductiveConstructor{zero}\AgdaSpace{}%
\AgdaSymbol{=}\AgdaSpace{}%
\AgdaBound{lt1}\<%
\\
%
\>[4]\AgdaSymbol{...}\AgdaSpace{}%
\AgdaSymbol{|}\AgdaSpace{}%
\AgdaInductiveConstructor{suc}\AgdaSpace{}%
\AgdaBound{n}\AgdaSpace{}%
\AgdaSymbol{=}\AgdaSpace{}%
\AgdaBound{lt2}\<%
\\
%
\\[\AgdaEmptyExtraSkip]%
%
\\[\AgdaEmptyExtraSkip]%
\>[0]\AgdaFunction{max'-LUB}\AgdaSpace{}%
\AgdaBound{lt1}\AgdaSpace{}%
\AgdaBound{lt2}\AgdaSpace{}%
\AgdaSymbol{=}\AgdaSpace{}%
\AgdaFunction{max'-Mono}\AgdaSpace{}%
\AgdaBound{lt1}\AgdaSpace{}%
\AgdaBound{lt2}\AgdaSpace{}%
\AgdaOperator{\AgdaFunction{≤⨟}}\AgdaSpace{}%
\AgdaFunction{max'-Idem}\<%
\\
%
\\[\AgdaEmptyExtraSkip]%
%
\\[\AgdaEmptyExtraSkip]%
%
\\[\AgdaEmptyExtraSkip]%
%
\\[\AgdaEmptyExtraSkip]%
%
\\[\AgdaEmptyExtraSkip]%
\>[0]\AgdaFunction{limSwap}\AgdaSpace{}%
\AgdaSymbol{:}\AgdaSpace{}%
\AgdaSymbol{∀}\AgdaSpace{}%
\AgdaSymbol{\{}\AgdaBound{c1}\AgdaSpace{}%
\AgdaBound{c2}\AgdaSpace{}%
\AgdaSymbol{\}}\AgdaSpace{}%
\AgdaSymbol{\{}\AgdaBound{f}\AgdaSpace{}%
\AgdaSymbol{:}\AgdaSpace{}%
\AgdaBound{El}\AgdaSpace{}%
\AgdaBound{c1}\AgdaSpace{}%
\AgdaSymbol{→}\AgdaSpace{}%
\AgdaBound{El}\AgdaSpace{}%
\AgdaBound{c2}\AgdaSpace{}%
\AgdaSymbol{→}\AgdaSpace{}%
\AgdaRecord{SMBTree}\AgdaSymbol{\}}\AgdaSpace{}%
\AgdaSymbol{→}\AgdaSpace{}%
\AgdaSymbol{(}\AgdaFunction{Lim}\AgdaSpace{}%
\AgdaBound{c1}\AgdaSpace{}%
\AgdaSymbol{λ}\AgdaSpace{}%
\AgdaBound{x}\AgdaSpace{}%
\AgdaSymbol{→}\AgdaSpace{}%
\AgdaFunction{Lim}\AgdaSpace{}%
\AgdaBound{c2}\AgdaSpace{}%
\AgdaSymbol{λ}\AgdaSpace{}%
\AgdaBound{y}\AgdaSpace{}%
\AgdaSymbol{→}\AgdaSpace{}%
\AgdaBound{f}\AgdaSpace{}%
\AgdaBound{x}\AgdaSpace{}%
\AgdaBound{y}\AgdaSymbol{)}\AgdaSpace{}%
\AgdaOperator{\AgdaRecord{≤}}\AgdaSpace{}%
\AgdaFunction{Lim}\AgdaSpace{}%
\AgdaBound{c2}\AgdaSpace{}%
\AgdaSymbol{λ}\AgdaSpace{}%
\AgdaBound{y}\AgdaSpace{}%
\AgdaSymbol{→}\AgdaSpace{}%
\AgdaFunction{Lim}\AgdaSpace{}%
\AgdaBound{c1}\AgdaSpace{}%
\AgdaSymbol{λ}\AgdaSpace{}%
\AgdaBound{x}\AgdaSpace{}%
\AgdaSymbol{→}\AgdaSpace{}%
\AgdaBound{f}\AgdaSpace{}%
\AgdaBound{x}\AgdaSpace{}%
\AgdaBound{y}\<%
\\
\>[0]\AgdaFunction{limSwap}\AgdaSpace{}%
\AgdaSymbol{=}\AgdaSpace{}%
\AgdaFunction{≤-limLeast}\AgdaSpace{}%
\AgdaSymbol{(λ}\AgdaSpace{}%
\AgdaBound{x}\AgdaSpace{}%
\AgdaSymbol{→}\AgdaSpace{}%
\AgdaFunction{≤-limLeast}\AgdaSpace{}%
\AgdaSymbol{λ}\AgdaSpace{}%
\AgdaBound{y}\AgdaSpace{}%
\AgdaSymbol{→}\AgdaSpace{}%
\AgdaFunction{≤-limUpperBound}\AgdaSpace{}%
\AgdaBound{x}\AgdaSpace{}%
\AgdaOperator{\AgdaFunction{≤⨟}}\AgdaSpace{}%
\AgdaFunction{≤-limUpperBound}\AgdaSpace{}%
\AgdaBound{y}%
\>[86]\AgdaSymbol{)}\<%
\\
%
\\[\AgdaEmptyExtraSkip]%
\>[0]\AgdaFunction{max-swapL}\AgdaSpace{}%
\AgdaSymbol{:}\AgdaSpace{}%
\AgdaSymbol{∀}\AgdaSpace{}%
\AgdaSymbol{\{}\AgdaBound{c}\AgdaSymbol{\}}\AgdaSpace{}%
\AgdaSymbol{\{}\AgdaBound{f}\AgdaSpace{}%
\AgdaBound{g}\AgdaSpace{}%
\AgdaSymbol{:}\AgdaSpace{}%
\AgdaBound{El}\AgdaSpace{}%
\AgdaBound{c}\AgdaSpace{}%
\AgdaSymbol{→}\AgdaSpace{}%
\AgdaRecord{SMBTree}\AgdaSymbol{\}}\AgdaSpace{}%
\AgdaSymbol{→}%
\>[44]\AgdaFunction{Lim}\AgdaSpace{}%
\AgdaBound{c}\AgdaSpace{}%
\AgdaSymbol{(λ}\AgdaSpace{}%
\AgdaBound{k}\AgdaSpace{}%
\AgdaSymbol{→}\AgdaSpace{}%
\AgdaFunction{max}\AgdaSpace{}%
\AgdaSymbol{(}\AgdaBound{f}\AgdaSpace{}%
\AgdaBound{k}\AgdaSymbol{)}\AgdaSpace{}%
\AgdaSymbol{(}\AgdaBound{g}\AgdaSpace{}%
\AgdaBound{k}\AgdaSymbol{))}\AgdaSpace{}%
\AgdaOperator{\AgdaRecord{≤}}\AgdaSpace{}%
\AgdaFunction{max}\AgdaSpace{}%
\AgdaSymbol{(}\AgdaFunction{Lim}\AgdaSpace{}%
\AgdaBound{c}\AgdaSpace{}%
\AgdaBound{f}\AgdaSymbol{)}\AgdaSpace{}%
\AgdaSymbol{(}\AgdaFunction{Lim}\AgdaSpace{}%
\AgdaBound{c}\AgdaSpace{}%
\AgdaBound{g}\AgdaSymbol{)}\<%
\\
\>[0]\AgdaFunction{max-swapL}\AgdaSpace{}%
\AgdaSymbol{\{}\AgdaBound{c}\AgdaSymbol{\}}\AgdaSpace{}%
\AgdaSymbol{\{}\AgdaBound{f}\AgdaSymbol{\}}\AgdaSpace{}%
\AgdaSymbol{\{}\AgdaBound{g}\AgdaSymbol{\}}\AgdaSpace{}%
\AgdaSymbol{=}\AgdaSpace{}%
\AgdaFunction{≤-extLim}\AgdaSpace{}%
\AgdaSymbol{(λ}\AgdaSpace{}%
\AgdaBound{k}\AgdaSpace{}%
\AgdaSymbol{→}\AgdaSpace{}%
\AgdaFunction{max≤max'}\AgdaSymbol{)}\AgdaSpace{}%
\AgdaOperator{\AgdaFunction{≤⨟}}\AgdaSpace{}%
\AgdaFunction{limSwap}\AgdaSpace{}%
\AgdaOperator{\AgdaFunction{≤⨟}}\AgdaSpace{}%
\AgdaFunction{≤-extLim}\AgdaSpace{}%
\AgdaFunction{helper}\AgdaSpace{}%
\AgdaOperator{\AgdaFunction{≤⨟}}\AgdaSpace{}%
\AgdaFunction{max'≤max}\<%
\\
\>[0][@{}l@{\AgdaIndent{0}}]%
\>[2]\AgdaKeyword{where}\<%
\\
\>[2][@{}l@{\AgdaIndent{0}}]%
\>[4]\AgdaFunction{helper}\AgdaSpace{}%
\AgdaSymbol{:}\AgdaSpace{}%
\AgdaSymbol{(}\AgdaBound{k}\AgdaSpace{}%
\AgdaSymbol{:}\AgdaSpace{}%
\AgdaBound{El}\AgdaSpace{}%
\AgdaBound{Cℕ}\AgdaSymbol{)}\AgdaSpace{}%
\AgdaSymbol{→}\<%
\\
\>[4][@{}l@{\AgdaIndent{0}}]%
\>[6]\AgdaFunction{Lim}\AgdaSpace{}%
\AgdaBound{c}\AgdaSpace{}%
\AgdaSymbol{(λ}\AgdaSpace{}%
\AgdaBound{x}\AgdaSpace{}%
\AgdaSymbol{→}\AgdaSpace{}%
\AgdaFunction{if0}\AgdaSpace{}%
\AgdaSymbol{(}\AgdaField{Iso.fun}\AgdaSpace{}%
\AgdaBound{CℕIso}\AgdaSpace{}%
\AgdaBound{k}\AgdaSymbol{)}\AgdaSpace{}%
\AgdaSymbol{(}\AgdaBound{f}\AgdaSpace{}%
\AgdaBound{x}\AgdaSymbol{)}\AgdaSpace{}%
\AgdaSymbol{(}\AgdaBound{g}\AgdaSpace{}%
\AgdaBound{x}\AgdaSymbol{))}\AgdaSpace{}%
\AgdaOperator{\AgdaRecord{≤}}\<%
\\
%
\>[6]\AgdaFunction{if0}\AgdaSpace{}%
\AgdaSymbol{(}\AgdaField{Iso.fun}\AgdaSpace{}%
\AgdaBound{CℕIso}\AgdaSpace{}%
\AgdaBound{k}\AgdaSymbol{)}\AgdaSpace{}%
\AgdaSymbol{(}\AgdaFunction{Lim}\AgdaSpace{}%
\AgdaBound{c}\AgdaSpace{}%
\AgdaBound{f}\AgdaSymbol{)}\AgdaSpace{}%
\AgdaSymbol{(}\AgdaFunction{Lim}\AgdaSpace{}%
\AgdaBound{c}\AgdaSpace{}%
\AgdaBound{g}\AgdaSymbol{)}\<%
\\
%
\>[4]\AgdaFunction{helper}\AgdaSpace{}%
\AgdaBound{kn}\AgdaSpace{}%
\AgdaKeyword{with}\AgdaSpace{}%
\AgdaField{Iso.fun}\AgdaSpace{}%
\AgdaBound{CℕIso}\AgdaSpace{}%
\AgdaBound{kn}\<%
\\
%
\>[4]\AgdaSymbol{...}\AgdaSpace{}%
\AgdaSymbol{|}\AgdaSpace{}%
\AgdaInductiveConstructor{zero}\AgdaSpace{}%
\AgdaSymbol{=}\AgdaSpace{}%
\AgdaFunction{≤-refl}\<%
\\
%
\>[4]\AgdaSymbol{...}\AgdaSpace{}%
\AgdaSymbol{|}\AgdaSpace{}%
\AgdaInductiveConstructor{suc}\AgdaSpace{}%
\AgdaBound{n}\AgdaSpace{}%
\AgdaSymbol{=}\AgdaSpace{}%
\AgdaFunction{≤-refl}\<%
\\
%
\\[\AgdaEmptyExtraSkip]%
%
\\[\AgdaEmptyExtraSkip]%
\>[0]\AgdaFunction{max-swapR}\AgdaSpace{}%
\AgdaSymbol{:}\AgdaSpace{}%
\AgdaSymbol{∀}\AgdaSpace{}%
\AgdaSymbol{\{}\AgdaBound{c}\AgdaSymbol{\}}\AgdaSpace{}%
\AgdaSymbol{\{}\AgdaBound{f}\AgdaSpace{}%
\AgdaBound{g}\AgdaSpace{}%
\AgdaSymbol{:}\AgdaSpace{}%
\AgdaBound{El}\AgdaSpace{}%
\AgdaBound{c}\AgdaSpace{}%
\AgdaSymbol{→}\AgdaSpace{}%
\AgdaRecord{SMBTree}\AgdaSymbol{\}}\AgdaSpace{}%
\AgdaSymbol{→}\AgdaSpace{}%
\AgdaFunction{max}\AgdaSpace{}%
\AgdaSymbol{(}\AgdaFunction{Lim}\AgdaSpace{}%
\AgdaBound{c}\AgdaSpace{}%
\AgdaBound{f}\AgdaSymbol{)}\AgdaSpace{}%
\AgdaSymbol{(}\AgdaFunction{Lim}\AgdaSpace{}%
\AgdaBound{c}\AgdaSpace{}%
\AgdaBound{g}\AgdaSymbol{)}\AgdaSpace{}%
\AgdaOperator{\AgdaRecord{≤}}\AgdaSpace{}%
\AgdaFunction{Lim}\AgdaSpace{}%
\AgdaBound{c}\AgdaSpace{}%
\AgdaSymbol{(λ}\AgdaSpace{}%
\AgdaBound{k}\AgdaSpace{}%
\AgdaSymbol{→}\AgdaSpace{}%
\AgdaFunction{max}\AgdaSpace{}%
\AgdaSymbol{(}\AgdaBound{f}\AgdaSpace{}%
\AgdaBound{k}\AgdaSymbol{)}\AgdaSpace{}%
\AgdaSymbol{(}\AgdaBound{g}\AgdaSpace{}%
\AgdaBound{k}\AgdaSymbol{))}\<%
\\
\>[0]\AgdaFunction{max-swapR}\AgdaSpace{}%
\AgdaSymbol{\{}\AgdaBound{c}\AgdaSymbol{\}}\AgdaSpace{}%
\AgdaSymbol{\{}\AgdaBound{f}\AgdaSymbol{\}}\AgdaSpace{}%
\AgdaSymbol{\{}\AgdaBound{g}\AgdaSymbol{\}}\AgdaSpace{}%
\AgdaSymbol{=}\AgdaSpace{}%
\AgdaFunction{max≤max'}\AgdaSpace{}%
\AgdaOperator{\AgdaFunction{≤⨟}}\AgdaSpace{}%
\AgdaFunction{≤-extLim}\AgdaSpace{}%
\AgdaFunction{helper}\AgdaSpace{}%
\AgdaOperator{\AgdaFunction{≤⨟}}\AgdaSpace{}%
\AgdaFunction{limSwap}\AgdaSpace{}%
\AgdaOperator{\AgdaFunction{≤⨟}}\AgdaSpace{}%
\AgdaFunction{≤-extLim}\AgdaSpace{}%
\AgdaSymbol{(λ}\AgdaSpace{}%
\AgdaBound{k}\AgdaSpace{}%
\AgdaSymbol{→}\AgdaSpace{}%
\AgdaFunction{max'≤max}\AgdaSymbol{)}\<%
\\
\>[0][@{}l@{\AgdaIndent{0}}]%
\>[2]\AgdaKeyword{where}\<%
\\
\>[2][@{}l@{\AgdaIndent{0}}]%
\>[4]\AgdaFunction{helper}\AgdaSpace{}%
\AgdaSymbol{:}\AgdaSpace{}%
\AgdaSymbol{(}\AgdaBound{k}\AgdaSpace{}%
\AgdaSymbol{:}\AgdaSpace{}%
\AgdaBound{El}\AgdaSpace{}%
\AgdaBound{Cℕ}\AgdaSymbol{)}\AgdaSpace{}%
\AgdaSymbol{→}\<%
\\
\>[4][@{}l@{\AgdaIndent{0}}]%
\>[6]\AgdaFunction{if0}\AgdaSpace{}%
\AgdaSymbol{(}\AgdaField{Iso.fun}\AgdaSpace{}%
\AgdaBound{CℕIso}\AgdaSpace{}%
\AgdaBound{k}\AgdaSymbol{)}\AgdaSpace{}%
\AgdaSymbol{(}\AgdaFunction{Lim}\AgdaSpace{}%
\AgdaBound{c}\AgdaSpace{}%
\AgdaBound{f}\AgdaSymbol{)}\AgdaSpace{}%
\AgdaSymbol{(}\AgdaFunction{Lim}\AgdaSpace{}%
\AgdaBound{c}\AgdaSpace{}%
\AgdaBound{g}\AgdaSymbol{)}\AgdaSpace{}%
\AgdaOperator{\AgdaRecord{≤}}\<%
\\
%
\>[6]\AgdaFunction{Lim}\AgdaSpace{}%
\AgdaBound{c}\AgdaSpace{}%
\AgdaSymbol{(λ}\AgdaSpace{}%
\AgdaBound{z}\AgdaSpace{}%
\AgdaSymbol{→}\AgdaSpace{}%
\AgdaFunction{if0}\AgdaSpace{}%
\AgdaSymbol{(}\AgdaField{Iso.fun}\AgdaSpace{}%
\AgdaBound{CℕIso}\AgdaSpace{}%
\AgdaBound{k}\AgdaSymbol{)}\AgdaSpace{}%
\AgdaSymbol{(}\AgdaBound{f}\AgdaSpace{}%
\AgdaBound{z}\AgdaSymbol{)}\AgdaSpace{}%
\AgdaSymbol{(}\AgdaBound{g}\AgdaSpace{}%
\AgdaBound{z}\AgdaSymbol{))}\<%
\\
%
\>[4]\AgdaFunction{helper}\AgdaSpace{}%
\AgdaBound{kn}\AgdaSpace{}%
\AgdaKeyword{with}\AgdaSpace{}%
\AgdaField{Iso.fun}\AgdaSpace{}%
\AgdaBound{CℕIso}\AgdaSpace{}%
\AgdaBound{kn}\<%
\\
%
\>[4]\AgdaSymbol{...}\AgdaSpace{}%
\AgdaSymbol{|}\AgdaSpace{}%
\AgdaInductiveConstructor{zero}\AgdaSpace{}%
\AgdaSymbol{=}\AgdaSpace{}%
\AgdaFunction{≤-refl}\<%
\\
%
\>[4]\AgdaSymbol{...}\AgdaSpace{}%
\AgdaSymbol{|}\AgdaSpace{}%
\AgdaInductiveConstructor{suc}\AgdaSpace{}%
\AgdaBound{n}\AgdaSpace{}%
\AgdaSymbol{=}\AgdaSpace{}%
\AgdaFunction{≤-refl}\<%
\\
\>[0]\<%
\end{code}


\subsubsection{Well Founded Ordering on SMB-trees}
Our motivation for defining SMB-trees was defining well founded recursion,
so the final piece of our definition is a proof that the strict ordering of
SMB-trees is well founded.
Intuitively this should hold: if there are no infinite descending chains
of Brouwer trees, and there are fewer SMB-trees than Brouwer trees, then
there can be no infinite descending chains of SMB-trees.
The key lemma is that an SMB-tree is accessible if its underlying Brouwer tree is.
\begin{code}%
\>[0]\AgdaKeyword{open}\AgdaSpace{}%
\AgdaKeyword{import}\AgdaSpace{}%
\AgdaModule{Induction.WellFounded}\<%
\\
\>[0]\AgdaKeyword{opaque}\<%
\\
\>[0][@{}l@{\AgdaIndent{0}}]%
\>[2]\AgdaKeyword{unfolding}\AgdaSpace{}%
\AgdaFunction{↑}\<%
\\
%
\>[2]\AgdaFunction{sizeWF}\AgdaSpace{}%
\AgdaSymbol{:}\AgdaSpace{}%
\AgdaFunction{WellFounded}\AgdaSpace{}%
\AgdaOperator{\AgdaFunction{\AgdaUnderscore{}<\AgdaUnderscore{}}}\<%
\\
%
\>[2]\AgdaFunction{sizeWF}\AgdaSpace{}%
\AgdaBound{t}\AgdaSpace{}%
\AgdaSymbol{=}\AgdaSpace{}%
\AgdaFunction{sizeAcc}\AgdaSpace{}%
\AgdaSymbol{(}\AgdaPostulate{Brouwer.ordWF}\AgdaSpace{}%
\AgdaSymbol{(}\AgdaField{rawTree}\AgdaSpace{}%
\AgdaBound{t}\AgdaSymbol{))}\<%
\\
\>[2][@{}l@{\AgdaIndent{0}}]%
\>[4]\AgdaKeyword{where}\<%
\\
\>[4][@{}l@{\AgdaIndent{0}}]%
\>[6]\AgdaFunction{sizeAcc}\AgdaSpace{}%
\AgdaSymbol{:}\AgdaSpace{}%
\AgdaSymbol{∀}\AgdaSpace{}%
\AgdaSymbol{\{}\AgdaBound{t}\AgdaSymbol{\}}\<%
\\
\>[6][@{}l@{\AgdaIndent{0}}]%
\>[8]\AgdaSymbol{→}\AgdaSpace{}%
\AgdaDatatype{Acc}\AgdaSpace{}%
\AgdaOperator{\AgdaPostulate{Brouwer.\AgdaUnderscore{}<\AgdaUnderscore{}}}\AgdaSpace{}%
\AgdaSymbol{(}\AgdaField{rawTree}\AgdaSpace{}%
\AgdaBound{t}\AgdaSymbol{)}\<%
\\
%
\>[8]\AgdaSymbol{→}\AgdaSpace{}%
\AgdaDatatype{Acc}\AgdaSpace{}%
\AgdaOperator{\AgdaFunction{\AgdaUnderscore{}<\AgdaUnderscore{}}}\AgdaSpace{}%
\AgdaBound{t}\<%
\\
%
\>[6]\AgdaFunction{sizeAcc}\AgdaSpace{}%
\AgdaSymbol{\{}\AgdaBound{t}\AgdaSymbol{\}}\AgdaSpace{}%
\AgdaSymbol{(}\AgdaInductiveConstructor{acc}\AgdaSpace{}%
\AgdaBound{x}\AgdaSymbol{)}\<%
\\
\>[6][@{}l@{\AgdaIndent{0}}]%
\>[8]\AgdaSymbol{=}\AgdaSpace{}%
\AgdaInductiveConstructor{acc}\AgdaSpace{}%
\AgdaSymbol{((λ}\AgdaSpace{}%
\AgdaBound{y}\AgdaSpace{}%
\AgdaBound{lt}\AgdaSpace{}%
\AgdaSymbol{→}\AgdaSpace{}%
\AgdaFunction{sizeAcc}\AgdaSpace{}%
\AgdaSymbol{(}\AgdaBound{x}\AgdaSpace{}%
\AgdaSymbol{(}\AgdaField{rawTree}\AgdaSpace{}%
\AgdaBound{y}\AgdaSymbol{)}\AgdaSpace{}%
\AgdaSymbol{(}\AgdaField{get≤}\AgdaSpace{}%
\AgdaBound{lt}\AgdaSymbol{))))}\<%
\end{code}

Thus, we have an ordinal type with limits, a strictly monotone join,
and well founded recursion.


%The usefulness of Brouwer trees is in defining well-founded recursion, but first we need on ordering on trees.


% A strict order can be defined in terms of the successor function. This strict relation is a well quasi-order: it has no infinite descending chains, and hence
% can be used as a decreasing metric
% for recursive functions.

%     TODO compare with cubical,
%     TODO look up original trees

% \subsubsection{Brouwer Trees}
% \label{model:subsec:brouwer}
% Unfortunately, it was not immediately apparent that any of the
% ``off-the-shelf'' formulations of constructive ordinals satisfied our critera,
% so we built our own formulation. We use a refined version of Brouwer trees:
% There is a zero ordinal, a successor operator, and a limit ordinal that is the least upper bound
% of the image for a function from a code's type to ordinals.
% We borrow the trick of taking the limits over types (or in our case, codes) from \citet{ionchyMasters},
% since this lets us easily model the sizes of dependent functions and pairs.
% The ordering on these trees is defined following \citet{KrausFX21}:
% \begin{flalign*}
%   data\ \_\le_o\_ : Ord -> Ord -> \sType{}\ where\nl
%   \qquad {\le_o}Z : (o : Ord) -> OZ \le_o o  \nl
%   \qquad {\le_o}sucMono : (o_1 : Ord) -> (o_2 : Ord) -> o_1 \le_o o_2 -> O{\uparrow}\  o_1 \le_o O{\uparrow}\  o_2  \nl
%   \qquad {\le_o}cocone : (c : \bC\ \ell) -> (o : Ord) -> (f : El_{Approx}\ c -> Ord)
%     -> (k : El_{Approx}\ c)
%     \nl\qquad\qquad -> o \le_o f\ k  -> o \le_o OLim\ c\ f\nl
%     \qquad {\le_o}limiting : (o : Ord) -> (c : \bC\ \ell) -> (f : El_{Approx}\ c -> Ord)
%     \nl\qquad\qquad -> ((k : El_{Approx}\ c) -> f\ k \le_o o) -> OLim\ c\ f \le_o o\\\nl
%     %
%     o_1 <_o o_2 = O{\uparrow}\ o_1 \le_o o_2
%   \end{flalign*}
%   That is, zero is the smallest ordinal, the successor is monotone,
%   and the limit is actually the least upper bound of the function's image.
% Unlike \citet{KrausFX21}, we do not include transitivity as a rule, but we can prove
% it as a theorem.
% The maximum function on ordinals is defined as follows:
% \begin{flalign*}
%   max_o : Ord -> Ord -> Ord\nl
%   max_o\ OZ\ o = o \nl
%   max_o\ o\ OZ = o \nl
%   max_o\ (O{\uparrow}\ o_1)\ (O{\uparrow}\ o_2) = O{\uparrow}\ (max_o\ o_1\ o_2)\nl
%   max_o\ (OLim\ c\ f)\ o = OLim\ c\ (\lambda k \ldotp max_o\ (f\ k)\ o)\nl
%   max_o\ o\ (OLim\ c\ f) = OLim\ c\ (\lambda k \ldotp max_o\ o\ (f\ k))
% \end{flalign*}
% Long but straightforward proofs show that $max_{o}$ is monotone
% and computes and upper bound of its inputs.
% It reduces when given $\s{O{\uparrow}}$ for both inputs, so it is strictly monotone.
% However, we cannot prove that it is a least upper-bound.
% The problem is that limits are not well-behaved with respect to the maximum.
% We could instead construct the maximum using $\s{OLim}$, but this version
% would not be strictly monotone.

% \subsubsection{A Least Upper Bound}

% We solve the problems with $\s{max_{o}}$ using a type of sizes, which include only the subset of
% ordinals that are idempotent with respect to the maximum. We can then
% define a type of sizes with the same interface as ordinals.
% \begin{flalign*}
%   Size : \sType{} \nl
%   Size = (o : Ord) \times (max_o\ o\ o \le_o o)\\\nl
% %
%   \_\bigvee\_ : Size -> Size -> Size\nl
%   s_1 \bigvee s_2 = (max_o\ (fst\ s_1)\ (fst\ s_2), \ldots)\\\nl
%   %
%   SZ : Size\nl
%   SZ = (OZ , {\le_o}Z)\\\nl
%   S{\uparrow} : Size -> Size\nl
%   S{\uparrow}\ s =  (O{\uparrow}\ (fst\ s), {\le_s}sucMono\ (snd\ s))
% \end{flalign*}
% Critically, the sizes are closed under the maximum operation: if $\s{max_{o}\ o_{1}\ o_{1} \le_{o}\ o_{1}}$
% and $\s{max_{o}\ o_{2}\ o_{2} \le_{o}\ o_{2}}$, then
% $\s{max_{o}\ (max_{o}\ o_{1}\ o_{2})\ (max_{o}\ o_{1}\ o_{2}) \le (max_{o}\ o_{1}\ o_{2})}$.
% % We omit the proof term, because it is long but boring.
% Zero and a successor operation for sizes are easily implemented.
% The difficulty is constructing a limit operator for sizes, since
% the self-idempotent ordinals are not closed under $\s{OLim}$.
% Our trick is to take the limit of maxing an ordinal with itself.
% We assume we have a code $\s{C\bN}$ whose elements have an injection $\s{Cto\bN}$ into $\s{\bN}$.
% The natural numbers can be defined as an inductive type, but in our Agda development we add it as an
% extra code constructor.
% Having numbers lets us take the maximum of an ordinal with itself infinitely many times, resulting in an ordinal
% that is as large as the original but idempotent with respect to $\s{max_{o}}$.
% \begin{flalign*}
%   nmax : Ord -> \bN -> Ord \nl
%   nmax\ o\ Z\ = OZ\nl
%   nmax\ o\ (S\ n) = omax\ (nmax\ o\ n)\ o\\ \nl
%   %
%   max\infty : Ord -> Ord\nl
%   max\infty\ o = OLim\ C\bN\ (\lambda k \ldotp nmax\ o\ (Cto\bN\ k)) \\ \nl
%   %
%   max\infty Idem : \{ o : Ord \} -> max_o\ (max\infty\ o)\ (max\infty\ o) \le_o (max\infty\ o)\\\nl
%   %
%   SLim : (c : \bC\ \ell) -> (El_{Approx}\ c -> Size) -> Size\nl
%   SLim\ c\ f = (max\infty\ (OLim\ c\ (\lambda k \ldotp fst\ (f\ k))) ,\ max\infty Idem )
% \end{flalign*}

% Sizes satisfy all the same inequalities as raw ordinals,
% listed in \cref{model:fig:size-order}.
% The monotonicity of $\s\bigvee$ follows from the monotonicity of $\s{max_{o}}$,
% and the idempotence  of $\s\bigvee$ follows by the definition of $\s{Size}$.
% Monotonicity, idempotence, and transitivity of $\s{\le_{s}}$ together imply
% that $\s\bigvee$ is a least upper bound,
% and strict monotonicity follows from the strict monotonicity of $\s{max_{o}}$.
% \begin{figure}
%   \begin{flalign*}
%     \_\le_s\_ : Size -> Size -> Size\nl
%     s_1 \le_s s_2 = (fst\ s_1) \le_o (fst\ s_2)\\\nl
%     %
%     \_<_s\_ : Size -> Size -> Size\nl
%     s_1 <_s s_2 = (S{\uparrow}\ s_1) \le_s s_2\\\nl
%     %
%     {\le_s}trans : (s_1 : Size) -> (s_2 : Size) -> (s_3 : Size) ->\nl
%     \qquad (s_1 \le_s s_2) -> (s_2 \le_s s_3) -> (s_1 \le_s s_3)\nl
%     {\le_s}Z : (s : Size) -> SZ \le_s s  \nl
%     {\le_s}sucMono : (s_1 : Size) -> (s_2 : Size) -> s_1 \le_s s_2 -> S{\uparrow}\  s_1 \le_s S{\uparrow}\  s_2  \nl
%     {\le_s}cocone : (c : \bC\ \ell) -> (s : Size) -> (f : El_{Approx}\ c -> Size)
%     -> (k : El_{Approx}\ c)
%     \nl\qquad -> s \le_s f\ k  -> s \le_s SLim\ c\ f\nl
%     {\le_s}limiting : (s : Size) -> (c : \bC\ \ell) -> (f : El_{Approx}\ c -> Size)
%     \nl\qquad -> ((k : El_{Approx}\ c) -> f\ k \le_s s) -> SLim\ c\ f \le_s s\\\nl
%     %
%     \bigvee\le : (s_1 : Size) -> (s_2 : Size) -> (s_1 \le_s s_1 \bigvee s_2) \times (s_2 \le_2 s_1 \bigvee s_2)\nl
%     \bigvee mono : (s_1 : size) -> (s_2 : Size) -> (s'_1 : Size) -> (s'_2 : Size) \nl
%     \qquad -> (s_1 \le_s s'_1) -> (s_2 \le_s s'_2) -> (s_1 \bigvee s_2) \le_s (s'_1 \bigvee s'_2)\nl
%     \bigvee idem : (s : Size) -> (s \bigvee s) \le_s s\nl
%     \bigvee lub : (s_1 : size) -> (s_2 : size) -> (s : Size) \nl
%     \qquad -> (s_1 \le_s s) -> (s_2 \le_s s) -> (s_1 \bigvee s_2 \le_s s)
%   \end{flalign*}
%   \caption{Ordering on Sizes}
%   \label{model:fig:size-order}
% \end{figure}
