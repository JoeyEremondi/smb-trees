% !TEX root =  main.tex
% !TEX root =  main.tex
\section{Brouwer Trees: An Introduction}
\begin{code}[hide]%
%
\>[2]\AgdaKeyword{open}\AgdaSpace{}%
\AgdaKeyword{import}\AgdaSpace{}%
\AgdaModule{Data.Nat}\AgdaSpace{}%
\AgdaKeyword{hiding}\AgdaSpace{}%
\AgdaSymbol{(}\AgdaOperator{\AgdaDatatype{\AgdaUnderscore{}≤\AgdaUnderscore{}}}\AgdaSymbol{)}\<%
\\
%
\>[2]\AgdaKeyword{open}\AgdaSpace{}%
\AgdaKeyword{import}\AgdaSpace{}%
\AgdaModule{Relation.Binary.PropositionalEquality}\<%
\\
%
\>[2]\AgdaKeyword{open}\AgdaSpace{}%
\AgdaKeyword{import}\AgdaSpace{}%
\AgdaModule{Data.Product}\<%
\\
%
\>[2]\AgdaKeyword{open}\AgdaSpace{}%
\AgdaKeyword{import}\AgdaSpace{}%
\AgdaModule{Relation.Nullary}\<%
\\
%
\>[2]\AgdaKeyword{open}\AgdaSpace{}%
\AgdaKeyword{import}\AgdaSpace{}%
\AgdaModule{Iso}\<%
\\
%
\>[2]\AgdaKeyword{module}\AgdaSpace{}%
\AgdaModule{Tree}\AgdaSpace{}%
\AgdaKeyword{where}\<%
\\
\>[0]\<%
\end{code}

Brouwer trees  are a simple but elegant tool for proving termination of higher-order procedures.
Traditionally, they are defined as follows:
\begin{code}%
\>[0][@{}l@{\AgdaIndent{1}}]%
\>[2]\AgdaKeyword{data}\AgdaSpace{}%
\AgdaDatatype{SmallTree}\AgdaSpace{}%
\AgdaSymbol{:}\AgdaSpace{}%
\AgdaPrimitive{Set}\AgdaSpace{}%
\AgdaKeyword{where}\<%
\\
\>[2][@{}l@{\AgdaIndent{0}}]%
\>[4]\AgdaInductiveConstructor{Z'}\AgdaSpace{}%
\AgdaSymbol{:}\AgdaSpace{}%
\AgdaDatatype{SmallTree}\<%
\\
%
\>[4]\AgdaInductiveConstructor{↑'}\AgdaSpace{}%
\AgdaSymbol{:}\AgdaSpace{}%
\AgdaDatatype{SmallTree}\AgdaSpace{}%
\AgdaSymbol{→}\AgdaSpace{}%
\AgdaDatatype{SmallTree}\<%
\\
%
\>[4]\AgdaInductiveConstructor{Lim'}\AgdaSpace{}%
\AgdaSymbol{:}\AgdaSpace{}%
\AgdaSymbol{(}\AgdaDatatype{ℕ}\AgdaSpace{}%
\AgdaSymbol{→}\AgdaSpace{}%
\AgdaDatatype{SmallTree}\AgdaSymbol{)}\AgdaSpace{}%
\AgdaSymbol{→}\AgdaSpace{}%
\AgdaDatatype{SmallTree}\<%
\end{code}
Under this definition, a Brouwer tree is either zero, the successor of another Brouwer tree, or the limit of a countable sequence of Brouwer trees. However, these are quite weak, in that they can only take the limit of countable sequences.
To represent the limits of uncountable sequences, we can paramterize our definition over some Universe \ala Tarski:

\begin{code}%
%
\>[2]\AgdaKeyword{module}\AgdaSpace{}%
\AgdaModule{\AgdaUnderscore{}}\AgdaSpace{}%
\AgdaSymbol{\{}\AgdaBound{ℓ}\AgdaSymbol{\}}\<%
\\
\>[2][@{}l@{\AgdaIndent{0}}]%
\>[4]\AgdaSymbol{(}\AgdaBound{ℂ}\AgdaSpace{}%
\AgdaSymbol{:}\AgdaSpace{}%
\AgdaPrimitive{Set}\AgdaSpace{}%
\AgdaBound{ℓ}\AgdaSymbol{)}\<%
\\
%
\>[4]\AgdaSymbol{(}\AgdaBound{El}\AgdaSpace{}%
\AgdaSymbol{:}\AgdaSpace{}%
\AgdaBound{ℂ}\AgdaSpace{}%
\AgdaSymbol{→}\AgdaSpace{}%
\AgdaPrimitive{Set}\AgdaSpace{}%
\AgdaBound{ℓ}\AgdaSymbol{)}\<%
\\
%
\>[4]\AgdaSymbol{(}\AgdaBound{Cℕ}\AgdaSpace{}%
\AgdaSymbol{:}\AgdaSpace{}%
\AgdaBound{ℂ}\AgdaSymbol{)}\AgdaSpace{}%
\AgdaSymbol{(}\AgdaBound{CℕIso}\AgdaSpace{}%
\AgdaSymbol{:}\AgdaSpace{}%
\AgdaRecord{Iso}\AgdaSpace{}%
\AgdaSymbol{(}\AgdaBound{El}\AgdaSpace{}%
\AgdaBound{Cℕ}\AgdaSymbol{)}\AgdaSpace{}%
\AgdaDatatype{ℕ}\AgdaSpace{}%
\AgdaSymbol{)}\AgdaSpace{}%
\AgdaKeyword{where}\<%
\end{code}

We then generalize limits to any function whose domain is the interpretation of some code.
\begin{code}%
%
\>[4]\AgdaKeyword{data}\AgdaSpace{}%
\AgdaDatatype{Tree}\AgdaSpace{}%
\AgdaSymbol{:}\AgdaSpace{}%
\AgdaPrimitive{Set}\AgdaSpace{}%
\AgdaBound{ℓ}\AgdaSpace{}%
\AgdaKeyword{where}\<%
\\
\>[4][@{}l@{\AgdaIndent{0}}]%
\>[6]\AgdaInductiveConstructor{Z}\AgdaSpace{}%
\AgdaSymbol{:}\AgdaSpace{}%
\AgdaDatatype{Tree}\<%
\\
%
\>[6]\AgdaInductiveConstructor{↑}\AgdaSpace{}%
\AgdaSymbol{:}\AgdaSpace{}%
\AgdaDatatype{Tree}\AgdaSpace{}%
\AgdaSymbol{→}\AgdaSpace{}%
\AgdaDatatype{Tree}\<%
\\
%
\>[6]\AgdaInductiveConstructor{Lim}\AgdaSpace{}%
\AgdaSymbol{:}\AgdaSpace{}%
\AgdaSymbol{∀}%
\>[15]\AgdaSymbol{(}\AgdaBound{c}\AgdaSpace{}%
\AgdaSymbol{:}\AgdaSpace{}%
\AgdaBound{ℂ}\AgdaSpace{}%
\AgdaSymbol{)}\AgdaSpace{}%
\AgdaSymbol{→}\AgdaSpace{}%
\AgdaSymbol{(}\AgdaBound{f}\AgdaSpace{}%
\AgdaSymbol{:}\AgdaSpace{}%
\AgdaBound{El}\AgdaSpace{}%
\AgdaBound{c}\AgdaSpace{}%
\AgdaSymbol{→}\AgdaSpace{}%
\AgdaDatatype{Tree}\AgdaSymbol{)}\AgdaSpace{}%
\AgdaSymbol{→}\AgdaSpace{}%
\AgdaDatatype{Tree}\<%
\end{code}

\begin{code}[hiding]%
\>[0]\<%
\end{code}


Our module is paramterized over a universe level, a type $\bC$ of \textit{codes}, and an ``elements-of'' interpretation
function $\mathit{El}$, which computes the type represented by each code.
We require that there be a code whose interpretation is isomorphic to the natural numbers,
as this is essential to our construction in \cref{sec:TODO}.
Increasingly larger trees can be obtained by setting $\bC := \AgdaPrimitive{Set} \ \ell$ and
$\mathit{El} := \mathit{id}$ for increasing $\ell$.
However, by defining an inductive-recursive universe,
one can still capture limits over some non-countable types, since
 $\AgdaDatatype{Tree}$ is in $\AgdaPrimitive{Set}$ whenever $\bC$ is.

The small limit constructor can be recovered from the natural-number code
\begin{code}%
\>[0]\<%
\\
\>[0][@{}l@{\AgdaIndent{1}}]%
\>[4]\AgdaFunction{ℕLim}\AgdaSpace{}%
\AgdaSymbol{:}\AgdaSpace{}%
\AgdaSymbol{(}\AgdaDatatype{ℕ}\AgdaSpace{}%
\AgdaSymbol{→}\AgdaSpace{}%
\AgdaDatatype{Tree}\AgdaSymbol{)}\AgdaSpace{}%
\AgdaSymbol{→}\AgdaSpace{}%
\AgdaDatatype{Tree}\<%
\\
%
\>[4]\AgdaFunction{ℕLim}\AgdaSpace{}%
\AgdaBound{f}\AgdaSpace{}%
\AgdaSymbol{=}\AgdaSpace{}%
\AgdaInductiveConstructor{Lim}\AgdaSpace{}%
\AgdaBound{Cℕ}%
\>[21]\AgdaSymbol{(λ}\AgdaSpace{}%
\AgdaBound{cn}\AgdaSpace{}%
\AgdaSymbol{→}\AgdaSpace{}%
\AgdaBound{f}\AgdaSpace{}%
\AgdaSymbol{(}\AgdaField{Iso.fun}\AgdaSpace{}%
\AgdaBound{CℕIso}\AgdaSpace{}%
\AgdaBound{cn}\AgdaSymbol{))}\<%
\end{code}

Brouwer trees are a the quintessential example of a higher-order inductive type.%
\footnote{Not to be confused with Higher Inductive Types (HITs) from Homotopy Type Theory~\citep{hottbook}}:
Each tree is built using smaller trees or functions producing smaller trees, which is essentially
a way of storing a possibly infinite number of smaller trees.

\subsection{Ordering Trees}

Our ultimate goal is to have a well-founded ordering%
\footnote{Technically, this is a well-founded quasi-ordering because there are pairs of
  trees which are related by both $\leq$ and $\qeq$, but which are not propositionally equal.},
so we define a relation to order Brouwer trees.

\begin{code}%
%
\>[4]\AgdaKeyword{data}\AgdaSpace{}%
\AgdaOperator{\AgdaDatatype{\AgdaUnderscore{}≤\AgdaUnderscore{}}}\AgdaSpace{}%
\AgdaSymbol{:}\AgdaSpace{}%
\AgdaDatatype{Tree}\AgdaSpace{}%
\AgdaSymbol{→}\AgdaSpace{}%
\AgdaDatatype{Tree}\AgdaSpace{}%
\AgdaSymbol{→}\AgdaSpace{}%
\AgdaPrimitive{Set}\AgdaSpace{}%
\AgdaBound{ℓ}\AgdaSpace{}%
\AgdaKeyword{where}\<%
\\
\>[4][@{}l@{\AgdaIndent{0}}]%
\>[6]\AgdaInductiveConstructor{≤-Z}\AgdaSpace{}%
\AgdaSymbol{:}\AgdaSpace{}%
\AgdaSymbol{∀}\AgdaSpace{}%
\AgdaSymbol{\{}\AgdaBound{t}\AgdaSymbol{\}}\AgdaSpace{}%
\AgdaSymbol{→}\AgdaSpace{}%
\AgdaInductiveConstructor{Z}\AgdaSpace{}%
\AgdaOperator{\AgdaDatatype{≤}}\AgdaSpace{}%
\AgdaBound{t}\<%
\\
%
\>[6]\AgdaInductiveConstructor{≤-sucMono}\AgdaSpace{}%
\AgdaSymbol{:}\AgdaSpace{}%
\AgdaSymbol{∀}\AgdaSpace{}%
\AgdaSymbol{\{}\AgdaBound{t1}\AgdaSpace{}%
\AgdaBound{t2}\AgdaSymbol{\}}\<%
\\
\>[6][@{}l@{\AgdaIndent{0}}]%
\>[8]\AgdaSymbol{→}\AgdaSpace{}%
\AgdaBound{t1}\AgdaSpace{}%
\AgdaOperator{\AgdaDatatype{≤}}\AgdaSpace{}%
\AgdaBound{t2}\<%
\\
%
\>[8]\AgdaSymbol{→}\AgdaSpace{}%
\AgdaInductiveConstructor{↑}\AgdaSpace{}%
\AgdaBound{t1}\AgdaSpace{}%
\AgdaOperator{\AgdaDatatype{≤}}\AgdaSpace{}%
\AgdaInductiveConstructor{↑}\AgdaSpace{}%
\AgdaBound{t2}\<%
\\
%
\>[6]\AgdaInductiveConstructor{≤-cocone}\AgdaSpace{}%
\AgdaSymbol{:}\AgdaSpace{}%
\AgdaSymbol{∀}%
\>[20]\AgdaSymbol{\{}\AgdaBound{t}\AgdaSymbol{\}}\AgdaSpace{}%
\AgdaSymbol{\{}\AgdaBound{c}\AgdaSpace{}%
\AgdaSymbol{:}\AgdaSpace{}%
\AgdaBound{ℂ}\AgdaSymbol{\}}\AgdaSpace{}%
\AgdaSymbol{(}\AgdaBound{f}\AgdaSpace{}%
\AgdaSymbol{:}\AgdaSpace{}%
\AgdaBound{El}\AgdaSpace{}%
\AgdaBound{c}%
\>[43]\AgdaSymbol{→}\AgdaSpace{}%
\AgdaDatatype{Tree}\AgdaSymbol{)}\AgdaSpace{}%
\AgdaSymbol{(}\AgdaBound{k}\AgdaSpace{}%
\AgdaSymbol{:}\AgdaSpace{}%
\AgdaBound{El}\AgdaSpace{}%
\AgdaBound{c}\AgdaSymbol{)}\<%
\\
\>[6][@{}l@{\AgdaIndent{0}}]%
\>[8]\AgdaSymbol{→}\AgdaSpace{}%
\AgdaBound{t}\AgdaSpace{}%
\AgdaOperator{\AgdaDatatype{≤}}\AgdaSpace{}%
\AgdaBound{f}\AgdaSpace{}%
\AgdaBound{k}\<%
\\
%
\>[8]\AgdaSymbol{→}\AgdaSpace{}%
\AgdaBound{t}\AgdaSpace{}%
\AgdaOperator{\AgdaDatatype{≤}}\AgdaSpace{}%
\AgdaInductiveConstructor{Lim}\AgdaSpace{}%
\AgdaBound{c}\AgdaSpace{}%
\AgdaBound{f}\<%
\\
%
\>[6]\AgdaInductiveConstructor{≤-limiting}\AgdaSpace{}%
\AgdaSymbol{:}\AgdaSpace{}%
\AgdaSymbol{∀}%
\>[23]\AgdaSymbol{\{}\AgdaBound{t}\AgdaSymbol{\}}\AgdaSpace{}%
\AgdaSymbol{\{}\AgdaBound{c}\AgdaSpace{}%
\AgdaSymbol{:}\AgdaSpace{}%
\AgdaBound{ℂ}\AgdaSymbol{\}}\<%
\\
\>[6][@{}l@{\AgdaIndent{0}}]%
\>[8]\AgdaSymbol{→}\AgdaSpace{}%
\AgdaSymbol{(}\AgdaBound{f}\AgdaSpace{}%
\AgdaSymbol{:}\AgdaSpace{}%
\AgdaBound{El}\AgdaSpace{}%
\AgdaBound{c}\AgdaSpace{}%
\AgdaSymbol{→}\AgdaSpace{}%
\AgdaDatatype{Tree}\AgdaSymbol{)}\<%
\\
%
\>[8]\AgdaSymbol{→}\AgdaSpace{}%
\AgdaSymbol{(∀}\AgdaSpace{}%
\AgdaBound{k}\AgdaSpace{}%
\AgdaSymbol{→}\AgdaSpace{}%
\AgdaBound{f}\AgdaSpace{}%
\AgdaBound{k}\AgdaSpace{}%
\AgdaOperator{\AgdaDatatype{≤}}\AgdaSpace{}%
\AgdaBound{t}\AgdaSymbol{)}\<%
\\
%
\>[8]\AgdaSymbol{→}\AgdaSpace{}%
\AgdaInductiveConstructor{Lim}\AgdaSpace{}%
\AgdaBound{c}\AgdaSpace{}%
\AgdaBound{f}\AgdaSpace{}%
\AgdaOperator{\AgdaDatatype{≤}}\AgdaSpace{}%
\AgdaBound{t}\<%
\\
\>[0]\<%
\end{code}
The ordering is based on the one presented by \citet{KRAUS2023113843}, but we modify it
so that transitivity can be proven constructively, rather than adding it as a constructor
for the relation. This allows us to prove well-foundedness of the relation without needing
quotient types or other advanced features.

\begin{code}%
\>[0]\<%
\\
\>[0][@{}l@{\AgdaIndent{1}}]%
\>[4]\AgdaFunction{≤-refl}\AgdaSpace{}%
\AgdaSymbol{:}\AgdaSpace{}%
\AgdaSymbol{∀}\AgdaSpace{}%
\AgdaBound{t}\AgdaSpace{}%
\AgdaSymbol{→}\AgdaSpace{}%
\AgdaBound{t}\AgdaSpace{}%
\AgdaOperator{\AgdaDatatype{≤}}\AgdaSpace{}%
\AgdaBound{t}\<%
\\
%
\>[4]\AgdaFunction{≤-refl}\AgdaSpace{}%
\AgdaInductiveConstructor{Z}\AgdaSpace{}%
\AgdaSymbol{=}\AgdaSpace{}%
\AgdaInductiveConstructor{≤-Z}\<%
\\
%
\>[4]\AgdaFunction{≤-refl}\AgdaSpace{}%
\AgdaSymbol{(}\AgdaInductiveConstructor{↑}\AgdaSpace{}%
\AgdaBound{t}\AgdaSymbol{)}\AgdaSpace{}%
\AgdaSymbol{=}\AgdaSpace{}%
\AgdaInductiveConstructor{≤-sucMono}\AgdaSpace{}%
\AgdaSymbol{(}\AgdaFunction{≤-refl}\AgdaSpace{}%
\AgdaBound{t}\AgdaSymbol{)}\<%
\\
%
\>[4]\AgdaFunction{≤-refl}\AgdaSpace{}%
\AgdaSymbol{(}\AgdaInductiveConstructor{Lim}\AgdaSpace{}%
\AgdaBound{c}\AgdaSpace{}%
\AgdaBound{f}\AgdaSymbol{)}\AgdaSpace{}%
\AgdaSymbol{=}\AgdaSpace{}%
\AgdaInductiveConstructor{≤-limiting}\AgdaSpace{}%
\AgdaBound{f}\AgdaSpace{}%
\AgdaSymbol{(λ}\AgdaSpace{}%
\AgdaBound{k}\AgdaSpace{}%
\AgdaSymbol{→}\AgdaSpace{}%
\AgdaInductiveConstructor{≤-cocone}\AgdaSpace{}%
\AgdaBound{f}\AgdaSpace{}%
\AgdaBound{k}\AgdaSpace{}%
\AgdaSymbol{(}\AgdaFunction{≤-refl}\AgdaSpace{}%
\AgdaSymbol{(}\AgdaBound{f}\AgdaSpace{}%
\AgdaBound{k}\AgdaSymbol{)))}\<%
\\
\>[0]\<%
\end{code}

\begin{code}%
\>[0]\<%
\\
%
\\[\AgdaEmptyExtraSkip]%
\>[0][@{}l@{\AgdaIndent{1}}]%
\>[4]\AgdaFunction{≤-reflEq}\AgdaSpace{}%
\AgdaSymbol{:}\AgdaSpace{}%
\AgdaSymbol{∀}\AgdaSpace{}%
\AgdaSymbol{\{}\AgdaBound{t1}\AgdaSpace{}%
\AgdaBound{t2}\AgdaSymbol{\}}\AgdaSpace{}%
\AgdaSymbol{→}\AgdaSpace{}%
\AgdaBound{t1}\AgdaSpace{}%
\AgdaOperator{\AgdaDatatype{≡}}\AgdaSpace{}%
\AgdaBound{t2}\AgdaSpace{}%
\AgdaSymbol{→}\AgdaSpace{}%
\AgdaBound{t1}\AgdaSpace{}%
\AgdaOperator{\AgdaDatatype{≤}}\AgdaSpace{}%
\AgdaBound{t2}\<%
\\
%
\>[4]\AgdaFunction{≤-reflEq}\AgdaSpace{}%
\AgdaInductiveConstructor{refl}\AgdaSpace{}%
\AgdaSymbol{=}\AgdaSpace{}%
\AgdaFunction{≤-refl}\AgdaSpace{}%
\AgdaSymbol{\AgdaUnderscore{}}\<%
\\
\>[0]\<%
\end{code}

\begin{code}%
\>[0]\<%
\\
%
\\[\AgdaEmptyExtraSkip]%
\>[0][@{}l@{\AgdaIndent{1}}]%
\>[4]\AgdaFunction{≤-trans}\AgdaSpace{}%
\AgdaSymbol{:}\AgdaSpace{}%
\AgdaSymbol{∀}\AgdaSpace{}%
\AgdaSymbol{\{}\AgdaBound{t1}\AgdaSpace{}%
\AgdaBound{t2}\AgdaSpace{}%
\AgdaBound{t3}\AgdaSymbol{\}}\AgdaSpace{}%
\AgdaSymbol{→}\AgdaSpace{}%
\AgdaBound{t1}\AgdaSpace{}%
\AgdaOperator{\AgdaDatatype{≤}}\AgdaSpace{}%
\AgdaBound{t2}\AgdaSpace{}%
\AgdaSymbol{→}\AgdaSpace{}%
\AgdaBound{t2}\AgdaSpace{}%
\AgdaOperator{\AgdaDatatype{≤}}\AgdaSpace{}%
\AgdaBound{t3}\AgdaSpace{}%
\AgdaSymbol{→}\AgdaSpace{}%
\AgdaBound{t1}\AgdaSpace{}%
\AgdaOperator{\AgdaDatatype{≤}}\AgdaSpace{}%
\AgdaBound{t3}\<%
\\
%
\>[4]\AgdaFunction{≤-trans}\AgdaSpace{}%
\AgdaInductiveConstructor{≤-Z}\AgdaSpace{}%
\AgdaBound{p23}\AgdaSpace{}%
\AgdaSymbol{=}\AgdaSpace{}%
\AgdaInductiveConstructor{≤-Z}\<%
\\
%
\>[4]\AgdaFunction{≤-trans}\AgdaSpace{}%
\AgdaSymbol{(}\AgdaInductiveConstructor{≤-sucMono}\AgdaSpace{}%
\AgdaBound{p12}\AgdaSymbol{)}\AgdaSpace{}%
\AgdaSymbol{(}\AgdaInductiveConstructor{≤-sucMono}\AgdaSpace{}%
\AgdaBound{p23}\AgdaSymbol{)}\AgdaSpace{}%
\AgdaSymbol{=}\AgdaSpace{}%
\AgdaInductiveConstructor{≤-sucMono}\AgdaSpace{}%
\AgdaSymbol{(}\AgdaFunction{≤-trans}\AgdaSpace{}%
\AgdaBound{p12}\AgdaSpace{}%
\AgdaBound{p23}\AgdaSymbol{)}\<%
\\
%
\>[4]\AgdaCatchallClause{\AgdaFunction{≤-trans}}\AgdaSpace{}%
\AgdaCatchallClause{\AgdaBound{p12}}\AgdaSpace{}%
\AgdaCatchallClause{\AgdaSymbol{(}}\AgdaCatchallClause{\AgdaInductiveConstructor{≤-cocone}}\AgdaSpace{}%
\AgdaCatchallClause{\AgdaBound{f}}\AgdaSpace{}%
\AgdaCatchallClause{\AgdaBound{k}}\AgdaSpace{}%
\AgdaCatchallClause{\AgdaBound{p23}}\AgdaCatchallClause{\AgdaSymbol{)}}\AgdaSpace{}%
\AgdaSymbol{=}\AgdaSpace{}%
\AgdaInductiveConstructor{≤-cocone}\AgdaSpace{}%
\AgdaBound{f}\AgdaSpace{}%
\AgdaBound{k}\AgdaSpace{}%
\AgdaSymbol{(}\AgdaFunction{≤-trans}\AgdaSpace{}%
\AgdaBound{p12}\AgdaSpace{}%
\AgdaBound{p23}\AgdaSymbol{)}\<%
\\
%
\>[4]\AgdaCatchallClause{\AgdaFunction{≤-trans}}\AgdaSpace{}%
\AgdaCatchallClause{\AgdaSymbol{(}}\AgdaCatchallClause{\AgdaInductiveConstructor{≤-limiting}}\AgdaSpace{}%
\AgdaCatchallClause{\AgdaBound{f}}\AgdaSpace{}%
\AgdaCatchallClause{\AgdaBound{x}}\AgdaCatchallClause{\AgdaSymbol{)}}\AgdaSpace{}%
\AgdaCatchallClause{\AgdaBound{p23}}\AgdaSpace{}%
\AgdaSymbol{=}\AgdaSpace{}%
\AgdaInductiveConstructor{≤-limiting}\AgdaSpace{}%
\AgdaBound{f}\AgdaSpace{}%
\AgdaSymbol{(λ}\AgdaSpace{}%
\AgdaBound{k}\AgdaSpace{}%
\AgdaSymbol{→}\AgdaSpace{}%
\AgdaFunction{≤-trans}\AgdaSpace{}%
\AgdaSymbol{(}\AgdaBound{x}\AgdaSpace{}%
\AgdaBound{k}\AgdaSymbol{)}\AgdaSpace{}%
\AgdaBound{p23}\AgdaSymbol{)}\<%
\\
%
\>[4]\AgdaFunction{≤-trans}\AgdaSpace{}%
\AgdaSymbol{(}\AgdaInductiveConstructor{≤-cocone}\AgdaSpace{}%
\AgdaBound{f}\AgdaSpace{}%
\AgdaBound{k}\AgdaSpace{}%
\AgdaBound{p12}\AgdaSymbol{)}\AgdaSpace{}%
\AgdaSymbol{(}\AgdaInductiveConstructor{≤-limiting}\AgdaSpace{}%
\AgdaDottedPattern{\AgdaSymbol{.}}\AgdaDottedPattern{\AgdaBound{f}}\AgdaSpace{}%
\AgdaBound{x}\AgdaSymbol{)}\AgdaSpace{}%
\AgdaSymbol{=}\AgdaSpace{}%
\AgdaFunction{≤-trans}\AgdaSpace{}%
\AgdaBound{p12}\AgdaSpace{}%
\AgdaSymbol{(}\AgdaBound{x}\AgdaSpace{}%
\AgdaBound{k}\AgdaSymbol{)}\<%
\\
\>[0]\<%
\end{code}

\begin{code}%
\>[0]\<%
\\
%
\\[\AgdaEmptyExtraSkip]%
\>[0][@{}l@{\AgdaIndent{1}}]%
\>[4]\AgdaKeyword{infixr}\AgdaSpace{}%
\AgdaNumber{10}\AgdaSpace{}%
\AgdaOperator{\AgdaFunction{\AgdaUnderscore{}≤⨟\AgdaUnderscore{}}}\<%
\\
\>[0]\<%
\end{code}

\begin{code}%
\>[0]\<%
\\
%
\\[\AgdaEmptyExtraSkip]%
\>[0][@{}l@{\AgdaIndent{1}}]%
\>[4]\AgdaOperator{\AgdaFunction{\AgdaUnderscore{}≤⨟\AgdaUnderscore{}}}\AgdaSpace{}%
\AgdaSymbol{:}%
\>[12]\AgdaSymbol{∀}\AgdaSpace{}%
\AgdaSymbol{\{}\AgdaBound{t1}\AgdaSpace{}%
\AgdaBound{t2}\AgdaSpace{}%
\AgdaBound{t3}\AgdaSymbol{\}}\AgdaSpace{}%
\AgdaSymbol{→}\AgdaSpace{}%
\AgdaBound{t1}\AgdaSpace{}%
\AgdaOperator{\AgdaDatatype{≤}}\AgdaSpace{}%
\AgdaBound{t2}\AgdaSpace{}%
\AgdaSymbol{→}\AgdaSpace{}%
\AgdaBound{t2}\AgdaSpace{}%
\AgdaOperator{\AgdaDatatype{≤}}\AgdaSpace{}%
\AgdaBound{t3}\AgdaSpace{}%
\AgdaSymbol{→}\AgdaSpace{}%
\AgdaBound{t1}\AgdaSpace{}%
\AgdaOperator{\AgdaDatatype{≤}}\AgdaSpace{}%
\AgdaBound{t3}\<%
\\
%
\>[4]\AgdaBound{lt1}\AgdaSpace{}%
\AgdaOperator{\AgdaFunction{≤⨟}}\AgdaSpace{}%
\AgdaBound{lt2}\AgdaSpace{}%
\AgdaSymbol{=}\AgdaSpace{}%
\AgdaFunction{≤-trans}\AgdaSpace{}%
\AgdaBound{lt1}\AgdaSpace{}%
\AgdaBound{lt2}\<%
\\
\>[0]\<%
\end{code}

\begin{code}%
\>[0]\<%
\\
%
\\[\AgdaEmptyExtraSkip]%
\>[0][@{}l@{\AgdaIndent{1}}]%
\>[4]\AgdaOperator{\AgdaFunction{\AgdaUnderscore{}<o\AgdaUnderscore{}}}\AgdaSpace{}%
\AgdaSymbol{:}\AgdaSpace{}%
\AgdaDatatype{Tree}\AgdaSpace{}%
\AgdaSymbol{→}\AgdaSpace{}%
\AgdaDatatype{Tree}\AgdaSpace{}%
\AgdaSymbol{→}\AgdaSpace{}%
\AgdaPrimitive{Set}\AgdaSpace{}%
\AgdaBound{ℓ}\<%
\\
%
\>[4]\AgdaBound{t1}\AgdaSpace{}%
\AgdaOperator{\AgdaFunction{<o}}\AgdaSpace{}%
\AgdaBound{t2}\AgdaSpace{}%
\AgdaSymbol{=}\AgdaSpace{}%
\AgdaInductiveConstructor{↑}\AgdaSpace{}%
\AgdaBound{t1}\AgdaSpace{}%
\AgdaOperator{\AgdaDatatype{≤}}\AgdaSpace{}%
\AgdaBound{t2}\<%
\\
%
\\[\AgdaEmptyExtraSkip]%
%
\>[4]\AgdaFunction{≤↑t}\AgdaSpace{}%
\AgdaSymbol{:}\AgdaSpace{}%
\AgdaSymbol{∀}\AgdaSpace{}%
\AgdaBound{t}\AgdaSpace{}%
\AgdaSymbol{→}\AgdaSpace{}%
\AgdaBound{t}\AgdaSpace{}%
\AgdaOperator{\AgdaDatatype{≤}}\AgdaSpace{}%
\AgdaInductiveConstructor{↑}\AgdaSpace{}%
\AgdaBound{t}\<%
\\
%
\>[4]\AgdaFunction{≤↑t}\AgdaSpace{}%
\AgdaInductiveConstructor{Z}\AgdaSpace{}%
\AgdaSymbol{=}\AgdaSpace{}%
\AgdaInductiveConstructor{≤-Z}\<%
\\
%
\>[4]\AgdaFunction{≤↑t}\AgdaSpace{}%
\AgdaSymbol{(}\AgdaInductiveConstructor{↑}\AgdaSpace{}%
\AgdaBound{t}\AgdaSymbol{)}\AgdaSpace{}%
\AgdaSymbol{=}\AgdaSpace{}%
\AgdaInductiveConstructor{≤-sucMono}\AgdaSpace{}%
\AgdaSymbol{(}\AgdaFunction{≤↑t}\AgdaSpace{}%
\AgdaBound{t}\AgdaSymbol{)}\<%
\\
%
\>[4]\AgdaFunction{≤↑t}\AgdaSpace{}%
\AgdaSymbol{(}\AgdaInductiveConstructor{Lim}\AgdaSpace{}%
\AgdaBound{c}\AgdaSpace{}%
\AgdaBound{f}\AgdaSymbol{)}\AgdaSpace{}%
\AgdaSymbol{=}\AgdaSpace{}%
\AgdaInductiveConstructor{≤-limiting}\AgdaSpace{}%
\AgdaBound{f}\AgdaSpace{}%
\AgdaSymbol{λ}\AgdaSpace{}%
\AgdaBound{k}\AgdaSpace{}%
\AgdaSymbol{→}\AgdaSpace{}%
\AgdaFunction{≤-trans}\AgdaSpace{}%
\AgdaSymbol{(}\AgdaFunction{≤↑t}\AgdaSpace{}%
\AgdaSymbol{(}\AgdaBound{f}\AgdaSpace{}%
\AgdaBound{k}\AgdaSymbol{))}\AgdaSpace{}%
\AgdaSymbol{(}\AgdaInductiveConstructor{≤-sucMono}\AgdaSpace{}%
\AgdaSymbol{(}\AgdaInductiveConstructor{≤-cocone}\AgdaSpace{}%
\AgdaBound{f}\AgdaSpace{}%
\AgdaBound{k}\AgdaSpace{}%
\AgdaSymbol{(}\AgdaFunction{≤-refl}\AgdaSpace{}%
\AgdaSymbol{(}\AgdaBound{f}\AgdaSpace{}%
\AgdaBound{k}\AgdaSymbol{))))}\<%
\\
%
\\[\AgdaEmptyExtraSkip]%
%
\\[\AgdaEmptyExtraSkip]%
%
\>[4]\AgdaFunction{<-in-≤}\AgdaSpace{}%
\AgdaSymbol{:}\AgdaSpace{}%
\AgdaSymbol{∀}\AgdaSpace{}%
\AgdaSymbol{\{}\AgdaBound{x}\AgdaSpace{}%
\AgdaBound{y}\AgdaSymbol{\}}\AgdaSpace{}%
\AgdaSymbol{→}\AgdaSpace{}%
\AgdaBound{x}\AgdaSpace{}%
\AgdaOperator{\AgdaFunction{<o}}\AgdaSpace{}%
\AgdaBound{y}\AgdaSpace{}%
\AgdaSymbol{→}\AgdaSpace{}%
\AgdaBound{x}\AgdaSpace{}%
\AgdaOperator{\AgdaDatatype{≤}}\AgdaSpace{}%
\AgdaBound{y}\<%
\\
%
\>[4]\AgdaFunction{<-in-≤}\AgdaSpace{}%
\AgdaBound{pf}\AgdaSpace{}%
\AgdaSymbol{=}\AgdaSpace{}%
\AgdaFunction{≤-trans}\AgdaSpace{}%
\AgdaSymbol{(}\AgdaFunction{≤↑t}\AgdaSpace{}%
\AgdaSymbol{\AgdaUnderscore{})}\AgdaSpace{}%
\AgdaBound{pf}\<%
\\
%
\\[\AgdaEmptyExtraSkip]%
%
\\[\AgdaEmptyExtraSkip]%
%
\>[4]\AgdaComment{--\ https://cj-xu.github.io/agda/constructive-ordinals-in-hott/BrouwerTree.Code.Results.html\#3168}\<%
\\
%
\>[4]\AgdaComment{--\ TODO:\ proper\ credit}\<%
\\
%
\>[4]\AgdaFunction{<∘≤-in-<}\AgdaSpace{}%
\AgdaSymbol{:}\AgdaSpace{}%
\AgdaSymbol{∀}\AgdaSpace{}%
\AgdaSymbol{\{}\AgdaBound{x}\AgdaSpace{}%
\AgdaBound{y}\AgdaSpace{}%
\AgdaBound{z}\AgdaSymbol{\}}\AgdaSpace{}%
\AgdaSymbol{→}\AgdaSpace{}%
\AgdaBound{x}\AgdaSpace{}%
\AgdaOperator{\AgdaFunction{<o}}\AgdaSpace{}%
\AgdaBound{y}\AgdaSpace{}%
\AgdaSymbol{→}\AgdaSpace{}%
\AgdaBound{y}\AgdaSpace{}%
\AgdaOperator{\AgdaDatatype{≤}}\AgdaSpace{}%
\AgdaBound{z}\AgdaSpace{}%
\AgdaSymbol{→}\AgdaSpace{}%
\AgdaBound{x}\AgdaSpace{}%
\AgdaOperator{\AgdaFunction{<o}}\AgdaSpace{}%
\AgdaBound{z}\<%
\\
%
\>[4]\AgdaFunction{<∘≤-in-<}\AgdaSpace{}%
\AgdaBound{x<y}\AgdaSpace{}%
\AgdaBound{y≤z}\AgdaSpace{}%
\AgdaSymbol{=}\AgdaSpace{}%
\AgdaFunction{≤-trans}\AgdaSpace{}%
\AgdaBound{x<y}\AgdaSpace{}%
\AgdaBound{y≤z}\<%
\\
%
\\[\AgdaEmptyExtraSkip]%
%
\>[4]\AgdaComment{--\ https://cj-xu.github.io/agda/constructive-ordinals-in-hott/BrouwerTree.Code.Results.html\#3168}\<%
\\
%
\>[4]\AgdaComment{--\ TODO:\ proper\ credit}\<%
\\
%
\>[4]\AgdaFunction{≤∘<-in-<}\AgdaSpace{}%
\AgdaSymbol{:}\AgdaSpace{}%
\AgdaSymbol{∀}\AgdaSpace{}%
\AgdaSymbol{\{}\AgdaBound{x}\AgdaSpace{}%
\AgdaBound{y}\AgdaSpace{}%
\AgdaBound{z}\AgdaSymbol{\}}\AgdaSpace{}%
\AgdaSymbol{→}\AgdaSpace{}%
\AgdaBound{x}\AgdaSpace{}%
\AgdaOperator{\AgdaDatatype{≤}}\AgdaSpace{}%
\AgdaBound{y}\AgdaSpace{}%
\AgdaSymbol{→}\AgdaSpace{}%
\AgdaBound{y}\AgdaSpace{}%
\AgdaOperator{\AgdaFunction{<o}}\AgdaSpace{}%
\AgdaBound{z}\AgdaSpace{}%
\AgdaSymbol{→}\AgdaSpace{}%
\AgdaBound{x}\AgdaSpace{}%
\AgdaOperator{\AgdaFunction{<o}}\AgdaSpace{}%
\AgdaBound{z}\<%
\\
%
\>[4]\AgdaFunction{≤∘<-in-<}\AgdaSpace{}%
\AgdaSymbol{\{}\AgdaBound{x}\AgdaSymbol{\}}\AgdaSpace{}%
\AgdaSymbol{\{}\AgdaBound{y}\AgdaSymbol{\}}\AgdaSpace{}%
\AgdaSymbol{\{}\AgdaBound{z}\AgdaSymbol{\}}\AgdaSpace{}%
\AgdaBound{x≤y}\AgdaSpace{}%
\AgdaBound{y<z}\AgdaSpace{}%
\AgdaSymbol{=}\AgdaSpace{}%
\AgdaFunction{≤-trans}\AgdaSpace{}%
\AgdaSymbol{(}\AgdaInductiveConstructor{≤-sucMono}\AgdaSpace{}%
\AgdaBound{x≤y}\AgdaSymbol{)}\AgdaSpace{}%
\AgdaBound{y<z}\<%
\\
%
\\[\AgdaEmptyExtraSkip]%
%
\>[4]\AgdaComment{--\ underLim\ :\ ∀\ \ \ \{c\ :\ ℂ\}\ (k\ :\ ℂ)\ t\ →\ \ (f\ :\ El\ c\ →\ Tree)\ →\ (∀\ k\ →\ t\ ≤\ f\ k)\ →\ t\ ≤\ Lim\ c\ f}\<%
\\
%
\>[4]\AgdaComment{--\ underLim\ \{c\ =\ c\}\ k\ t\ f\ all\ =\ ≤-trans\ (≤-cocone\ (λ\ \AgdaUnderscore{}\ →\ t)\ \{!!\}\ (≤-refl\ t))\ (≤-limiting\ (λ\ \AgdaUnderscore{}\ →\ t)\ (λ\ k\ →\ ≤-cocone\ f\ k\ (all\ k)))}\<%
\\
%
\\[\AgdaEmptyExtraSkip]%
%
\>[4]\AgdaFunction{extLim}\AgdaSpace{}%
\AgdaSymbol{:}\AgdaSpace{}%
\AgdaSymbol{∀}%
\>[17]\AgdaSymbol{\{}\AgdaBound{c}\AgdaSpace{}%
\AgdaSymbol{:}\AgdaSpace{}%
\AgdaBound{ℂ}\AgdaSymbol{\}}\AgdaSpace{}%
\AgdaSymbol{→}%
\>[28]\AgdaSymbol{(}\AgdaBound{f1}\AgdaSpace{}%
\AgdaBound{f2}\AgdaSpace{}%
\AgdaSymbol{:}\AgdaSpace{}%
\AgdaBound{El}\AgdaSpace{}%
\AgdaBound{c}\AgdaSpace{}%
\AgdaSymbol{→}\AgdaSpace{}%
\AgdaDatatype{Tree}\AgdaSymbol{)}\AgdaSpace{}%
\AgdaSymbol{→}\AgdaSpace{}%
\AgdaSymbol{(∀}\AgdaSpace{}%
\AgdaBound{k}\AgdaSpace{}%
\AgdaSymbol{→}\AgdaSpace{}%
\AgdaBound{f1}\AgdaSpace{}%
\AgdaBound{k}\AgdaSpace{}%
\AgdaOperator{\AgdaDatatype{≤}}\AgdaSpace{}%
\AgdaBound{f2}\AgdaSpace{}%
\AgdaBound{k}\AgdaSymbol{)}\AgdaSpace{}%
\AgdaSymbol{→}\AgdaSpace{}%
\AgdaInductiveConstructor{Lim}\AgdaSpace{}%
\AgdaBound{c}\AgdaSpace{}%
\AgdaBound{f1}\AgdaSpace{}%
\AgdaOperator{\AgdaDatatype{≤}}\AgdaSpace{}%
\AgdaInductiveConstructor{Lim}\AgdaSpace{}%
\AgdaBound{c}\AgdaSpace{}%
\AgdaBound{f2}\<%
\\
%
\>[4]\AgdaFunction{extLim}\AgdaSpace{}%
\AgdaSymbol{\{}\AgdaArgument{c}\AgdaSpace{}%
\AgdaSymbol{=}\AgdaSpace{}%
\AgdaBound{c}\AgdaSymbol{\}}\AgdaSpace{}%
\AgdaBound{f1}\AgdaSpace{}%
\AgdaBound{f2}\AgdaSpace{}%
\AgdaBound{all}\AgdaSpace{}%
\AgdaSymbol{=}\AgdaSpace{}%
\AgdaInductiveConstructor{≤-limiting}\AgdaSpace{}%
\AgdaBound{f1}\AgdaSpace{}%
\AgdaSymbol{(λ}\AgdaSpace{}%
\AgdaBound{k}\AgdaSpace{}%
\AgdaSymbol{→}\AgdaSpace{}%
\AgdaInductiveConstructor{≤-cocone}\AgdaSpace{}%
\AgdaBound{f2}\AgdaSpace{}%
\AgdaBound{k}\AgdaSpace{}%
\AgdaSymbol{(}\AgdaBound{all}\AgdaSpace{}%
\AgdaBound{k}\AgdaSymbol{))}\<%
\\
%
\\[\AgdaEmptyExtraSkip]%
%
\\[\AgdaEmptyExtraSkip]%
%
\>[4]\AgdaFunction{existsLim}\AgdaSpace{}%
\AgdaSymbol{:}\AgdaSpace{}%
\AgdaSymbol{∀}%
\>[19]\AgdaSymbol{\{}\AgdaBound{c1}\AgdaSpace{}%
\AgdaSymbol{:}\AgdaSpace{}%
\AgdaBound{ℂ}\AgdaSymbol{\}}\AgdaSpace{}%
\AgdaSymbol{\{}\AgdaBound{c2}\AgdaSpace{}%
\AgdaSymbol{:}\AgdaSpace{}%
\AgdaBound{ℂ}\AgdaSymbol{\}}\AgdaSpace{}%
\AgdaSymbol{→}%
\>[40]\AgdaSymbol{(}\AgdaBound{f1}\AgdaSpace{}%
\AgdaSymbol{:}\AgdaSpace{}%
\AgdaBound{El}\AgdaSpace{}%
\AgdaBound{c1}%
\>[53]\AgdaSymbol{→}\AgdaSpace{}%
\AgdaDatatype{Tree}\AgdaSymbol{)}\AgdaSpace{}%
\AgdaSymbol{(}\AgdaBound{f2}\AgdaSpace{}%
\AgdaSymbol{:}\AgdaSpace{}%
\AgdaBound{El}%
\>[71]\AgdaBound{c2}%
\>[75]\AgdaSymbol{→}\AgdaSpace{}%
\AgdaDatatype{Tree}\AgdaSymbol{)}\AgdaSpace{}%
\AgdaSymbol{→}\AgdaSpace{}%
\AgdaSymbol{(∀}\AgdaSpace{}%
\AgdaBound{k1}\AgdaSpace{}%
\AgdaSymbol{→}\AgdaSpace{}%
\AgdaFunction{Σ[}\AgdaSpace{}%
\AgdaBound{k2}\AgdaSpace{}%
\AgdaFunction{∈}\AgdaSpace{}%
\AgdaBound{El}%
\>[105]\AgdaBound{c2}\AgdaSpace{}%
\AgdaFunction{]}\AgdaSpace{}%
\AgdaBound{f1}\AgdaSpace{}%
\AgdaBound{k1}\AgdaSpace{}%
\AgdaOperator{\AgdaDatatype{≤}}\AgdaSpace{}%
\AgdaBound{f2}\AgdaSpace{}%
\AgdaBound{k2}\AgdaSymbol{)}\AgdaSpace{}%
\AgdaSymbol{→}\AgdaSpace{}%
\AgdaInductiveConstructor{Lim}%
\>[132]\AgdaBound{c1}\AgdaSpace{}%
\AgdaBound{f1}\AgdaSpace{}%
\AgdaOperator{\AgdaDatatype{≤}}\AgdaSpace{}%
\AgdaInductiveConstructor{Lim}%
\>[145]\AgdaBound{c2}\AgdaSpace{}%
\AgdaBound{f2}\<%
\\
%
\>[4]\AgdaFunction{existsLim}\AgdaSpace{}%
\AgdaSymbol{\{}\AgdaBound{æ1}\AgdaSymbol{\}}\AgdaSpace{}%
\AgdaSymbol{\{}\AgdaBound{æ2}\AgdaSymbol{\}}\AgdaSpace{}%
\AgdaBound{f1}\AgdaSpace{}%
\AgdaBound{f2}\AgdaSpace{}%
\AgdaBound{allex}\AgdaSpace{}%
\AgdaSymbol{=}\AgdaSpace{}%
\AgdaInductiveConstructor{≤-limiting}%
\>[50]\AgdaBound{f1}\AgdaSpace{}%
\AgdaSymbol{(λ}\AgdaSpace{}%
\AgdaBound{k}\AgdaSpace{}%
\AgdaSymbol{→}\AgdaSpace{}%
\AgdaInductiveConstructor{≤-cocone}\AgdaSpace{}%
\AgdaBound{f2}\AgdaSpace{}%
\AgdaSymbol{(}\AgdaField{proj₁}\AgdaSpace{}%
\AgdaSymbol{(}\AgdaBound{allex}\AgdaSpace{}%
\AgdaBound{k}\AgdaSymbol{))}\AgdaSpace{}%
\AgdaSymbol{(}\AgdaField{proj₂}\AgdaSpace{}%
\AgdaSymbol{(}\AgdaBound{allex}\AgdaSpace{}%
\AgdaBound{k}\AgdaSymbol{)))}\<%
\\
%
\\[\AgdaEmptyExtraSkip]%
%
\\[\AgdaEmptyExtraSkip]%
%
\>[4]\AgdaFunction{¬Z<↑o}\AgdaSpace{}%
\AgdaSymbol{:}\AgdaSpace{}%
\AgdaSymbol{∀}%
\>[15]\AgdaBound{t}\AgdaSpace{}%
\AgdaSymbol{→}\AgdaSpace{}%
\AgdaOperator{\AgdaFunction{¬}}\AgdaSpace{}%
\AgdaSymbol{((}\AgdaInductiveConstructor{↑}\AgdaSpace{}%
\AgdaBound{t}\AgdaSymbol{)}\AgdaSpace{}%
\AgdaOperator{\AgdaDatatype{≤}}\AgdaSpace{}%
\AgdaInductiveConstructor{Z}\AgdaSymbol{)}\<%
\\
%
\>[4]\AgdaFunction{¬Z<↑o}\AgdaSpace{}%
\AgdaBound{t}\AgdaSpace{}%
\AgdaSymbol{()}\<%
\\
%
\\[\AgdaEmptyExtraSkip]%
%
\\[\AgdaEmptyExtraSkip]%
%
\>[4]\AgdaKeyword{open}\AgdaSpace{}%
\AgdaKeyword{import}\AgdaSpace{}%
\AgdaModule{Induction.WellFounded}\<%
\\
%
\>[4]\AgdaFunction{access}\AgdaSpace{}%
\AgdaSymbol{:}\AgdaSpace{}%
\AgdaSymbol{∀}\AgdaSpace{}%
\AgdaSymbol{\{}\AgdaBound{x}\AgdaSymbol{\}}\AgdaSpace{}%
\AgdaSymbol{→}\AgdaSpace{}%
\AgdaDatatype{Acc}\AgdaSpace{}%
\AgdaOperator{\AgdaFunction{\AgdaUnderscore{}<o\AgdaUnderscore{}}}\AgdaSpace{}%
\AgdaBound{x}\AgdaSpace{}%
\AgdaSymbol{→}\AgdaSpace{}%
\AgdaFunction{WfRec}\AgdaSpace{}%
\AgdaOperator{\AgdaFunction{\AgdaUnderscore{}<o\AgdaUnderscore{}}}\AgdaSpace{}%
\AgdaSymbol{(}\AgdaDatatype{Acc}\AgdaSpace{}%
\AgdaOperator{\AgdaFunction{\AgdaUnderscore{}<o\AgdaUnderscore{}}}\AgdaSymbol{)}\AgdaSpace{}%
\AgdaBound{x}\<%
\\
%
\>[4]\AgdaFunction{access}\AgdaSpace{}%
\AgdaSymbol{(}\AgdaInductiveConstructor{acc}\AgdaSpace{}%
\AgdaBound{r}\AgdaSymbol{)}\AgdaSpace{}%
\AgdaSymbol{=}\AgdaSpace{}%
\AgdaBound{r}\<%
\\
%
\\[\AgdaEmptyExtraSkip]%
%
\>[4]\AgdaComment{--\ https://cj-xu.github.io/agda/constructive-ordinals-in-hott/BrouwerTree.Code.Results.html\#3168}\<%
\\
%
\>[4]\AgdaComment{--\ TODO:\ proper\ credit}\<%
\\
%
\>[4]\AgdaFunction{smaller-accessible}\AgdaSpace{}%
\AgdaSymbol{:}\AgdaSpace{}%
\AgdaSymbol{(}\AgdaBound{x}\AgdaSpace{}%
\AgdaSymbol{:}\AgdaSpace{}%
\AgdaDatatype{Tree}\AgdaSymbol{)}\AgdaSpace{}%
\AgdaSymbol{→}\AgdaSpace{}%
\AgdaDatatype{Acc}\AgdaSpace{}%
\AgdaOperator{\AgdaFunction{\AgdaUnderscore{}<o\AgdaUnderscore{}}}\AgdaSpace{}%
\AgdaBound{x}\AgdaSpace{}%
\AgdaSymbol{→}\AgdaSpace{}%
\AgdaSymbol{∀}\AgdaSpace{}%
\AgdaBound{y}\AgdaSpace{}%
\AgdaSymbol{→}\AgdaSpace{}%
\AgdaBound{y}\AgdaSpace{}%
\AgdaOperator{\AgdaDatatype{≤}}\AgdaSpace{}%
\AgdaBound{x}\AgdaSpace{}%
\AgdaSymbol{→}\AgdaSpace{}%
\AgdaDatatype{Acc}\AgdaSpace{}%
\AgdaOperator{\AgdaFunction{\AgdaUnderscore{}<o\AgdaUnderscore{}}}\AgdaSpace{}%
\AgdaBound{y}\<%
\\
%
\>[4]\AgdaFunction{smaller-accessible}\AgdaSpace{}%
\AgdaBound{x}\AgdaSpace{}%
\AgdaBound{isAcc}\AgdaSpace{}%
\AgdaBound{y}\AgdaSpace{}%
\AgdaBound{x<y}\AgdaSpace{}%
\AgdaSymbol{=}\AgdaSpace{}%
\AgdaInductiveConstructor{acc}\AgdaSpace{}%
\AgdaSymbol{(λ}\AgdaSpace{}%
\AgdaBound{y'}\AgdaSpace{}%
\AgdaBound{y'<y}\AgdaSpace{}%
\AgdaSymbol{→}\AgdaSpace{}%
\AgdaFunction{access}\AgdaSpace{}%
\AgdaBound{isAcc}\AgdaSpace{}%
\AgdaBound{y'}\AgdaSpace{}%
\AgdaSymbol{(}\AgdaFunction{<∘≤-in-<}\AgdaSpace{}%
\AgdaBound{y'<y}\AgdaSpace{}%
\AgdaBound{x<y}\AgdaSymbol{))}\<%
\\
%
\\[\AgdaEmptyExtraSkip]%
%
\>[4]\AgdaComment{--\ https://cj-xu.github.io/agda/constructive-ordinals-in-hott/BrouwerTree.Code.Results.html\#3168}\<%
\\
%
\>[4]\AgdaComment{--\ TODO:\ proper\ credit}\<%
\\
%
\>[4]\AgdaFunction{ordWF}\AgdaSpace{}%
\AgdaSymbol{:}\AgdaSpace{}%
\AgdaFunction{WellFounded}\AgdaSpace{}%
\AgdaOperator{\AgdaFunction{\AgdaUnderscore{}<o\AgdaUnderscore{}}}\<%
\\
%
\>[4]\AgdaFunction{ordWF}\AgdaSpace{}%
\AgdaInductiveConstructor{Z}\AgdaSpace{}%
\AgdaSymbol{=}\AgdaSpace{}%
\AgdaInductiveConstructor{acc}\AgdaSpace{}%
\AgdaSymbol{λ}\AgdaSpace{}%
\AgdaSymbol{\AgdaUnderscore{}}\AgdaSpace{}%
\AgdaSymbol{()}\<%
\\
%
\>[4]\AgdaFunction{ordWF}\AgdaSpace{}%
\AgdaSymbol{(}\AgdaInductiveConstructor{↑}\AgdaSpace{}%
\AgdaBound{x}\AgdaSymbol{)}\AgdaSpace{}%
\AgdaSymbol{=}\AgdaSpace{}%
\AgdaInductiveConstructor{acc}\AgdaSpace{}%
\AgdaSymbol{(λ}\AgdaSpace{}%
\AgdaSymbol{\{}\AgdaSpace{}%
\AgdaBound{y}\AgdaSpace{}%
\AgdaSymbol{(}\AgdaInductiveConstructor{≤-sucMono}\AgdaSpace{}%
\AgdaBound{y≤x}\AgdaSymbol{)}\AgdaSpace{}%
\AgdaSymbol{→}\AgdaSpace{}%
\AgdaFunction{smaller-accessible}\AgdaSpace{}%
\AgdaBound{x}\AgdaSpace{}%
\AgdaSymbol{(}\AgdaFunction{ordWF}\AgdaSpace{}%
\AgdaBound{x}\AgdaSymbol{)}\AgdaSpace{}%
\AgdaBound{y}\AgdaSpace{}%
\AgdaBound{y≤x}\AgdaSymbol{\})}\<%
\\
%
\>[4]\AgdaFunction{ordWF}\AgdaSpace{}%
\AgdaSymbol{(}\AgdaInductiveConstructor{Lim}\AgdaSpace{}%
\AgdaBound{c}\AgdaSpace{}%
\AgdaBound{f}\AgdaSymbol{)}\AgdaSpace{}%
\AgdaSymbol{=}\AgdaSpace{}%
\AgdaInductiveConstructor{acc}\AgdaSpace{}%
\AgdaFunction{helper}\<%
\\
\>[4][@{}l@{\AgdaIndent{0}}]%
\>[6]\AgdaKeyword{where}\<%
\\
\>[6][@{}l@{\AgdaIndent{0}}]%
\>[8]\AgdaFunction{helper}\AgdaSpace{}%
\AgdaSymbol{:}\AgdaSpace{}%
\AgdaSymbol{(}\AgdaBound{y}\AgdaSpace{}%
\AgdaSymbol{:}\AgdaSpace{}%
\AgdaDatatype{Tree}\AgdaSymbol{)}\AgdaSpace{}%
\AgdaSymbol{→}\AgdaSpace{}%
\AgdaSymbol{(}\AgdaBound{y}\AgdaSpace{}%
\AgdaOperator{\AgdaFunction{<o}}\AgdaSpace{}%
\AgdaInductiveConstructor{Lim}\AgdaSpace{}%
\AgdaBound{c}\AgdaSpace{}%
\AgdaBound{f}\AgdaSymbol{)}\AgdaSpace{}%
\AgdaSymbol{→}\AgdaSpace{}%
\AgdaDatatype{Acc}\AgdaSpace{}%
\AgdaOperator{\AgdaFunction{\AgdaUnderscore{}<o\AgdaUnderscore{}}}\AgdaSpace{}%
\AgdaBound{y}\<%
\\
%
\>[8]\AgdaFunction{helper}\AgdaSpace{}%
\AgdaBound{y}\AgdaSpace{}%
\AgdaSymbol{(}\AgdaInductiveConstructor{≤-cocone}\AgdaSpace{}%
\AgdaDottedPattern{\AgdaSymbol{.}}\AgdaDottedPattern{\AgdaBound{f}}\AgdaSpace{}%
\AgdaBound{k}\AgdaSpace{}%
\AgdaBound{y<fk}\AgdaSymbol{)}\AgdaSpace{}%
\AgdaSymbol{=}\AgdaSpace{}%
\AgdaFunction{smaller-accessible}\AgdaSpace{}%
\AgdaSymbol{(}\AgdaBound{f}\AgdaSpace{}%
\AgdaBound{k}\AgdaSymbol{)}\AgdaSpace{}%
\AgdaSymbol{(}\AgdaFunction{ordWF}\AgdaSpace{}%
\AgdaSymbol{(}\AgdaBound{f}\AgdaSpace{}%
\AgdaBound{k}\AgdaSymbol{))}\AgdaSpace{}%
\AgdaBound{y}\AgdaSpace{}%
\AgdaSymbol{(}\AgdaFunction{<-in-≤}\AgdaSpace{}%
\AgdaBound{y<fk}\AgdaSymbol{)}\<%
\\
%
\\[\AgdaEmptyExtraSkip]%
%
\\[\AgdaEmptyExtraSkip]%
\>[0]\<%
\end{code}


\begin{code}%
\>[0][@{}l@{\AgdaIndent{1}}]%
\>[4]\AgdaFunction{limMax}\AgdaSpace{}%
\AgdaSymbol{:}\AgdaSpace{}%
\AgdaDatatype{Tree}\AgdaSpace{}%
\AgdaSymbol{→}\AgdaSpace{}%
\AgdaDatatype{Tree}\AgdaSpace{}%
\AgdaSymbol{→}\AgdaSpace{}%
\AgdaDatatype{Tree}\<%
\\
%
\>[4]\AgdaFunction{limMax}\AgdaSpace{}%
\AgdaBound{t1}\AgdaSpace{}%
\AgdaBound{t2}\AgdaSpace{}%
\AgdaSymbol{=}\AgdaSpace{}%
\AgdaInductiveConstructor{Lim}\AgdaSpace{}%
\AgdaBound{Cℕ}\AgdaSpace{}%
\AgdaSymbol{λ}\AgdaSpace{}%
\AgdaBound{k}\AgdaSpace{}%
\AgdaSymbol{→}\AgdaSpace{}%
\AgdaFunction{if0}\AgdaSpace{}%
\AgdaSymbol{(}\AgdaField{Iso.fun}\AgdaSpace{}%
\AgdaBound{CℕIso}\AgdaSpace{}%
\AgdaBound{k}\AgdaSymbol{)}\AgdaSpace{}%
\AgdaBound{t1}\AgdaSpace{}%
\AgdaBound{t2}\<%
\\
\>[4][@{}l@{\AgdaIndent{0}}]%
\>[6]\AgdaKeyword{where}\<%
\\
\>[6][@{}l@{\AgdaIndent{0}}]%
\>[8]\AgdaFunction{if0}\AgdaSpace{}%
\AgdaSymbol{:}\AgdaSpace{}%
\AgdaDatatype{ℕ}\AgdaSpace{}%
\AgdaSymbol{→}\AgdaSpace{}%
\AgdaDatatype{Tree}\AgdaSpace{}%
\AgdaSymbol{→}\AgdaSpace{}%
\AgdaDatatype{Tree}\AgdaSpace{}%
\AgdaSymbol{→}\AgdaSpace{}%
\AgdaDatatype{Tree}\<%
\\
%
\>[8]\AgdaFunction{if0}\AgdaSpace{}%
\AgdaInductiveConstructor{zero}\AgdaSpace{}%
\AgdaBound{z}\AgdaSpace{}%
\AgdaBound{s}\AgdaSpace{}%
\AgdaSymbol{=}\AgdaSpace{}%
\AgdaBound{z}\<%
\\
%
\>[8]\AgdaFunction{if0}\AgdaSpace{}%
\AgdaSymbol{(}\AgdaInductiveConstructor{suc}\AgdaSpace{}%
\AgdaBound{n}\AgdaSymbol{)}\AgdaSpace{}%
\AgdaBound{z}\AgdaSpace{}%
\AgdaBound{s}\AgdaSpace{}%
\AgdaSymbol{=}\AgdaSpace{}%
\AgdaBound{s}\<%
\end{code}


% \begin{code}


%     private
%       data IndMaxView : Tree → Tree → Set ℓ where
%         IndMaxZ-L : ∀ {t} → IndMaxView Z o
%         IndMaxZ-R : ∀ {t} → IndMaxView t Z
%         IndMaxLim-L : ∀ {o } {c : ℂ} {f : El c → Tree} → IndMaxView (Lim c f) o
%         IndMaxLim-R : ∀ {o } {c : ℂ} {f : El c → Tree}
%           → (∀   {c' : ℂ} {f' : El c' → Tree} → ¬ (o ≡ Lim  c' f'))
%           → IndMaxView t (Lim c f)
%         IndMaxLim-Suc : ∀  {t1 t2 } → IndMaxView (↑ t1) (↑ t2)

%       indMaxView : ∀ t1 t2 → IndMaxView t1 t2
%       indMaxView Z t2 = IndMaxZ-L
%       indMaxView (Lim c f) t2 = IndMaxLim-L
%       indMaxView (↑ t1) Z = IndMaxZ-R
%       indMaxView (↑ t1) (Lim c f) = IndMaxLim-R λ ()
%       indMaxView (↑ t1) (↑ t2) = IndMaxLim-Suc

%     abstract
%       indMax : Tree → Tree → Tree
%       indMax' : ∀ {t1 t2} → IndMaxView t1 t2 → Tree

%       indMax t1 t2 = indMax' (indMaxView t1 t2)

%       indMax' {.Z} {t2} IndMaxZ-L = t2
%       indMax' {t1} {.Z} IndMaxZ-R = t1
%       indMax' {(Lim c f)} {t2} IndMaxLim-L = Lim c λ x → indMax (f x) t2
%       indMax' {t1} {(Lim c f)} (IndMaxLim-R _) = Lim c (λ x → indMax t1 (f x))
%       indMax' {(↑ t1)} {(↑ t2)} IndMaxLim-Suc = ↑ (indMax t1 t2)

%       indMax-≤L : ∀ {t1 t2} → t1 ≤ indMax t1 t2
%       indMax-≤L {t1} {t2} with indMaxView t1 t2
%       ... | IndMaxZ-L = ≤-Z
%       ... | IndMaxZ-R = ≤-refl _
%       ... | IndMaxLim-L {f = f} = extLim f (λ x → indMax (f x) t2) (λ k → indMax-≤L)
%       ... | IndMaxLim-R {f = f} _ = underLim {!!} t1 (λ x → indMax t1 (f x)) (λ k → indMax-≤L)
%       ... | IndMaxLim-Suc = ≤-sucMono indMax-≤L


%       indMax-≤R : ∀ {t1 t2} → t2 ≤ indMax t1 t2
%       indMax-≤R {t1} {t2} with indMaxView t1 t2
%       ... | IndMaxZ-R = ≤-Z
%       ... | IndMaxZ-L = ≤-refl _
%       ... | IndMaxLim-R {f = f} _ = extLim f (λ x → indMax t1 (f x)) (λ k → indMax-≤R {t1 = t1} {f k})
%       ... | IndMaxLim-L {f = f} = underLim {!!} t2 (λ x → indMax (f x) t2) (λ k → indMax-≤R {t1 = f k} {t2 = t2})
%       ... | IndMaxLim-Suc {t1} {t2} = ≤-sucMono (indMax-≤R {t1 = t1} {t2 = t2})




%       indMax-monoR : ∀ {t1 t2 t2'} → t2 ≤ t2' → indMax t1 t2 ≤ indMax t1 t2'
%       indMax-monoR' : ∀ {t1 t2 t2'} → t2 <o t2' → indMax t1 t2 <o indMax (↑ t1) t2'

%       indMax-monoR {t1} {t2} {t2'} lt with indMaxView t1 t2 in eq1 | indMaxView t1 t2' in eq2
%       ... | IndMaxZ-L  | v2  = ≤-trans lt (≤-reflEq (cong indMax' eq2))
%       ... | IndMaxZ-R  | v2  = ≤-trans indMax-≤L (≤-reflEq (cong indMax' eq2))
%       ... | IndMaxLim-L {f = f1} |  IndMaxLim-L  = extLim _ _ λ k → indMax-monoR {t1 = f1 k} lt
%       indMax-monoR {t1} {(Lim _ f')} {.(Lim _ f)} (≤-cocone f k lt) | IndMaxLim-R neq  | IndMaxLim-R neq'
%         = ≤-limiting (λ x → indMax t1 (f' x)) (λ y → ≤-cocone (λ x → indMax t1 (f x)) k (indMax-monoR {t1 = t1} {t2 = f' y} (≤-trans (≤-cocone _ y (≤-refl _)) lt)))
%       indMax-monoR {t1} {.(Lim _ _)} {t2'} (≤-limiting f x₁) | IndMaxLim-R x  | v2  =
%         ≤-trans (≤-limiting (λ x₂ → indMax t1 (f x₂)) λ k → indMax-monoR {t1 = t1} (x₁ k)) (≤-reflEq (cong indMax' eq2))
%       indMax-monoR {(↑ t1)} {.(↑ _)} {.(↑ _)} (≤-sucMono lt) | IndMaxLim-Suc  | IndMaxLim-Suc  = ≤-sucMono (indMax-monoR {t1 = t1} lt)
%       indMax-monoR {(↑ t1)} {(↑ t2)} {(Lim _ f)} (≤-cocone f k lt) | IndMaxLim-Suc  | IndMaxLim-R x
%         = ≤-trans (indMax-monoR' {t1 = t1} {t2 = t2} {t2' = f k} lt) (≤-cocone (λ x₁ → indMax (↑ t1) (f x₁)) k (≤-refl _)) --indMax-monoR' {!lt!}

%       indMax-monoR' {t1} {t2} {t2'}  (≤-sucMono lt) = ≤-sucMono ( (indMax-monoR {t1 = t1} lt))
%       indMax-monoR' {t1} {t2} {.(Lim _ f)} (≤-cocone f k lt)
%         = ≤-cocone _ k (indMax-monoR' {t1 = t1} lt)


%       indMax-monoL : ∀ {t1 t1' t2} → t1 ≤ t1' → indMax t1 t2 ≤ indMax t1' t2
%       indMax-monoL' : ∀ {t1 t1' t2} → t1 <o t1' → indMax t1 t2 <o indMax t1' (↑ t2)
%       indMax-monoL {t1} {t1'} {t2} lt with indMaxView t1 t2 in eq1 | indMaxView t1' t2 in eq2
%       ... | IndMaxZ-L | v2 = ≤-trans (indMax-≤R {t1 = t1'}) (≤-reflEq (cong indMax' eq2))
%       ... | IndMaxZ-R | v2 = ≤-trans lt (≤-trans (indMax-≤L {t1 = t1'}) (≤-reflEq (cong indMax' eq2)))
%       indMax-monoL {.(Lim _ _)} {.(Lim _ f)} {t2} (≤-cocone f k lt) | IndMaxLim-L  | IndMaxLim-L
%         = ≤-cocone (λ x → indMax (f x) t2) k (indMax-monoL lt)
%       indMax-monoL {.(Lim _ _)} {t1'} {t2} (≤-limiting f lt) | IndMaxLim-L |  v2
%         = ≤-limiting (λ x₁ → indMax (f x₁) t2) λ k → ≤-trans (indMax-monoL (lt k)) (≤-reflEq (cong indMax' eq2))
%       indMax-monoL {.Z} {.Z} {.(Lim _ _)} ≤-Z | IndMaxLim-R neq  | IndMaxZ-L  = ≤-refl _
%       indMax-monoL  {.(Lim _ f)} {.Z} {.(Lim _ _)} (≤-limiting f x) | IndMaxLim-R neq | IndMaxZ-L
%         with () ← neq refl
%       indMax-monoL {t1} {.(Lim _ _)} {.(Lim _ _)} (≤-cocone _ k lt) | IndMaxLim-R {f = f} neq | IndMaxLim-L {f = f'}
%         = ≤-limiting (λ x → indMax t1 (f x)) (λ y → ≤-cocone (λ x → indMax (f' x) (Lim _ _)) k
%           (≤-trans (indMax-monoL lt) (indMax-monoR {t1 = f' k} (≤-cocone f y (≤-refl _)))))
%       ... | IndMaxLim-R neq | IndMaxLim-R {f = f} neq' = extLim (λ x → indMax t1 (f x)) (λ x → indMax t1' (f x)) (λ k → indMax-monoL lt)
%       indMax-monoL {.(↑ _)} {.(↑ _)} {.(↑ _)} (≤-sucMono lt) | IndMaxLim-Suc  | IndMaxLim-Suc
%         = ≤-sucMono (indMax-monoL lt)
%       indMax-monoL {.(↑ _)} {.(Lim _ f)} {.(↑ _)} (≤-cocone f k lt) | IndMaxLim-Suc  | IndMaxLim-L
%         = ≤-cocone (λ x → indMax (f x) (↑ _)) k (indMax-monoL' lt)

%       indMax-monoL' {t1} {t1'} {t2} lt with indMaxView t1 t2 in eq1 | indMaxView t1' t2 in eq2
%       indMax-monoL' {t1} {.(↑ _)} {t2} (≤-sucMono lt) | v1 | v2 = ≤-sucMono (≤-trans (≤-reflEq (cong indMax' (sym eq1))) (indMax-monoL lt))
%       indMax-monoL' {t1} {.(Lim _ f)} {t2} (≤-cocone f k lt) | v1 | v2
%         = ≤-cocone _ k (≤-trans (≤-sucMono (≤-reflEq (cong indMax' (sym eq1)))) (indMax-monoL' lt))


%       indMax-mono : ∀ {t1 t2 t1' t2'} → t1 ≤ t1' → t2 ≤ t2' → indMax t1 t2 ≤ indMax t1' t2'
%       indMax-mono {t1' = t1'} lt1 lt2 = ≤-trans (indMax-monoL lt1) (indMax-monoR {t1 = t1'} lt2)

%       indMax-strictMono : ∀ {t1 t2 t1' t2'} → t1 <o t1' → t2 <o t2' → indMax t1 t2 <o indMax t1' t2'
%       indMax-strictMono lt1 lt2 = indMax-mono lt1 lt2


%       indMax-sucMono : ∀ {t1 t2 t1' t2'} → indMax t1 t2 ≤ indMax t1' t2' → indMax t1 t2 <o indMax (↑ t1') (↑ t2')
%       indMax-sucMono lt = ≤-sucMono lt


%       -- indMax-Z : ∀ t → indMax t Z ≡ o
%       -- indMax-Z Z = refl
%       -- indMax-Z (↑ t) = refl
%       -- indMax-Z (Lim c f) = cong (Lim c) {!!} -- cong (Lim c) (funExt (λ x → indMax-Z (f x)))

%       indMax-Z : ∀ t → indMax t Z ≤ o
%       indMax-Z Z = ≤-Z
%       indMax-Z (↑ t) = ≤-refl (indMax (↑ t) Z)
%       indMax-Z (Lim c f) = extLim (λ x → indMax (f x) Z) f (λ k → indMax-Z (f k))

%       indMax-↑ : ∀ {t1 t2} → indMax (↑ t1) (↑ t2) ≡ ↑ (indMax t1 t2)
%       indMax-↑ = refl

%       indMax-≤Z : ∀ t → indMax t Z ≤ o
%       indMax-≤Z Z = ≤-refl _
%       indMax-≤Z (↑ t) = ≤-refl _
%       indMax-≤Z (Lim c f) = extLim _ _ (λ k → indMax-≤Z (f k))

%       -- indMax-oneL : ∀ {t} → indMax T1 (↑ t) ≤ ↑ t
%       -- indMax-oneL  = ≤-refl _

%       -- indMax-oneR : ∀ {t} → indMax (↑ t) T1 ≤ ↑ t
%       -- indMax-oneR {Z} = ≤-sucMono (≤-refl _)
%       -- indMax-oneR {↑ t} = ≤-sucMono (≤-refl _)
%       -- indMax-oneR {Lim c f} = ≤-sucMono (substPath (λ x → x ≤ Lim c f) (sym (indMax-Z (Lim c f))) (≤-refl (Lim c f))) -- rewrite ctop (indMax-Z (Lim c f))= ≤-refl _


%       indMax-limR : ∀   {c : ℂ} (f : El  c  → Tree) t → indMax t (Lim c f) ≤ Lim c (λ k → indMax t (f k))
%       indMax-limR f Z = ≤-refl _
%       indMax-limR f (↑ t) = extLim _ _ λ k → ≤-refl _
%       indMax-limR f (Lim c f₁) = ≤-limiting _ λ k → ≤-trans (indMax-limR f (f₁ k)) (extLim _ _ (λ k2 → indMax-monoL {t1 = f₁ k} {t1' = Lim c f₁} {t2 = f k2}  (≤-cocone _ k (≤-refl _))))


%       indMax-commut : ∀ t1 t2 → indMax t1 t2 ≤ indMax t2 t1
%       indMax-commut t1 t2 with indMaxView t1 t2
%       ... | IndMaxZ-L = indMax-≤L
%       ... | IndMaxZ-R = ≤-refl _
%       ... | IndMaxLim-R {f = f} x = extLim _ _ (λ k → indMax-commut t1 (f k))
%       ... | IndMaxLim-Suc {t1 = t1} {t2 = t2} = ≤-sucMono (indMax-commut t1 t2)
%       ... | IndMaxLim-L {c = c} {f = f} with indMaxView t2 t1
%       ... | IndMaxZ-L = extLim _ _ λ k → indMax-Z (f k)
%       ... | IndMaxLim-R x = extLim _ _ (λ k → indMax-commut (f k) t2)
%       ... | IndMaxLim-L {c = c2} {f = f2} =
%         ≤-trans (extLim _ _ λ k → indMax-limR f2 (f k))
%         (≤-trans (≤-limiting _ (λ k → ≤-limiting _ λ k2 → ≤-cocone _ k2 (≤-cocone _ k (≤-refl _))))
%         (≤-trans (≤-refl (Lim c2 λ k2 → Lim c λ k → indMax (f k) (f2 k2)))
%         (extLim _ _ (λ k2 → ≤-limiting _ λ k1 → ≤-trans (indMax-commut (f k1) (f2 k2)) (indMax-monoR {t1 = f2 k2} {t2 = f k1} {t2' = Lim c f} (≤-cocone _ k1 (≤-refl _)))))))


%       indMax-assocL : ∀ t1 t2 t3 → indMax t1 (indMax t2 t3) ≤ indMax (indMax t1 t2) t3
%       indMax-assocL t1 t2 t3 with indMaxView t2 t3 in eq23
%       ... | IndMaxZ-L = indMax-monoL {t1 = t1} {t1' = indMax t1 Z} {t2 = t3} indMax-≤L
%       ... | IndMaxZ-R = indMax-≤L
%       ... | m with indMaxView t1 t2
%       ... | IndMaxZ-L rewrite sym eq23 = ≤-refl _
%       ... | IndMaxZ-R rewrite sym eq23 = ≤-refl _
%       ... | IndMaxLim-R {f = f} x rewrite sym eq23 = ≤-trans (indMax-limR (λ x → indMax (f x) t3) t1) (extLim _ _ λ k → indMax-assocL t1 (f k) t3) -- f,indMax-limR f t1
%       indMax-assocL .(↑ _) .(↑ _) .Z | IndMaxZ-R  | IndMaxLim-Suc = ≤-refl _
%       indMax-assocL t1 t2 .(Lim _ _) | IndMaxLim-R {f = f} x   | IndMaxLim-Suc = extLim _ _ λ k → indMax-assocL t1 t2 (f k)
%       indMax-assocL (↑ t1) (↑ t2) (↑ t3) | IndMaxLim-Suc  | IndMaxLim-Suc = ≤-sucMono (indMax-assocL t1 t2 t3)
%       ... | IndMaxLim-L {f = f} rewrite sym eq23 = extLim _ _ λ k → indMax-assocL (f k) t2 t3



%       indMax-assocR : ∀ t1 t2 t3 →  indMax (indMax t1 t2) t3 ≤ indMax t1 (indMax t2 t3)
%       indMax-assocR t1 t2 t3 = ≤-trans (indMax-commut (indMax t1 t2) t3) (≤-trans (indMax-monoR {t1 = t3} (indMax-commut t1 t2))
%         (≤-trans (indMax-assocL t3 t2 t1) (≤-trans (indMax-commut (indMax t3 t2) t1) (indMax-monoR {t1 = t1} (indMax-commut t3 t2)))))


%       indMax-swap4 : ∀ {t1 t1' t2 t2'} → indMax (indMax t1 t1') (indMax t2 t2') ≤ indMax (indMax t1 t2) (indMax t1' t2')
%       indMax-swap4 {t1}{t1'}{t2 }{t2'} =
%         indMax-assocL (indMax t1 t1') t2 t2'
%         ≤⨟ indMax-monoL {t1 = indMax (indMax t1 t1') t2} {t2 = t2'}
%           (indMax-assocR t1 t1' t2 ≤⨟ indMax-monoR {t1 = t1} (indMax-commut t1' t2) ≤⨟ indMax-assocL t1 t2 t1')
%         ≤⨟ indMax-assocR (indMax t1 t2) t1' t2'

%       indMax-swap6 : ∀ {t1 t2 t3 t1' t2' t3'} → indMax (indMax t1 t1') (indMax (indMax t2 t2') (indMax t3 t3')) ≤ indMax (indMax t1 (indMax t2 t3)) (indMax t1' (indMax t2' t3'))
%       indMax-swap6 {t1} {t2} {t3} {t1'} {t2'} {t3'} =
%         indMax-monoR {t1 = indMax t1 t1'} (indMax-swap4 {t1 = t2} {t1' = t2'} {t2 = t3} {t2' = t3'})
%         ≤⨟ indMax-swap4 {t1 = t1} {t1' = t1'}

%       indMax-lim2L :
%         ∀
%         {c1 : ℂ}
%         (f1 : El  c1 → Tree)
%         {c2 : ℂ}
%         (f2 : El  c2 → Tree)
%         → Lim  c1 (λ k1 → Lim  c2 (λ k2 → indMax (f1 k1) (f2 k2))) ≤ indMax (Lim  c1 f1) (Lim  c2 f2)
%       indMax-lim2L f1 f2 = ≤-limiting  _ (λ k1 → ≤-limiting  _ λ k2 → indMax-mono (≤-cocone  f1 k1 (≤-refl _)) (≤-cocone  f2 k2 (≤-refl _)))

%       indMax-lim2R :
%         ∀
%         {c1 : ℂ}
%         (f1 : El  c1 → Tree)
%         {c2 : ℂ}
%         (f2 : El  c2 → Tree)
%         →  indMax (Lim  c1 f1) (Lim  c2 f2) ≤ Lim  c1 (λ k1 → Lim  c2 (λ k2 → indMax (f1 k1) (f2 k2)))
%       indMax-lim2R f1 f2 = extLim  _ _ (λ k1 → indMax-limR  _ (f1 k1))

%     --Attempt to have an idempotent version of indMax

%       nindMax : Tree → ℕ → Tree
%       nindMax t ℕ.zero = Z
%       nindMax t (ℕ.suc n) = indMax (nindMax t n) o

%       nindMax-mono : ∀ {t1 t2 } n → t1 ≤ t2 → nindMax t1 n ≤ nindMax t2 n
%       nindMax-mono ℕ.zero lt = ≤-Z
%       nindMax-mono {t1 = t1} {t2} (ℕ.suc n) lt = indMax-mono {t1 = nindMax t1 n} {t2 = t1} {t1' = nindMax t2 n} {t2' = t2} (nindMax-mono n lt) lt

%     --
%       indMax∞ : Tree → Tree
%       indMax∞ t = OℕLim (λ n → nindMax t n)


%       indMax-∞lt1 : ∀ t → indMax (indMax∞ o) t ≤ indMax∞ o
%       indMax-∞lt1 t = ≤-limiting  _ λ k → helper (Iso.fun CℕIso k)
%         where
%           helper : ∀ n → indMax (nindMax t n) t ≤ indMax∞ o
%           helper n = ≤-cocone  _ (Iso.inv CℕIso (ℕ.suc n)) (subst (λ sn → nindMax t (ℕ.suc n) ≤ nindMax t sn) (sym (Iso.rightInv CℕIso (suc n))) (≤-refl _))
%         -- helper (ℕ.suc n) = ≤-cocone  _ (CℕfromNat (ℕ.suc (ℕ.suc n))) (subst (λ sn → indMax (indMax (nindMax t n) o) t ≤ nindMax t sn) (sym (Cℕembed (ℕ.suc n)))
%         --   {!!})
%         --

%       -- nindMax-idem-absorb : ∀ t n → indMax t o ≤ t → nindMax t n ≤ o
%       -- nindMax-idem-absorb t ℕ.zero lt = ≤-Z
%       -- nindMax-idem-absorb t (ℕ.suc n) lt = indMax-monoL (nindMax-idem-absorb t n lt) ≤⨟ lt
%       -- indMax∞-idem-absorb : ∀ {t} → indMax t o ≤ t → indMax∞ t ≤ o
%       -- indMax∞-idem-absorb lt = ≤-limiting  (λ x → nindMax _ (CℕtoNat x)) (λ k → nindMax-idem-absorb _ (CℕtoNat k) lt)

%       indMax-∞ltn : ∀ n t → indMax (indMax∞ o) (nindMax t n) ≤ indMax∞ o
%       indMax-∞ltn ℕ.zero t = indMax-≤Z (indMax∞ o)
%       indMax-∞ltn (ℕ.suc n) t =
%         ≤-trans (indMax-monoR {t1 = indMax∞ o} (indMax-commut (nindMax t n) o))
%         (≤-trans (indMax-assocL (indMax∞ o) t (nindMax t n))
%         (≤-trans (indMax-monoL {t1 = indMax (indMax∞ o) o} {t2 = nindMax t n} (indMax-∞lt1 o)) (indMax-∞ltn n o)))

%       indMax∞-idem : ∀ t → indMax (indMax∞ o) (indMax∞ o) ≤ indMax∞ o
%       indMax∞-idem t = ≤-limiting  _ λ k → ≤-trans (indMax-commut (nindMax t (Iso.fun CℕIso k)) (indMax∞ o)) (indMax-∞ltn (Iso.fun CℕIso k) o)


%       indMax∞-self : ∀ t → t ≤ indMax∞ o
%       indMax∞-self t = ≤-cocone  _ (Iso.inv CℕIso 1) (subst (λ x → t ≤ nindMax t x) (sym (Iso.rightInv CℕIso 1)) (≤-refl _))

%       indMax∞-idem∞ : ∀ t → indMax t o ≤ indMax∞ o
%       indMax∞-idem∞ t = ≤-trans (indMax-mono (indMax∞-self o) (indMax∞-self o)) (indMax∞-idem o)

%       indMax∞-mono : ∀ {t1 t2} → t1 ≤ t2 → (indMax∞ t1) ≤ (indMax∞ t2)
%       indMax∞-mono lt = extLim  _ _ λ k → nindMax-mono (Iso.fun CℕIso k) lt



%       nindMax-≤ : ∀ {t} n → indMax t o ≤ t → nindMax t n ≤ o
%       nindMax-≤ ℕ.zero lt = ≤-Z
%       nindMax-≤ {o = o} (ℕ.suc n) lt = ≤-trans (indMax-monoL {t1 = nindMax t n} {t2 = o} (nindMax-≤ n lt)) lt

%       indMax∞-≤ : ∀ {t} → indMax t o ≤ t → indMax∞ t ≤ o
%       indMax∞-≤ lt = ≤-limiting  _ λ k → nindMax-≤ (Iso.fun CℕIso k) lt

%       -- Convenient helper for turing < with indMax∞ into < without
%       indMax<-∞ : ∀ {t1 t2 o} → indMax (indMax∞ (t1)) (indMax∞ t2) <o t → indMax t1 t2 <o o
%       indMax<-∞ lt = ≤∘<-in-< (indMax-mono (indMax∞-self _) (indMax∞-self _)) lt

%       indMax-<Ls : ∀ {t1 t2 t1' t2'} → indMax t1 t2 <o indMax (↑ (indMax t1 t1')) (↑ (indMax t2 t2'))
%       indMax-<Ls {t1} {t2} {t1'} {t2'} = indMax-sucMono {t1 = t1} {t2 = t2} {t1' = indMax t1 t1'} {t2' = indMax t2 t2'}
%         (indMax-mono {t1 = t1} {t2 = t2} (indMax-≤L) (indMax-≤L))

%       indMax∞-<Ls : ∀ {t1 t2 t1' t2'} → indMax t1 t2 <o indMax (↑ (indMax (indMax∞ t1) t1')) (↑ (indMax (indMax∞ t2) t2'))
%       indMax∞-<Ls {t1} {t2} {t1'} {t2'} =  <∘≤-in-< (indMax-<Ls {t1} {t2} {t1'} {t2'})
%         (indMax-mono {t1 = ↑ (indMax t1 t1')} {t2 = ↑ (indMax t2 t2')}
%           (≤-sucMono (indMax-monoL (indMax∞-self t1)))
%           (≤-sucMono (indMax-monoL (indMax∞-self t2))))


%       indMax∞-lub : ∀ {t1 t2 o} → t1 ≤ indMax∞ t → t2 ≤ indMax∞  t → indMax t1 t2 ≤ (indMax∞ o)
%       indMax∞-lub {t1 = t1} {t2 = t2} lt1 lt2 = indMax-mono {t1 = t1} {t2 = t2} lt1 lt2 ≤⨟ indMax∞-idem _

%       indMax∞-absorbL : ∀ {t1 t2 o} → t2 ≤ t1 → t1 ≤ indMax∞ t → indMax t1 t2 ≤ indMax∞ o
%       indMax∞-absorbL lt12 lt1 = indMax∞-lub lt1 (lt12 ≤⨟ lt1)

%       indMax∞-distL : ∀ {t1 t2} → indMax (indMax∞ t1) (indMax∞ t2) ≤ indMax∞ (indMax t1 t2)
%       indMax∞-distL {t1} {t2} =
%         indMax∞-lub {t1 = indMax∞ t1} {t2 = indMax∞ t2} (indMax∞-mono indMax-≤L) (indMax∞-mono (indMax-≤R {t1 = t1}))


%       indMax∞-distR : ∀ {t1 t2} → indMax∞ (indMax t1 t2) ≤ indMax (indMax∞ t1) (indMax∞ t2)
%       indMax∞-distR {t1} {t2} = ≤-limiting  _ λ k → helper {n = Iso.fun CℕIso k}
%         where
%         helper : ∀ {t1 t2 n} → nindMax (indMax t1 t2) n ≤ indMax (indMax∞ t1) (indMax∞ t2)
%         helper {t1} {t2} {ℕ.zero} = ≤-Z
%         helper {t1} {t2} {ℕ.suc n} =
%           indMax-monoL {t1 = nindMax (indMax t1 t2) n} (helper {t1 = t1} {t2} {n})
%           ≤⨟ indMax-swap4 {indMax∞ t1} {indMax∞ t2} {t1} {t2}
%           ≤⨟ indMax-mono {t1 = indMax (indMax∞ t1) t1} {t2 = indMax (indMax∞ t2) t2} {t1' = indMax∞ t1} {t2' = indMax∞ t2}
%             (indMax∞-lub {t1 = indMax∞ t1} (≤-refl _) (indMax∞-self _))
%             (indMax∞-lub {t1 = indMax∞ t2} (≤-refl _) (indMax∞-self _))


%       indMax∞-cocone : ∀  {c : ℂ} (f : El c → Tree) k →
%         f k ≤ indMax∞ (Lim  c f)
%       indMax∞-cocone f k =  indMax∞-self _ ≤⨟ indMax∞-mono (≤-cocone  _ k (≤-refl _))

%       -- indMax* : ∀ {n} → Vec Tree n → Tree
%       -- indMax* [] = Z
%       -- indMax* (x ∷ os) = indMax x (indMax* os)

%       -- indMax*-≤L : ∀ {n o} {os : Vec Tree n} → t ≤ indMax* (o ∷ os)
%       -- indMax*-≤L = indMax-≤L

%       -- indMax*-≤R : ∀ {n o} {os : Vec Tree n} → indMax* os ≤ indMax* (o ∷ os)
%       -- indMax*-≤R {o = o} = indMax-≤R {t1 = o}

%       -- indMax*-≤-n : ∀ {n} {os : Vec Tree n} (f : Fin n) → lookup f os ≤ indMax* os
%       -- indMax*-≤-n {os = t ∷ os} Fin.zero = indMax*-≤L {o = o} {os = os}
%       -- indMax*-≤-n {os = t ∷ os} (Fin.suc f) = indMax*-≤-n f ≤⨟ (indMax*-≤R {o = o} {os = os})

%       -- indMax*-swap : ∀ {n} {os1 os2 : Vec Tree n} → indMax* (zipWith indMax os1 os2) ≤ indMax (indMax* os1) (indMax* os2)
%       -- indMax*-swap {n = ℕ.zero} {[]} {[]} = ≤-Z
%       -- indMax*-swap {n = ℕ.suc n} {t1 ∷ os1} {t2 ∷ os2} = indMax-monoR {t1 = indMax t1 t2} (indMax*-swap {n = n}) ≤⨟ indMax-swap4 {t1 = t1} {t1' = t2} {t2 = indMax* os1} {t2' = indMax* os2}

%       -- indMax*-mono : ∀ {n} {os1 os2 : Vec Tree n} → foldr (λ (t1 , t2) rest → (t1 ≤ t2) × rest) Unit (zipWith _,_ os1 os2) → indMax* os1 ≤ indMax* os2
%       -- indMax*-mono {ℕ.zero} {[]} {[]} lt = ≤-Z
%       -- indMax*-mono {ℕ.suc n} {t1 ∷ os1} {t2 ∷ os2} (lt , rest) = indMax-mono {t1 = t1} {t1' = t2} lt (indMax*-mono {os1 = os1} {os2 = os2} rest)

%     -- orec : ∀  (P : Tree → Set ℓ)
%     --   → ((x : Tree) → (rec : (y : Tree) → (_ : ∥ y <o x ∥₁) → P y ) → P x)
%     --   → ∀ {t} → P o
%     -- orec P f = induction (λ x rec → f x rec) _
%     --   where open WFI (ordWFProp)


%     -- oPairRec : ∀  (P : Tree → Tree → Set ℓ)
%     --   → ((x1 x2 : Tree) → (rec : (y1 y2 : Tree) → (_ : (y1 , y2) <oPair (x1 , x2)) → P y1 y2 ) → P x1 x2)
%     --   → ∀ {t1 t2} → P t1 t2
%     -- oPairRec P f = induction (λ (x1 , x2) rec → f x1 x2 λ y1 y2 → rec (y1 , y2)) _
%     --   where open WFI (oPairWF)
% \end{code}





The usefulness of Brouwer trees is in defining well-founded recursion, but first we need on ordering on trees.


A strict order can be defined in terms of the successor function. This strict relation is a well quasi-order: it has no infinite descending chains, and hence
can be used as a decreasing metric
for recursive functions.

    TODO compare with cubical,
    TODO look up original trees

\subsubsection{Brouwer Trees}
\label{model:subsec:brouwer}
Unfortunately, it was not immediately apparent that any of the
``off-the-shelf'' formulations of constructive ordinals satisfied our critera,
so we built our own formulation. We use a refined version of Brouwer trees:
There is a zero ordinal, a successor operator, and a limit ordinal that is the least upper bound
of the image for a function from a code's type to ordinals.
We borrow the trick of taking the limits over types (or in our case, codes) from \citet{ionchyMasters},
since this lets us easily model the sizes of dependent functions and pairs.
The ordering on these trees is defined following \citet{KrausFX21}:
\begin{flalign*}
  data\ \_\le_o\_ : Ord -> Ord -> \sType{}\ where\nl
  \qquad {\le_o}Z : (o : Ord) -> OZ \le_o o  \nl
  \qquad {\le_o}sucMono : (o_1 : Ord) -> (o_2 : Ord) -> o_1 \le_o o_2 -> O{\uparrow}\  o_1 \le_o O{\uparrow}\  o_2  \nl
  \qquad {\le_o}cocone : (c : \bC\ \ell) -> (o : Ord) -> (f : El_{Approx}\ c -> Ord)
    -> (k : El_{Approx}\ c)
    \nl\qquad\qquad -> o \le_o f\ k  -> o \le_o OLim\ c\ f\nl
    \qquad {\le_o}limiting : (o : Ord) -> (c : \bC\ \ell) -> (f : El_{Approx}\ c -> Ord)
    \nl\qquad\qquad -> ((k : El_{Approx}\ c) -> f\ k \le_o o) -> OLim\ c\ f \le_o o\\\nl
    %
    o_1 <_o o_2 = O{\uparrow}\ o_1 \le_o o_2
  \end{flalign*}
  That is, zero is the smallest ordinal, the successor is monotone,
  and the limit is actually the least upper bound of the function's image.
Unlike \citet{KrausFX21}, we do not include transitivity as a rule, but we can prove
it as a theorem.
The maximum function on ordinals is defined as follows:
\begin{flalign*}
  max_o : Ord -> Ord -> Ord\nl
  max_o\ OZ\ o = o \nl
  max_o\ o\ OZ = o \nl
  max_o\ (O{\uparrow}\ o_1)\ (O{\uparrow}\ o_2) = O{\uparrow}\ (max_o\ o_1\ o_2)\nl
  max_o\ (OLim\ c\ f)\ o = OLim\ c\ (\lambda k \ldotp max_o\ (f\ k)\ o)\nl
  max_o\ o\ (OLim\ c\ f) = OLim\ c\ (\lambda k \ldotp max_o\ o\ (f\ k))
\end{flalign*}
Long but straightforward proofs show that $max_{o}$ is monotone
and computes and upper bound of its inputs.
It reduces when given $\s{O{\uparrow}}$ for both inputs, so it is strictly monotone.
However, we cannot prove that it is a least upper-bound.
The problem is that limits are not well-behaved with respect to the maximum.
We could instead construct the maximum using $\s{OLim}$, but this version
would not be strictly monotone.

\subsubsection{A Least Upper Bound}

We solve the problems with $\s{max_{o}}$ using a type of sizes, which include only the subset of
ordinals that are idempotent with respect to the maximum. We can then
define a type of sizes with the same interface as ordinals.
\begin{flalign*}
  Size : \sType{} \nl
  Size = (o : Ord) \times (max_o\ o\ o \le_o o)\\\nl
%
  \_\bigvee\_ : Size -> Size -> Size\nl
  s_1 \bigvee s_2 = (max_o\ (fst\ s_1)\ (fst\ s_2), \ldots)\\\nl
  %
  SZ : Size\nl
  SZ = (OZ , {\le_o}Z)\\\nl
  S{\uparrow} : Size -> Size\nl
  S{\uparrow}\ s =  (O{\uparrow}\ (fst\ s), {\le_s}sucMono\ (snd\ s))
\end{flalign*}
Critically, the sizes are closed under the maximum operation: if $\s{max_{o}\ o_{1}\ o_{1} \le_{o}\ o_{1}}$
and $\s{max_{o}\ o_{2}\ o_{2} \le_{o}\ o_{2}}$, then
$\s{max_{o}\ (max_{o}\ o_{1}\ o_{2})\ (max_{o}\ o_{1}\ o_{2}) \le (max_{o}\ o_{1}\ o_{2})}$.
% We omit the proof term, because it is long but boring.
Zero and a successor operation for sizes are easily implemented.
The difficulty is constructing a limit operator for sizes, since
the self-idempotent ordinals are not closed under $\s{OLim}$.
Our trick is to take the limit of maxing an ordinal with itself.
We assume we have a code $\s{C\bN}$ whose elements have an injection $\s{Cto\bN}$ into $\s{\bN}$.
The natural numbers can be defined as an inductive type, but in our Agda development we add it as an
extra code constructor.
Having numbers lets us take the maximum of an ordinal with itself infinitely many times, resulting in an ordinal
that is as large as the original but idempotent with respect to $\s{max_{o}}$.
\begin{flalign*}
  nmax : Ord -> \bN -> Ord \nl
  nmax\ o\ Z\ = OZ\nl
  nmax\ o\ (S\ n) = omax\ (nmax\ o\ n)\ o\\ \nl
  %
  max\infty : Ord -> Ord\nl
  max\infty\ o = OLim\ C\bN\ (\lambda k \ldotp nmax\ o\ (Cto\bN\ k)) \\ \nl
  %
  max\infty Idem : \{ o : Ord \} -> max_o\ (max\infty\ o)\ (max\infty\ o) \le_o (max\infty\ o)\\\nl
  %
  SLim : (c : \bC\ \ell) -> (El_{Approx}\ c -> Size) -> Size\nl
  SLim\ c\ f = (max\infty\ (OLim\ c\ (\lambda k \ldotp fst\ (f\ k))) ,\ max\infty Idem )
\end{flalign*}

Sizes satisfy all the same inequalities as raw ordinals,
listed in \cref{model:fig:size-order}.
The monotonicity of $\s\bigvee$ follows from the monotonicity of $\s{max_{o}}$,
and the idempotence  of $\s\bigvee$ follows by the definition of $\s{Size}$.
Monotonicity, idempotence, and transitivity of $\s{\le_{s}}$ together imply
that $\s\bigvee$ is a least upper bound,
and strict monotonicity follows from the strict monotonicity of $\s{max_{o}}$.
\begin{figure}
  \begin{flalign*}
    \_\le_s\_ : Size -> Size -> Size\nl
    s_1 \le_s s_2 = (fst\ s_1) \le_o (fst\ s_2)\\\nl
    %
    \_<_s\_ : Size -> Size -> Size\nl
    s_1 <_s s_2 = (S{\uparrow}\ s_1) \le_s s_2\\\nl
    %
    {\le_s}trans : (s_1 : Size) -> (s_2 : Size) -> (s_3 : Size) ->\nl
    \qquad (s_1 \le_s s_2) -> (s_2 \le_s s_3) -> (s_1 \le_s s_3)\nl
    {\le_s}Z : (s : Size) -> SZ \le_s s  \nl
    {\le_s}sucMono : (s_1 : Size) -> (s_2 : Size) -> s_1 \le_s s_2 -> S{\uparrow}\  s_1 \le_s S{\uparrow}\  s_2  \nl
    {\le_s}cocone : (c : \bC\ \ell) -> (s : Size) -> (f : El_{Approx}\ c -> Size)
    -> (k : El_{Approx}\ c)
    \nl\qquad -> s \le_s f\ k  -> s \le_s SLim\ c\ f\nl
    {\le_s}limiting : (s : Size) -> (c : \bC\ \ell) -> (f : El_{Approx}\ c -> Size)
    \nl\qquad -> ((k : El_{Approx}\ c) -> f\ k \le_s s) -> SLim\ c\ f \le_s s\\\nl
    %
    \bigvee\le : (s_1 : Size) -> (s_2 : Size) -> (s_1 \le_s s_1 \bigvee s_2) \times (s_2 \le_2 s_1 \bigvee s_2)\nl
    \bigvee mono : (s_1 : size) -> (s_2 : Size) -> (s'_1 : Size) -> (s'_2 : Size) \nl
    \qquad -> (s_1 \le_s s'_1) -> (s_2 \le_s s'_2) -> (s_1 \bigvee s_2) \le_s (s'_1 \bigvee s'_2)\nl
    \bigvee idem : (s : Size) -> (s \bigvee s) \le_s s\nl
    \bigvee lub : (s_1 : size) -> (s_2 : size) -> (s : Size) \nl
    \qquad -> (s_1 \le_s s) -> (s_2 \le_s s) -> (s_1 \bigvee s_2 \le_s s)
  \end{flalign*}
  \caption{Ordering on Sizes}
  \label{model:fig:size-order}
\end{figure}

