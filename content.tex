% !TEX root =  main.tex


\subsubsection{Brouwer Trees}
\label{model:subsec:brouwer}
Unfortunately, it was not immediately apparent that any of the
``off-the-shelf'' formulations of constructive ordinals satisfied our critera,
so we built our own formulation. We use a refined version of Brouwer trees:
\begin{agda}
  data\ Ord\ : \sType{}\ where\nl
  \qquad OZ : Ord\nl
  \qquad O{\uparrow} : Ord -> Ord\nl
  \qquad OLim : (c : \bC\ \ell) -> (El_{Approx}\ c -> Ord) -> Ord
\end{agda}
There is a zero ordinal, a successor operator, and a limit ordinal that is the least upper bound
of the image for a function from a code's type to ordinals.
We borrow the trick of taking the limits over types (or in our case, codes) from \citet{ionchyMasters},
since this lets us easily model the sizes of dependent functions and pairs.
The ordering on these trees is defined following \citet{KrausFX21}:
\begin{agda}
  data\ \_\le_o\_ : Ord -> Ord -> \sType{}\ where\nl
  \qquad {\le_o}Z : (o : Ord) -> OZ \le_o o  \nl
  \qquad {\le_o}sucMono : (o_1 : Ord) -> (o_2 : Ord) -> o_1 \le_o o_2 -> O{\uparrow}\  o_1 \le_o O{\uparrow}\  o_2  \nl
  \qquad {\le_o}cocone : (c : \bC\ \ell) -> (o : Ord) -> (f : El_{Approx}\ c -> Ord)
    -> (k : El_{Approx}\ c)
    \nl\qquad\qquad -> o \le_o f\ k  -> o \le_o OLim\ c\ f\nl
    \qquad {\le_o}limiting : (o : Ord) -> (c : \bC\ \ell) -> (f : El_{Approx}\ c -> Ord)
    \nl\qquad\qquad -> ((k : El_{Approx}\ c) -> f\ k \le_o o) -> OLim\ c\ f \le_o o\\\nl
    %
    o_1 <_o o_2 = O{\uparrow}\ o_1 \le_o o_2
  \end{agda}
  That is, zero is the smallest ordinal, the successor is monotone,
  and the limit is actually the least upper bound of the function's image.
Unlike \citet{KrausFX21}, we do not include transitivity as a rule, but we can prove
it as a theorem.
The maximum function on ordinals is defined as follows:
\begin{agda}
  max_o : Ord -> Ord -> Ord\nl
  max_o\ OZ\ o = o \nl
  max_o\ o\ OZ = o \nl
  max_o\ (O{\uparrow}\ o_1)\ (O{\uparrow}\ o_2) = O{\uparrow}\ (max_o\ o_1\ o_2)\nl
  max_o\ (OLim\ c\ f)\ o = OLim\ c\ (\lambda k \ldotp max_o\ (f\ k)\ o)\nl
  max_o\ o\ (OLim\ c\ f) = OLim\ c\ (\lambda k \ldotp max_o\ o\ (f\ k))
\end{agda}
Long but straightforward proofs show that $max_{o}$ is monotone
and computes and upper bound of its inputs.
It reduces when given $\s{O{\uparrow}}$ for both inputs, so it is strictly monotone.
However, we cannot prove that it is a least upper-bound.
The problem is that limits are not well-behaved with respect to the maximum.
We could instead construct the maximum using $\s{OLim}$, but this version
would not be strictly monotone.

\subsubsection{A Least Upper Bound}

We solve the problems with $\s{max_{o}}$ using a type of sizes, which include only the subset of
ordinals that are idempotent with respect to the maximum. We can then
define a type of sizes with the same interface as ordinals.
\begin{agda}
  Size : \sType{} \nl
  Size = (o : Ord) \times (max_o\ o\ o \le_o o)\\\nl
%
  \_\sansbigvee\_ : Size -> Size -> Size\nl
  s_1 \sansbigvee s_2 = (max_o\ (fst\ s_1)\ (fst\ s_2), \ldots)\\\nl
  %
  SZ : Size\nl
  SZ = (OZ , {\le_o}Z)\\\nl
  S{\uparrow} : Size -> Size\nl
  S{\uparrow}\ s =  (O{\uparrow}\ (fst\ s), {\le_s}sucMono\ (snd\ s))
\end{agda}
Critically, the sizes are closed under the maximum operation: if $\s{max_{o}\ o_{1}\ o_{1} \le_{o}\ o_{1}}$
and $\s{max_{o}\ o_{2}\ o_{2} \le_{o}\ o_{2}}$, then
$\s{max_{o}\ (max_{o}\ o_{1}\ o_{2})\ (max_{o}\ o_{1}\ o_{2}) \le (max_{o}\ o_{1}\ o_{2})}$.
% We omit the proof term, because it is long but boring.
Zero and a successor operation for sizes are easily implemented.
The difficulty is constructing a limit operator for sizes, since
the self-idempotent ordinals are not closed under $\s{OLim}$.
Our trick is to take the limit of maxing an ordinal with itself.
We assume we have a code $\s{C\bN}$ whose elements have an injection $\s{Cto\bN}$ into $\s{\bN}$.
The natural numbers can be defined as an inductive type, but in our Agda development we add it as an
extra code constructor.
Having numbers lets us take the maximum of an ordinal with itself infinitely many times, resulting in an ordinal
that is as large as the original but idempotent with respect to $\s{max_{o}}$.
\begin{agda}
  nmax : Ord -> \bN -> Ord \nl
  nmax\ o\ Z\ = OZ\nl
  nmax\ o\ (S\ n) = omax\ (nmax\ o\ n)\ o\\ \nl
  %
  max\infty : Ord -> Ord\nl
  max\infty\ o = OLim\ C\bN\ (\lambda k \ldotp nmax\ o\ (Cto\bN\ k)) \\ \nl
  %
  max\infty Idem : \{ o : Ord \} -> max_o\ (max\infty\ o)\ (max\infty\ o) \le_o (max\infty\ o)\\\nl
  %
  SLim : (c : \bC\ \ell) -> (El_{Approx}\ c -> Size) -> Size\nl
  SLim\ c\ f = (max\infty\ (OLim\ c\ (\lambda k \ldotp fst\ (f\ k))) ,\ max\infty Idem )
\end{agda}

Sizes satisfy all the same inequalities as raw ordinals,
listed in \cref{model:fig:size-order}.
The monotonicity of $\s\bigvee$ follows from the monotonicity of $\s{max_{o}}$,
and the idempotence  of $\s\bigvee$ follows by the definition of $\s{Size}$.
Monotonicity, idempotence, and transitivity of $\s{\le_{s}}$ together imply
that $\s\bigvee$ is a least upper bound,
and strict monotonicity follows from the strict monotonicity of $\s{max_{o}}$.
\begin{figure}
  \begin{agda}
    \_\le_s\_ : Size -> Size -> Size\nl
    s_1 \le_s s_2 = (fst\ s_1) \le_o (fst\ s_2)\\\nl
    %
    \_<_s\_ : Size -> Size -> Size\nl
    s_1 <_s s_2 = (S{\uparrow}\ s_1) \le_s s_2\\\nl
    %
    {\le_s}trans : (s_1 : Size) -> (s_2 : Size) -> (s_3 : Size) ->\nl
    \qquad (s_1 \le_s s_2) -> (s_2 \le_s s_3) -> (s_1 \le_s s_3)\nl
    {\le_s}Z : (s : Size) -> SZ \le_s s  \nl
    {\le_s}sucMono : (s_1 : Size) -> (s_2 : Size) -> s_1 \le_s s_2 -> S{\uparrow}\  s_1 \le_s S{\uparrow}\  s_2  \nl
    {\le_s}cocone : (c : \bC\ \ell) -> (s : Size) -> (f : El_{Approx}\ c -> Size)
    -> (k : El_{Approx}\ c)
    \nl\qquad -> s \le_s f\ k  -> s \le_s SLim\ c\ f\nl
    {\le_s}limiting : (s : Size) -> (c : \bC\ \ell) -> (f : El_{Approx}\ c -> Size)
    \nl\qquad -> ((k : El_{Approx}\ c) -> f\ k \le_s s) -> SLim\ c\ f \le_s s\\\nl
    %
    \sansbigvee\le : (s_1 : Size) -> (s_2 : Size) -> (s_1 \le_s s_1 \sansbigvee s_2) \times (s_2 \le_2 s_1 \sansbigvee s_2)\nl
    \sansbigvee mono : (s_1 : size) -> (s_2 : Size) -> (s'_1 : Size) -> (s'_2 : Size) \nl
    \qquad -> (s_1 \le_s s'_1) -> (s_2 \le_s s'_2) -> (s_1 \sansbigvee s_2) \le_s (s'_1 \sansbigvee s'_2)\nl
    \sansbigvee idem : (s : Size) -> (s \sansbigvee s) \le_s s\nl
    \sansbigvee lub : (s_1 : size) -> (s_2 : size) -> (s : Size) \nl
    \qquad -> (s_1 \le_s s) -> (s_2 \le_s s) -> (s_1 \sansbigvee s_2 \le_s s)
  \end{agda}
  \caption{Ordering on Sizes}
  \label{model:fig:size-order}
\end{figure}

